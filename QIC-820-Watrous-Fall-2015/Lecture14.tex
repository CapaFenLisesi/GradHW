\documentclass{article}
\usepackage[]{amsmath}
\usepackage[]{dsfont}
%\usepackage[]{bbmath}
\newtheorem{theorem}{Theorem}
\begin{document}
\section{Entanglement}

We define entanglement in terms of what it is not. We will only discuss
bipartite entanglement. Consider two registers X and Y. We will define an
operator $ P \in Pos(\mathcal{X} \otimes \mathcal{Y}) $  is separable (or not
entangled if and only if there exists a collection of positive integer m and
collections $\{Q_1, \ldots, Q_m\} \subset Pos(\mathcal{X})$ and
$\{R_1,\cdots,R_m\} \subset Pos(\mathcal{Y})$ such that 
\[ 
        P = \sum_{k=1}^m Q_k \otimes R_k 
\]

Introducing some notation: Sep($\mathcal{X}:\mathcal{Y}$) denotes
the set of all such operators, and $SepD(\mathcal{X}:\mathcal{Y}) =
Sep(\mathcal{X}:\mathcal{Y} \cap D(\mathcal{X} \otimes \mathcal{Y}$.

Suppose $\rho \in Sep D(\mathcal{X}:\mathcal{Y})$. It must
be possible, then, to write 
\[ 
        \rho = \sum_{k=1}^m p(k) \sigma_k \otimes \xi_k 
\]

for p a probability vector and 
\begin{align*}
    \{\sigma_1,\ldots,\sigma_m\} \subseteq D(\mathcal{X}) \\
    \{\xi_1,\cdots,\xi_m \} \subseteq D(\mathcal{Y})
\end{align*}

We could write $\rho \in SepD(\mathcal{X}:\mathcal{Y})$ as 
\[ 
        \rho = \sum_{k=1}^N q(k) x_k x_k^* \otimes y_k y_k^* 
\]

for q a probability vector and 
\begin{align*}
    x_1,\ldots,x_N \in S(\mathcal{X}) \\
    y_1,\cdots,y_N \in S(\mathcal{Y})
\end{align*}

In other words we could write 
\[
        SepD(\mathcal{X}:\mathcal{Y}) = conv\{ x x^* \otimes y y^*: x \in
        S(\mathcal{X}), y \in S(\mathcal{Y}) \}
\]

In general, an expression of the form $\rho = \sum_{k=1}^N q(k) x_k x_k^*
\otimes y_k y_k^*$ will always exist for $N \le
dim(L(\mathcal{X}\otimes\mathcal{Y})) = dim(\mathcal{X})^2dim(\mathcal{Y})^2$

This follows from Carathedory's theorem. In general, we cann take $N \le
dim(\mathcal{X} \otimes \mathcal{Y})$.

There are some nice properties for $SepD(\mathcal{X}:\mathcal{Y})$
\begin{enumerate}
    \item convex
    \item compact
\end{enumerate}

The set $Sep(\mathcal{X}:\mathcal{Y})$ also has some nice properties:
\begin{enumerate}
    \item It's a convex cone (I can multiple it by any real number and it will
        still be separable.
    \item It's closed. 
\end{enumerate}

We will now address a basic fact from convex analysis: The Horodecki criterion.

\subsection{The Horodecki Criterion}
Consider some $ K \subseteq Herm(\mathcal{X}) $ is a closed, convex cone and $A
\in Herm(\mathcal{X)}$ such that $A \notin K$. From this we can conclude that
there must exist some Hermitian operator $H \in Herm(\mathcal{X})$ such that
\begin{enumerate}
    \item $\langle H , X \rangle \ge 0 \quad \forall X \in K$
    \item $\langle H , A \rangle < 0$
\end{enumerate}

This is as an example of a separating hyperplane theorem.

\begin{theorem}[Horodecki Criterion]
$\mathcal{X}$ and $\mathcal{Y}$ are complex Euclidean spaces 
        $P \in Pos(\mathcal{X} \otimes \mathcal{Y})$. It holds that $P \in
    Sep(\mathcal{X}:\mathcal{Y})$ if and only if:
    \[ 
            (\Phi \otimes \mathds{1}_{L(\mathcal{Y})}(P) \in Pos(\mathcal{Y}
                \otimes \mathcal{Y}) 
    \]
    for all positive and unital maps $\Phi \in T(\mathcal{X},\mathcal{Y})$.

    Proof: Suppose that $P \in Sep(\mathcal{X}:\mathcal{Y})$ so that 
    \[ 
            P = \sum_{k=1}^m Q_k \otimes R_k 
    \]
    
    for $Q_1, \ldots, Q_m \in Pos(\mathcal{X})$ and $R_1,\ldots,R_m \in
    Pos(\mathcal{Y})$. If $\Phi \in T(\mathcal{X},\mathcal{Y})$ is positive then
    \[ 
            \Phi \otimes \mathds{1}_{L(\mathcal{Y})}(P) = \sum_{k=1}^m \Phi(Q_k)
            \otimes R_k \in Sep(\mathcal{Y}:\mathcal{Y}) \subseteq
            Pos(\mathcal{Y}:\mathcal{Y}) 
    \]
    
    It must be the case that $\Phi(Q_k)$ is positive semidefinite.

    Now let us prove the other implication. We will prove this using the
    contrapositive form:

    If
    \[ 
            P\notin Sep(\mathcal{X}:\mathcal{Y}) \text{then} (\Phi \otimes
                \mathds{1}_{L(\mathcal{Y})}(P) \notin Pos(\mathcal{Y}\otimes
                \mathcal{Y}) 
    \]

    for at least one choice of a positive unital map $\Phi$.
    
    For now, let's forget about the unital condition.

    Assume $P \notin Sep(\mathcal{X}:\mathcal{Y})$. There must exist $H \in
    Herm(\mathcal{X}\otimes \mathcal{Y})$ such that:

    \begin{enumerate}
        \item $\langle H , P \rangle < 0$
        \item $\langle H , Q \otimes R \rangle \ge 0 \quad \forall Q\in
            Pos(\mathcal{X}) , R \in Pos(\mathcal{Y})$
    \end{enumerate}
    
    Let $\Phi \in T(\mathcal{Y},\mathcal{X})$ be the map for which:
    \[ 
            J(\Phi) = H 
    \]
    
    Assume, $Q, R \ge 0$.

    \begin{align*}
        0 & \le \langle H , Q \otimes R \rangle \\
          &= \langle J(\Phi) , Q \otimes R \rangle \\
          &= \langle \left( \Phi \otimes \mathds{1}_{L(\mathcal{Y})} \right)
        \left( vec(\mathds{1}_{\mathcal{Y}} vec(\mathds{1}_{\mathcal{Y}})^*
            \right) , Q \otimes R \rangle \\
            &= \langle vec(\mathds{1}_{\mathcal{Y}}
            vec(\mathds{1}_{\mathcal{Y}})^* , \Phi^*(Q) \otimes R \rangle  \\
            &= vec(\mathds{1}_{\mathcal{Y}}^* \left( \Phi^*(Q) \otimes R \right)
            vec(\mathds{1}) \\
            = Tr \left( \Phi^*(Q) R^T \right) &= \langle \overline{R} , \Phi^*(Q) \rangle 
    \end{align*}
    
    Thus, $\Phi^*(Q)$ is positive semidefinite (for every $Q \in
        Pos(\mathcal{X})$ and $\Phi^*$ is positive.

        Now consider
        \[ 
                (\Phi^* \otimes \mathds{1}_{L(\mathcal{Y})}(P) 
        \]
        
        Hopefully this is not positive semidefinite. It suffices to find $u \in
        \mathcal{Y} \otimes \mathcal{Y}$ such that 
        \[ 
                u^* \left( \Phi^* \otimes \mathds{1}_{L(\mathcal{Y})}(P) u < 0 \right) 
        \]

        Take $u = vec(\mathds{1})$.

        \begin{align*}
            &vec(\mathds{1}) \left( \Phi^* \otimes
            \mathds{1}_{L(\mathcal{Y})}(P) vec(\mathds{1}) \right) \\
            &= \langle vec(\mathds{1})vec(\mathds{1})^* , \left( \Phi* \otimes
        \mathds{1}_{L(\mathcal{Y})} \right)(P) \rangle \\
        &= \langle J(\Phi) , P \rangle \\
        &= \langle H , P \rangle \
        < 0
        \end{align*}
        And, we're done. For the unital condition see the book or notes.
\end{theorem}
        \subsection{Useful things this theorem}
        Let's now consider a separable ball around identity.
        \[ 
                \omega = \mathds{1}_{\mathcal{X}}/n \otimes
                \mathds{1}_{\mathcal{Y}}/m 
        \]

        There exists a ball around $\omega$ where everything is separable.

        Lemma: Let $\mathcal{X}$ and $ \mathcal{Y} = \mathds{C}^\Sigma$ be
        complex Euclidean spaces. Let
        \[ 
                A = \sum_{a,b \in \Sigma} A_{a,b} \otimes E_{a,b} \in
                L(\mathcal{X},\mathcal{Y}) 
        \]
        
        be any operator. It holds that $\left| \left| A \right| \right|^2 \le
        \sum_{a,b} \left| \left| A_{a,b} \right| \right|^2$.

        Another lemma (Russo-Dye):

        Let $\Phi \in T(\mathcal{X},\mathcal{Y})$ be a positive and unital map.
        It holds that 
        \[\left| \left| \Phi(X) \right| \right| \le \left| \left| X
        \right| \right| \quad \forall X \in L(\mathcal{X})\] 

        Consider another theorem: $A \in Herm(\mathcal{X} \otimes \mathcal{Y})$
        where $\left| \left| A \right| \right|_2 \le 1$. It holds that 
        \[ 
                \mathds{1}_{\mathcal{X}} \otimes \mathds{1}_{\mathcal{X}} - A
                \in Sep(\mathcal{X}:\mathcal{Y}) 
        \]
        
        Let $\Phi \in T(\mathcal{X},\mathcal{Y})$ be any positive and unital
        map. We can write
        \[ 
                A = \sum_{a,b} A_{a,b} \otimes E_{a,b} 
        \]
        
        so that

        \[ 
                \left( \Phi \otimes \mathds{1} \right)(A) = \sum_{a,b}
                \Phi(A_{a,b}) \otimes E_{a,b}
        \]
        
        We know that 
        \[ 
                \left| \left| \left(\Phi \otimes \mathds{1}\right)(A) \right|
                \right| ^2 \le \sum_{a,b} \left| \left| \Phi(A_{a,b}) \right|
            \right|^2
        \]
        
        by the first lemma. So this last expression bears the following
        relationships with the next expressions:

        \begin{align*}
            & \le \left| \left| A_{a,b} \right| \right|^2 \\
            & \le \left| \left| A_{a,b} \right| \right|^2_2 \\
            & = \left| \left| A \right| \right|^2
            & \le 1
        \end{align*}
        So, finaly,

        \[ 
                \left( \Phi \otimes \mathds{1} \right) (A) \le \mathds{1}
                \otimes \mathds{1}
        \]
        
        
\end{document}
