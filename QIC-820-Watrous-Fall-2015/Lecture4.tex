\begin{document}

\begin{section}[Choi Representation]

    We will define a linear map J: T(\scriptx,\scripty) \mapsto
    L(\scripty \tensor \scriptx) as follows (assuming \scriptx =
    \mathC^\Sigma:

    J(\Phi) = \sum_{a,b \elementsof \Sigma} \Phi(E_{a,b})\tensor
    E_{a,b}. This is the Choi representation of \Phi.

    Alternatively, J(\Phi) = (\Phi \tensor
    \math1_{L(\scriptx)}(vec(\math1_{\scriptx})vec(\math1_{\scriptx})^*)
    These are equivalent because vec(\math1) = vec(\sum_{a\elementsof
    \Sigma} E_{a,a}) = \sum_{a\elementsof \Sigma} e_a \tensor e_a

    vec(\math1_\scriptx)vec(\math1_\scriptx)^* = \sum_{a,b} E_{a,b} \tensor
    E_{a,b} \elementsof L(\scriptx \tensor \scriptx)

    (\Phi \tensor
    \math1_{L(\scriptx)})(vec(\math1_{\scriptx})vec(\math1_\scriptx)^*)\elementsof
    L(\scripty \tensor \scriptx)

    J(\Phi) \element L(\scripty \tensor \scriptx) is a linear mapping. It
    does not implement \Phi directly but it includes all the information
    about \Phi. It applies \Phi in a different way (it doesn't vec the
    objects \Phi acts on).

    We cane recover the action of \Phi from J(\Phi) like this: \Phi(X) =
    Tr_\scriptx(J(\Phi)(\math1_\scripty \tensor X^T)). This is a
    basis-dependent representation. With a different choice of basis you
    would get a different operator.

    If you want to know if \Phi is completely positive you can check to
    make sure that J is positive semidefinite. They are if and only if
    conditions.

    e.g. \Delta \element T(\scriptx) \elementof T(\scriptx,\scriptx). This
    operator will leave the diagonal alone and zero-out everything else.

    It's the action on E_{a,b} = {E_{a,a} if a = b, 0 if a \ne b}
    So, \Delta {\alpha \beta; \gamma \delta} = {\alpha 0 ; 0 \delta}

    J(\Delta) = \Sigma_{a,b \elementsof \Sigma} \Delta(E_{a,b}) \tensor
    E_{a,b} = \Sigma_{a\elementsof \Sigma} E_{a,a}\tensor E_{a,a}
\end{section}
\begin{section}[Kraus Representations]

    Suppose \Phi \elementof T(\scriptx,\scripty), and suppose we have
    {A_a : a \elementof \Sigma} \containedin {B_a : a \element \Sigma}
    \containedin L(\scriptx,\scripty) so that \sum_{a\elementof \Sigma}
    A_a X B_a^* = \Phi(X) (for all X \element L(\scriptx)). Then, the
    sum would be a Kraus representation of \Phi. A popular choice of
    representation will be that where A_a = B_a for all a \element
    \Sigma. This is possible if and only if \Phi is completely positive. 
\end{section}
\begin{section}[Stinespring Representations]
    Suppose \Phi \element T(\scriptx,\scripty) and also suppose A,B
    \element L(\scriptx, \scripty \tensor \scriptz) (for \scriptz a
    complex Euclidean space) so that \Phi(X) = Tr_\scriptz(AXB^*) (for
    all X \element L(\scriptx)). Then, this a Stinespring representation
    of X.
\end{section}

Proposition: Let \Phi \element T(\scriptx,\scripty) for complex
Euclidean spaces \scriptx and \scripty and let {A_a: a \element \Sigma},
{B_\a : a \element \Sigma} \containedin L(\scriptx, \scripty) be
operators (for any alphabet \Sigma). The following are equivalent:

1. \Phi(X) = \sum_{a\element \Sigma} A_a X B_a^* for all X \element
L(\scriptx)
2. For \scriptz = \mathc^\Sigma and A,B \elementsof L(\scriptx,\scripty
\tensor \scriptz) given by A = \sum_{a \elementsof \Sigma} A_a \tensor
e_a, B = \sum_{a\element \Sigma} B_a \tensor e_a we have \Phi(X) =
Tr_\scriptz(A X B^*) for all X \element L(\scriptx).
3. It holds that K(\Phi) = \sum_{a \element \Sigma} A_a \tensor
\overbar{B_a}
4. It holds that J(\Phi) = \sum_{a \element \Sigma} vec(A_a)vec(B_a)^*.

To prove the above proposition use the formula vec(C X D^*) = C \tensor
\overbar{D}vec(X). Specifically, vec(C) = (C \tensor \math1)vec(\math1).
To show 1 implies 2 is not so hard. To show that 3 implies 4 is not that
hard.

\begin{theorem}[Characterization of Completely Positive Maps]
    Let \scriptx and \scripty be complex Euclidean spaces and let \Phi
    \element T(\scriptx,\scripty). The following are equivalent:
    1. \Phi is completely positive.  \Phi \tensor \math1_{L(\scriptz)}
    is positive for all choice of \scriptz
    2. \Phi \tensor \math1_{L(\scriptx)} is positive.
    3. J(\Phi) \element Pos(\scripty \tensor \scriptx) (is positive
    semi-definite)
    4. There exists operators {A_a: a \element \Sigma} \containedin
    L(\scriptx,\scripty) so that \Phi(X) = \sum_{a \element \Sigma} A_a
    X A_a^* (for all X \element L(\scriptx))
    5. Statement 4 holds if |\Sigma| = rank(J(\Phi))
    6. There exists an operator A \element L(\scriptx, \scripty \tensor
    \scriptz) such that \Phi(X) = Tr_\scriptz (A X A^*) for all X
    \element L(\scriptx)
    7. Statement 6 holds for dim(\scriptz) = rank(J(\Phi))

\end{theorem}
Proof of the above statements:

(1) \rightarrow (2) \rightarrow (3) \rightarrow (5) \rightarrow (4)
\rightarrow (1)
(5) \rightarrow (7) \rightarrow (6) \rightarrow (1)

To show (2) \rightarrow (3) you can use the Choi representation to show
that a positive operator * semidefinite operator begets a positive
semidefinite operator.

(4)\rightarrow (1): Suppose \scriptw is any complex euclidean space and
P \element Pos(\scriptx \tensor \scriptw). Assuming that 94) is true we
have \Phi \tensor \math1_{L(\scriptw)}(P) = \sum_{a \element \Sigma}
(A_a \tensor \math1_\scriptw) P (A_a \tensor \math1_\scriptw)^*.

By the form of the operators in the sum, we have a positive semidefinite
operator since positive semidefinite operators are closed over summation
(something about cones and higher dimensional spaces and convexity).

(3) \rightarrow (5) ((6) \rightarrow (1) is similar): J(\Phi) \element
Pos(\scripty \tensor \scriptx). By the spectral theorem we can write
J(\Phi) = \sum_{k = 1}^{r = rank(J(\Phi))} u_k u_k^*, u_1,\cdots,u_r
\element \scripty \tensor \scriptx. The eigenvalues have been sucked
into the u_k's. Now we write u_k = vec(A_k) for A_1,\cdots,A_r \element
L(\scriptx,\scripty). J(\Phi) = \sum_{k=1}^r vec(A_k)vec(A_k)^8. So,
\Phi(X) = \sum_{k=1}^r A_k X A_k^* for all X \element L(\scriptx) by the
proposition.

\begin{theorem}
    \Phi \element T(\scriptx,\scripty)
    The following are equivalent:
    1. \Phi is trace-preserving
    2. \Phi^* is a unital map (a map that maps identity to identity)
    3. Tr_\scripty(J(\Phi)) = \math1_\scriptx
    4. For every (any) Kraus representation \Phi(X) = \sum_{a \element
    \Sigma} A_a X B_a^* we have \sum_{a \element \Sigma} B_a^* A_a =
    \math1_\scriptx
    5. For every (any) Stinespring representation: \Phi(X) = Tr_\scriptz
    (A X B^*) it holds that B^*A = \math1_\scriptx
\end{theorem}

Corollary: The following are equivalent:
1. \Phi \element C(\scriptx,\scripty) (\Phi is a channel)
2. J(\Phi) \element Pos(\scripty \tensor \scriptx) and
Tr_\scripty(J(\Phi)) = \math1_{scriptx}
3. \Phi(X) = \sum_{a \element \Sigma} A_a X A_a^* and \sum_{a \element
\Sigma} A_a^* A_a = \math1_\scriptx
4. \Phi(X) = Tr_\scriptz(A X A^*) and A^*A = \math1_\scriptx

In general if you have two spaces where dim(\scripty) \ge dim(\scriptx)
and A \element L(\scriptx,\scripty) satisfies A^*A = \math1_\scriptx
then we say that A is an isometry (or linear isometry).

Imagine we have two vectiors x,y \element \scriptx then <Ax, Ay> =
<A^*Ax,y> = <x,y> so it preserves the inner product. Additionally ||Ax||
= ||x|| (you can see this by allowing x = y). U(\scriptx, \scripty)
will, in this course, denote all isometries from \scriptx \mapsto
\scripty.

Example: We have already talked about \Delta \element T(\scriptx) (the
completely dephasing map): J(\Delta) = \sum_{a\element \Sigma} E_{a,a}
\tensor E_{a,a}. It's obvious that it preserves trace. If you want a
Kraus representation you could write this:

\Delta(X) = \sum_{a \element \Sigma} E_{a,a} X E_{a,a}^* (the adjoint
star being unnecessary). This will throw away the off-diagonal entries
and leave the diagonal entries alone.

The natural representation of \Delta looks like: K(\Delta) =
\sum_{a\element \Sigma} E_{a,a} \tensor E_{a,a}. In this case the Choi
and natural representations are equal.

If X = \mathc^\scriptz then 
K(\Delta) = {1 0 0 0; 0 0 0 0; 0 0 0 0; 0 0 0 1} vec(\alpha \beta;
\gamma \delta) \rightarrow {1 0 0 0; 0 0 0 0; 0 0 0 0; 0 0 0 1} {\alpha;
\beta; \gamma; \delta} = {\alpha; 0; 0; \delta}.

Consider, now, a unitary channel. That is, one that follows: \Phi(X) = U
X U^* for U \element \mathu(\scriptx). This is a Kraus representation
because U^*U = \math1_\scriptx. This is also a Stinespring
representation where \scriptz is one-dimensional. What does the Choi representation of
this channel look like?
J(\Phi) = vec(U)vec(U)^*. To check to make sure this works: Tr_\scripty
(vec(U)vec(U)^*) = (U^* U)^T = \math1^T = \math1.
