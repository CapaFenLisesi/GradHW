\documentclass{article}
\usepackage[]{amsmath,amssymb,dsfont,amsthm}
\newtheorem{theorem}{Theorem} 
\begin{document}
\section*{Nielsen's Theorem}
Nielsen's theorem tells us when we can take a pure state $ uu^* $  and use an LOCC to turn
it into $vv^*$.

\begin{theorem}
    Let $\mathcal{X}$ and $ \mathcal{Y} $ be complex Euclidean spaces ancd let $
    u,v \in \mathcal{X}\otimes \mathcal{Y}$  be unit vectors. The following
    statements are equivalent.
    \begin{enumerate}
        \item There exists an alphabet $\Sigma$ and 2 collections of operators:
            \begin{enumerate}
                \item $ \{A_a: a \in \Sigma \} \subset L(\mathcal{X}) $ such that
                    $ \sum_{a\in\Sigma} A_a^* A_a = \mathds{1}_{\mathcal{X}} $
                \item $ \{V_a: a\in \Sigma\} \subset U(\mathcal{Y}) $  
            \end{enumerate}
            such that $vv^* = \sum_{a\in\Sigma} (A_a \otimes V_a) u u^* (A_a
            \otimes V_a)^*$
        \item Similar to number 1, there exist collections:
            \begin{enumerate}
                \item $ \{B_a: a \in \Sigma\} \subset L(\mathcal{Y})$ such that $
                    \sum_{a\in\Sigma} B^*_a B_a = \mathds{1}_{\mathcal{Y}} $
                \item $ \{U_a : a \in \Sigma \} \subset U(\mathcal{X}) $ 
            \end{enumerate}
            such that $ vv^* = \sum_{a\in\Sigma} \left( U_a \otimes B_a
            \right)u u^* \left( U_a \otimes B_a \right)^* $
        \item There exists a separable channel 
            \[ 
                \Phi \in SepC(\mathcal{X}:\mathcal{Y}) \equiv
                SepC(\mathcal{X},\mathcal{X}:\mathcal{Y},\mathcal{Y}) 
            \]
            such that 
            \[ 
                v v^* = \Phi(uu^*) 
            \]
        \item   $ Tr_{\mathcal{Y}} \left( u u^* \right) \prec Tr_{\mathcal{Y}}(vv^*) $
            \\ (equivalent to $ Tr_{\mathcal{X}}\left( uu^* \right) \prec
            Tr_{\mathcal{X}}\left( vv^* \right) $ )
    \end{enumerate}


\end{theorem}
    1 implies 3 is trivial. 2 implies 3 is trivial. 4 implies 1 and 4
    implies 2 are equivalent by symmetry. We will just prove 4 implies 2, first,
    and 3 implies 4, second.

    Proving that $4 \rightarrow 2$:
\begin{proof}

    First, let $X,Y \in L(\mathcal{Y},\mathcal{X})$ be the uniquely defined
    operators for which
    \[ 
        u = vec(X) \quad , \quad v = vec(Y) 
    \]
    
    we have 

    \[ 
        Tr_{\mathcal{X}} \left( uu^* \right) = X X^* \quad , \quad
        Tr_{\mathcal{Y}} \left( vv^* \right) = Y Y^*
    \]

    Singular value decomposition of $X: X = \sum_{k=1}^r s_k x_k y_k^*$.

    Now to show $4 \rightarrow 2$:

    Statement 4 is equivalent to:

    \[ 
        XX^* \prec YY^* 
    \]
    
    This means there must exist a mixed-unitary channel mapping $YY^*$ to $ XX^* $ 

    \[ 
        \sum_{a\in\Sigma} p(a)W_a YY^*W_a^* = XX^*
    \]
    
    for $ p \in P(\Sigma), \{W_a, a\in\Sigma\} \subset U(\mathcal{X})$. Let $ Z
    = \mathcal{C}^{\Sigma}$ and define $ M \in L(\mathcal{Y}\otimes \mathcal{Z},
    \mathcal{X})$ as:

    \[ 
        M = \sum_{a\in\Sigma} \sqrt{p(a)} W_a Y \otimes e_a^* 
    \]
    
    It's not obvious that we should define some operator M in this way. This is
    the trick we're using to make it all work. This is the creative part of the
    proof. We're looking at something we care about (the expression involving
    $p(a)W_a YY^*$ and we're defining something that's like the square root of
    that expression. One way to see why $M$ is nice is to look at $MM^*$:

    \[ 
        MM^* = \left( \sum_a \sqrt{p(a)} W_a Y \otimes e_a^* \right)
        \left( \sum_a \sqrt{p(a)} Y^* W_a^* \otimes e_a \right) 
    \]
    
    We have no cross terms because of orthogonality, so this gives us:

    \[ 
        \sum_{a\in\Sigma} p(a) W_a YY^* W_a = XX^* 
    \]
    
    Since $MM^* = XX^*$ we are motivated to think about a singular value
    decomposition of X. M must have singular value decomposition:

\[ 
    M = \sum_{k=1}^r s_k x_k w_k^* 
\]

for some orthonormal set $\{ w_1,\ldots,w_r\} \subset \mathcal{Y}\otimes
\mathcal{Z}$. Since the $w$s are in a larger space than the $y$s used in the
spectral decomposition of X we know there must exist some isometry that maps $X$
to $M$. Thus, we'll consider some isometry $V \in
U(\mathcal{Y},\mathcal{Y}\otimes \mathcal{Z})$ such that 
\[ 
    V y_k = w_k \quad, \enskip \text{for } k=1,\ldots,r 
\]

Note $ XV^* = M $. Now, we'll define:

\[ 
    U_a = W_a^* \quad,\quad B_a = \left( \mathds{1}_{\mathcal{Y}}\otimes e_a^*
    \right) \overline{V} 
\]

for each $a \in \Sigma$.

Let us verify that these choices work (i.e. they satisfy statement 2).

\begin{align*}
    \sum_{a\in\Sigma} B_a^* B_a &= \sum_{a\in\Sigma} V^T \left(
    \mathds{1}_{\mathcal{Y}} \otimes e_a \right) \left( \mathds{1}_{\mathcal{Y}}
\otimes e_a^*\right)\overline{V} \\
&= V^T \left( \mathds{1}_{\mathcal{Y}} \otimes   \mathds{1}_{\mathcal{X}}
\right) \overline{V} \\
&= V^T \overline{V} \\
&= \mathds{1} \quad \text{Because V is an isometry}
\end{align*}


Note that:

\begin{align*}
    W_a^* X B_a^T &= W_a^* XV^* \left( \mathds{1} \otimes e_a \right) \\
                  &= W_a^* M \left( \mathds{1}\otimes e_a \right)\\
                  &= \sqrt{p(a)} W_a^* W_a Y \\
                  &= \sqrt{p(a) Y}
\end{align*}

\begin{align*}
    \sum_{a \in \sigma} \left( U_a \otimes B_a \right) u u^* \left( U_a \otimes
    B_a\right)^* &= \sum_{a\in\Sigma} \left( W_a^* \otimes B_a \right) vec(X)
        vec(X)^* \left( W_a^* \otimes B_a \right)^* \\
        &= \sum_{a\in\Sigma} vec \left( W_a^* X B_a^T \right) vec(W_a^* X
        B_a^T)^* \\
        &= \sum_a p(a) vec(Y) vec(Y)^* \\
        &= vv^*
\end{align*}

and we're done.
\end{proof}

Now we'll prove that $3 \rightarrow 4$.

\begin{proof}

    Let the separable map $ \Phi $ be written:

    \[ 
        \Phi(Z) = \sum_{a \in \Sigma} \left( A_a \otimes B_a \right) Z \left(
        A_a \otimes B_a \right)^*  \quad,\quad \forall z\in L(\mathcal{X}\otimes
        \mathcal{Y})
    \]

    We assume that:

    \[ 
        vec(Y)vec(Y^*) = \Phi(vec(X)vec(X^*)) 
    \]
    
    We can expand this to mean:

    \begin{align*}
        vec(Y)vec(Y^*) &= \sum_{a\in\Sigma} \left( A_a \otimes B_a \right)
        vec(X)vec(X^*) \left( A_a \otimes B_a \right)^* \\
        &= \sum_{a\in\Sigma} vec(A_a X B_a^T) vec(A_a X B_a^T)^*
    \end{align*}

    it must be the case that $vec(A_a X X_a^T) = \alpha_a Y$ for some
    choice of $\{ \alpha_a : a \in \Sigma\} \subset \mathbb{C}$.

    So suppose that we have 
    \[ 
        w w^* = \sum_{k=1}^N z_k z_k^* 
    \]
    
    it must be that $ Z_k = a_k w \enskip,\enskip \forall k $.

    To understand why this must be the case consider:

    \[ 
        u w w^* u^* = \left| \langle w , u \rangle \right|^2
    \]
    
    We could substitute $ ww^* = \sum_{k=1}^N z_k z_k^*$:

    \begin{align*}
        u \sum_{k=1}^N z_k z_k^* u^* &= \sum_{k=1}^N \left| \langle u , z_k
    \rangle \right|^2	
    \end{align*}
    
    If $ u $ is perpendicular to $w$ then it must be the case that $\langle w ,
    u \rangle = 0$. Thus, $ \langle u , z_k \rangle = 0$ for all $z_k$. Thus,
    $z_k = \alpha w$ for all $k$.

    Returning to the proof, we want to show that $XX^* \prec YY^*$ which is
    equivalent to:
    \[ 
        \sum_{k=1}^m \lambda_k \left( YY^* \right) \ge \sum_{k=1}^m \lambda_k
        \left( X X^* \right)
    \]
    
    or:

    \[ 
        \sum_{k=m}^n \lambda_k(YY^*) \le \sum_{k=m}^n \lambda_k \left( XX^* \right) 
    \]
    
    for $1 \le m \le n = dim(\mathcal{X})$.

    Introducing some notation:

    \[ 
        X_m = \sum_{k=m}^r s_k x_k y_k^* 
    \]
    
    When $m>r$ the sum is identically zero.

    Now:

    \begin{align*}
        \sum_{k=m}^n \lambda_k(YY^*) &= \sum_{k=m}^n \sum_{a\in\Sigma} p(a)
        \lambda_k \left( YY^* \right) \\
        &= \sum_{k=m}^n \sum^{}_{a\in\Sigma} \lambda_k \left( p(a) YY^* \right)
        \\
        &= \sum^{}_{a\in\Sigma} \sum^{n}_{k=m}  \lambda_k \left( A_a X B_a^T
    \overline{B_a} X^* A_a^* \right)
    \end{align*}
    
    Now, we're summing the $n-m+1$ smallest eigenvalues of the operator. This
    can be any smaller than:
    \[ 
        \sum^{}_{a\in\Sigma} \langle \Pi_{a,m}, A_a X B_a^T \overline{B_a} X^* A_a^* \rangle 
    \]

    for any choice of $\{\Pi_{a,m}\}$ with $rank \left( \Pi_{a,m} \right) \ge
    n-m+1$.

    Because we know this is true for any $\Pi$ with at least that rank then we
    can choose any $\Pi$ which is good for us. Let's take $\Pi_{a,m}$ to be the
    projection onto the orthogonal complement of:
    
    \[ 
        span\{A_a x_1, \ldots, A_a x_{m-1} \} 
    \]
    
    So, we have:

    \begin{align*}
        \langle \Pi_{a,m}, A_a X B_a^T \overline{B_a} X^* A_a^* \rangle &=
        \langle \Pi_{a,m}, A_a X_m B_a^T \overline{B_a} X_m^* A_a^* \rangle \\
        &\le Tr \left( A_a X_m B_a^T \overline{B_a} X_m^* A_a^* \right) \\
        &= Tr \left( \Phi \left( vec(X_m) vec(X_m)^* \right) \right) \\
        &= Tr \left( vec(X_m) vec(X_m)^* \right) \\
        &= Tr \left( X_m X_m^* \right) \\
        &= \sum_{k=m}^n s_k^2 \\
        &= \sum^{n}_{k=m} \lambda_k \left( X X^* \right)
    \end{align*}
\end{proof}
    
\end{document}
