\documentclass{article}
\usepackage[]{amsmath,dsfont,braket,amsthm}
\newtheorem{theorem}{Theorem}
\newtheorem{lemma}{Lemma}
\newtheorem{corollary}{Corollary}[theorem]
\begin{document}
\section*{Channel Discrimination}
Let's imagine a situation where we have two channels $\Phi_0, \Phi_1 \in
C(\mathcal{X},\mathcal{Y})$. Suppose we are tasked with distinguishing between
$\Phi_0$ and $\Phi_1$. We are only allowed \textbf{one use} of $\Phi_a$ (for $a
\in \{0,1\}$) and we want to determine a.

Recall the Holevo-Holstrom theorem for states. If we were given two states
$\rho_0,\rho_1 \in D(\mathcal{X}), \lambda \in [0,1]$:

\[ 
    \max_{\mu : \{0,1\} \rightarrow Pos(\mathcal{X})} \left( \lambda \langle \mu(0) , \rho_0 \rangle + \left( 1-\lambda
\right) \langle \mu(1) , \rho_1 \rangle  \right) = \frac{1}{2} \left( 1 + \left|
\left| \lambda \rho_0 + \left( 1-\lambda \right) \rho_1 \right|\right| \right) 
\]

The question is: Is there a similar relationship between channels? To discover
this, we need to consider first using the channel once with some state $\sigma
\in D(\mathcal{X})$ obtaining either $\Phi_0(\sigma)$ or $\Phi_1(\sigma)$ after
applying the channel once. Now, we can use the Holevo-Holtrom theorem to
distinguish between $\Phi_0(\sigma) = \rho_0$ and $ \Phi_1(\sigma) = \rho_1 $.
Thus, the problem is reduced to:

\[ 
    \max_{\sigma \in D(\mathcal{X})} \left\{ \frac{1}{2} \left( 1 + \left| \left|
    \lambda \Phi_0(\sigma) - (1-\lambda) \Phi_1(\sigma) \right| \right| \right) \right\} 
\]

Another way to attack this problem would be to choose to apply to the channel
some state $\rho$ which is entangled with the environment such that $\rho \in
D(\mathcal{X}) \otimes D(\mathcal{Z})$ is piped into $\Phi_a \in D(\mathcal{X})$
so that what comes out is $ \left( \Phi_a \otimes \mathds{1}_{L(\mathcal{Z})}
\right) (\rho) $. We could consider this setup for any choice of $\mathcal{Z}$
and $\rho \in D(\mathcal{X} \otimes \mathcal{Z})$.

So, consider the following examples where $n \ge 2$.

\[
    \Phi_0(X)= \frac{1}{n+1} \left( Tr(\mathcal{X})\mathds{1} + X^T \right)
\]

\[ 
    \Phi_1(X) = \frac{1}{n-1} \left( Tr(X)\mathds{1} - X^T \right)
\]

These are known as the Werner-Holevo channels. You can show that these are
completely positive and trace preserving.

\[ 
    J(\Phi_0) = \frac{1}{n+1} \left( \mathds{1}\otimes \mathds{1} + W \right) =
    \frac{2}{n+1}\Pi_0
\]

\[ 
    J(\Phi_1) = \frac{1}{n-1} \left( \mathds{1}\otimes \mathds{1} - W \right) =
    \frac{2}{n-1}\Pi_1
\]

where $\Pi_0$ and $\Pi_1$ are the projections onto the symmetric and the
antisymmetric subspaces, respectively.

Suppose we evaluate $ \Phi_0 $ and $ \Phi_1 $ on:

\begin{align*}
    \tau =& \frac{1}{n} vec(\mathds{1}) vec(\mathds{1})^*  \\
    =& \frac{1}{n} \sum_{a,b} E_{a,b} \otimes E_{a,b}
\end{align*}

Now:

\begin{align*}
    \left( \Phi_0 \otimes \mathds{1}_{L(\mathcal{X})}\right)(\tau) =& \frac{2}{n(n+1)}
\Pi_0   \\
\left( \Phi_1 \otimes \mathds{1}_{L(\mathcal{X})}\right)(\tau) =&
\frac{2}{n(n-1)} \Pi_1
\end{align*}

These can be distinguished perfectly because of the projection onto orthogonal
subspaces. The states are orthogonal because of the projection so by
Holevo-Holstrom the states can be distinguished perfectly.

On the other hand, using the naive strategy:

\[ 
    \Phi_0(\sigma) - \Phi_1(\sigma) = \frac{-2}{n^2-1} \mathds{1} +
    \frac{2n}{n^2-1} \sigma^T
\]

Now, 

\[ 
    \left| \left| \Phi_0(\sigma) - \Phi_1(\sigma) \right| \right|_1 \le
    \frac{4}{n+1} 
\]

Note that for very large N it becomes almost impossible to distinguish the
states.  So, this example demonstrates that you really should use the auxiliary
system $\mathcal{Z}$ in order to distinguish the channels really well.

\section*{Completely Bounded Trace Norm (Diamond Norm)}

We will begin by defining a norm that isn't exactly the norm we want to consider
(the diamond norm). Consider the induced trace norm: Imagine you have some map
(not necessarily a channel) $\Phi \in T(\mathcal{X},\mathcal{Y})$.

\[
    \left| \left| \Phi  \right| \right|_1 = \max \{ \left| \left| \Phi(X) \right|
            \right|_1 : X \in L(\mathcal{X}), \left| \left| X \right| \right|_1
    \le 1 \}
\]

This norm has a few basic properties:

\begin{enumerate}
    \item The norm is submultiplicative:
        \[
            \left| \left| \Psi \Phi \right| \right|_1 \le \left| \left| \Psi
            \right| \right|_1 \left| \left| \Phi \right| \right|_1
        \]
    \item By convexity, we have:
        \[ 
            \left| \left| \Phi \right| \right|_1 = \max \{\left| \left|
            \Phi(uv^*) \right| \right|_1 : u,v \in S(\mathcal{X}) \}
        \]
    \item $ \left| \left| \Phi(X) \right| \right|_1 \le \left| \left| \Phi
        \right| \right|_1 \left| \left| X \right| \right| $ 
\end{enumerate}

So, this norm doesn't look bad but it is bad (it's terrible). One reason it's
particularly bad is that it does not ``get along well''/respect tensor products.

For example, consider the transpose mapping $T(X)=X^T$ for $T \in
T(\mathcal{X})$.

\[ 
    \left| \left| T \right| \right|_1 = 1 
\]

Consider $\mathds{1}_{L(\mathcal{X})} \in T(\mathcal{X})$.

\[ 
    \left| \left| \mathds{1}_{L(\mathcal{X})} \right| \right|_1 = 1 
\]

Now, consider $\left| \left| T \otimes \mathds{1}_{L(\mathcal{X})} \right|
\right|_1 = dim(\mathcal{X})$. Demonstrating this:

\begin{align*}
    \left| \left| \left( T \otimes \mathds{1}_{L(\mathcal{X})} (\tau) \right)
    \right| \right|_1 =& \enskip \left| \left| \frac{1}{n} W \right| \right|_1 \\
            =& \enskip \frac{1}{n} n^2 \\
            =& \enskip n
\end{align*}

Another reason this is a bad norm is that, in general, calculating the norm is
NP-hard (John acknowledges that he doesn't have a reference for this, he
believes that it is).

The completely bounded trace norm is a ``stabilized'' version of the induce
trace norm.

Definition: $\Phi \in T(\mathcal{X},\mathcal{Y})$ is:

\[ 
    \left| \left| \left| \Phi \right| \right| \right|_1 \equiv \left| \left|
    \Phi \otimes \mathds{1}_{L(\mathcal{X})} \right| \right|_1 
\]

where the identity lives in the input space of $ \Phi $. It's not clear why the
identity is only on the input space of $\Phi$ and not on multiple copies of this
space or something else. We will show that this now works well with the tensor
product and is easy to compute and has almost every property that you can ask of
a norm. This is also called the ``diamond norm'' leading to the alternative
notation: $ \left| \left| \Phi \right| \right|_\diamond $.

\begin{lemma}
Consider the following lemma: Consider the following operator and the following
space $\Phi \in T(\mathcal{X},\mathcal{Y}),
\mathcal{Z}$. For any choice of unit vectors:

\[ 
    u, v \in \mathcal{X} \otimes \mathcal{Z} 
\]

there exist unit vectors

\[ 
    x,y \in \mathcal{X} \otimes \mathcal{X} 
\]

such that

\[ 
    \left| \left| \left( \Phi \otimes \mathds{1}_{L(\mathcal{Z})} (uv)^* \right)
    \right| \right|_1 = \left| \left| \Phi \otimes \mathds{1}_{L(\mathcal{X})}
    (xy^*) \right| \right|_1
\]

If $u=v$ we may take $x=y$.
\end{lemma}

\begin{proof}
    Consider Schmidt decompositions of $u$ and $v$.

    \[ 
    u = \sum_{a\in\Sigma} \sqrt{p(a)} x_a \otimes z_a 
    \]
    
    \[ 
    v = \sum_{a\in\Sigma} \sqrt{q(a)} y_a \otimes w_a 
    \]

    Take:

    \[ 
        x = \sum_{a \in \Sigma} \sqrt{p(a)} x_a \otimes x_a 
    \]
    
    \[ 
        y = \sum_{a \in \Sigma} \sqrt{q(a)} y_a \otimes y_a 
    \]
 
    We can now consider:

    \[ 
        \sum_{a,b} \sqrt{p(a)q(b)} \Phi(x_a y_b^*) \otimes z_a w_b^* 
    \]
    
    
    \[ 
        \sum_{a,b} \sqrt{p(a)q(b)} \Phi(x_a y_b^*) \otimes x_a y_b^* 
    \]

    But, these are the same up to an isometry so the trace norms must be the
    same. Thus, the fact that the lemma is true has been proven.
\end{proof}

\begin{theorem}
    For any $\Phi \in T(\mathcal{X},\mathcal{Y})$ and $\mathcal{Z}$ with
    $dim(\mathcal{Z}) \ge dim(\mathcal{X})$ we have

    \[
        \Phi \otimes \mathds{1}_{L(\mathcal{Z})} = \left| \left| \left| \Phi
        \right| \right| \right|_1
    \]
\end{theorem}

\begin{corollary}
    Consider some $\Phi_0 \in T(\mathcal{X}_0,\mathcal{Y}_0)$ and $\Phi_1 \in
    T(\mathcal{X}_1,\mathcal{Y}_1)$. Then,
    $ \left| \left| \left| \Phi_0 \otimes \Phi_1 \right| \right| \right| =
    \left| \left| \left| \Phi_0 \right| \right|\right|_1 \left| \left| \left|
    \Phi_1 \right| \right| \right|_1 $.
\end{corollary}

\begin{proof}
    Choose $X_0 \in L(\mathcal{X}_0 \otimes \mathcal{Z}_0)$, $ X_1 \in
    L(\mathcal{X}_1,\mathcal{Z}_1) $ so that

    \[ 
        \left| \left| X_0 \right| \right|_1 = 1 = \left| \left| X_1 \right|
        \right|_1 
    \]
    
    and

    \[ 
        \left| \left| \left| \Phi_0 \right| \right| \right|_1 = \left| \left|
        \left( \Phi_0 \otimes \mathds{1}_{L(\mathcal{Z}_0)}\right) (X_0) 
\right| \right|_1
    \]
    
    
    \[ 
        \left| \left| \left| \Phi_1 \right| \right| \right|_1 = \left| \left|
        \left( \Phi_1 \otimes \mathds{1}_{L(\mathcal{Z}_1)}\right) (X_1) 
\right| \right|_1
    \]

    Now, we have:

    \begin{align*}
        \left| \left| \left| \Phi_0 \otimes \Phi_1 \right| \right| \right|_1
        \ge& \left| \left| \Phi_0 \otimes \Phi_1 \otimes
\mathds{1}_{L(\mathcal{Z}_0 \otimes \mathcal{Z}_1)} \right| \right|_1 \\
=& \left| \left| \Phi_0 \otimes \mathds{1}_{L(\mathcal{Z}_0)} \otimes \Phi_1
\otimes \mathds{1}_{L(\mathcal{Z}_1)} \right| \right|_1 \\
\ge& \left| \left| \left( \Phi_0 \otimes \mathds{1}_{L(\mathcal{Z}_0)} (X_0)
\otimes \left( \Phi_1 \otimes \mathds{1}_{L(\mathcal{Z}_1)}(X_1) \right)\right)
\right| \right|_1 \\
=& \left| \left| \left( \Phi_0 \otimes \mathds{1}_{L(\mathcal{Z}_0)} \right)
(X_0) \right| \right|_1 \cdot \left| \left| \left( \Phi_1 \otimes
\mathds{1}_{L(\mathcal{Z}_1)} \right) (X_1) \right| \right|_1 \\
=& \left| \left| \left| \Phi_0 \right| \right| \right|_1 \cdot \left| \left|
\left| \Phi_1 \right| \right| \right|_1
    \end{align*}
   
    Going the other way: 
    \begin{align*}
        \left| \left| \left| \Phi_0 \otimes \Phi_1 \right|
                \right| \right|_1 =& \left| \left| \Phi_0 \otimes \Phi_1 \otimes
            \mathds{1}_{L(\mathcal{X}_0)} \otimes \mathds{1}_{L(\mathcal{X}_1)} \right|
        \right|_1 \\
        =& \left| \left| \left( \Phi_0 \otimes \mathds{1}_{L(\mathcal{Y}_1)} \otimes
    \mathds{1}_{L(\mathcal{X}_0)} \otimes \mathds{1}_{L(\mathcal{X}_1)} \right)
    \left( \mathds{1}_{L(\mathcal{X}_0)} \otimes \Phi_1 \otimes
    \mathds{1}_{L(\mathcal{X}_0)} \otimes \mathds{1}_{L(\mathcal{X}_1)} \right)
\right| \right|_1 \\
\le& \left| \left| \Phi_0 \otimes \mathds{1}_{L(\mathcal{Y}_1)} \otimes
\mathds{1}_{L(\mathcal{X}_0)} \otimes \mathds{1}_{L(\mathcal{X}_1)}\right|
\right|_1 \cdot \left| \left| \mathds{1}_{L(\mathcal{X}_0)} \otimes \Phi_1
    \otimes \mathds{1}_{L(\mathcal{X}_0)} \otimes \mathds{1}_{L(\mathcal{X}_1)}
\right| \right|_1 \\
\le& \left| \left| \left| \Phi_0 \right| \right| \right|_1 \cdot \left| \left|
\left| \Phi_1 \right| \right| \right|
    \end{align*}
\end{proof}
\end{document}
