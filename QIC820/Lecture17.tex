\documentclass{article}
\usepackage[]{amsmath,amssymb,dsfont,amsthm,braket}
\newtheorem{theorem}{Theorem} 
\begin{document}
\section{PPT Operators (positive partial transpose)}
We will define $\mathcal{X},\mathcal{Y}$ as complex Euclidean space.

\[ 
    PPT(\mathcal{X}:\mathcal{Y}) = \{ P \in Pos(\mathcal{X}\otimes \mathcal{Y})
            : \left( T \otimes \mathds{1}_{L(\mathcal{Y})} \right) (P) \in
            Pos(\mathcal{X}\otimes \mathcal{Y})
\]

Consider $ T \in T(\mathcal{X}), T(X) = X^T$. $T_{\mathcal{X}}, T_{\mathcal{Y}}$
denotes the partial transpose over $\mathcal{x}$ or $\mathcal{Y}$.

Now, 

\begin{align*}
    T_{\mathcal{X}}(P) \ge 0 &\leftrightarrow T_{\mathcal{X}\otimes \mathcal{Y}}
    \left( T_{\mathcal{X}} \left( P \right)\right) \ge 0 \\
    &\leftrightarrow T_{\mathcal{Y}} \left(
        P\right) \ge 0.
\end{align*}

Note these facts:

\begin{enumerate}
    \item If $P \in Sep(\mathcal{X}:\mathcal{Y})$, then $P \in
        PPT(\mathcal{X}:\mathcal{Y})$. 
        \begin{align*}
            P = \sum_{k=1}^m Q_k \otimes R_k \quad,\enskip &Q_1,\ldots,Q_m \in
            Pos(\mathcal{X}) \\
            & R_1, \ldots, R_m \in Pos(\mathcal{Y})
        \end{align*}
        
        \begin{align*}
            \left( T \otimes \mathds{1}_{L \left( \mathcal{Y} \right)}  \right)
            \left( P \right) = \sum_{k=1}^m Q_k^T \otimes R_k \ge 0
        \end{align*}
        
    \item If $P \in Pos(\mathcal{X}\otimes \mathcal{Y}$ is \textbf{not}
        separable, then it might be the case that $P \notin
        PPT(\mathcal{X}:\mathcal{Y})$.

        \[ 
            u = \frac{1}{\sqrt{2}} \left( e_0 \otimes e_0 + e_1 \otimes e_1 \right)
        \]

        \[ 
            \tau = u u^* = \frac{1}{2} \begin{pmatrix}
                1 & 0 & 0 & 1 \\ 0 & 0 & 0 & 0 \\
                0 & 0 & 0 & 0 \\ 1 & 0 & 0 & 1 
            \end{pmatrix} 
        \]
        
        \[ 
            \left( T \otimes \mathds{1}_{L \left( \mathcal{Y} \right)} \right)
            \left( \tau \right) = \frac{1}{2} \begin{pmatrix}
                1 & 0 & 0 & 0 \\ 0 & 0 & 1 & 0 \\
                0 & 1 & 0 & 0 \\ 0 & 0 & 0 & 1 
            \end{pmatrix} \ngeq 0
        \]
        
    \item   There do exist non-separable PPT states
        
        \[ 
            SepD(\mathcal{X}:\mathcal{Y}) \subset PPT(\mathcal{X}:\mathcal{Y})
            \cap D(\mathcal{X}\otimes \mathcal{Y}) \subset
            D(\mathcal{X}:\mathcal{Y})
        \]
\end{enumerate}

For item 3, consider the following example: Allow $ \mathcal{X} =
\mathbb{C}^\mathbb{Z}_3 = \mathcal{Y} $ 

\begin{align*}
    \intertext{Define $u_1,\ldots,u_5 \in \mathcal{X}\otimes \mathcal{Y}$ unit
    vectors as: }
    u_1 &= \ket{0} \otimes \left( \ket{0} - \ket{1} \right) / \sqrt{2} \\
    u_2 &= \ket{2} \otimes \left( \frac{\ket{1}-\ket{2}}{\sqrt{2}} \right) \\
    u_3 &= \left( \frac{\ket{0}-\ket{1}}{\sqrt{2}} \right) \otimes \ket{2} \\
    u_4 &= \left( \frac{\ket{1}-\ket{2}}{\sqrt{2}} \right) \otimes \ket{0} \\
    u_5 &= \left( \frac{\ket{0}+\ket{1}+\ket{2}}{\sqrt{3}} \right) \otimes
    \left( \frac{\ket{0} + \ket{1} + \ket{2}}{\sqrt{3}} \right)
\end{align*}

We can make the following observations:

\begin{enumerate}
    \item $\{u_1,\ldots,u_5\}$ is an orthonormal set.
    \item Each of $u_1,\ldots,u_5 \}$ is a product vector:

        \[ 
            u_k = x_k \otimes y_k 
        \]
        
    \item There does \textbf{not} exist a nonzero vector $k \in x \otimes y$
        that is both a product vector and orthogonal to $\{u_1, \ldots, u_5 \}$.
        Imagine 
        \[ 
            \langle v\otimes w , u_k \rangle \quad,\enskip k=1,\ldots,5 
        \]

        This implies either that $v=0$ or that $w=0$.

        So, $\{u_1,\ldots,u_5\}$ is an unextendable product set (UPB).
\end{enumerate}

Now define:

\[ 
    P = \mathds{1} \otimes \mathds{1} - \sum_{k=1}^5 u_k u_k^* 
\]

(i.e., P is the projection onto the space orthogonal to $ \{u_1,\ldots,u_5\} $).

Let us show that 

\begin{enumerate}
    \item $P \notin Sep(\mathcal{X}:\mathcal{Y})$
    \item $P \in PPT(\mathcal{X}:\mathcal{Y})$
\end{enumerate}

The claim is that $P \in PPT(\mathcal{X}:\mathcal{Y})$

\begin{align*}
    \left( T \otimes \mathds{1}_{L(\mathcal{Y})} \right) \left( P \right)
    &= \left( T \otimes \mathds{1}_{L(\mathcal{Y})} \right) \left( \mathds{1}
\otimes \mathds{1} \right) - \sum_{k=1}^5 T(x_k x_k^*) \otimes y_k y_k^* \\
&= \mathds{1} \otimes \mathds{1} - \sum_{k=1}^5 T(x_k x_k^*) \otimes y_k y_k^*
\\
&= P \in Pos(\mathcal{X}:\mathcal{Y})
\end{align*}

Thus, $P \in PPT(\mathcal{X}:\mathcal{Y})$.

Now, let's claim $ P \in Sep(\mathcal{X}:\mathcal{Y})$, By contradiction, we'll
assume that $P$ \textbf{is} separable.

\[ 
    P = \sum_{k=1}^N v_k v_k^* \otimes w_k w_k^* 
\]

for any choice of $v_1,\ldots,v_N \in \mathcal{X}$, $w_1,\ldots,w_N \in
\mathcal{Y}$ (not necessarily unit vectors).

It holds that

\[ 
    0 = u_j^* P u_j = \sum_{k=1}^N \left| \langle u_j , v_k \otimes w_k \rangle
    \right|^2 
\]

Thus,

\[ 
    \langle u_j , v_k \otimes w_k \rangle  = 0 \quad,\enskip \forall
    j=1,\ldots,5 \enskip,\enskip k=1,\ldots,N 
\]

Finally,

\[ 
v_k \otimes w_k = 0 \quad,\enskip \text{for}\enskip k=1,\ldots,N \quad
\rightarrow \quad P = 0 
\]

But, this contradicts our original premise that P is separable. Thus, P is not
separable.

Consider that X and Y are registers and that $\rho \in
PPT(\mathcal{X}:\mathcal{Y}) \cap D(\mathcal{X}\otimes \mathcal{Y})$ . Then, we
can conclude from this that

\[ 
    E_D(X:Y)_\rho = 0 
\]


We know that $SepD(\mathcal{X}:\mathcal{Y}) \subset PPT(\mathcal{X}:\mathcal{Y})
\subset D(\mathcal{X}:\mathcal{Y})$. But, we can also conclude that $E_D
(\mathcal{X}:\mathcal{Y}) \supset PPT(\mathcal{X}:\mathcal{Y})$. We will show
this. PPT states can be entangled. But, we can show that this entanglement is
useless for certain things, like teleportation. You can't use these to even
approximate a maximally entangled state. For this reason, people sometimes call
the states that exist in $\left( E_D (\mathcal{X}:\mathcal{Y}) \cup
PPT(\mathcal{X}:\mathcal{Y}) \right) \cap SepD(\mathcal{X}:\mathcal{Y})$ ``bound
entangled'' states. It is an open question if $E_D(\mathcal{X}:\mathcal{Y}) =
PPT(\mathcal{X}:\mathcal{Y})$. Don't attempt solving this. You (John) will waste
all of your time.

Let's consider some theorems:

\begin{theorem}
    \[P \in PPT(\mathcal{X}:\mathcal{Y})\]

    \[ 
        \Phi \in SepCP(\mathcal{X},\mathcal{Z}:\mathcal{Y},\mathcal{W})
    \]
    
    Then $\Phi(P) \in PPT(\mathcal{Z}:\mathcal{W})$
\end{theorem}

\begin{proof}
    \[
        \left( T_{\mathcal{Z}} \otimes \mathds{1}_{L(\mathcal{W})} \right) \left(
        \left( A \otimes B\right) X \left( A \otimes B \right)^* \right)
        \quad,\enskip A \in L(\mathcal{X},\mathcal{Z}) \enskip \text{and} \enskip
        B \in L(\mathcal{Y},\mathcal{W})
    \]

    \[ 
        = \left( \overline{A} \otimes B \right) \left( \left( T_{\mathcal{X}}
        \otimes \mathds{1}_{L(\mathcal{Y})} \right)\left( X \right) \right) 
        \left( \overline{A} \otimes B \right)^*
    \]

    $\Phi(P) \in PPT(\mathcal{Z}:\mathcal{W})$ follows \dots.
    
\end{proof}

To see the above more clearly consider the following case:

\begin{align*}
    \left( T \otimes \mathds{1} \right) \left( \left( A \otimes B \right) \left(
    Y \otimes Z\right) \left( A \otimes B \right)^* \right)
    &= \left( T \otimes \mathds{1} \right) \left( A Y A^* \otimes B Z B^* \right)\\
    &= \left( AYA^* \right)^T \otimes BZB^* \\
    &= \overline{A} Y^T \left( \overline{A} \right)^* \otimes B Z B^* \\
    &= \left( \overline{A} \otimes B \right) \left( Y^T \otimes Z \right) \left(
\overline{A} \otimes B\right)^*
\end{align*}

This extends to all X by linearity.

Consider the following proposition

\begin{theorem}
    Let $\mathcal{Z} = \mathbb{C}^\Sigma$ and $ W = \mathbb{C}^\Sigma $.
    Assume $\rho \in PPT(\mathcal{Z}: \mathcal{W}) \cap D(\mathcal{Z}\otimes
    \mathcal{W})$. Also, define
    \[
        \tau = \frac{1}{\left| \Sigma \right|} \sum_{a,b \in \Sigma} E_{a,b}
        \otimes E_{a,b}
    \]

    It holds that $\langle \tau , \rho \rangle \le \frac{1}{\left| \Sigma
    \right|}$.
\end{theorem}

\begin{proof}
    Observe that

    \[ 
        T_{\mathcal{X}}(\tau) = \frac{1}{\left| \Sigma \right|} \sum_{a,b \in \Sigma}
        E_{b,a} \otimes E_{a,b} = \frac{1}{\left| \Sigma \right|} W
    \]
    
    $W \in U(\mathcal{X} \otimes \mathcal{Y})$ is the swap operator. Now:

    \[ 
        |\langle \tau , \rho \rangle|= |\langle T_{\mathcal{X}} (\tau) ,
        T_{\mathcal{X}} (\rho) \rangle |= |\frac{1}{\left| \Sigma \right|} \langle
        W, T_{\mathcal{X}} (\rho) \rangle | \le \frac{1}{\left| \Sigma \right|}
        \left| \left| T_{\mathcal{X}} (\rho) \right| \right| = \frac{1}{\left|
        \Sigma \right|}
    \]
    
    The last equality holds because $T_{\mathcal{X}}(\rho)$ is a density operator.
\end{proof} 

\begin{theorem}
    Consider X,Y as registers and allow $\rho \in D(\mathcal{X} \otimes
\mathcal{Y}) \cap PPT(\mathcal{X}:\mathcal{Y})$. It can be shown that $E_D (X:Y)_\rho = 0$.
\end{theorem}

\begin{proof}
    Allow $\mathcal{Z} = \mathbb{C}^{\mathbb{Z}_2} = W$, $\tau = u$, $u =
    \frac{1}{\sqrt{2}} \left( e_0 \otimes e_0 + e_1 \otimes e_1 \right)$.

    We have that $\rho \in PPT(\mathcal{X}: \mathcal{Y})$ so that 

    \[ 
        \rho^{\otimes n} \in PPT(\mathcal{X}^{\otimes n} : \mathcal{Y}^{\otimes
        n} )
    \]
    
    Considering some $\Phi \in LOCC(\mathcal{X}^{\otimes n}, \mathcal{Z}:
    \mathcal{Y}^{\otimes n}, W)$.

    \[ 
        \Phi(\rho^{\otimes n}) \in PPT( \mathcal{Z}:\mathcal{W}) \cap
        D(\mathcal{Z} \otimes \mathcal{W})
    \]
    
    So,

    \[ 
        \langle \tau , \Phi(\rho^{\otimes n} \rangle \le \frac{1}{2}
    \]
    
    Thus, Alice and Bob can't distill even a single qubit.
\end{proof}

Below is an alternative construction to PPT states.

We will define four spaces :

\begin{align*}
    \mathcal{X}_2 &= \mathbb{C}^{\mathbb{Z}_2} \\
    \mathcal{Y}_2 &= \mathbb{C}^{\mathbb{Z}_2} \\
    \mathcal{X}_3 &= \mathbb{C}^{\mathbb{Z}_3} \\
    \mathcal{Y}_3 &= \mathbb{C}^{\mathbb{Z}_3} \\
    \tau_2 &= D(\mathcal{X}_2 \otimes \mathcal{Y}_2) \\
    \tau_3 &= D(\mathcal{X}_3 \otimes \mathcal{X}_3)
\end{align*}

The $\tau$s are maximally mixed states. Now, consider some:

\[ 
    P(\alpha) = \left( \mathds{1}_{\mathcal{X}_2} \otimes
    \mathds{1}_{\mathcal{Y}_2} - \tau_2 \right) \otimes \left(
\mathds{1}_{\mathcal{X}_3} \otimes \mathds{1}_{\mathcal{Y}_3} - \tau_3 \right) +
\alpha \tau_2 \otimes \tau_3
\]

Now:

\begin{align*}
    P(\alpha) & \notin Sep(\mathcal{X}_2 \otimes \mathcal{X}_3: \mathcal{Y}_2
    \otimes \mathcal{Y}_3) \quad,\enskip \forall \alpha > 0 \\
    P(\alpha) & \in PPT(\mathcal{X}_2 \otimes \mathcal{X}_3: \mathcal{Y}_2
    \otimes \mathcal{Y}_3) \quad,\enskip \forall \alpha \in [0,4]
\end{align*}

\end{document}
