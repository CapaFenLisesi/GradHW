\begin{homeworkProblem}

\begin{homeworkSection}{(a)}
To construct the Hamiltonian describing time evolution of the dynamical varibales we can simply consider the total energy of the system as a constant of motion however we firmly follow a more general approach. we first try to redrive the second order differential equation which determines equation of motion based on Lagrange equation. Primary KVL and KCL equations manifest themself in defining conjugate varibales. Quite generally we should first pick up minimum number of varibles which can completely describe the working point in configuration space. We choose $\mathcal{Q}=v$ as the sole dynamical quantity. $i$ is related to $v$ by:
\begin{equation}
i=C\frac{dv}{dt}=C\dot{\mathcal{Q}}
\end{equation}
\begin{figure}[!h]
\centering
% Generated with LaTeXDraw 2.0.8
% Thu Feb 28 09:11:00 GMT-06:00 2013
% \usepackage[usenames,dvipsnames]{pstricks}
% \usepackage{epsfig}
% \usepackage{pst-grad} % For gradients
% \usepackage{pst-plot} % For axes
\scalebox{0.7} % Change this value to rescale the drawing.
{
\begin{pspicture}(0,-1.5479687)(4.9146876,1.5879687)
\psline[linewidth=0.028222222cm](3.8764584,0.9753869)(3.8864996,0.56300914)
\rput{-88.605156}(3.507715,4.24715){\psarc[linewidth=0.028222222](3.9297698,0.32649234){0.2}{0.0}{180.0}}
\rput{-88.605156}(3.916978,3.8667367){\psarc[linewidth=0.028222222](3.9395065,-0.07338913){0.2}{0.0}{180.0}}
\rput{-88.605156}(4.3262405,3.486323){\psarc[linewidth=0.028222222](3.9492435,-0.4732706){0.2}{0.0}{180.0}}
\rput{-88.605156}(4.7366757,3.073829){\psarc[linewidth=0.028222222](3.9431307,-0.8897928){0.18375}{0.0}{180.0}}
\psline[linewidth=0.028222222cm](3.9272726,-1.1114945)(3.9373138,-1.5238723)
\psline[linewidth=0.04cm](3.936875,-1.5279688)(1.356875,-1.5279688)
\psline[linewidth=0.04cm](3.876875,0.97203124)(1.296875,0.97203124)
\psline[linewidth=0.054cm](1.756875,-0.098767914)(0.916875,-0.098767914)
\psline[linewidth=0.054cm](1.756875,-0.38796875)(0.916875,-0.38796875)
\psline[linewidth=0.04cm](1.316875,0.95203125)(1.336875,-0.04796875)
\psline[linewidth=0.04cm](1.376875,-1.5279688)(1.376875,-0.38796875)
\psline[linewidth=0.04cm,arrowsize=0.133cm 2.36,arrowlength=1.4,arrowinset=0.4]{<-}(2.236875,0.9521436)(2.576875,0.9519189)
\usefont{T1}{ptm}{m}{n}
\rput(2.1842186,-0.27296874){\large $C$}
\usefont{T1}{ptm}{m}{n}
\rput(4.534219,-0.29296875){\large $L$}
\usefont{T1}{ptm}{m}{n}
\rput(2.5142188,1.3670312){\large $i$}
\usefont{T1}{ptm}{m}{n}
\rput(0.28421876,-0.25296876){\large $v$}
\end{pspicture} 
}

\caption{\small parallel LC network}
\end{figure}
Please note that we could also choose the cahrge of the capacitor ($Q$) and total magnetic flux inisde the inductor ($\Phi$) as the dynamical variables. We define the action as the difference of magnetic and electrical energies as:
\begin{equation}
\mathcal{L}=\frac{1}{2}Li^2-\frac{1}{2}Cv^2=\frac{1}{2}C\left[\frac{1}{\omega_0^2}\dot{\mathcal{Q}}^2-\mathcal{Q}^2\right]
\end{equation}  
where $\omega_0$ is defined as 
$$\omega_0=\frac{1}{\sqrt{LC}}$$
Lagrange equation for this action is:
\begin{equation}
\frac{d}{dt}\pd{\M{L}(\M{Q},\dot{\M{Q}})}{\dot{\M{Q}}}-\pd{\M{L}(\M{Q},\dot{\M{Q}})}{\M{Q}}=0\quad\Lrw\quad \ddot{\M{Q}}=-\omega_0^2\M{Q}\quad\Lrw\quad \frac{d^2v}{dt^2}=-\omega_0^2 v
\end{equation}
This equation is exactly same as the equation which can be derived based on KVL and KCL equations. Thus our Lagrangian works well and we can go ahead to cunstruct the Hamiltonian. First note that the conjugate varible corresponde to  $\M{Q}$ is:
\begin{equation}
\M{P}=\pd{\M{L}}{\dot{\M{Q}}}=\frac{C\dot{\M{Q}}}{\omega_0^2}=LC i
\end{equation}
Now the Hamiltonian is:
\begin{equation}\label{P1-H}
\M{H}=\left.\M{P}\dot{\M{Q}}-\M{L}(\M{Q},\dot{\M{Q}})\right|_{\dot{\M{Q}}=\omega_0^2\M{P}/C}
=\frac{\omega_0^2\M{P}^2}{2C}+\frac{1}{2}C\M{Q}^2
\end{equation}
Interestingly canonical equations of motion in phase space are:
\begin{align}
\dot{\M{Q}}=\pd{\M{H}(\M{Q},\M{P})}{\M{P}}=\frac{\omega_0^2\M{P}}{C}\quad  &\Lrw \quad   C\frac{dv}{dt}=i \quad\text{(KCL)}\\
\dot{\M{P}}=-\pd{\M{H}(\M{Q},\M{P})}{\M{Q}}=-C\M{Q}\quad &\Lrw \quad  L\frac{di}{dt}=-v \quad \text{(KVL)}
\end{align}
\end{homeworkSection}
\begin{homeworkSection}{(b)}
The Hamiltonian describing time evolution in a one dimensional SHO is:
\begin{equation}\label{P1-SHO}
H=\frac{p^2}{2m}+\frac{1}{2}m\omega^2x^2
\end{equation}
where $x$ and $p$ are conjugate varibles. In both \eqref{P1-H} and \eqref{P1-SHO} Hamiltonians are elipitical functions of two conjugate varibles. In fact both Hamiltonians are equal if we do the following replacements:
\begin{equation}\label{P1-Sub}
H\leftrightarrow \M{H}\qquad x\leftrightarrow\M{Q}\qquad p\leftrightarrow\M{P}\qquad m\leftrightarrow LC^2\qquad\omega\leftrightarrow \omega_0
\end{equation}

 \end{homeworkSection}

\begin{homeworkSection}{(c)}
 The conh=jugate varible to $\M{Q}=v$ has been calculated in part (a) as we showed:
 \begin{equation}
 \M{P}=\pd{\M{L}}{\dot{\M{Q}}}=\frac{C\dot{\M{Q}}}{\omega_0^2}=LC i
 \end{equation}
 \end{homeworkSection}

%--------d--------------
\begin{homeworkSection}{(d)}
To quantize the LC circuit we start form the Driac's quantization rules. We first take a look at fundamental Poison backets in the classical problem:
\begin{equation}
\left\{\M{Q},\M{P}\right\}=\pd{\M{Q}}{\M{Q}}\pd{\M{P}}{\M{P}}-\pd{\M{P}}{\M{Q}}\pd{\M{Q}}{\M{P}}=1
\end{equation}   
According to the Dirac's quantization rule we should just replace:
\begin{equation}
\frac{1}{j\hbar}\left[\quad ,\quad \right]\leftrightarrow\left\{\quad,\quad\right\}
\end{equation}
Please note that we have used $j$ as the unit imaginary number. This quantization implies that $\M{Q}$ and $\M{P}$ should be treated as operators. So we have:
\begin{equation}
\left[\A{\M{Q}},\A{\M{P}}\right]=\left[\A{v},LC\A{i}\right]=j\hbar
\end{equation}
This commutation relation is the starting point to develop the whole \textit{Lie algebra}. So we should have the same quantization and generally algebra in a SHO and quantum LC circuit.
\end{homeworkSection}



%------------e-------------
\begin{homeworkSection}{(e)} 
As it's explained in the previous part we have the same algebra in SHO and quantum LC network. So we can simply use the result of our discussions in SHO. we should just use the substitutions given in \eqref{P1-Sub}. So we have:
\begin{align}
&a_{SHO}=\sqrt{\frac{m\omega}{2\hbar}}\left(x+\frac{jp}{m\omega}\right)\quad\Lrw\quad 
\M{A}=\sqrt{\frac{LC^2\omega_0}{2\hbar}}\left(\M{Q}+\frac{j\M{P}}{LC^2\omega_0}\right)\\
&a^\dagger_{SHO}=\sqrt{\frac{m\omega}{2\hbar}}\left(x-\frac{jp}{m\omega}\right)\quad\Lrw\quad 
\M{A}^{\dagger}=\sqrt{\frac{LC^2\omega_0}{2\hbar}}\left(\M{Q}-\frac{j\M{P}}{LC^2\omega_0}\right)
\end{align}
 or
 \begin{align}
 \M{A}&=\sqrt{\frac{C}{2\hbar\omega_0}}\left(\A{v}+\frac{j\A{i}}{C\omega_0}\right)\qquad \text{annihilation operator}\label{P1-111}\\
 \M{A}^{\dagger}&=\sqrt{\frac{C}{2\hbar\omega_0}}\left(\A{v}-\frac{j\A{i}}{C\omega_0}\right)\qquad\text{creation operator}\label{P1-112}
 \end{align}
The Hamiltonian can be rewriten in terms of creation and annihilation operators as:
\begin{equation}
\M{H}=\hbar \omega_0\left(\M{A}^{\dagger}\M{A}+\frac{1}{2}\right)
\end{equation}
Numbering operator is defined as:
\begin{equation}
\M{N}=\M{A}^{\dagger}\M{A}
\end{equation}

\end{homeworkSection}
%-----f-----------
\begin{homeworkSection}{(f)}
from \eqref{P1-111} and \eqref{P1-112}
\begin{align}
\A{v}&=\sqrt{\frac{\hbar\omega_0}{2C}}\left(\M{A}+\M{A}^{\dagger}\right)\\
\A{i}&=-j\sqrt{\frac{\hbar\omega_0}{2L}}\left(\M{A}-\M{A}^{\dagger}\right)
\end{align}

\end{homeworkSection}
%-----g-----
\begin{homeworkSection}{(g)}
Using the results of the quantization of SHO we can write:
\begin{equation}
E_n=\left(n+\frac{1}{2}\right)\hbar\omega_0=\left(n+\frac{1}{2}\right)\hbar\frac{1}{\sqrt{LC}}
\end{equation}
\end{homeworkSection}

%------h------
\begin{homeworkSection}{(h)}
Current and voltage as conjugate dynamical varibles are incompatible. So the following uncertainty relations prohibite the precise measurment of both $v$ and $i$ simultaniously:
\begin{equation}
\langle(\Delta v)^2\rangle\langle (\Delta i)^2\rangle\geq\frac{1}{4}\left|\langle\left[\A{v},\A{i}\right]\rangle\right|^2=\frac{\hbar^2}{4L^2C^2}
\end{equation}
Paricularly uncertainty in nth energy eigenstate is:
\begin{align}
&\langle(\Delta v)^2\rangle_n=\bra{n} v^2\ket{n} -\bra{n} v\ket{n}^2=\frac{\hbar\omega_0}{2C}\bra{n}\M{A}^2+\M{A}^{\dagger}^2+\M{A}\M{A}^{\dagger}+\M{A}^{\dagger}\M{A}\ket{n}-\frac{\hbar\omega_0}{2C}\bra{n}\M{A}+\M{A}^{\dagger}\ket{n}^2\\
&\langle(\Delta i)^2\rangle_n=\bra{n} i^2\ket{n} -\bra{n} i\ket{n}^2=\frac{\hbar\omega_0}{2L}\bra{n}\M{A}^2+\M{A}^{\dagger}^2+\M{A}\M{A}^{\dagger}+\M{A}^{\dagger}\M{A}\ket{n}-\frac{\hbar\omega_0}{2L}\bra{n}\M{A}+\M{A}^{\dagger}\ket{n}^2
\end{align}
Hence:
\begin{equation}\label{P1-220}
\left.
\begin{array}{l}
\langle(\Delta v)^2\rangle_n =\frac{\hbar\omega_0}{2C}(2n+1)\\
\langle(\Delta i)^2\rangle_n =\frac{\hbar\omega_0}{2L}(2n+1)
\end{array}
\right\}\quad\Lrw\quad \langle(\Delta v)^2\rangle_n\langle(\Delta i)^2\rangle_n=\frac{\hbar^2}{4L^2C^2}(2n+1)^2
\end{equation}
Clearly qunatum uncertainty which reprsents quantum mechanical effects is proportional to $\omega_0^4$ this menas that quantum mechanical effects are more important at higher frequencies comparable to $\omega_0$. Dimensional analysis is a direct method to show this limit.  
\end{homeworkSection}

\begin{homeworkSection}{(i)}
To evaluate quantum flactuation level in an eigenstate of energy we use equation \eqref{P1-220}. We have:
\begin{align}
&\sqrt{\langle(\Delta v)^2\rangle_n}=\sqrt{\frac{\hbar\omega_0}{2C}(2n+1)}\approx 1.28 \mathrm{\mu V}\sqrt{2n+1}\\
&\sqrt{\langle(\Delta i)^2\rangle_n}=\sqrt{\frac{\hbar\omega_0}{2L}(2n+1)}\approx 40.7 \mathrm{nA}\sqrt{2n+1}
\end{align}
\end{homeworkSection}
%-----j----------------------------
\begin{homeworkSection}{(j)}
We assume that quanta of thermal radiation is $k_BT$. $k_B$ is Boltzman's constant. To measure the the quantized energy levels we should conduct our experiment in an enviornment in which termal energy quanta is lower than the half of the first energy level:
\begin{equation}
k_BT<\frac{\hbar\omega_0}{2}\quad\Lrw\quad T<0.12\mathrm{K}
\end{equation} 
%--------
\end{homeworkSection}
%----k----
\begin{homeworkSection}{(k)}
Assume that the voltage wave function is reptresented by $\Psi(v,t)$. since $\M{Q}$ and $\M{P}$ commutation relation is the same as $x-p$ commutation relation we can use the same math. So we can just replace $\M{P}\leftrightarrow -j\hbar\fpds{\M{Q}}$. Hamiltonian is also generator of time translation so we can write:
\begin{equation}
j\hbar\pd{\Psi(v,t)}{t}=-\frac{\omega_0^2\hbar^2}{2C}\pdt{\Psi(v,t)}{v}+\frac{1}{2}Cv^2\Psi(v,t)
\end{equation}
\end{homeworkSection}
%-----l------
\begin{homeworkSection}{(l)}
The voltage wavefunction associated with the ground state of the LC network is similar to x-space wave function of SHO. we should just use \eqref{P1-Sub}. Actually we have:
\begin{equation}
\M{A}\ket{0}=0\quad\Lrw\quad v\Psi_0(v)+\frac{\hbar}{LC^2\omega_0}\pd{\Psi_{0}(v)}{v}=0
\end{equation}
And explicitly:
\begin{equation}
\Psi_0(v)=\left({\frac{LC^2\omega_0}{\pi\hbar}}\right)^{1/4}\exp\left[-\frac{1}{2}\frac{LC^2\omega_0 v^2}{\hbar}\right]
\end{equation}

\end{homworkSection}


\end{homeworkProblem}
