\begin{homeworkProblem}

\begin{homeworkSection}{(a)}
A simple harmonic oscilator is initially in a state describied by a wavefunction $\ket{\Psi}$ as:
\begin{equation}
\ket{\Psi}=A\sum_{n=0}^{\infty}c^n\ket{\psi_n}
\end{equation}
where $\ket{\psi_n}$ are normalized energy eigenfunctions of SHO. The normalization constant $A$ can be calculated as a function of $c$. Using orthogonality of igenfunctions we can write:
\begin{equation} \label{P3-1}
\bracket{\Psi}{\Psi}=\abs{A}^2\sum_{m=0}^{\infty}\sum_{n=0}^{\infty}c^{n}{c^*}^m\bracket{\psi_m}{\psi_n}=\abs{A}^2\sum_{n=0}^{\infty}\abs{c}^{2n}=1
\end{equation}
From \eqref{P3-1} we can write:
\begin{equation}
\abs{A}^2=\frac{1}{\sum_{n=0}^{\infty}\abs{c}^{2n}}=1-\abs{c}^2\quad\Lrw\quad A=\sqrt{1-\abs{c}^2}
\end{equation}
\end{homeworkSection}
%------b--------------
\begin{homeworkSection}{(b)}
The wavefunction of the system at a later time $t$ can be determined by applying time evolution operator:
\begin{equation}
\ket{\Psi(t)}=\exp\left(\frac{-iHt}{\hbar}\right)\ket{\Psi(0)}=\sqrt{1-\abs{c}^2}\sum_{n=0}^{\infty}c^n\exp\left(-\frac{iE_nt}{\hbar}\right)\ket{\psi_n}
\end{equation}
From the lecture the  energy eigenvalues of SHO are 
$$E_n=\left(n+\frac{1}{2}\right)\hbar\omega_0$$ 
so:
\begin{equation}
\ket{\Psi(t)}=
\sqrt{1-\abs{c}^2}\sum_{n=0}^{\infty}c^n\exp\left[-i\left(n+\frac{1}{2}\right)\omega_0t\right]\ket{\psi_n}
\end{equation}


\end{homeworkSection}
\begin{homeworkSection}{(c)}
The probability of finding the system again in its initial state at aleter time $t$ can be simply calculated as below:
\begin{equation}
P=\abs{\bracket{\Psi(t=0)}{\Psi(t)}}^2=\left|\left(1-\abs{c}^2\right)\sum_{n=0}^{\infty}\abs{c}^{2n}\exp\left[\left(n+\frac{1}{2}\right)\omega_0 t\right]\right|^2
=\left|\frac{1-\abs{c}^2}{1-\abs{c}^2e^{-i\omega_0 t}}\right|^2
\end{equation}
This expression can be simplified more:
\begin{equation}
P=\frac{1+\abs{c}^4-2\abs{c}^2}{1+\abs{c}^4-2\cos(\omega_0 t)\abs{c}^2}
\end{equation}




\end{homeworkSection}
%----d--------
\begin{homeworkSection}{(d)}
The expectation value of the total energy of the system is:
\begin{equation}
\langle H\rangle_t=\bra{\Psi(t)}H\ket{\Psi(t)}
\end{equation}
Note that energy is a \textit{constant of motion} and consequently expectation value of the total energy is indepenent of time. So $ \leftangle H\rightangle$ can be calculated at the initial state:
\begin{equation}
\langle H\rangle_t=\langle H\rangle_{t=0}=\bra{\Psi(t=0)}H\ket{\Psi(t=0)}
\end{equation}
Using 
$$H\ket{\psi_n}=\left(n+\frac{1}{2}\right)\hbar\omega_0\ket{\psi_n}$$
we obtain:
\begin{equation}
\langle H\rangle=\left(1-\abs{c}^2\right)\sum_{n=0}^{\infty}\left(n+\frac{1}{2}\right)\hbar\omega_0\abs{c}^{2n}
\end{equation}

To evaluate this seris we first prove the follwoing equality:
\begin{equation}\label{P3-H}
\sum_{n=0}^{\infty}nq^{n}=\frac{q}{(1-q)^2}\qquad \abs{q}\leq 1
\end{equation}
Proof:
\begin{equation}\label{P3-identity}
\sum_{n=0}^{\infty}nq^{n}=\lim_{N\to\infty}q\fpds{q}\sum_{n=0}^{N}q^{n+1}=q\fpds{q}\frac{q}{1-q}=\frac{q}{(1-q)^2}
\end{equation}
Using \eqref{P3-identity} and after some simple algebraic manipulations we get:
\begin{equation}
\langle H\rangle=\hbar\omega_0\left(1-\abs{c}^2\right)\left\{\frac{\abs{c}^2}{\left(1-\abs{c}^2\right)^2}+\frac{1}{2}\frac{1}{1-\abs{c}^2}\right\}=\frac{\hbar\omega}{2}\frac{1+\abs{c}^2}{1-\abs{c}^2}
\end{equation}

\end{homeworkSection}




























\end{homewoorkProblem}