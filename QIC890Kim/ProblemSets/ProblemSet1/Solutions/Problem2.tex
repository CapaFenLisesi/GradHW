\begin{homeworkProblem}[Problem 2: Ergodicity]
   \problemStatement{We learned ergodicity in the mean and in the
   autocorrelation in class.}
   \subsection{Problem 2a}
   \problemStatement{Please write down an example of the processes which are ergodic in the mean
   and/or the autocorrelation.}

   A process that would be ergodic in the mean would be the sampling of numbers
   uniformly over some interval $ [a, b] $, $ a,b \in \mathds{R} $ at each time
   $ t $. This process would also be ergodic in the
   autocorrelation. I will demonstrate it as follows.
   \begin{align*}
   \overline{x(t)} = \frac{b+a}{2}
   \end{align*}
   That is, the time-averaged mean is just the average of the range. This is
   intuitively obvious.
   In the case of an ensemble, it is equally as obvious. Considering a
   sufficiently large ensemble, there will be an equal number of all values
   between $ a $ and $ b $ such that the mean is
   \begin{align*}
      \langle x(t) \rangle
      &= \frac{Ma+M(a+\delta)+M(a+2\delta)+\ldots+M(a+N \delta))}{M(N+1)} \\
      &= \frac{Ma(N+1)}{M(N+1)} + \frac{\delta M \frac{N(N+1)}{2}}{M(N+1)} \\
      &= a + \frac{N \delta}{2}
      \intertext{But, $ b = a + N \delta $. So, $ \frac{b+a}{2} =
      \frac{2a + N \delta}{2} $}
      &= \frac{b+a}{2}
   \end{align*}
Thus, as asserted earlier, this process is ergodic in the mean. This is also
ergodic in the autocorrelation. Since the numbers are selected uniformly at
random, each time instant corresponds to an independent sample.  Thus, it is the
same to consider a sample point from a time-shifted version of $ x(t) $ (at $ t
+ t_{n} $, for example) or another sample point $ x^{(j)}(t_{n}) $ from the jth
sample function of the ensemble. Thus, the autocorrelation of the ensemble and
of the time sequence must be the same.

\subsection{Problem 2b}
\problemStatement{ Suppose the autocorrelation of a noise process is ergodic. Is the process ergodic in
the mean? Please justify your answers.}

If a noise process is ergodic in the autocorrelation, then that means
\[
   \lim_{T \to \infty} \frac{1}{T} \int_{-T/2}^{T/2} x^{(i)}(t)x^{(i)}(t+
   (t_{2}-t_{1})) dt =
   \int_{-\infty}^{\infty}x_{1}x_{2}p_{2}(x_{1},x_{2}; t_{1},t_{2})
   dx_{1}dx_{2}
\]

If a noise process is ergodic in the mean, then that means that \[ \lim_{T \to
\infty} \frac{1}{T} \int_{-T/2}^{T/2} x^{(i)}(t) dt = \lim_{N \to \infty}
\frac{1}{N} \sum^{N}_{i=1} x_{1}p_{1}(x_{1}; t_{1}) dx_{1} \] The question is:
If the noise process is ergodic in the autocorrelation is it ergodic in the
mean? In general, the equality of the first pair of integrals does not guarantee
the equality of the second pair of integrals. So, no, just because a noise
process is ergodic in the autocorrelation does not mean that the process is
ergodic in the mean. But, we have to prove this. So, let's assume that the noise
process is ergodic in the autocorrelation. If we find a contradiction, then
we're good.  The logical statement to violate then is the following: ``If the
noise process is ergodic in the autocorrelation, then the noise process is
ergodic in the mean.'' This has the logical form ``P $ \to $ Q''. To prove this
false, all we have to do is find some P for which Q is false. That is, we must
find some process which is ergodic in the autocorrelation but is not ergodic in
the mean.

After much struggle (thanks to Arnaud Carignan-Dugas for the assistance), I have
found such an example. Consider the random process consisting of selecting some
number $ a $ with probability 1/2 and another number $ -a $ with probability
1/2. The process will be expressed mathematically as follows:
\[
   x^{(i)}(t) = \begin{cases}
      a &\text{ if } i = 1, \\
      -a &\text{ if } i = 2,
   \end{cases}
   \]
   where $ i $ is a random variable with probability .5 that $ i=1 $ and
   .5 that $ i=2 $.

The time-average for the random process is conditional upon the value of $ i $.
If $ i = 1 $, $ \overline{x^{(1)}(t)} = a $. If $ i = 2 $, $
\overline{x^{(2)}(t)} = -a $. However, the ensemble average $ \langle x(t)
\rangle = 0 $. This is because the ensemble average would be calculated as
\[
   \langle x(t) \rangle = \frac{1}{2}a + \frac{1}{2}(-a) = 0
\]
So, this process is not ergodic in the mean. If it is ergodic in the
autocorrelation then we're good!

The calculation of the time-associated autocorrelation, again, depends on the value of $
i $. If $ i=1 $:
\begin{align*}
   \phi_{x}^{(1)}(\tau) &=
   \lim_{T \to \infty}\frac{1}{T} \int_{-T/2}^{T/2} x^{(1)}(t)x^{(1)}(t+\tau) dt
   \\
   &= \lim_{T \to \infty} \frac{1}{T}\int_{-T/2}^{T/2}a^2 dt \\
   &= a^2
\end{align*}
However, if $ i=2 $
\begin{align*}
   \phi_{x}^{(2)}(\tau) &=
   \lim_{T \to \infty}\frac{1}{T} \int_{-T/2}^{T/2} x^{(2)}(t)x^{(2)}(t+\tau) dt
   \\
   &= \lim_{T \to \infty} \frac{1}{T}\int_{-T/2}^{T/2}(-a)(-a) dt \\
   &= a^2
\end{align*}
However, in the case of the ensemble-associated covariance:
\begin{align*}
   \langle x(t) x(t +\tau) \rangle &= \frac{1}{2}\phi_{x}^{(1)}(\tau) +
   \frac{1}{2} \phi_{x}^{(2)}(\tau) \\
   &= \frac{a^2}{2} + \frac{a^2}{2} \\
   &= a^2
\end{align*}
The ensemble-associated covariance equals the time-associated
autocorrelation for each sample function in the ensemble. Thus, it has been
demonstrated that ergodicity in the autocorrelation does not imply ergodicity in
the mean. But, this is intuitively obvious from the above definitions of the
mean and the autocorrelation/covariance.
\end{homeworkProblem}
