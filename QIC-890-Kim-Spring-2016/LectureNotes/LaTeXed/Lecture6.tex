\documentclass{article}
\usepackage[]{amsmath}
\begin{document}
\section{Carson's Theorem}
\label{sec:carson_s_theorem}
Imagine you have a random pulse train characterized by the following stochastic
process
\[
   x(t) = \sum^{K}_{k=1} a_{k}f(t-t_{k}) \enskip.
\]
By considering the power spectral density $ S_{x}(\omega) = \lim_{T \to \infty}
\frac{2 \langle |X(i \omega) |^2 \rangle}{T}$ and the Fourier transform $
X(i\omega) = \mathcal{F}(x(t)) $ we can show that
\[
   S_{x}(\omega) =2 \nu \langle a^{2} \rangle \langle |F(i\omega)|^{2} \rangle +
   4 \pi \left( \nu \langle a \rangle \int_{-\infty}^{\infty} f(t) dt
   \right)^{2} \delta(\omega) \enskip.
\]
Above, $ \langle a^{2} \rangle = \frac{1}{K} \langle \sum^{K}_{k=1}  a_{k}^{2}
   \rangle $ and $ \nu = \frac{K}{T} $, $ T $ being introduced in the spectral
   density and $ K $ being the number of pulses. Thus, $ \nu $ represents the
   average rate of pulses. The above expression for $ S_{x}(\omega) $ is
   Carson's theorem.

\section{Shot Noise}
\label{sec:shot_noise}
Consider the case of a vacuum diode. Note that shot noise is not intrinsic noise
and it's not equilibrium noise. It arises due to independent electron emission
fluctuations.

Consider a simple circuit consisting of an ideal voltage source in series with a
source resistance in series with a diode (which possesses a cathode and anode).
This circuit can be made to operate in two modes: One mode, constant voltage
mode, keeps the voltage constant in time, but allows the current to vary. The
other mode keeps the current constant allowing the current to vary.

Establishing a voltage across the diode establishes an electric field which
draws charge from anode to cathode. If the cathode and anode are separated by a
distance $ d $, then we can write the current as
\[
   i(t) = \frac{q}{t} = \frac{q}{\frac{d}{v(t)}} = \frac{q}{d} v(t) \enskip,
\]
where, $ v(t) $ is the velocity of the charge carriers as determined by the
strength of the applied voltage. We can thus consider two time scales for this
problem. The first is the transit time $ \tau_{t} = \frac{d}{v(t)} $. The other
is the relaxation time associated with the external circuit. Imagining the
cathode and anode to form some capacitor, we can write the circuit time scale as
$ \tau_{c} = R_{s}C $. These two times scales yield three possible relationships
we need to consider
\begin{equation*}
\begin{cases}
   \tau_{t} \ll \tau_{c} \enskip,\text{$ v(t) $ is constant} \\
   \tau_{t} \ll \tau_{c} \enskip,\text{$ v(t) $ is accelerating by electric
   field.} \\
   \tau_{t} \gg \tau_{c} \enskip,\text{impulse events}
\end{cases}
\end{equation*}

\subsection{Case 1}
\label{sub:case_1}


Let's determine the dynamics for case 1. First, we should find the surface
charge on these plates. Initially, $ V $ is generated across the cathode and
anode. The charge
built up on the cathode would then be $ Q_{c} = -CV $. The charge built up on
the anode would be $ Q_{a} = +CV $. The current then would be
\[
   i(t) = \begin{cases}
      \frac{qv}{d}, & 0 < t < \frac{d}{v}  \\
      0, &\text{otherwise}
   \end{cases}
\]

We can write the charge on the cathode as a function of time as
\[
   Q_{c}(t) =
   \begin{cases}
       -CV + q - \int_{0}^{t} i(t') dt' & \enskip, 0 < t <
      \frac{d}{v} \\
      -CV & \enskip,\text{otherwise}
   \end{cases} .
\]
Considering the anode's charge as a function of time, we have
\[
   Q_{a}(t) =
   \begin{cases}
       +CV + \int_{0}^{t} i(t') dt' & \enskip, 0 < t <
      \frac{d}{v} \\
      +CV & \enskip,\text{otherwise}
   \end{cases} .
\]
\subsection{Case 2}
\label{sub:case_2}
Considering the case where the charge is accelerating due to the electric field
we have
\[
   i(t) = \begin{cases}
      \frac{q}{d}v(t) &\enskip, 0 < t < \frac{d}{v(t)} \\
      0 &\enskip, \text{otherwise} \\
   \end{cases}.
\]
If we consider integrating along the length of the diode, we have
\[
   \int_{0}^{d} dx = \int_{0}^{\tau_{t}} dt' v(t') = \int_{0}^{\tau_{t}}dt'
   \frac{qV}{md} t' = \frac{qV}{md}\tau_{t}^{2} = d \enskip.
\]
We used the fact that the velocity of the electrons is $ v(t) =
\int_{-\infty}^{t}a(t') dt' = \frac{qE}{m}t = \frac{qV}{md}t $. This allows us
to write the transit time as $ \tau_{t} = \sqrt{\frac{m d^{2}}{qV}} $. Now, we
can write the charge built up on the cathode as
\[
   Q_{c}(t) =
   \begin{cases}
      -CV + q - \int_{0}^{t} \frac{q^{2} V}{m d^{2}} t' dt' &\enskip, 0 < t <
      \frac{d}{v(t)} \\
      0 &\enskip, \text{otherwise}
   \end{cases}
\]

\subsection{Case 3}
\label{sub:case_3}
In the case where $ \tau_{c} \gg \tau_{t} $ the charge motion is an impulse
event. We can write, then, the charge on the anode as
\[
   Q_{a}(t) = \begin{cases}
      CV - q e^{-\frac{t}{R_{s}C}}, & t > 0 \\
      CV , & \text{otherwise}
   \end{cases}.
\]
The cathode charge is $ Q_{c}(t) = -Q_{a}(t) $.

\section{Current Noise}
\label{sec:current_noise}

\subsection{Case 1 Noise}
\label{sub:case_1_noise}


Now, we're going to consider the current noise associated with case 1. In case
1, the current is a pulse train. We want to use Carson's theorem, so we first
discover
\[
   F(i\omega) = \int_{-\infty}^{\infty} f(t) \exp(-i \omega t) dt = q
   \frac{e^{-\frac{i \omega d}{2 v}}}{\frac{\omega d}{2 v}} \sin(\frac{\omega
   d}{2 v}).
\]
Then, we need
\[
   \int_{-\infty}^{\infty} f(t) dt = \int_{0}^{d/v} \frac{qv}{d} dt = q.
\]
Then,
\[
   S_{I}(\omega) = 2q \langle I \rangle \frac{\sin^2(\frac{\omega d }{2
      v})}{\left( \frac{\omega d}{2 v} \right)^2} + 4 \pi \langle I
      \rangle^{2}\delta(\omega)
\]

\subsection{Case 2 Noise}
\label{sub:case_2_noise}

In case 2, the current isn't constant in time, so the pulse train $ f(t-t_{k}) $
is defined as
\[
   f(t) = \begin{cases}
      t-t_{k}, & 0 < t < d/v \\
      0, & \text{otherwise}
   \end{cases}.
\]
Now, the spectral density is
\[
   S_{I}(\omega) = 2 \nu \left( \frac{q^{2}V}{md^2} \right)^2 \langle
   |F(i\omega)|^2 \rangle + 4 \pi \nu^{2}q^{2} \delta(\omega).
\]
Determining $ F(i\omega) $,
\[
   F(i\omega) = \int_{0}^{d/v}t e^{-i \omega t} dt =  \frac{1 - e^{-i \omega
   \frac{d}{v}}}{\omega^{2}}.
\]
So, $ \langle | F(i\omega)|^{2} \rangle $ is now
\[
   \langle | F(i\omega)|^{2} \rangle  = \frac{1}{\omega^{4}} \left( 2 +
      \omega^{2} (\frac{d}{v})^{2} - 2 \omega \frac{d}{v}\sin(\omega
   \frac{d}{v}) - 2 \cos(\omega \frac{d}{v}) \right).
\]
We can now write $ S_{I}(\omega) $ as
\[
   S_{I}(\omega) = 2 \nu \left( \frac{q^{2}V}{m d^{2}} \frac{1}{\omega^{4}} \right)
\left( 2 +
      \omega^{2} (\frac{d}{v})^{2} - 2 \omega \frac{d}{v}\sin(\omega
      \frac{d}{v}) - 2 \cos(\omega \frac{d}{v}) \right) + 4 \pi
      \nu^{2}q^{2}\delta(\omega).
\]

The low frequency limit yields that $ S_{I}(\omega \approx 0) = 2 q \langle I
\rangle $.

\subsection{Case 3 Noise}
\label{sub:case_3_noise}
Now, we have a different form for our impulse train.
\[
   f(t-t_{k}) = \begin{cases}
      \frac{q}{R_{s}C} e^{-\frac{t-t_{k}}{R_{s}C}}, & t > 0 \\
      0, & \text{otherwise}
   \end{cases}.
\]
The spectral density $ S_{I}(\omega) $ can now be written as
\[
   S_{I}(\omega) = 2 q \langle I \rangle \frac{1}{1 + \omega^{2} R_{s}C} + 4 \pi
   \langle I \rangle^{2} \delta(\omega).
\]
Again, the low frequency limit yields $ S_{I}(\omega) = 2 q \langle I \rangle $.

\section{Discussion of Shot Noise}
The origin of shot noise in a vacuum diode is that the electrons are emitted as
independent events of a random pulse train. However, in reality there exists
some dependence between emissions. This results in some ``negative'' feedback
occurring in the electron process. This gives rise to the space-charge effect,
that pushes the likelihood of the system to exist in the $ \tau_{t} \gg \tau_{c}
$ limit. It also gives rise to a memory effect in the external circuit, namely
that $ \tau_{c} \gg \tau_{t} $ which results in ``slow voltage recovery''. This
results in sub-shot noise in the case of constant current operation.

\section{Partition Noise - Mesoscopic Conductors}
In the mesoscopic limit, the charge carriers exist in a state where we must
consider both their wave and particle nature, simultaneously, unfortunately. In
the case where charge carriers are traveling through a mesoscopic conductor they
have some probability of overcoming some potential barrier or, alternatively,
reflecting off of the potential barrier. In this case, the power spectral
density is
\[
   S_{I}(\omega \to 0) = 2q \langle I \rangle \left( 1 - P_{T} \right)
\]

\section{$\frac{1}{f}$/Pink/Flicker}
\label{sec:1_f_pink_flicker}
This noise can arise from some defect at an interface which traps and releases
charge carriers randomly. We will begin by establishing the relationship between
$ 1/f $ noise and a random telegraph signal. We will posit that $ 1/f $ noise is
the result of an ensemble of random telegraph signals.

What are the characteristics of $ 1/f $ noise?
\begin{enumerate}
   \item It should be scale invariant. To illustrate this, consider $
      S_{f}(\omega) = \frac{c}{\omega} $. Considering $ P(\omega_{1},\omega_{2})
      = \int_{\omega_{1}}^{\omega_{2}} S_{f}(\omega) d\omega = C
      \ln(\omega_{2}/\omega_{1}) $. Note that the power only depends on the
      ratio of $ \omega_{2} $ and $ \omega_{1} $.
   \item Considering its stationarity:
      \[
         S_{f}(\omega) = \begin{cases}
            c/\omega, & \omega_{1} \le \omega \le \omega_{2} \\
            0, & \text{otherwise}.
         \end{cases}
      \]
      Using the Wiener-Khintchine theorem, we can compute the autocorrelation
      function as
      \[
         \phi_{f}(\tau) = \frac{c}{2 \pi} \int_{\omega_{1}}^{\omega_{2}}
         \frac{\cos(\omega \tau)}{\omega} d\omega = \frac{c}{2 \pi}
      \bigl(\mathcal{C}(\omega_{2}\tau) - \mathcal{C}(\omega_{1}\tau)\bigr).
      \]
      This result, however, only results from the nature of our filter function
      (the bandpass filter). So, we can't guarantee stationarity in general.
\end{enumerate}

\end{document}
