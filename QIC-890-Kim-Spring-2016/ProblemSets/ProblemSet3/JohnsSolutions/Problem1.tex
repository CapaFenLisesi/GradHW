\begin{homeworkProblem}[Two Important Theorems of the Fourier Transform]
%   \problemStatement{
%   }
   % Problem 1.1
   \subsection{Differentiation Theorem}
   \label{sub:differentiation_theorem}
%   \problemStatement{
%   }

   Explicitly, the Fourier transform of the derivative of a function is
   \[
      \int_{-\infty}^{\infty} \frac{dx(t)}{dt} \exp(- i \omega t) dt \enskip.
   \]
   Integrating this by parts yields
   \[
      \int_{-\infty}^{\infty} \frac{dx(t)}{dt} \exp(- i \omega t) dt =
      x(t) \exp(- i \omega t) \bigg|_{-\infty}^{\infty} + i \omega
      \int_{-\infty}^{\infty} x(t) \exp(-i\omega t) dt \enskip.
   \]
   Now, the second integral is clearly $ X(\omega) $, the Fourier transform of $
   x(t) $. The first term, however, must be zero in order that $ x(t) $ possess
   a valid Fourier transform (that is, it's in $ L^{2} $). So, finally,
   \[
      \int_{-\infty}^{\infty} \frac{dx(t)}{dt} \exp(- i \omega t) dt =
      i \omega X(\omega) \enskip.
   \]
   This is not the desired result. However, if I change the lower bound to $ 0 $
   instead of $ -\infty $ I would obtain the desired result.

   However, the problem does not state that $ x(t < 0) = 0 $. It merely says
   that $ x(t) $ is related to some function $ y(t) $ in that $ y(t_{0}) $ is
   related to $ x(t) $ only for $ t \le t_{0} $. This does not impose that $
   x(t<0) =0 $.
   % Problem 1.2
   \subsection{Integration Theorem}
   \label{sub:integration_theorem}
%   \problemStatement{
%   }

   The previous result can be expressed in the following way,
   \[
      \mathcal{F}(d x(t) dt) = i \omega \mathcal{F}(x(t)) \enskip.
   \]
   Allow, $ y(t) = \frac{dx(t)}{dt} $ such that $ x(t) = \int_{-\infty}^{t} y(t')
   dt'$. Then,
   \[
      \mathcal{F}(y(t)) = i \omega
      \mathcal{F}\Bigl(\int_{-\infty}^{t}y(t')dt' + C \Bigr) \enskip.
   \]
   C, above, is some constant that's included to account for the fact that if we
   consider two ``versions'' of $ x(t) $, $ x_{1}(t) $ and $ x_{2}(t) $, which
   only differ in that $ x_{2}(t) - x_{1}(t) = k $, $k$ being some constant, their
   derivatives would be equal, $ \frac{dx_{1}(t)}{dt} = \frac{dx_{2}(t)}{dt} $.
   The value of the integration constant is $ C = x(t) -
   \int_{-\infty}^{t}y(t')dt' $. By linearity of the Fourier transform, however,
   we can break the above up into two parts and rearrange for what we want:
   \[
      \mathcal{F}\bigl(\int_{-\infty}^{t}y(t') dt'\bigr) =
      \frac{\mathcal{F}(y(t))}{i\omega} - \mathcal{F}(C) \enskip.
   \]
   But, we know that the Fourier transform of a constant C is just $ 2 \pi
   \delta(\omega) C $. So, the end result is that
   \[
      \mathcal{F}\left(\int_{-\infty}^{t}y(t')dt'\right) = \frac{\mathcal{F}(y(t))}{i\omega}
         - 2 \pi \delta(\omega)C \enskip.
   \]

   I'm not sure how to relate the constant of integration to $ X(0) $.
   Hopefully, that will be something the official solutions will teach me.
%   \subsection{Integration Theorem (Integration by Parts Method)}
%   \label{sub:integration_theorem_integration_by_parts_method_}
%   The integral to solve is
%   \[
%      \int_{-\infty}^{\infty} \int_{0}^{t}x(t')dt' \exp(-i\omega t) dt \enskip.
%   \]
%   We can break this into two separate terms using integration by parts. To
%   facilitate this, we will call $ \int_{0}^{t}x(t')dt' = y(t) $. Using the form
%   \[
%      \int uv' = uv - \int u'v \enskip,
%   \]
%   with $ u = y(t) $ and $ v = \exp(-i \omega t) $ :
%   \[
%      \int_{-\infty}^{\infty} y(t) \exp(-i\omega t) dt  =
%      \frac{1}{-i \omega} \lim_{t \to \infty} y(t) \exp(-i\omega t) -
%      \frac{1}{-i \omega} \lim_{t \to -\infty} y(t) \exp(-i\omega t) -
%      \int_{-\infty}^{\infty} x(t) \exp(-i \omega t) dt \enskip.
%   \]
%   Evaluating this limit explicitly yields
%   \[
%      \frac{1}{-i \omega} \lim_{t \to \infty} \int_{0}^{t} x(t') dt' \exp(-i\omega t) -
%      \frac{1}{-i \omega} \lim_{t \to -\infty} \int_{0}^{t} x(t') dt' \exp(-i\omega t) -
%      \int_{-\infty}^{\infty} x(t) \exp(-i \omega t) dt
%   \]
%   There is no added simplifications that can be made to this expression since $
%   \lim_{t \to \infty} \exp(-i \omega t) $ doesn't exist. All we can do is
%   recognize the third term as $ X(\omega) $.
%   \[
%      \frac{1}{-i \omega} \lim_{t \to \infty} \int_{0}^{t} x(t') dt' \exp(-i\omega t) -
%      \frac{1}{-i \omega} \lim_{t \to -\infty} \int_{0}^{t} x(t') dt' \exp(-i\omega t) -
%      X(\omega).
%   \]
\end{homeworkProblem}
