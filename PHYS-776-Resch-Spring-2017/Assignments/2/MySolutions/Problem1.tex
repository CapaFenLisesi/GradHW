% Problem 1
\begin{homeworkProblem}
    The Michelson visibility of a source, as a function of the delay $ \tau $
    has been expressed as
    \[
        V(\tau) =
        \frac{I_{\textrm{max}}(\tau) -
        I_{\textrm{min}}(\tau)}{I_{\textrm{max}}(\tau) + I_{\textrm{min}}(\tau)}
        \enskip.
    \]
    Now, this expression is not a good expression. We will demonstrate why in
    this problem and offer a superior expression at the end. The visibility
    attempts to quantify the degree of variation in the signal intensity as the
    delay changes over small values of $ \tau $. In most experiments with
    Michelson interferometry, Increasing the delay by large amounts tends to, on
    average, reduce the intensity. However, increasing the delay by small
    amounts has the potential to swing the intensity by much larger amounts than
    would happen (on the average) with large increases in the delay. In other
    words. $ V(\tau) $ tends to have a quickly oscillating function of $ \tau $
    and a slow decaying function of $ \tau $. When describing the visibility we
    are interested in the variation of the oscillatory function according to the
    slow envelope which guides its decay. In order to get our hands on these
    envelope and oscillatory functions we should calculated $ I(\tau) $, the
    intensity as a function of delay.

    \begin{align}
        I(\tau) &= \epsilon_{0} c |t r|^2 E_{0}^{2} \left( 1 + \Re(g^{(1)}(\tau) \right)
    \end{align}
    We are interested in a source which has been Doppler-broadened. The
    first-order coherence of this source is given in Eq.~4.34 of the course
    notes and is
    \[
        g^{(1)}(\tau) = e^{-i \omega_{0} \tau - \frac{1}{2}\tau^{2} \sigma^{2}}
        \enskip.
    \]
    Thus, $ I(\tau) = \epsilon_{0} c |t r|^2 E_{0}^{2} \left( 1 +
    e^{-\frac{1}{2}\tau^2 \sigma^2}\cos(\omega_{0}\tau) \right) $. Now, $
    \sigma^{2} = \omega_{0}^{2} \frac{k T}{m c^2} $. Assuming a room-temperature
    source (\SI{25}{\celsius}), a HeNe gas laser (weighted mass of $ \approx
    \SI{1.07e-26}{\kilo\gram} $, wavelength of \SI{632.8}{\nano\meter}), and $
    \epsilon_{0} c |tr|^2 E_{0}^2 = 1 $ we can plot the visibility of this
    source as a function of delay.

    \begin{figure}[ht]
        \centering
        \begin{tikzpicture}[]
\begin{axis}[height = {101.6mm}, ylabel = {$I(\textrm{\tau})$}, xmin = {0.0}, xmax = {6.332221480987324}, ymax = {2}, xlabel = {$\tau$ (femtoseconds)}, {unbounded coords=jump, xticklabel style={rotate = 0}, yticklabel style={rotate = 0},     xshift = 0.0mm,
    yshift = 0.0mm,
    axis background/.style={fill={rgb,1:red,1.00000000;green,1.00000000;blue,1.00000000}}
, grid = major}, ymin = {0}, width = {152.4mm}]\addplot+ [color = {rgb,1:red,0.00000000;green,0.60560316;blue,0.97868012},
draw opacity=1.0,
line width=1,
solid,mark = none,
mark size = 2.0,
mark options = {
    color = {rgb,1:red,0.00000000;green,0.00000000;blue,0.00000000}, draw opacity = 1.0,
    fill = {rgb,1:red,0.00000000;green,0.60560316;blue,0.97868012}, fill opacity = 1.0,
    line width = 1,
    rotate = 0,
    solid
},forget plot]coordinates {
(0.0, 2.0)
(0.021107404936624417, 1.9980267284282631)
(0.04221480987324883, 1.9921147013144442)
(0.06332221480987324, 1.9822872507286138)
(0.08442961974649767, 1.9685831611285)
(0.10553702468312208, 1.951056516294952)
(0.12664442961974648, 1.929776485887968)
(0.14775183455637092, 1.9048270524656443)
(0.16885923949299533, 1.8763066800433887)
(0.18996664442961975, 1.8443279255014362)
(0.21107404936624416, 1.8090169943742624)
(0.23218145430286857, 1.7705132427749999)
(0.25328885923949296, 1.7289686274205227)
(0.2743962641761174, 1.684547105927709)
(0.29550366911274184, 1.637423989747632)
(0.3166110740493662, 1.5877852522913534)
(0.33771847898599067, 1.5358267949778353)
(0.35882588392261505, 1.4817536741005364)
(0.3799332888592395, 1.4257792915639045)
(0.4010406937958639, 1.3681245526835526)
(0.4221480987324883, 1.3090169943739007)
(0.4432555036691127, 1.2486898871639263)
(0.46436290860573715, 1.1873813145849568)
(0.48547031354236153, 1.125333233563743)
(0.5065777184789859, 1.0627905195290073)
(0.5276851234156104, 1.0000000000000002)
(0.5487925283522348, 0.9372094804710462)
(0.5698999332888592, 0.8746667664364696)
(0.5910073382254837, 0.8126186854155194)
(0.612114743162108, 0.7513101128369166)
(0.6332221480987325, 0.690983005627408)
(0.6543295530353569, 0.6318754473183179)
(0.6754369579719813, 0.5742207084386197)
(0.6965443629086057, 0.5182463259027276)
(0.7176517678452301, 0.4641732050262488)
(0.7387591727818545, 0.4122147477136243)
(0.759866577718479, 0.3625760102583061)
(0.7809739826551034, 0.3154528940792475)
(0.8020813875917278, 0.27103137258750254)
(0.8231887925283522, 0.22948675723413514)
(0.8442961974649766, 0.190983005636014)
(0.8654036024016011, 0.15567207451000387)
(0.8865110073382254, 0.1236933199692265)
(0.9076184122748499, 0.09517294754814809)
(0.9287258172114743, 0.07022351412699157)
(0.9498332221480987, 0.0489434837211552)
(0.9709406270847231, 0.031416838888724596)
(0.9920480320213475, 0.017712749289686136)
(1.0131554369579718, 0.007885298704878863)
(1.0342628418945963, 0.0019732715920203203)
(1.0553702468312207, 2.1170287745064797e-11)
(1.0764776517678452, 0.001973271593710635)
(1.0975850567044696, 0.007885298708239397)
(1.118692461641094, 0.01771274929467692)
(1.1397998665777185, 0.031416838895286014)
(1.160907271514343, 0.04894348372920865)
(1.1820146764509674, 0.07022351413643968)
(1.2031220813875916, 0.09517294755887495)
(1.224229486324216, 0.12369331998109923)
(1.2453368912608405, 0.15567207452287335)
(1.266444296197465, 0.19098300564971527)
(1.2875517011340893, 0.2294867572484891)
(1.3086591060707138, 0.271031372602317)
(1.3297665110073382, 0.31545289409431865)
(1.3508739159439627, 0.3625760102734191)
(1.371981320880587, 0.4122147477285564)
(1.3930887258172113, 0.464173205040768)
(1.4141961307538358, 0.5182463259165978)
(1.4353035356904602, 0.5742207084515991)
(1.4564109406270846, 0.6318754473301638)
(1.477518345563709, 0.6909830056378747)
(1.4986257505003335, 0.7513101128457607)
(1.519733155436958, 0.8126186854225003)
(1.5408405603735824, 0.8746667664413512)
(1.5619479653102069, 0.9372094804735975)
(1.583055370246831, 0.9999999999999998)
(1.6041627751834555, 1.0627905195262417)
(1.62527018012008, 1.1253332335580106)
(1.6463775850567044, 1.1873813145760703)
(1.6674849899933288, 1.248689887151711)
(1.6885923949299533, 1.3090169943581997)
(1.7096997998665777, 1.3681245526642247)
(1.7308072048032022, 1.425779291540829)
(1.7519146097398264, 1.481753674073611)
(1.7730220146764508, 1.53582679494698)
(1.7941294196130753, 1.5877852522565103)
(1.8152368245496997, 1.6374239897087675)
(1.8363442294863241, 1.6845471058848118)
(1.8574516344229486, 1.7289686273736076)
(1.878559039359573, 1.770513242724106)
(1.8996664442961975, 1.809016994319455)
(1.920773849232822, 1.8443279254428067)
(1.9418812541694461, 1.8763066799810546)
(1.9629886591060706, 1.904827052399749)
(1.984096064042695, 1.9297764858186812)
(2.0052034689793197, 1.9510565162224691)
(2.0263108739159437, 1.9685831610530407)
(2.047418278852568, 1.9822872506504234)
(2.0685256837891925, 1.9921147012337912)
(2.089633088725817, 1.9980267283454394)
(2.1107404936624414, 1.9999999999153188)
(2.131847898599066, 1.9980267283420587)
(2.1529553035356903, 1.9921147012270706)
(2.1740627084723148, 1.982287250640442)
(2.195170113408939, 1.9685831610399176)
(2.2162775183455636, 1.9510565162063624)
(2.237384923282188, 1.9297764857997857)
(2.2584923282188125, 1.9048270523782955)
(2.279599733155437, 1.8763066799573092)
(2.3007071380920614, 1.8443279254170681)
(2.321814543028686, 1.8090169942920524)
(2.3429219479653103, 1.7705132426953976)
(2.3640293529019347, 1.7289686273439777)
(2.385136757838559, 1.6845471058546693)
(2.406244162775183, 1.63742398967854)
(2.4273515677118076, 1.5877852522266473)
(2.448458972648432, 1.535826794917942)
(2.4695663775850565, 1.4817536740458712)
(2.490673782521681, 1.4257792915148704)
(2.5117811874583054, 1.3681245526405343)
(2.53288859239493, 1.3090169943372667)
(2.5539959973315542, 1.2486898871340224)
(2.5751034022681787, 1.1873813145621082)
(2.596210807204803, 1.1253332335482478)
(2.6173182121414276, 1.0627905195211385)
(2.638425617078052, 1.0000000000000002)
(2.6595330220146765, 0.9372094804791289)
(2.680640426951301, 0.8746667664528142)
(2.7017478318879253, 0.8126186854402743)
(2.7228552368245498, 0.7513101128701918)
(2.743962641761174, 0.6909830056692766)
(2.765070046697798, 0.6318754473688191)
(2.7861774516344227, 0.5742207084977513)
(2.807284856571047, 0.5182463259704491)
(2.8283922615076715, 0.4641732051024777)
(2.849499666444296, 0.41221474779824074)
(2.8706070713809204, 0.36257601035114806)
(2.891714476317545, 0.3154528941801126)
(2.9128218812541693, 0.27103137269614663)
(2.9339292861907937, 0.22948675735027746)
(2.955036691127418, 0.19098300575932936)
(2.9761440960640426, 0.15567207464013166)
(2.997251501000667, 0.12369332010576706)
(3.0183589059372915, 0.09517294769066498)
(3.039466310873916, 0.07022351427501305)
(3.0605737158105404, 0.048943483874174576)
(3.081681120747165, 0.0314168390462044)
(3.1027885256837893, 0.01771274945105772)
(3.1238959306204137, 0.007885298869545254)
(3.1450033355570377, 0.0019732717593581306)
(3.166110740493662, 1.9053258970558318e-10)
(3.1872181454302866, 0.001973271764428852)
(3.208325550366911, 0.007885298879626523)
(3.2294329553035355, 0.017712749466030298)
(3.25054036024016, 0.03141683906588888)
(3.2716477651767844, 0.048943483898334805)
(3.292755170113409, 0.07022351430335672)
(3.3138625750500332, 0.09517294772284501)
(3.3349699799866577, 0.12369332014138523)
(3.356077384923282, 0.15567207467873956)
(3.3771847898599066, 0.19098300580043404)
(3.398292194796531, 0.22948675739333946)
(3.4193995997331554, 0.2710313727405915)
(3.44050700466978, 0.31545289422532596)
(3.4616144096064043, 0.3625760103964889)
(3.4827218145430288, 0.41221474784303713)
(3.5038292194796528, 0.46417320514603655)
(3.524936624416277, 0.5182463260120582)
(3.5460440293529016, 0.574220708536689)
(3.567151434289526, 0.6318754474043542)
(3.5882588392261505, 0.6909830057006757)
(3.609366244162775, 0.7513101128967223)
(3.6304736490993994, 0.8126186854612172)
(3.651581054036024, 0.874666766467459)
(3.6726884589726483, 0.9372094804867831)
(3.6937958639092727, 0.9999999999999978)
(3.714903268845897, 1.0627905195128422)
(3.7360106737825216, 1.1253332335310526)
(3.757118078719146, 1.1873813145354482)
(3.7782254836557705, 1.2486898870973768)
(3.799332888592395, 1.3090169942901613)
(3.8204402935290194, 1.3681245525825507)
(3.841547698465644, 1.4257792914456415)
(3.8626551034022683, 1.481753673965094)
(3.8837625083388922, 1.5358267948253754)
(3.9048699132755167, 1.5877852521221203)
(3.925977318212141, 1.6374239895619476)
(3.9470847231487656, 1.6845471057259789)
(3.96819212808539, 1.7289686272032325)
(3.9892995330220145, 1.770513242542717)
(4.010406937958639, 1.8090169941276315)
(4.031514342895264, 1.8443279252411804)
(4.052621747831887, 1.8763066797703076)
(4.073729152768512, 1.9048270521806097)
(4.094836557705136, 1.9297764855919262)
(4.115943962641761, 1.9510565159889133)
(4.137051367578385, 1.9685831608135398)
(4.1581587725150095, 1.9822872504058702)
(4.179266177451634, 1.9921147009851117)
(4.200373582388258, 1.9980267280935875)
(4.221480987324883, 1.9999999996612754)
(4.242588392261507, 1.9980267280868267)
(4.263695797198132, 1.9921147009716698)
(4.284803202134756, 1.9822872503859075)
(4.305910607071381, 1.9685831607872943)
(4.327018012008005, 1.9510565159566995)
(4.3481254169446295, 1.9297764855541346)
(4.369232821881254, 1.9048270521377033)
(4.390340226817878, 1.8763066797228167)
(4.411447631754503, 1.844327925189703)
(4.432555036691127, 1.8090169940728265)
(4.453662441627752, 1.7705132424852996)
(4.474769846564376, 1.7289686271439735)
(4.495877251501001, 1.6845471056656938)
(4.516984656437625, 1.6374239895014946)
(4.5380920613742495, 1.5877852520623916)
(4.559199466310874, 1.5358267947672983)
(4.580306871247498, 1.4817536739096147)
(4.601414276184123, 1.4257792913937242)
(4.622521681120747, 1.36812455253517)
(4.643629086057372, 1.3090169942482952)
(4.664736490993996, 1.2486898870619998)
(4.685843895930621, 1.1873813145075236)
(4.706951300867245, 1.1253332335115267)
(4.728058705803869, 1.0627905195026341)
(4.749166110740494, 1.0000000000000004)
(4.770273515677118, 0.9372094804978455)
(4.791380920613742, 0.8746667664903883)
(4.812488325550366, 0.8126186854967616)
(4.833595730486991, 0.7513101129455827)
(4.854703135423615, 0.690983005763483)
(4.87581054036024, 0.6318754474816679)
(4.896917945296864, 0.5742207086289951)
(4.9180253502334885, 0.5182463261197605)
(4.939132755170113, 0.46417320526945716)
(4.960240160106737, 0.4122147479824074)
(4.981347565043362, 0.362576010551947)
(5.002454969979986, 0.31545289439691393)
(5.023562374916611, 0.2710313729282523)
(5.044669779853235, 0.22948675759691484)
(5.06577718478986, 0.19098300601966278)
(5.086884589726484, 0.15567207491325674)
(5.1079919946631085, 0.12369332039072078)
(5.129099399599733, 0.09517294798642506)
(5.150206804536357, 0.07022351458050324)
(5.171314209472982, 0.04894348418826777)
(5.192421614409606, 0.03141683936772566)
(5.213529019346231, 0.01771274977879178)
(5.234636424282855, 0.00788529920223835)
(5.25574382921948, 0.0019732720957239547)
(5.276851234156104, 5.292570826043175e-10)
(5.2979586390927285, 0.0019732721041751944)
(5.319066044029353, 0.007885299219040687)
(5.340173448965977, 0.017712749803745487)
(5.361280853902602, 0.031416839400533636)
(5.382388258839226, 0.04894348422853445)
(5.403495663775851, 0.07022351462774201)
(5.424603068712475, 0.09517294804005827)
(5.4457104736490995, 0.12369332045008385)
(5.466817878585724, 0.15567207497760482)
(5.487925283522348, 0.19098300608816954)
(5.509032688458973, 0.22948675766868565)
(5.530140093395596, 0.27103137300232616)
(5.551247498332221, 0.31545289447227043)
(5.572354903268845, 0.3625760106275122)
(5.59346230820547, 0.41221474805706504)
(5.614569713142094, 0.4641732053420523)
(5.635677118078719, 0.5182463261891117)
(5.656784523015343, 0.5742207086938922)
(5.6778919279519675, 0.6318754475408941)
(5.698999332888592, 0.690983005815816)
(5.720106737825216, 0.7513101129898027)
(5.741214142761841, 0.8126186855316693)
(5.762321547698465, 0.8746667665147944)
(5.78342895263509, 0.9372094805106037)
(5.804536357571714, 0.9999999999999976)
(5.825643762508339, 1.06279051948881)
(5.846751167444963, 1.1253332334828658)
(5.8678585723815875, 1.1873813144630896)
(5.888965977318212, 1.248689887000923)
(5.910073382254836, 1.3090169941697904)
(5.931180787191461, 1.3681245524385304)
(5.952288192128085, 1.4257792912783438)
(5.97339559706471, 1.481753673774987)
(5.994503002001334, 1.535826794613023)
(6.015610406937959, 1.5877852518881792)
(6.036717811874583, 1.637423989307171)
(6.0578252168112074, 1.684547105451208)
(6.078932621747832, 1.728968626909397)
(6.100040026684456, 1.7705132422308318)
(6.121147431621081, 1.8090169937987908)
(6.142254836557705, 1.844327924896557)
(6.16336224149433, 1.8763066794111474)
(6.184469646430954, 1.9048270518082284)
(6.2055770513675785, 1.9297764852077008)
(6.226684456304203, 1.9510565155942836)
(6.247791861240827, 1.9685831604099977)
(6.268899266177451, 1.982287249994955)
(6.290006671114075, 1.9921147005684054)
(6.3111140760507, 1.9980267276727077)
(6.332221480987324, 1.9999999992378699)
};
\end{axis}

\end{tikzpicture}

        \caption{Intensity as a function of delay time (x axis normalized to $
            \omega_0 \approx \SI{2.98}{\tera\radian\per\second}) (f_0 \approx
        \SI{474}{\tera\hertz})$.}
        \label{fig:Figure1a}
    \end{figure}

    Quite clearly, the visibility appears to be 1 for very small $ \tau $ (since
    the maximum value is close to 2 and the minimum value is close to 0).
    However, there should be a Gaussian envelope guiding the intensity function.
    This is not visible because of the time scale of this decay.

    In order for a detector to be able to reproduce the theoretical expectation
    of $ I(\tau) $ a detector would need to be able to measure/average the
    intensity over an average of $ \approx \SI{.1}{\femto\second} $. This is
    entirely impractical given the bandwidth/response times of contemporary
    photodiodes (e.g. Thorlabs FDS100 has a \SI{35}{\pico\second} rise time). In
    fact, it would be nearly impossible for the detector to average the measured
    intensity over any length of time that is significant relative to $ \omega_0
    $.

    Now, we want to quantify how the local extrema (maxima and minima) vary as a
    function of $ \tau $. Since the oscillations jjkk at such a short time
    scale relative to the envelope, we can consider the oscillations of any one
    period as being of fixed amplitude. Then, we can take the maximum as the
    maximum of the $\cos$ function in $ 1 + e^{-\frac{1}{2}\tau^2
    \sigma^2}\cos(\omega_{0}\tau) $. The minimum, correspondingly, can be given
    by $ 1 - e^{-\frac{1}{2}\tau^2 \sigma^2} $. Thus, the visibility, according
    to the above expression would be
    \[
        V(\tau) = \frac{\left( 1 + e^{-\frac{1}{2}\tau^2 \sigma^2}\right) -
        \left( 1 - e^{-\frac{1}{2}\tau^2 \sigma^2}\right)}
        {\left( 1 + e^{-\frac{1}{2}\tau^2 \sigma^2}\right) + \left( 1 -
        e^{-\frac{1}{2}\tau^2 \sigma^2}\right)}
        =
        \frac{2 e^{-\frac{1}{2}\tau^2 \sigma^2}}{2}
        =
        e^{-\frac{1}{2}\tau^2 \sigma^2}
    \]
    Note that by using the positive envelope and the negative envelope and
    the average of $ I(\tau) $ we can write the visibility, for this case,
    equivalently as
    \[
        V(\tau) = (I_{\textrm{+env}}(\tau) - I_{\textrm{-env}}(\tau)) /
        2 \overline{I} \enskip,
    \]
    where $ I_{\textrm{+/-env}} $ is the positive/negative envelope function of
    the intensity and $ \overline{I} $ is the mean of $ I(\tau) $ over all $
    \tau $. Calculating the visibility trivially reproduces the same result as
    the first expression. In fact, if the visibility function is symmetric about
    the mean, then the positive envelope will be related to the negative
    envelope: $ I_{\textrm{-env}} = 2 \overline{I} - I_{\textrm{+env}} $. Thus,
    the visibility function can, once again, be rewritten as
    \[
        V(\tau) = (2 I_{\textrm{+env}}(\tau) - 2 \overline{I}) /
        (2 \overline{I})
        = \frac{I_{\textrm{+env}}(\tau)}{\overline{I}} - 1
    \]

    The advantage of an expression like this one is two-fold: 1) Most intensity
    correlations ($ I(\tau) $) will be symmetric about the mean and 2) it's
    practically measurable. A plot of the two envelopes of the intensity
    functions and the mean is shown in Fig.~\ref{fig:Figure1b}.

    \begin{figure}[ht]
        \centering
        \begin{tikzpicture}[]
\begin{axis}[height = {101.6mm}, ylabel = {$I(\textrm{\tau})$}, xmin = {0.0}, xmax = {422.1480987324883}, ymax = {2}, xlabel = {$\tau$ (picoseconds)}, {unbounded coords=jump, xticklabel style={rotate = 0}, yticklabel style={rotate = 0},     xshift = 0.0mm,
    yshift = 0.0mm,
    axis background/.style={fill={rgb,1:red,1.00000000;green,1.00000000;blue,1.00000000}}
, grid = major}, ymin = {0}, width = {152.4mm}]\addplot+ [color = {rgb,1:red,0.00000000;green,0.60560316;blue,0.97868012},
draw opacity=1.0,
line width=1,
solid,mark = none,
mark size = 2.0,
mark options = {
    color = {rgb,1:red,0.00000000;green,0.00000000;blue,0.00000000}, draw opacity = 1.0,
    fill = {rgb,1:red,0.00000000;green,0.60560316;blue,0.97868012}, fill opacity = 1.0,
    line width = 1,
    rotate = 0,
    solid
}]coordinates {
(0.0, 2.0)
(2.1107404936624414, 1.9999153224589539)
(4.221480987324883, 1.999661332855102)
(6.332221480987323, 1.9992381602098828)
(8.442961974649766, 1.9986460194376572)
(10.553702468312206, 1.9978852111638075)
(12.664442961974647, 1.9969561214703764)
(14.77518345563709, 1.9958592215695476)
(16.88592394929953, 1.9945950674053574)
(18.996664442961972, 1.9931642991841039)
(21.107404936624413, 1.9915676408340128)
(23.218145430286853, 1.9898058993947985)
(25.328885923949294, 1.987879964337837)
(27.439626417611738, 1.9857908068177574)
(29.55036691127418, 1.9835394788563283)
(31.66110740493662, 1.9811271124596055)
(33.77184789859906, 1.9785549186693716)
(35.8825883922615, 1.975824186549985)
(37.993328885923944, 1.9729362821118221)
(40.104069379586385, 1.969892647172572)
(42.214809873248825, 1.966694798157714)
(44.325550366911266, 1.9633443248415712)
(46.436290860573706, 1.9598428890304067)
(48.54703135423615, 1.9561922231890874)
(50.65777184789859, 1.9523941290129052)
(52.768512341561035, 1.9484504759461991)
(54.879252835223475, 1.9443631996494908)
(56.989993328885916, 1.9401343004168832)
(59.10073382254836, 1.9357658415455403)
(61.2114743162108, 1.9312599476591017)
(63.32221480987324, 1.9266188029869429)
(65.43295530353568, 1.9218446496012251)
(67.54369579719813, 1.916939785613732)
(69.65443629086056, 1.911906563334512)
(71.765176784523, 1.9067473873943983)
(73.87591727818544, 1.9014647128334932)
(75.98665777184789, 1.8960610431577458)
(78.09739826551032, 1.8905389283657699)
(80.20813875917277, 1.8849009629480746)
(82.3188792528352, 1.879149783860901)
(84.42961974649765, 1.87328806847687)
(86.5403602401601, 1.8673185325146702)
(88.65110073382253, 1.8612439279500106)
(90.76184122748498, 1.8550670409100842)
(92.87258172114741, 1.8487906895537862)
(94.98332221480986, 1.8424177219399298)
(97.0940627084723, 1.8359510138857065)
(99.20480320213474, 1.8293934668176295)
(101.31554369579717, 1.8227480056171923)
(103.42628418945962, 1.816017576463462)
(105.53702468312207, 1.8092051446748167)
(107.6477651767845, 1.8023136925520151)
(109.75850567044695, 1.7953462172247734)
(111.86924616410938, 1.7883057285039954)
(113.97998665777183, 1.7811952467417833)
(116.09072715143427, 1.774017800701324)
(118.20146764509671, 1.76677642543872)
(120.31220813875915, 1.7594741601987982)
(122.4229486324216, 1.7521140463268974)
(124.53368912608404, 1.7446991251985948)
(126.64442961974648, 1.7372324361692961)
(128.75517011340892, 1.7297170145455687)
(130.86591060707136, 1.7221558895800535)
(132.9766511007338, 1.7145520824917502)
(135.08739159439625, 1.7069086045134112)
(137.19813208805868, 1.6992284549677448)
(139.30887258172112, 1.6915146193740616)
(141.41961307538355, 1.6837700675869578)
(143.530353569046, 1.6759977519685636)
(145.64109406270845, 1.6682006055958358)
(147.75183455637088, 1.6603815405043114)
(149.86257505003334, 1.652543445969684)
(151.97331554369578, 1.6446891868285003)
(154.0840560373582, 1.6368216018392197)
(156.19479653102064, 1.6289435020848042)
(158.3055370246831, 1.6210576694179633)
(160.41627751834554, 1.6131668549500935)
(162.52701801200797, 1.6052737775849053)
(164.6377585056704, 1.5973811225976506)
(166.74849899933287, 1.5894915402608125)
(168.8592394929953, 1.5816076445170415)
(170.96997998665773, 1.5737320117000653)
(173.0807204803202, 1.5658671793042247)
(175.19146097398263, 1.558015644803234)
(177.30220146764506, 1.5501798645186857)
(179.4129419613075, 1.5423622525387597)
(181.52368245496996, 1.5345651796875364)
(183.6344229486324, 1.5267909725452369)
(185.74516344229482, 1.5190419125196586)
(187.8559039359573, 1.511320234969005)
(189.96664442961972, 1.50362812837625)
(192.07738492328215, 1.4959677335751072)
(194.1881254169446, 1.4883411430276199)
(196.29886591060705, 1.4807504001533238)
(198.40960640426948, 1.4731974987098724)
(200.52034689793192, 1.4656843822249606)
(202.63108739159435, 1.4582129434793196)
(204.7418278852568, 1.450785024040504)
(206.85256837891924, 1.4434024138471333)
(208.96330887258168, 1.4360668508431975)
(211.07404936624414, 1.4287800206619847)
(213.18478985990657, 1.421543556359135)
(215.295530353569, 1.414359038194279)
(217.40627084723144, 1.407227993460668)
(219.5170113408939, 1.4001518963621562)
(221.62775183455634, 1.3931321679368558)
(223.73849232821877, 1.3861701760267335)
(225.8492328218812, 1.3792672352923867)
(227.95997331554366, 1.372424607272185)
(230.0707138092061, 1.3656435004849408)
(232.18145430286853, 1.3589250705752165)
(234.292194796531, 1.352270420500365)
(236.40293529019343, 1.3456806007583446)
(238.51367578385586, 1.3391566096553364)
(240.6244162775183, 1.332699393612155)
(242.73515677118075, 1.3263098475084183)
(244.8458972648432, 1.319988815063418)
(246.95663775850562, 1.3137370892526072)
(249.06737825216808, 1.307555412758606)
(251.17811874583052, 1.301444478455597)
(253.28885923949295, 1.2954049299259762)
(255.39959973315538, 1.289437362008102)
(257.51034022681785, 1.2835423213739738)
(259.6210807204803, 1.277720307135656)
(261.7318212141427, 1.2719717714792664)
(263.84256170780515, 1.2662971203253184)
(265.9533022014676, 1.2606967140142205)
(268.06404269513, 1.255170868015719)
(270.1747831887925, 1.2497198536610736)
(272.28552368245494, 1.244343898896747)
(274.39626417611737, 1.2390431890584037)
(276.5070046697798, 1.2338178676639968)
(278.61774516344224, 1.228668037224741)
(280.72848565710467, 1.2235937600727653)
(282.8392261507671, 1.2185950592042532)
(284.9499666444296, 1.2136719191368777)
(287.060707138092, 1.2088242867803538)
(289.17144763175446, 1.2040520723189452)
(291.2821881254169, 1.1993551501047606)
(293.3929286190793, 1.1947333595607066)
(295.50366911274176, 1.190186506091964)
(297.6144096064042, 1.1857143620048785)
(299.7251501000667, 1.1813166674321696)
(301.8358905937291, 1.1769931312633823)
(303.94663108739155, 1.1727434320795258)
(306.057371581054, 1.1685672190908558)
(308.1681120747164, 1.1644641130767925)
(310.27885256837885, 1.1604337073269722)
(312.3895930620413, 1.1564755685824655)
(314.5003335557038, 1.15258923797621)
(316.6110740493662, 1.1487742319717367)
(318.72181454302864, 1.145030043299286)
(320.8325550366911, 1.1413561418884426)
(322.9432955303535, 1.1377519757964387)
(325.05403602401594, 1.1342169721313038)
(327.1647765176784, 1.1307505379690657)
(329.2755170113408, 1.127352061264234)
(331.3862575050033, 1.1240209117528275)
(333.49699799866573, 1.1207564418472264)
(335.60773849232817, 1.1175579875221748)
(337.7184789859906, 1.1144248691912655)
(339.82921947965303, 1.1113563925732861)
(341.93995997331547, 1.108351849547825)
(344.0507004669779, 1.1054105189995636)
(346.1614409606404, 1.1025316676507138)
(348.2721814543028, 1.0997145508810877)
(350.38292194796526, 1.0969584135353088)
(352.4936624416277, 1.0942624907167113)
(354.6044029352901, 1.0916260085674976)
(356.71514342895256, 1.089048185034748)
(358.825883922615, 1.0865282306219137)
(360.9366244162775, 1.084065349125443)
(363.0473649099399, 1.0816587383562228)
(365.15810540360235, 1.0793075908455403)
(367.2688458972648, 1.077011094535303)
(369.3795863909272, 1.0747684334522698)
(371.49032688458965, 1.0725787883660851)
(373.6010673782521, 1.0704413374309214)
(375.7118078719146, 1.0683552568105674)
(377.822548365577, 1.066319721286821)
(379.93328885923944, 1.0643339048510667)
(382.0440293529019, 1.0623969812789495)
(384.1547698465643, 1.0605081246880654)
(386.26551034022674, 1.0586665100786259)
(388.3762508338892, 1.0568713138570645)
(390.4869913275516, 1.0551217143425788)
(392.5977318212141, 1.0534168922566192)
(394.70847231487653, 1.0517560311953587)
(396.81921280853896, 1.0501383180851935)
(398.9299533022014, 1.0485629436213435)
(401.04069379586383, 1.0470291026896386)
(403.15143428952626, 1.0455359947715972)
(405.2621747831887, 1.0440828243329139)
(407.3729152768512, 1.042668801195494)
(409.4836557705136, 1.0412931408931836)
(411.59439626417605, 1.0399550650113605)
(413.7051367578385, 1.0386538015105655)
(415.8158772515009, 1.0373885850343625)
(417.92661774516336, 1.036158657201633)
(420.0373582388258, 1.0349632668835178)
(422.1480987324883, 1.0338016704652349)
};
\addlegendentry{$I_{\textrm{+env}}(\tau)$}
\addplot+ [color = {rgb,1:red,0.88887350;green,0.43564919;blue,0.27812294},
draw opacity=1.0,
line width=1,
solid,mark = none,
mark size = 2.0,
mark options = {
    color = {rgb,1:red,0.00000000;green,0.00000000;blue,0.00000000}, draw opacity = 1.0,
    fill = {rgb,1:red,0.88887350;green,0.43564919;blue,0.27812294}, fill opacity = 1.0,
    line width = 1,
    rotate = 0,
    solid
}]coordinates {
(0.0, 0.0)
(2.1107404936624414, 8.467754104624081e-5)
(4.221480987324883, 0.0003386671448979417)
(6.332221480987323, 0.0007618397901171248)
(8.442961974649766, 0.0013539805623427048)
(10.553702468312206, 0.002114788836192427)
(12.664442961974647, 0.0030438785296236226)
(14.77518345563709, 0.004140778430452352)
(16.88592394929953, 0.005404932594642586)
(18.996664442961972, 0.00683570081589624)
(21.107404936624413, 0.00843235916598728)
(23.218145430286853, 0.01019410060520154)
(25.328885923949294, 0.012120035662162798)
(27.439626417611738, 0.014209193182242563)
(29.55036691127418, 0.01646052114367169)
(31.66110740493662, 0.018872887540394512)
(33.77184789859906, 0.02144508133062839)
(35.8825883922615, 0.02417581345001496)
(37.993328885923944, 0.027063717888177963)
(40.104069379586385, 0.030107352827428047)
(42.214809873248825, 0.033305201842286025)
(44.325550366911266, 0.036655675158428935)
(46.436290860573706, 0.0401571109695934)
(48.54703135423615, 0.043807776810912524)
(50.65777184789859, 0.0476058709870949)
(52.768512341561035, 0.05154952405380098)
(54.879252835223475, 0.05563680035050933)
(56.989993328885916, 0.05986569958311683)
(59.10073382254836, 0.06423415845445968)
(61.2114743162108, 0.0687400523408982)
(63.32221480987324, 0.07338119701305712)
(65.43295530353568, 0.07815535039877486)
(67.54369579719813, 0.0830602143862681)
(69.65443629086056, 0.08809343666548797)
(71.765176784523, 0.09325261260560169)
(73.87591727818544, 0.09853528716650684)
(75.98665777184789, 0.1039389568422543)
(78.09739826551032, 0.10946107163423013)
(80.20813875917277, 0.11509903705192526)
(82.3188792528352, 0.12085021613909908)
(84.42961974649765, 0.12671193152312987)
(86.5403602401601, 0.13268146748532972)
(88.65110073382253, 0.1387560720499894)
(90.76184122748498, 0.14493295908991577)
(92.87258172114741, 0.15120931044621366)
(94.98332221480986, 0.1575822780600702)
(97.0940627084723, 0.16404898611429353)
(99.20480320213474, 0.1706065331823705)
(101.31554369579717, 0.1772519943828077)
(103.42628418945962, 0.18398242353653793)
(105.53702468312207, 0.19079485532518325)
(107.6477651767845, 0.19768630744798488)
(109.75850567044695, 0.2046537827752266)
(111.86924616410938, 0.21169427149600462)
(113.97998665777183, 0.2188047532582167)
(116.09072715143427, 0.22598219929867602)
(118.20146764509671, 0.23322357456128018)
(120.31220813875915, 0.2405258398012019)
(122.4229486324216, 0.24788595367310262)
(124.53368912608404, 0.2553008748014052)
(126.64442961974648, 0.2627675638307039)
(128.75517011340892, 0.2702829854544314)
(130.86591060707136, 0.2778441104199465)
(132.9766511007338, 0.2854479175082498)
(135.08739159439625, 0.2930913954865887)
(137.19813208805868, 0.3007715450322552)
(139.30887258172112, 0.30848538062593844)
(141.41961307538355, 0.3162299324130423)
(143.530353569046, 0.3240022480314363)
(145.64109406270845, 0.33179939440416406)
(147.75183455637088, 0.3396184594956886)
(149.86257505003334, 0.34745655403031617)
(151.97331554369578, 0.3553108131714996)
(154.0840560373582, 0.36317839816078035)
(156.19479653102064, 0.3710564979151958)
(158.3055370246831, 0.37894233058203675)
(160.41627751834554, 0.3868331450499064)
(162.52701801200797, 0.3947262224150947)
(164.6377585056704, 0.4026188774023495)
(166.74849899933287, 0.4105084597391875)
(168.8592394929953, 0.4183923554829584)
(170.96997998665773, 0.42626798829993473)
(173.0807204803202, 0.4341328206957753)
(175.19146097398263, 0.4419843551967658)
(177.30220146764506, 0.4498201354813143)
(179.4129419613075, 0.4576377474612403)
(181.52368245496996, 0.46543482031246364)
(183.6344229486324, 0.4732090274547631)
(185.74516344229482, 0.4809580874803415)
(187.8559039359573, 0.4886797650309951)
(189.96664442961972, 0.49637187162374996)
(192.07738492328215, 0.5040322664248928)
(194.1881254169446, 0.5116588569723801)
(196.29886591060705, 0.5192495998466762)
(198.40960640426948, 0.5268025012901276)
(200.52034689793192, 0.5343156177750394)
(202.63108739159435, 0.5417870565206804)
(204.7418278852568, 0.549214975959496)
(206.85256837891924, 0.5565975861528666)
(208.96330887258168, 0.5639331491568024)
(211.07404936624414, 0.5712199793380154)
(213.18478985990657, 0.5784564436408651)
(215.295530353569, 0.585640961805721)
(217.40627084723144, 0.592772006539332)
(219.5170113408939, 0.5998481036378437)
(221.62775183455634, 0.6068678320631442)
(223.73849232821877, 0.6138298239732664)
(225.8492328218812, 0.6207327647076133)
(227.95997331554366, 0.6275753927278149)
(230.0707138092061, 0.6343564995150592)
(232.18145430286853, 0.6410749294247835)
(234.292194796531, 0.6477295794996348)
(236.40293529019343, 0.6543193992416554)
(238.51367578385586, 0.6608433903446636)
(240.6244162775183, 0.667300606387845)
(242.73515677118075, 0.6736901524915817)
(244.8458972648432, 0.680011184936582)
(246.95663775850562, 0.6862629107473928)
(249.06737825216808, 0.692444587241394)
(251.17811874583052, 0.6985555215444031)
(253.28885923949295, 0.7045950700740238)
(255.39959973315538, 0.7105626379918978)
(257.51034022681785, 0.7164576786260262)
(259.6210807204803, 0.722279692864344)
(261.7318212141427, 0.7280282285207337)
(263.84256170780515, 0.7337028796746817)
(265.9533022014676, 0.7393032859857795)
(268.06404269513, 0.7448291319842809)
(270.1747831887925, 0.7502801463389265)
(272.28552368245494, 0.755656101103253)
(274.39626417611737, 0.7609568109415964)
(276.5070046697798, 0.7661821323360032)
(278.61774516344224, 0.7713319627752591)
(280.72848565710467, 0.7764062399272347)
(282.8392261507671, 0.7814049407957467)
(284.9499666444296, 0.7863280808631224)
(287.060707138092, 0.7911757132196462)
(289.17144763175446, 0.7959479276810548)
(291.2821881254169, 0.8006448498952394)
(293.3929286190793, 0.8052666404392934)
(295.50366911274176, 0.809813493908036)
(297.6144096064042, 0.8142856379951215)
(299.7251501000667, 0.8186833325678305)
(301.8358905937291, 0.8230068687366177)
(303.94663108739155, 0.8272565679204742)
(306.057371581054, 0.8314327809091442)
(308.1681120747164, 0.8355358869232075)
(310.27885256837885, 0.8395662926730278)
(312.3895930620413, 0.8435244314175346)
(314.5003335557038, 0.8474107620237901)
(316.6110740493662, 0.8512257680282633)
(318.72181454302864, 0.8549699567007141)
(320.8325550366911, 0.8586438581115574)
(322.9432955303535, 0.8622480242035614)
(325.05403602401594, 0.8657830278686962)
(327.1647765176784, 0.8692494620309343)
(329.2755170113408, 0.8726479387357658)
(331.3862575050033, 0.8759790882471726)
(333.49699799866573, 0.8792435581527736)
(335.60773849232817, 0.8824420124778252)
(337.7184789859906, 0.8855751308087346)
(339.82921947965303, 0.8886436074267139)
(341.93995997331547, 0.891648150452175)
(344.0507004669779, 0.8945894810004364)
(346.1614409606404, 0.8974683323492861)
(348.2721814543028, 0.9002854491189122)
(350.38292194796526, 0.9030415864646912)
(352.4936624416277, 0.9057375092832887)
(354.6044029352901, 0.9083739914325023)
(356.71514342895256, 0.910951814965252)
(358.825883922615, 0.9134717693780863)
(360.9366244162775, 0.9159346508745569)
(363.0473649099399, 0.9183412616437773)
(365.15810540360235, 0.9206924091544596)
(367.2688458972648, 0.9229889054646969)
(369.3795863909272, 0.9252315665477302)
(371.49032688458965, 0.9274212116339149)
(373.6010673782521, 0.9295586625690786)
(375.7118078719146, 0.9316447431894326)
(377.822548365577, 0.9336802787131792)
(379.93328885923944, 0.9356660951489332)
(382.0440293529019, 0.9376030187210505)
(384.1547698465643, 0.9394918753119347)
(386.26551034022674, 0.9413334899213742)
(388.3762508338892, 0.9431286861429354)
(390.4869913275516, 0.9448782856574212)
(392.5977318212141, 0.946583107743381)
(394.70847231487653, 0.9482439688046413)
(396.81921280853896, 0.9498616819148065)
(398.9299533022014, 0.9514370563786565)
(401.04069379586383, 0.9529708973103614)
(403.15143428952626, 0.9544640052284028)
(405.2621747831887, 0.9559171756670861)
(407.3729152768512, 0.9573311988045059)
(409.4836557705136, 0.9587068591068164)
(411.59439626417605, 0.9600449349886395)
(413.7051367578385, 0.9613461984894345)
(415.8158772515009, 0.9626114149656374)
(417.92661774516336, 0.9638413427983671)
(420.0373582388258, 0.9650367331164822)
(422.1480987324883, 0.9661983295347653)
};
\addlegendentry{$I_{\textrm{-env}}(\tau)$}
\addplot+ [color = {rgb,1:red,0.24222430;green,0.64327509;blue,0.30444865},
draw opacity=1.0,
line width=1,
solid,mark = none,
mark size = 2.0,
mark options = {
    color = {rgb,1:red,0.00000000;green,0.00000000;blue,0.00000000}, draw opacity = 1.0,
    fill = {rgb,1:red,0.24222430;green,0.64327509;blue,0.30444865}, fill opacity = 1.0,
    line width = 1,
    rotate = 0,
    solid
}]coordinates {
(0.0, 1.0)
(2.1107404936624414, 1.0)
(4.221480987324883, 1.0)
(6.332221480987323, 1.0)
(8.442961974649766, 1.0)
(10.553702468312206, 1.0)
(12.664442961974647, 1.0)
(14.77518345563709, 1.0)
(16.88592394929953, 1.0)
(18.996664442961972, 1.0)
(21.107404936624413, 1.0)
(23.218145430286853, 1.0)
(25.328885923949294, 1.0)
(27.439626417611738, 1.0)
(29.55036691127418, 1.0)
(31.66110740493662, 1.0)
(33.77184789859906, 1.0)
(35.8825883922615, 1.0)
(37.993328885923944, 1.0)
(40.104069379586385, 1.0)
(42.214809873248825, 1.0)
(44.325550366911266, 1.0)
(46.436290860573706, 1.0)
(48.54703135423615, 1.0)
(50.65777184789859, 1.0)
(52.768512341561035, 1.0)
(54.879252835223475, 1.0)
(56.989993328885916, 1.0)
(59.10073382254836, 1.0)
(61.2114743162108, 1.0)
(63.32221480987324, 1.0)
(65.43295530353568, 1.0)
(67.54369579719813, 1.0)
(69.65443629086056, 1.0)
(71.765176784523, 1.0)
(73.87591727818544, 1.0)
(75.98665777184789, 1.0)
(78.09739826551032, 1.0)
(80.20813875917277, 1.0)
(82.3188792528352, 1.0)
(84.42961974649765, 1.0)
(86.5403602401601, 1.0)
(88.65110073382253, 1.0)
(90.76184122748498, 1.0)
(92.87258172114741, 1.0)
(94.98332221480986, 1.0)
(97.0940627084723, 1.0)
(99.20480320213474, 1.0)
(101.31554369579717, 1.0)
(103.42628418945962, 1.0)
(105.53702468312207, 1.0)
(107.6477651767845, 1.0)
(109.75850567044695, 1.0)
(111.86924616410938, 1.0)
(113.97998665777183, 1.0)
(116.09072715143427, 1.0)
(118.20146764509671, 1.0)
(120.31220813875915, 1.0)
(122.4229486324216, 1.0)
(124.53368912608404, 1.0)
(126.64442961974648, 1.0)
(128.75517011340892, 1.0)
(130.86591060707136, 1.0)
(132.9766511007338, 1.0)
(135.08739159439625, 1.0)
(137.19813208805868, 1.0)
(139.30887258172112, 1.0)
(141.41961307538355, 1.0)
(143.530353569046, 1.0)
(145.64109406270845, 1.0)
(147.75183455637088, 1.0)
(149.86257505003334, 1.0)
(151.97331554369578, 1.0)
(154.0840560373582, 1.0)
(156.19479653102064, 1.0)
(158.3055370246831, 1.0)
(160.41627751834554, 1.0)
(162.52701801200797, 1.0)
(164.6377585056704, 1.0)
(166.74849899933287, 1.0)
(168.8592394929953, 1.0)
(170.96997998665773, 1.0)
(173.0807204803202, 1.0)
(175.19146097398263, 1.0)
(177.30220146764506, 1.0)
(179.4129419613075, 1.0)
(181.52368245496996, 1.0)
(183.6344229486324, 1.0)
(185.74516344229482, 1.0)
(187.8559039359573, 1.0)
(189.96664442961972, 1.0)
(192.07738492328215, 1.0)
(194.1881254169446, 1.0)
(196.29886591060705, 1.0)
(198.40960640426948, 1.0)
(200.52034689793192, 1.0)
(202.63108739159435, 1.0)
(204.7418278852568, 1.0)
(206.85256837891924, 1.0)
(208.96330887258168, 1.0)
(211.07404936624414, 1.0)
(213.18478985990657, 1.0)
(215.295530353569, 1.0)
(217.40627084723144, 1.0)
(219.5170113408939, 1.0)
(221.62775183455634, 1.0)
(223.73849232821877, 1.0)
(225.8492328218812, 1.0)
(227.95997331554366, 1.0)
(230.0707138092061, 1.0)
(232.18145430286853, 1.0)
(234.292194796531, 1.0)
(236.40293529019343, 1.0)
(238.51367578385586, 1.0)
(240.6244162775183, 1.0)
(242.73515677118075, 1.0)
(244.8458972648432, 1.0)
(246.95663775850562, 1.0)
(249.06737825216808, 1.0)
(251.17811874583052, 1.0)
(253.28885923949295, 1.0)
(255.39959973315538, 1.0)
(257.51034022681785, 1.0)
(259.6210807204803, 1.0)
(261.7318212141427, 1.0)
(263.84256170780515, 1.0)
(265.9533022014676, 1.0)
(268.06404269513, 1.0)
(270.1747831887925, 1.0)
(272.28552368245494, 1.0)
(274.39626417611737, 1.0)
(276.5070046697798, 1.0)
(278.61774516344224, 1.0)
(280.72848565710467, 1.0)
(282.8392261507671, 1.0)
(284.9499666444296, 1.0)
(287.060707138092, 1.0)
(289.17144763175446, 1.0)
(291.2821881254169, 1.0)
(293.3929286190793, 1.0)
(295.50366911274176, 1.0)
(297.6144096064042, 1.0)
(299.7251501000667, 1.0)
(301.8358905937291, 1.0)
(303.94663108739155, 1.0)
(306.057371581054, 1.0)
(308.1681120747164, 1.0)
(310.27885256837885, 1.0)
(312.3895930620413, 1.0)
(314.5003335557038, 1.0)
(316.6110740493662, 1.0)
(318.72181454302864, 1.0)
(320.8325550366911, 1.0)
(322.9432955303535, 1.0)
(325.05403602401594, 1.0)
(327.1647765176784, 1.0)
(329.2755170113408, 1.0)
(331.3862575050033, 1.0)
(333.49699799866573, 1.0)
(335.60773849232817, 1.0)
(337.7184789859906, 1.0)
(339.82921947965303, 1.0)
(341.93995997331547, 1.0)
(344.0507004669779, 1.0)
(346.1614409606404, 1.0)
(348.2721814543028, 1.0)
(350.38292194796526, 1.0)
(352.4936624416277, 1.0)
(354.6044029352901, 1.0)
(356.71514342895256, 1.0)
(358.825883922615, 1.0)
(360.9366244162775, 1.0)
(363.0473649099399, 1.0)
(365.15810540360235, 1.0)
(367.2688458972648, 1.0)
(369.3795863909272, 1.0)
(371.49032688458965, 1.0)
(373.6010673782521, 1.0)
(375.7118078719146, 1.0)
(377.822548365577, 1.0)
(379.93328885923944, 1.0)
(382.0440293529019, 1.0)
(384.1547698465643, 1.0)
(386.26551034022674, 1.0)
(388.3762508338892, 1.0)
(390.4869913275516, 1.0)
(392.5977318212141, 1.0)
(394.70847231487653, 1.0)
(396.81921280853896, 1.0)
(398.9299533022014, 1.0)
(401.04069379586383, 1.0)
(403.15143428952626, 1.0)
(405.2621747831887, 1.0)
(407.3729152768512, 1.0)
(409.4836557705136, 1.0)
(411.59439626417605, 1.0)
(413.7051367578385, 1.0)
(415.8158772515009, 1.0)
(417.92661774516336, 1.0)
(420.0373582388258, 1.0)
(422.1480987324883, 1.0)
};
\addlegendentry{$\overline{I}$}
\end{axis}

\end{tikzpicture}

        \caption{Envelopes and mean of the intensity function (x axis normalized to $
            \omega_0 \approx \SI{2.98}{\tera\radian\per\second}) (f_0 \approx
        \SI{474}{\tera\hertz})$.}
        \label{fig:Figure1b}
    \end{figure}
\end{homeworkProblem}
