% Problem 3
\begin{homeworkProblem}
    \begin{homeworkSection}{a)}
        Understanding a Poissonian photon distribution to mean a coherent state,
        we can express the state after an application of a linear loss channel
        as
        \[
            e^{-\frac{\left| \alpha \right|^2}{2}} e^{\alpha a_1^\dagger } \otimes
            \mathds{1}_2
            \ket{0}_{1} \ket{0}_{2}
            \to e^{-\frac{\left| \alpha \right|^2}{2}} e^{\alpha \left(
                    \sqrt{T} a^\dagger_3 \otimes \tilde{\mathds{1}}_4 +
                    \sqrt{1-T} \left( \mathds{1}_3
                \otimes \tilde{a}^\dagger_4 \right) \right)} \ket{0}_3 \otimes
                \ket{0}_4 \enskip.
        \]
        Above, we have used the beamsplitter mode transformations and loss model
        provided in the course notes (Eq.~4.156 and Sec.~4.15). Applying the
        operators to the states yields
        \[
            e^{-\frac{\left| \alpha \right|^2}{2}} e^{\alpha \sqrt{T}
            a^\dagger_3 \otimes \tilde{\mathds{1}}_4} \ket{0}_3
            \otimes
            e^{\alpha \sqrt{1-T} \mathds{1}_3 \otimes \tilde{a}^\dagger_4 }
            \ket{0}_4 = \ket{\alpha \sqrt{T}} \otimes \ket{\alpha \sqrt{1-T}} \enskip.
        \]
    \end{homeworkSection}
    \begin{homeworkSection}{b)}
        A single-mode coherent state incident upon a beamsplitter can be
        expressed in terms of a combined input state incident upon both input
        beams of the beamsplitter as
        \[
            \ket{\psi} = \ket{\alpha}_1 \otimes \ket{0}_2 \enskip.
        \]
        Using the beamsplitter mode transformations provided in the course notes
        we can express the output state as
        \[
            \ket{\alpha}_1 \otimes \ket{0}_2 = (D(\alpha)_1 \otimes \mathds{1}_2) \ket{0}_1 \otimes
            \ket{0}_2 \to e^{-\left| \alpha \right|^2/2} e^{\alpha \left( t a_2
            + r a_4 \right)} \ket{0}_3 \otimes \ket{0}_4
            = \ket{\alpha t}_3 \otimes \ket{\alpha r}_4
        \]
    \end{homeworkSection}
\end{homeworkProblem}
