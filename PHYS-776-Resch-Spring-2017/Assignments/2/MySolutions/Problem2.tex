% Problem 2
\begin{homeworkProblem}
    The second-order coherence of a quantum state can be calculated as
    \[
        g^{(2)}(\tau) =
        \frac{\braket{a^\dagger a^\dagger a a }}
        {\braket{a^\dagger a}^2} \enskip.
    \]
    The single-mode squeezed vacuum state can be expressed in the Fock basis as
    \[
        \ket{\psi} = \frac{1}{\sqrt{\cosh(r)}} \sum_{n=0}^{\infty} (-1)^n
        \frac{\sqrt{(2n)!}}{2^n n!} e^{i n \theta} \tanh^n(r) \ket{2n} \enskip.
    \]
    The density matrix representation of this state is given by
    \begin{align}
        \rho = \ket{\psi}\bra{\psi} &=
        \frac{1}{\sqrt{\cosh(r)}} \sum_{n=0}^{\infty} (-1)^n
        \frac{\sqrt{(2n)!}}{2^n n!} e^{i n \theta} \tanh^n(r) \ket{2n}
        \frac{1}{\sqrt{\cosh(r)}} \sum^{\infty}_{m=0}  (-1)^m \frac{\sqrt{(2m)!}}{2^m
        m!} e^{-i m \theta} \tanh^m(r) \bra{2m} \\
        \ket{\psi}\bra{\psi} &=
        \frac{1}{\cosh(r)} \sum_{n=0}^{\infty}\sum^{\infty}_{m=0}
        (-1)^{n+m}
        \frac{\sqrt{(2n)!(2m)!}}{2^{n+m} n! m!} e^{i (n-m) \theta} \tanh^{n+m}(r)
        \ket{2n} \bra{2m}
    \end{align}
    Calculate the numerator of the second-order coherence by taking the trace
    over the Fock basis yields
    \begin{align}
        \Braket{a^\dagger a^\dagger a a}
        &= \sum_{k=0}^\infty \bra{k}
        a^\dagger a^\dagger a a
        \frac{1}{\cosh(r)} \sum_{n=0}^{\infty}\sum^{\infty}_{m=0}
        (-1)^{n+m}
        \frac{\sqrt{(2n)!(2m)!}}{2^{n+m} n! m!} e^{i (n-m) \theta} \tanh^{n+m}(r)
        \ket{2n} \bra{2m}
        \ket{k}
        \intertext{Applying the operators to the $ \ket{2n} $ ket:}
        &= \sum_{k=0}^\infty \bra{k}
        \frac{1}{\cosh(r)} \sum_{n=0}^{\infty}\sum^{\infty}_{m=0}
        (-1)^{n+m}
        \frac{\sqrt{(2n)!(2m)!}}{2^{n+m} n! m!} e^{i (n-m) \theta} \tanh^{n+m}(r)
        2n(2n-1)\ket{2n} \bra{2m}
        \ket{k} \\
        &= \frac{1}{\cosh(r)}
        \sum_{\textrm{even $ k $}} (-1)^{k} \frac{\sqrt{k!k!}}{2^{k}
        \frac{k}{2}! \frac{k}{2}!} e^{i (\frac{k}{2} - \frac{k}{2}) \theta}
        \tanh^{\frac{k}{2} + \frac{k}{2}}(r)
        k(k-1)
        \intertext{Replacing even $ k $ with $ l = \frac{k}{2} $ yields}
        &= \frac{1}{\cosh(r)}
        \sum_{l=0}^\infty \frac{\sqrt{(2l)!^2}}{2^{2l} l!^2} 2l(2l-1)
        \tanh^{2l}(r)
    \end{align}
    Performing a similar calculation for the denominator yields
    \begin{align}
        \Braket{a^\dagger a}
        &= \sum_{k=0}^\infty \bra{k}
        a^\dagger a
        \frac{1}{\cosh(r)} \sum_{n=0}^{\infty}\sum^{\infty}_{m=0}
        (-1)^{n+m}
        \frac{\sqrt{(2n)!(2m)!}}{2^{n+m} n! m!} e^{i (n-m) \theta} \tanh^{n+m}(r)
        \ket{2n} \bra{2m}
        \ket{k} \\
        &= \sum_{k=0}^\infty
        \frac{1}{\cosh(r)} \sum_{n=0}^{\infty}\sum^{\infty}_{m=0}
        (-1)^{n+m}
        \frac{\sqrt{(2n)!(2m)!}}{2^{n+m} n! m!} e^{i (n-m) \theta} \tanh^{n+m}(r)
        2n \ket{2n} \bra{2m}
        \ket{k} \\
        &= \sum_{\textrm{even $ k $}}
        \frac{1}{\cosh(r)}
        (-1)^{k}
        \frac{\sqrt{k!^2}}{2^{k} \frac{k}{2}!^2} \tanh^{k}(r)
        k \\
        \intertext{Replacing $ l = \frac{k}{2} $:}
        &= \frac{1}{\cosh(r)}
        \sum_{l=0}^\infty
        \frac{\sqrt{(2l)!^2}}{2^{2 l} l!^2} 2l \tanh^{2 l}(r) \\
    \end{align}
    Now, we are interested in the second-order coherence of this state. In terms
    of these two expressions we have, now:
    \begin{align}
        g^{(2)}(\tau) &= \frac{\frac{1}{\cosh(r)}
            \sum_{l=0}^\infty \frac{\sqrt{(2l)!^2}}{2^{2l} l!^2} 2l(2l-1)
            \tanh^{2l}(r)}{\frac{1}{\cosh^2(r)}
            \sum_{l=0}^\infty
            \sum_{m=0}^\infty
            \frac{\sqrt{(2l)!^2}}{2^{2 l} l!^2} 2l \tanh^{2 l}(r)
        \frac{\sqrt{(2m)!^2}}{2^{2 m} m!^2} 2m \tanh^{2 m}(r) } \\
        \intertext{Calling $ f(l) = \frac{\sqrt{(2l)!^2}}{2^{2l} l!^2} 2l $
        we can rewrite this expression as}
        &= \frac{ \cosh(r)
        \sum_{l=0}^\infty f(l) (2l-1)
            \tanh^{2l}(r)}{
            \sum_{l=0}^\infty
            \sum_{m=0}^\infty
            f(l) \tanh^{2 l}(r) f(m) \tanh^{2 m}(r) }
            \intertext{Now, using Mathematica I discovered that $ \sum_l f(l)
                \tanh(r)^{2l} =
                \frac{\tanh^2(r)}{\sech^3(r)} $ and $\sum_l f(l) 2l
            \tanh(r)^{2l} = \frac{\sinh(r)^2(2+\tanh(r)^2)}{\sech(r)^3} $.}
            &= \frac{
            \cosh(r) \left( \frac{\sinh^{2}(r)\left( 2+\tanh^2(r)
                \right)}{\sech^3(r)} -
                \frac{\tanh^2(r)}{\sech^3(r)}
            \right)
            }{
            \frac{\tanh^{4}(r)}{\sech^{6}(r)}
        }\\
        &= \frac{
        \sech^2(r)
        \left( \sinh^{2}(r)\left( 2+\tanh^2(r) \right) - \tanh^2(r)
        \right)
        }{\tanh^{4}(r)}
        \\
        &= \frac{
        \sech^2(r)
        \left( \sinh^{2}(r)\left( 2+\tanh^2(r) \right) - \tanh^2(r)
        \right)
        }{\tanh^{4}(r)} \\
        &= 3 + \csch^2(r)
        \\
    \end{align}
    So, since $ \csch(r) \to 0 $ for $ r \to 0 $ the limit of large squeezing
    puts the second-order coherence $ \approx 3 $. But, for small values of
    squeezing $ \csch(r) \to \infty $. So, the ground state has a very large
    second-order coherence.

    Since there is no value of $ r $ for which the second-order coherence is
    less than 1 a squeezed vacuum state can never violate the classical bounds
    on the second-order coherence ($ \ge 1 $).

    %This has no obvious simplifications. In the limit of small squeezing, $ r
    %\to 0 $, $ \tanh^n(r) \to r^n $ and $ \cosh(r) \to 1 $. However, realizing
    %that small squeezing corresponds to a state that is ``close'' to the vacuum
    %state this brings both the denominator and the numerator to 0 ($ g^{2}(\tau)
    %= \frac{\braket{a^\dagger a^\dagger a a}}{\braket{a^\dagger a}^2} $).
    %Another way to consider the limit of small squeezing is to realize that this
    %state resembles the vacuum state and that the $ g^2(\tau) $ for Fock states
    %is
    %\[
        %g^2(\tau) = 1 - \frac{1}{n}
    %\]
    %This means that $ g^2(\tau) $ for the vacuum state is not only less than 1,
    %it would correspond to $ -\infty $. Indeed, a rigorous analysis of the
    %expression obtained above for the squeeze state yields the same thing when
    %evaluated using L'Hopital's theorem in the case of small $ r $.

    %In the limit of large $ r $, however, $ \tanh(r) \to 1 $ and $ \cosh(r) \to
    %\infty $. So, clearly this expression tends to $ \infty $.

    %Thus, there most definitely exists region where the classical bounds on $
    %g^2 $ ($ \ge 1 $) are violated by a single mode squeezed state.
\end{homeworkProblem}
