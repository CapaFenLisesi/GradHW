% Problem 8
\begin{homeworkProblem}
    A quantum state in terms of the P function (Glauber-Sudarshan P
    representation) is
    \[
        \rho = \int d^2\alpha P(\alpha) \ket{\alpha}\bra{\alpha} \enskip.
    \]
    If the state exhibits sub-Poissonian statistics then the function
    \[
        Q = \frac{\braket{{a^\dagger}^2 a^2}-\braket{a^\dagger
        a}^2}{\braket{a^\dagger a}}
    \]
    will be negative. However,
    \[
        \braket{{a^\dagger}^2 a^2} = \Tr(\int d^2\alpha P(\alpha) \ket{\alpha}
        \bra{\alpha} a^\dagger a^\dagger a a) = \int d^2\alpha P(\alpha)
        \bra{\alpha} {a^\dagger}^2 a^2 \ket{\alpha} = \int d^2\alpha P(\alpha)
        \left| \alpha \right|^4
    \]
    and
    \[
        \braket{a^\dagger a} = \int d^2\alpha P(\alpha) \bra{\alpha} a^\dagger
        a \ket{\alpha} = \int d^2\alpha P(\alpha) \left| \alpha \right|^2
        \enskip.
    \]
    This doesn't seem to prove anything until one realizes that
    \[
        \braket{{a^\dagger}^2 a^2} - {\braket{a^\dagger a}}^2 = \int d^2\alpha P(\alpha)
        \left| \alpha \right|^4 - \left( \int d^2\alpha P(\alpha) \left| \alpha
        \right|^2 \right)^2
    \]
    is nothing more than the variance of $ \left| \alpha \right|^2 $. This can
    never be negative. An alternative proof can be constructed in terms of the
    Cauchy-Schwarz inequality. But, that proof relies on the fact that $
    P(\alpha) \left| \alpha \right|^4 \in \mathcal{L}^2 $. Even though $ \int
    P(\alpha) = 1 $, in order that it be a probability distribution over $
    \alpha $, it's not guaranteed that $ \int P(\alpha) \left| \alpha \right|^4
    $ is bounded.
\end{homeworkProblem}
