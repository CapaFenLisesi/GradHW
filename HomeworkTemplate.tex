%---Packages------------------------------------------------------------------
 %------------------------------------------------------
\usepackage[T1]{fontenc}
\usepackage{lmodern}
\usepackage[english]{babel}
%\usepackage[utf8,latin1]{inputenc}
\usepackage[T1]{fontenc}
\usepackage[usenames,dvipsnames]{pstricks}
\usepackage{epsfig}
\usepackage{pst-grad} % For gradients \usepackage{pst-plot} % For axes
\usepackage{pifont}
\graphicspath{{IMG/}}
\usepackage[absolute,overlay]{textpos}
\usepackage{hyperref}
\usepackage{xcolor}
\usepackage{calc}
\usepackage{chngcntr}
\usepackage{microtype}
\usepackage[parfill]{parskip}
\usepackage{fancyhdr} % Required for custom headers
\usepackage{lastpage} % Required to determine the last page for the footer
\usepackage{extramarks} % Required for headers and footers
\usepackage{graphicx} % Required to insert images
\usepackage{lipsum} % Used for inserting dummy 'Lorem ipsum' text into the template
\usepackage{amsmath,amsthm,amsxtra,amsfonts}
\usepackage[]{dsfont}
\usepackage[toc,page]{appendix}
\usepackage[framemethod=TikZ]{mdframed}
\usepackage[]{siunitx}
\usepackage{circuitikz}
\usepackage{caption}
\usepackage[]{natbib}
\usetikzlibrary{arrows}
%\usepackage{tikz}
%\usetikzlibrary[]{external}
%\tikzexternalize{prefix=tikz/}
% Margins
\topmargin=-0.45in
\evensidemargin=0in
\oddsidemargin=0in
\textwidth=6.5in
\textheight=9.0in
\headsep=0.25in

\linespread{1.1} % Line spacing

% Set up the header and footer
\pagestyle{fancy}
\lhead{\hmwkAuthorName} % Top left header
\rhead{\courseTitle\ : \hmwkTitle} % Top center header
%\rhead{\firstxmark} % Top right header
\lfoot{\lastxmark} % Bottom left footer
\cfoot{} % Bottom center footer
\rfoot{Page\ \thepage\ of\ \pageref{LastPage}} % Bottom right footer
\renewcommand\headrulewidth{0.4pt} % Size of the header rule
\renewcommand\footrulewidth{0.4pt} % Size of the footer rule

\setlength\parindent{0pt} % Removes all indentation from paragraphs

%----------------------------------------------------------------------------------------
%	DOCUMENT STRUCTURE COMMANDS
%	Skip this unless you know what you're doing
%----------------------------------------------------------------------------------------

% Defines the problem answer command with thecontent as the only argument
\newcommand{\problemStatement}[1]{
\noindent
\begin{mdframed}[roundcorner=3pt]{
\begin{minipage}{0.98\columnwidth}
\begin{flushleft}#1\end{flushleft}
\end{minipage}}
\end{mdframed}
}

% Header and footer for when a page split occurs within a problem environment
\newcommand{\enterProblemHeader}[1]{
\nobreak\extramarks{#1}{#1 continued on next page\ldots}\nobreak
\nobreak\extramarks{#1 (continued)}{#1 continued on next page\ldots}\nobreak
}

% Header and footer for when a page split occurs between problem environments
\newcommand{\exitProblemHeader}[1]{
\nobreak\extramarks{#1 (continued)}{#1 continued on next page\ldots}\nobreak
\nobreak\extramarks{#1}{}\nobreak
}

%\renewcommand{\thesection}{}
%\renewcommand{\thesubsection}{\arabic{section}.\arabic{subsection}}
%\makeatletter
%\def\@seccntformat#1{\csname #1ignore\expandafter\endcsname\csname the#1\endcsname\quad}
%\let\sectionignore\@gobbletwo
%\let\latex@numberline\numberline
%\def\numberline#1{\if\relax#1\relax\else\latex@numberline{#1}\fi}
%\makeatother

% Try to hide the first digit in the section number
%\makeatletter
%\renewcommand{\thesection}{}
%\makeatother
%\setcounter{secnumdepth}{0} % Removes default section numbers
\newcounter{homeworkProblemCounter} % Creates a counter to keep track of the number of problems
\newcommand{\homeworkProblemName}{}
\newenvironment{homeworkProblem}[1][Problem \arabic{homeworkProblemCounter}]{ % Makes a new environment called homeworkProblem which takes 1 argument (custom name) but the default is "Problem #"
\stepcounter{homeworkProblemCounter} % Increase counter for number of problems
\renewcommand{\homeworkProblemName}{#1} % Assign \homeworkProblemName the name of the problem
\section*{\homeworkProblemName} % Make a section in the document with the custom problem count
\addcontentsline{toc}{section}{\homeworkProblemName}
\stepcounter{section} % Increase counter for section
\enterProblemHeader{Problem \arabic{homeworkProblemCounter}} % Header and footer within the environment
}{
\exitProblemHeader{Problem \arabic{homeworkProblemCounter}} % Header and footer after the environment
}

\newcommand{\problemAnswer}[1]{ % Defines the problem answer command with the content as the only argument
\noindent\framebox[\columnwidth][c]{\begin{minipage}{0.98\columnwidth}\begin{center}#1\end{center}\end{minipage}} % Makes the box around the problem answer and puts the content inside
}

\newcommand{\homeworkSectionName}{}
\newenvironment{homeworkSection}[1]{ % New environment for sections within homework problems, takes 1 argument - the name of the section
\renewcommand{\homeworkSectionName}{#1} % Assign \homeworkSectionName to the name of the section from the environment argument
\subsection*{\homeworkSectionName} % Make a subsection with the custom name of the subsection
\addcontentsline{toc}{subsection}{\homeworkSectionName}
\enterProblemHeader{\homeworkProblemName} % Header and footer within the environment
}{
\enterProblemHeader{\homeworkProblemName} % Header and footer after the environment
}

%%%%%%%%%%
\newcommand{\blu}[1]{\textcolor[rgb]{0,0,1}{#1}}
\newcommand{\bs}[1]{\boldsymbol{#1}}
%\newcommand{\V}[1]{\bm{#1}}
\newcommand{\V}[1]{\Vec{#1}}
\newcommand{\A}[1]{\Hat{#1}}
\newcommand{\W}[1]{\widehat{#1}}
\newcommand{\T}[1]{\widetilde{#1}}

\newcommand{\pd}[2]{\dfrac{\partial #1}{\partial #2}}
\newcommand {\ppds}[2]{\dfrac{\partial^2 {#1}}{\partial {#2}^2}}
\newcommand{\ppdss}[2]{\dfrac{\partial^2}{\partial #1 \partial #2}}
\newcommand{\pdtt}[3]{\dfrac{\partial^2 {#1}}{\partial {#2} \partial {#3}}}

\newcommand{\fpd}[2]{\frac{\partial #1}{\partial #2}}
\newcommand{\fpds}[1]{\frac{\partial}{\partial #1}}

\newcommand{\ignore}[1]{}

\newcommand{\der}[2]{\frac{d{#1}}{d{#2}}}
\newcommand{\vt}[1]{\Vec{\mathcal{#1}}}
\newcommand{\VP}[1]{\Vec{\mathbf{#1}}}
\newcommand{\vp}[1]{\mathbf{#1}}
\newcommand{\phas}[1]{\angle{#1}^{\circ}}
\newcommand{\er}{\epsilon_{r}}
\newcommand{\mr}{\mu_{r}}
\newcommand{\Lrw}{\Longrightarrow}
\newcommand{\refeq}[1]{(\ref{#1})}
\newcommand{\abs}[1]{\left| #1\right|}
%\newcommand{\ket}[1]{|#1\rangle}
%\newcommand{\bra}[1]{\langle #1| }
\newcommand{\intas}{\int\limits_{all\;space}}
\newcommand{\intsradial}[3]{\int\limits_{#1}^{#2} #3 r^2 \mathrm{d} r}
\newcommand{\intspolar}[3]{\int\limits_{#1}^{#2} #3 sin(\theta) \mathrm{d} \theta}
\newcommand{\intsazim}[3]{\int\limits_{#1}^{#2} #3 \mathrm{d} \phi}
\newcommand{\intcz}[3]{\int\limits_{#1}^{#2} #3 \mathrm{d} z}
\newcommand{\intcx}[3]{\int\limits_{#1}^{#2} #3 \mathrm{d} x}
\newcommand{\intcy}[3]{\int\limits_{#1}^{#2} #3 \mathrm{d} y}
\newcommand{\intavcart}[1]{\int \limits_{all\; space} #1 \, \mathrm{d} x \mathrm{d} y \mathrm{d} z}
\newcommand{\bracket}[2]{\langle#1|#2\rangle }
\newcommand*{\myalign}[2]{\multicolumn{1}{#1}{#2}}

\DeclareMathOperator{\Tr}{Tr}
% Not included in amsmath
\DeclareMathOperator{\sech}{sech}
\DeclareMathOperator{\csch}{csch}
\DeclareMathOperator{\arcsec}{arcsec}
\DeclareMathOperator{\arccot}{arcCot}
\DeclareMathOperator{\arccsc}{arcCsc}
\DeclareMathOperator{\arccosh}{arcCosh}
\DeclareMathOperator{\arcsinh}{arcsinh}
\DeclareMathOperator{\arctanh}{arctanh}
\DeclareMathOperator{\arcsech}{arcsech}
\DeclareMathOperator{\arccsch}{arcCsch}
\DeclareMathOperator{\arccoth}{arcCoth} 

% New definition of square root:
% it renames \sqrt as \oldsqrt
\let\oldsqrt\sqrt
% it defines the new \sqrt in terms of the old one
\def\sqrt{\mathpalette\DHLhksqrt}
\def\DHLhksqrt#1#2{%
\setbox0=\hbox{$#1\oldsqrt{#2\,}$}\dimen0=\ht0
\advance\dimen0-0.2\ht0
\setbox2=\hbox{\vrule height\ht0 depth -\dimen0}%
{\box0\lower0.4pt\box2}}

%%%---
\newcommand\ointint{\begingroup
\displaystyle \unitlength 1pt
\int\mkern-7.2mu
\begin{picture}(0,3)
\put(0,3){\oval(10,8)}
\end{picture}
\mkern-7mu\int\endgroup}
%%%----
\providecommand{\abs}[1]{\lvert#1\rvert}
\providecommand{\norm}[1]{\lVert#1\rVert}

%%%%%%%%%%%%%%%


% Matlab code section obtained from StackExchange: http://tex.stackexchange.com/questions/75116/what-can-i-use-to-typeset-matlab-code-in-my-document
\usepackage{listings}
\usepackage{color} %red, green, blue, yellow, cyan, magenta, black, white
\definecolor{mygreen}{RGB}{28,172,0} % color values Red, Green, Blue
\definecolor{mylilas}{RGB}{170,55,241}

\lstset{language=Matlab,%
    %basicstyle=\color{red},
    breaklines=true,%
    morekeywords={matlab2tikz},
    keywordstyle=\color{blue},%
    morekeywords=[2]{1}, keywordstyle=[2]{\color{black}},
    identifierstyle=\color{black},%
    stringstyle=\color{mylilas},
    commentstyle=\color{mygreen},%
    showstringspaces=false,%without this there will be a symbol in the places where there is a space
    numbers=left,%
    numberstyle={\tiny \color{black}},% size of the numbers
    numbersep=9pt, % this defines how far the numbers are from the text
    emph=[1]{for,end,break},emphstyle=[1]\color{red}, %some words to emphasise
    %emph=[2]{word1,word2}, emphstyle=[2]{style},
}

%----------------------------------------------------------------
\numberwithin{equation}{section}
%\numberwithin{equation}{chapter}
%\renewcommand{\theequation}{\arabic{equation}}



%%------------------------------------------------------------------------------------------
%	TITLE PAGE
%----------------------------------------------------------------------------------------

\title{
\vspace{2in}
\textmd{\textbf{\courseTitle\\ \vspace{0.5in}\hmwkTitle}}\\
\vspace{0.5in}\large{{\hmwkClassInstructor}}
\vspace{3in}
}
\author{\textbf{\hmwkAuthorName}\\
\date{\hmwkDueDate}}
