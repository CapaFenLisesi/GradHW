\begin{homeworkProblem}

\begin{homeworkSection}{a}
Charge multipole expansions are given in terms of an integral over the charge density as follows: \[ q_{lm} = \int\limits_{\text{all space}} Y_{lm}^*(\theta',\phi')r'^l \rho(\vec{x'})d\tau' \]

Now, this homework problem has a charge density which can be expressed as follows: \[ \rho(\vec{x'}) = \frac{q\delta(|\vec{x'}|)\delta(\theta'-\pi/2)}{r'^2sin^2\theta'} \big( \delta(\phi') + \delta(\phi'- \pi/2) - \delta(\phi - \pi) - \delta(\phi - 3\pi/2) \big) \]

Additionally, the spherical harmonics, $Y_{lm}$s, can be written in terms of associated Legendre polynomials as: \[Y_{lm}^*(\cos\theta) = \gamma_{lm}P_{lm}(\cos\theta)\exp(-im\phi) \]

Given all of this.

\begin{align*}
	q_{lm} = \int\limits_0^{2\pi} \int\limits_0^\pi \int\limits_0^\infty& Y_{lm}(\theta',\phi')r'^l \\ &\bigg( \frac{q\delta(|\vec{x'}|)\delta(\theta'-\pi/2)}{r'^2sin^2\theta'} \big( \delta(\phi') + \delta(\phi'- \pi/2) - \delta(\phi - \pi) - \delta(\phi - 3\pi/2) \big) \bigg) r'^2 sin^2\theta' \de r' \de\theta' \de\phi' \\
\end{align*}
\begin{align*}
	 &q_{lm} = q a^l Y_{lm}^*(\pi/2,0)+Y_{lm}^*(\pi/2,\pi/2) -Y_{lm}^*(\pi/2,\pi)-Y_{lm}^*(\pi/2,3\pi/2) \\
   &Y_{lm}(\theta,\phi) = \gamma_{lm} P_{lm}(\cos\theta)\exp(-i m \phi) \textit{,}\quad \textit{So,} \quad Y_{lm}(\pi/2,a) = \gamma_{lm} P_{lm}(0)\exp(-i m a)
\end{align*}
For a = $0,\pi/2,\pi,3\pi/2$ the expression for $\exp(-i m a)$ can be reduced as $1,(-i)^m,(-1)^m,i^m$, respectively. Given this, the expression for the multipoles can be recast as follows:

\[
    q_{lm} = q a^l \gamma_{lm} P_{lm}(0)\big( (1-(-1)^m) - (i^m - (-i)^m) \big)
\]

This can be easily seen to be zero for even $m$. Consider $f(m) =  (1-(-1)^m) - (i^m - (-i)^m $. $f(1) = 2-2i,\, f(3) = 2+2i,\, f(5) = 2-2i,\, f(7) = 2+2i,\, \cdots$. Additionally, though, $P_{lm}(x)$ is odd for any combination of $l$ and $m$ that is also odd. Since $m$ is constrained to be odd for nonzero $q_{lm}$ then $l$ must be constrained to be odd as well for nonvanishing $q_{lm}$ Thus, the final expression for $q_{lm}$ can be reduced:

\[
	q_{2j+1,2k+1} = 2q a^{2j+1} \big\{ 1+(-i)^{2k+1} \big\}\gamma_{1,1}P_{2j+1,2k+1}(0)
\]

To meet the problem demands
\begin{center}
	\begin{align*}
		q_{1,1} &= 2qa^3(1-i)\gamma_{1,1}P_{1,1}(0) = -\sqrt{\frac{3}{2\pi}}qa^3(1-i) \\
		q_{1,-1} &= 2qa^3(1-i)\gamma_{1,-1}P_{1,-1}(0) = \sqrt{\frac{3}{2\pi}}qa^3(1+i)
	\end{align*}
\end{center}

\end{homeworkSection}

\begin{homeworkSection}{b}
	In a similar fashion as before, we can construct $\rho(r,\theta,\phi) = \frac{q\delta(\phi')}{r'^2\sin^2\theta'}\big( \delta(\theta')\delta(|\vec{r'}-a\hat{z}|) + \delta(\theta'-\pi)\delta(|\vec{r'}+a\hat{z}|)- 2\delta(\theta')\delta(|\vec{r'}|)$. Thus, the $q_{lm}s$ are given by:
	\[
	\int\limits_{\text{all space}}\frac{qr'^l\delta(\phi')}{r'^2\sin^2\theta'} Y^*_{l,m}(\theta',\phi')\bigg(\delta(|\vec{r'}|-a)\delta(\theta')+\delta(|\vec{r'}|-a|)\delta(\theta'-\pi)-2\delta(r')\delta(\theta')\big)
	\]
	
	Reducing this expression yields:
	\[
	q_{lm} = -2q\delta_{l,0} + qa^l(Y^*_{l,m}(0,0)+Y^*_{l,m}(\pi,0))
	\]
	
	Since, $Y_{l,m}(\theta,\phi) = \gamma_{l,m}P_{l,m}(\cos\theta)\exp(-im\phi)$ then $Y_{l,m}(0,0)$ can be written as $\gamma_{l,m}P_{l,m}(1)$ and $Y_{l,m}(\pi,0)$ can be written as $\gamma_{l,m}P_{l,m}(-1)$.
	
	\[
	q_{l,m} = -2q\delta_{l,0}+qa^l\gamma_{l,m}(P_{l,m}(1)+P_{l,m}(-1))
	\]
	Now, $P_{l,m}$ is odd if $l+m$ is odd. Thus, in order to have nonvanishing $q_{l,m}$ it must be the case that the sum of $l$ and $m$ must be even. Additionally, it must be the case the $P_{l,m}(1) \text{\,and\,} P_{l,m}(-1) \ne 0$. Note that for all associated Legendre polynomials for which m is nonzero, $P_{l,m}(\pm 1) = 0$. This can be seen in that $P_{l,l} = (-1)^l(2l-1)!!(1-x^2)^{l/2}$. This is a restatement of the fact that this charge distribution exhibits azimuthal symmetry.  Thus, $q_{0,0}$ dies and \Leftrightarrow$q_{1,-1}$ and $q_{1,1}$ die (as a consequence of $P_{1,\pm 1}(\pm 1) = 0$). Note that $q_{1,0}$ dies, too. Thus, the first set of nonvanishing $q_{l,m}$ are:
\begin{center}
	\begin{align*}
	q_{2,0} &= qa^2\gamma_{2,0}(P_{2,0}(1) + P_{2,0}(-1)) = qa^2\sqrt{\frac{5}{4\pi}}(1+1) = qa^2\sqrt{5/\pi}
	\end{align*}
\end{center}
	
\end{homeworkSection}
%\begin{homeworkSection}
%\end{homeworkSection}

\end{homeworkProblem}