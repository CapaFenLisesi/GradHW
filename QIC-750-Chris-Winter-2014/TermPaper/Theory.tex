Consider the phase space representation of a classical system of particles. The experimentalist's ignorance as to the exact configuration of the system results in a probability distribution of the particles in phase space (where the phase space comprises the momentum and position of each particle). Now, the experimentalist's ignorance as to the quantum state of a system of interest, similarly, generates a probability distribution over the Hilbert space of that system. Typically, the representation of this state is given by the density matrix of the system. Briefly, if a quantum state is in any of k possible states (indexed as $\ket{i}$) with probability P(i), then the density matrix, $\rho$, is given as $\rho = \sum\limits_{i=0}^{k} P(i) \ketbra{i}$. It might be thought that with the proper definition of a statistical function, one can discuss the state of a quantum-mechanical system using the system's density matrix formalism in a way that is strongly analogous to that of the state-space analysis performed in classical mechanical systems. This statistical function that we define is the Wigner function $W(q,p)=\frac{1}{2\pi} \int\limits_{-\infty}^\infty e^{ipx} \bra{q-x/2}\rho\ket{q+x/2}dx$ \cite{Ulf}.

The Wigner function has many useful properties and alternative representations but these are irrelevant for this paper and will be omitted for succinctness. There are only a couple relevant notes with regards to the form of the Wigner function. One, the Wigner function is a probability density function over joint variables which are canonically conjugate (like position and momentum). But, the Wigner function can be, and often is, negative over portions of this phase space. The physical understanding of these negative regions is that the negative regions represent a resource useful for quantum computation; they represent the non-classical portions of the state \cite{Ulf}. Of course, though, this violates the typical understanding of probability density functions as strictly non-negative objects. Additionally, it should be noted that since the Wigner function is constructed using only the density matrix of the system of interest, the determination of the Wigner function determines the density matrix of the system, uniquely (and vice versa).

This is the idea behind Wigner state tomography. By making different measurements of a prepared quantum system one can determine the state of the system uniquely. Consider a quantum mechanical system defined in terms of the canonically conjugate position and momentum operators : $x=\frac{a+a^\dagger}{\sqrt{2}}$ and $p=\frac{a-a^\dagger}{i\sqrt{2}}$. One can then speak of an in-phase operator $x_\phi = p \sin(\phi)+x\cos(\phi)$ and a quadrature operator, $p_\phi = p\cos(\phi)-x\sin(\phi)$. As the Wigner distribution represents a joint probability distribution over $x(x_\phi,p_\phi)$ and $p(x_\phi,p_\phi)$ integration over $p_\phi$ results in a marginal probability distribution over $x_\phi$: $P_\phi(x_\phi)=\int\limits_{-\infty}^{\infty}W\big(x_\phi\cos(\phi)-p_\phi sin(\phi),x_\phi\sin(\phi)+p_\phi\cos(\phi)\big)dp_\phi$. Thus, to determine the Wigner function representing the particular state, it is only necessary to obtain the probability distribution over one quadrature (e.g. sweep $\phi$ and measure $x_\phi$) and take the derivative of that distribution with respect to the alternate quadrature ($p_\phi$). The physical implications behind sweeping the phase'' depends on the particular implementation. The first such successful application of Wigner state tomography to a quantum system was accomplished by Smithey, Beck, Raymer and Faridani. As their platform was an optical one, sweeping the phase corresponded to a physical rotation of reflective apparatuses. Now, this discussion has made the measurement of the Wigner function of some state sound relatively trivial. This is not the case, typically. In order to obtain useful results, one typically needs to apply some more mathematics (like maximal likelihood estimations and filtered back projections); but, the above gives the novice a rough idea as to the mathematical foundations underlying Wigner representations of quantum states.