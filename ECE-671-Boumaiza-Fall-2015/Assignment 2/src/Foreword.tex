\section{Foreword: A Designer's Note}
\label{sec:Foreword}
The design process is an iterative one. Sometimes the entire foundation of the
design (and, by extension, the design itself) must be completely re-worked when
it is realized that the initial conditions lead to a final design that does not
satisfy all design criteria. Thus, no justification will be given as to how the
final designs that are recorded here were obtained. Instead, they will stand on
their own as results that are provided to the reader. No analytic resources will
be provided that help a future designer improve the current design.

Although, the specific designs will (and maybe can not) be rationalized, I will
briefly describe the recommended workflow for a project such as this. These
designs were not made by adhering to this recommended workflow.

It should be noted that if bias points of a transistor are chosen using ideal
components, first, to meet specified design criteria it may make it impossible,
later, to find non-ideal components that provide satisfactory performance. The
act of replacing ideal components with real components (capacitors, transmission
lines, inductors, resistors) will affect every measurable amplifier parameter:
noise figure, stability, gain, insertion loss, return loss, etc. If the
non-ideal components differ significantly from the ideal components used to
establish the original bias conditions then it may, for example, make it
impossible to stabilize the transistor without introducing resistors that will
increase the noise figure to an unacceptable level. Continuing this example: It
may, then, be necessary to return to adjust the biasing conditions until a
satisfactory noise figure is obtained given the transistor is to remain stable.

In short, the design should be constructed first! That is, the topology of the
design should be established (components in place, values being variable). Then,
the various parameters of the layout can be adjusted until the required design
specifications can be met as well as possible. It may even be possible to add
more parameters than might otherwise be used (a shunt resistor at the output and
a series resistor at the output, for stability) to improve the performance of
the design beyond what could be possible by using a restricted layout (only a
shunt resistor at the output).
