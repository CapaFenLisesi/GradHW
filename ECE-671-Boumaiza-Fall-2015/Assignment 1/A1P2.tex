\section{Different Tuning Circuits}
\addtocounter{section}{1}
\setcounter{equation}{0}
\subsection{Type 1: Single Open Stub Tuner}

The goal is to match a load impedance to a transmission line by placing a stub
of a certain length $l_s$ a certain distance $l_l$ away from a load. To be
matched means that the input impedance looks like the characteristic impedance
$Z_c$. To the source, the two paths (the stub and the load) will seem to be
connected in parallel. Thus, quite generally:

\[ 
        Z_{in} = Z_{stub} || Z'_l
\]

where $Z'_l$ is the impedance of the load transformed by a certain length
$l_{l}$ down the line. A load of impedance $Z_l$ looks the following
impedance when we're a length ``l'' down the line:

\[ 
        Z'_l(l) = Z_c \frac{Z_l + j Z_c \tan \beta l}{Z_c + j Z_l \tan \beta l} 
\]

However, because we are combining parallel impedances it will be easier to deal
with admittances:

\[ 
        Y_{in} = Y_{stub} + Y'_l 
\]

where

\[ 
        Y'_l(l_l) = Y_c \frac{Y_l + j Y_c\tan\beta l_l}{Y_c + j Y_l \tan \beta
        l_l}
\]

Our goal is to make $\Re \left( Y_{in} \right) = Z_c$ and $ \Im \left( Y_{in}
\right) = 0$. We have an expression for $Y'_l$ already. We'd like an expression
for $Y_{stub}$. However, $Y_{stub}$ is just a infinite impedance (zero
admittance) load. Thus, we can use the same equation as before before:

\begin{align*}
    Y_{stub}(l_s) &= Y_c \frac{Y_s + j Y_c\tan\beta l_s}{Y_c + j Y_s \tan \beta
       l_s}  \\
       &= Y_c \frac{0 + jY_c\tan\beta l_s}{Y_c + j0\tan\beta l_s} \\
       &= j Y_c \tan \beta l_s
\end{align*}

Unfortunately, the load can not, in general, be simplified any further. Thus,
the input admittance is:
\begin{align*}
    Y_{in} = Y_c \left( \frac{Y_l + j Y_c\tan\beta l_l}{Y_c + j Y_l \tan \beta
    l_l} + j \tan \beta l_s \right)
\end{align*}

From this, we can consider the following two equations:

\begin{align*}
    1 = \Re \left( Y_{in} \right) \\
    0 = \Im \left( Y_{in} \right)
\end{align*}

These two equations can be solved simultaneously for pairs of $l_l$ and $l_s$
that satisfy them. Given the transcendental nature of these two solutions it is
not, in general, possible to find an analytic solution. However, computers are
more than capable, these days, of determining the roots of such equations with
much ease. Mathematica was employed to find the following roots:

\[ 
        \beta_{l_l} \approx .955~\text{radians \quad} \beta_{l_s} \approx
        2.53~\text{radians}
\]

These values were used in ADS to simulate the frequency response of the single
stub tuner.

%TODO: Make a plot of these for alpha = 2 (since the load is 100 ohms)
%TODO: Solve for l_s and l_l

\subsection{Type 2: LC Matching Network}
The goal in constructing a matching network for the load using an LC network is
going to be similar as that that was performed in the previous section. Namely,
our goal is going to be to make $Z_{in} = Z_c$ such that $\Re \left( Z_{in}
\right) = Z_c$ and $\Im \left( Z_{in} \right) = 0$. It is easy, in this case to
construct the input impedance as 

\[ 
        Z_{in} = Z_A + Z_B || Z_l 
\]

The only knowledge we have, currently, regarding $Z_A$ and $Z_B$ is that both
impedances are purely imaginary. We know that $Z_l$ is purely real ($
\SI{100}{\ohm}$). Thus, we start by considering $\Re \left( Z_{in} \right)$.

\begin{align*}
    \Re(Z_{in}) = Z_c &= \Re ( \frac{Z_B Z_l}{Z_B + Z_l} )\\
                      &= \Re ( \frac{j X_B R_l}{j X_B + R_l} ) \\
                      &= \frac{ X_B^2 R_l}{X_B^2 + R_l^2}
\end{align*}

Considering the imaginary part of $Z_{in}$

\begin{align}
    \Im \left( Z_{in} \right) = 0 &= \Im \left( j X_A + \frac{j X_B
    R_l}{j X_B + R_l} \right)	 \\
    &= \Im \left( j X_A + \frac{j X_B R_l \left( R_l - j X_B \right)}{
    X_B^2 + R_l^2} \right) \\
    &= X_A + \frac{ X_B R_l^2}{ X_B^2 + R_l^2}
\end{align}

The real part of $Z_{in}$ only involves $Z_B$. So, we can easily solve for $Z_B$
that way. Doing so yields:

\[ 
    X_B^2 = \frac{R_l^2 Z_c}{R_l-Z_c}
\]

Since we have constrained $X_B$ to be real (so that the impedance $Z_B$ is purely
reactive) we know that we must take the positive root. We also know that this
only works if $R > Z_c$. If the load is smaller than the characteristic
impedance than an LC matching network will not work. You must add some
resistance.

Now, $X_A$ is easily determined:

\begin{align*}
    X_A &= -\frac{X_B R_l^2}{X_B^2 + R_l^2} \\
        &= - \frac{X_B}{\frac{Z_c}{R_l - Z_c} + 1} \\
        &= -\frac{X_B \left( R_l - Z_c \right)}{Z_c + R_l - Z_c} \\
        &= -\frac{X_B \left( R_l - Z_c \right)}{R_l} \\
        &= -\sqrt{\frac{Z_c}{R_l - Z_c}} \left( R_l - Z_c \right) \\
        &= -\sqrt{Z_c \left( R_l-Z_c \right)}
\end{align*}

If one allows, again, $R_l = \alpha Z_c$ then

\[ 
    X_B^2 = \frac{\alpha^2 Z_c^2}{\alpha - 1} 
\]

and 

\[ 
        X_A = -Z_c\sqrt{\alpha - 1}
\]

%TODO; Determine C and L from the above equations.

Plugging in the relevant values $\alpha = \frac{R_l}{Z_c} = 2$ and $Z_c = 2$
yields the following $\SI{50}{\ohm}$

\[ 
        X_A = - Z_c = -\SI{50}{\ohm} \text{\quad and \quad} X_B = 2 Z_c = \SI{100}{\ohm}
\]

This implies that $Z_A$ is a capacitor of value $Z_A = j X_A = \frac{1}{j\omega
C_A}$ and $Z_B$ is an inductor of value $Z_B = j X_B = j \omega L_B$. Thus:

\[
\arraycolsep=20pt
\begin{array}{cc}
    C_A = -\frac{1}{\omega X_A} = 
        \SI[parse-numbers = false]{\frac{1}{50\omega}}{\farad} & 
    L_B = \frac{X_B}{\omega} = 
        \SI[parse-numbers = false]{\frac{100}{\omega}}{\henry}
\end{array}
\]

The reflection coefficient for the load and the matching network can be found by
considering $ \Gamma_{in} = \frac{Z_{in} - Z_c}{Z_{in} + Z_c} $. In this case,
$\Gamma_{in}$ is a function of frequency.

\begin{align*}
    Z_{in} &= Z_A + Z_B||Z_l \\
           &= j X_A + \frac{j X_B R_l }{jX_B + R_l} \\
           &= \frac{1}{j \omega \frac{1}{50 \omega_m}}~\Omega + \frac{j \omega
\frac{100}{\omega_m}R_l}{j \omega \frac{100}{\omega_m} + R_l}~\Omega \\
&= \frac{50 \omega_m}{j \omega}~\Omega + \frac{j \omega 100 R_l}{j \omega 100 +
\omega_m R_l}~\Omega
\end{align*}

In the above, $\omega_m$ is the frequency at which we have chosen to match our
network. Let's consider a parameter $\gamma = \frac{\omega}{\omega_m}$ which we
will adjust from $.5\omega_m \rightarrow 1.5\omega_m$. The previous expression
for $Z_{in}$ can be written, now, as:

\[ 
        Z_{in} = \frac{50}{j \gamma}~\Omega + \frac{j 100 \gamma R_l}{j 100 \gamma +
        R_l}~\Omega
\]

Since 100 bears a nice relationship with $R_l = 100 \Omega $ we can simplify
$Z_{in}$ even further in this particular case:

\[ 
        Z_{in} = \frac{50}{j \gamma}~\Omega + \frac{j \gamma }{j \gamma +
        1}~\Omega
\]

Let's assume an operation frequency of \SI{1}{\giga\hertz} as an example. This
determines a capacitance of \SI[parse-numbers=false]{\frac{1}{50\cdot 2\cdot \pi
\cdot 1 \cdot 10^{9}}}{\farad} $ \approx $ \SI{3.18}{\pico\farad} and an
inductance of \SI[parse-numbers=false]{\frac{100}{2\pi\cdot 10^{9}}}{\ohm}
$\approx$ \SI{15.9}{\nano\henry}.

These values were used in ADS to simulate the reflection coefficient as a
function of frequency. 
\subsection{Type 3: Double Open Stub Tuner}

This problem is very similar to Problem 2a with the single stub tuner. The only
difference, now is that we have two stubs instead of one. We will follow a very
similar path as we did for problem 2a to solve for the matching network.

\begin{align}
    Y_{in} = Y_{stub_2} + Y_{stub_1}' + Y_l'
\end{align}

Where, here, $Y'_{stub_1}$ is the impedance of the stub 1 transformed from the
end of stub 1 to the start of stub 2. Like before:

\begin{align*}
    Y_l'(l_{l_1}) &= Y_c \frac{Y_l + j Y_c \tan \beta l_{l_1}}{Y_c + j Y_l \tan
    \beta l_{l_1}} \\
\end{align*}

Where, unlike before, we have to transform the impedance of stub 1 (which is
$l_{s_1}$ long) and the load a length of $l_{l_1}$ along the line. $Y_{stub_1}(l_{s_1}) = j Y_c
\tan \beta l_{s_1}$. Right at the first stub, though, the admittance of both the
first stub and the load is:

\[ 
        Y_1 = j Y_c \tan \beta l_{s_1} + Y_c \frac{Y_l + j Y_c \tan \beta
        l_{l_1}}{Y_c + j Y_l \tan \beta l_{l_1}}
\]



Thus, the total admittance at the 2nd stub (closer to the source, farther from
the load) is:

\begin{align*}
    Y_2 =  Y_c \frac{Y_1(l_{s_1},l_{l_1}) + j Y_c \tan \beta l_{l_2}}{Y_c + j
        Y_1(l_{s_1},l_{l_1}) \tan \beta{l_{l_2}}}
\end{align*}


This admittance must relate to the admittance of stub 2 in the following way:

\begin{align*}
    \Re \left( Y_2 \right) &= Y_c \\
    \Im \left( Y_2 \right) &= -\Im \left( Y_{stub_2} \right) = - \Im \left( j
Y_c \tan \beta l_{s_2} \right)
\end{align*}
There are four unknowns in the above equation for the admittance: $l_{s_1}, l_{s_2},
l_{l_1} \text{and~} l_{l_2}$. We have been given, in addition to the problem
statement, the information that $l_{l_2} = l_{l_1} = \frac{\lambda}{8}$. I'm
going to assume that both stubs are the same length and allow the distance from
stub 1 to the load to vary. Using Mathematica, again, to solve for these roots
yields:

\[ 
        \beta l_{s_2} \approx 65.8 \degree \text{\quad and } \beta l_{s_1}
        \approx 54.1 \degree
\]

\subsection{Comparison of Different Matching Networks}

It seems that based on the bandwidths of the reflection coefficients alone, the
LC matching network is the best option.
