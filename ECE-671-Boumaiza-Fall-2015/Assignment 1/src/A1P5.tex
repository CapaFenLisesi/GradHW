\section{Series-Connected Two-Port Networks}
\setcounter{equation}{0}
\addtocounter{section}{1}

What's important to realize in this problem is that the same current that flows
out of the transistor's emitter flows into the inductor. Thus, the inductor (a
2-port device) is connected in series with the transistor (a 2-port device). A
nice way to solve for the `black box'' that represents series-connected two port
devices is to use the impedance matrix. The relationship between the impedance
matrix and the voltages and currents of the two port is:

\[ 
    \begin{pmatrix} V_1 \\ V_2 \end{pmatrix} =
    \begin{pmatrix} Z_{11} & Z_{12} \\ Z_{21} & Z_{22} \end{pmatrix} 
    \begin{pmatrix} I_1 \\ I_2 \end{pmatrix}
\]

So, if we consider two two-port networks given by two impedance matrices $Z_a$
and $Z_b$ respectively and we know that $I_{1_a} = I_{1_b}$ and that $I_{2_a} =
I_{2_b}$
then we can say that

\[ 
    \begin{pmatrix} V_{1} \\ V_{2} \end{pmatrix} =  \\
    \begin{pmatrix} V_{1_a} \\ V_{2_a} \end{pmatrix} + \begin{pmatrix} V_{1_b} \\
    V_{2_b} \end{pmatrix} =
    Z_a \begin{pmatrix} I_{1_a} \\ I_{2_a} \end{pmatrix} + Z_b \begin{pmatrix}
    I_{1_b} \\ I_{2_b} \end{pmatrix} =
    \left( Z_a+Z_b \right) \begin{pmatrix} I_{1} \\ I_{2} \end{pmatrix}
\]

We are given scattering parameters for the transistor and we don't know the
scattering parameters or the impedance parameters for the inductor so we might
seem stuck. However, as shown in the appendix it is possible to transform
%TODO: Type up tho S -> Z Derivation
between scattering parameters and impedance parameters. So, if we transform the
transistor parameters to impedance parameters and then add the to-be-determined
inductor impedance parameters we can add these two impedance matrices together
and go back to scattering parameters.

Determining the impedance parameters of the inductor is not too hard. To
determine the diagonal entries we need to drive one port of the inductor while
leaving the other port open. The voltage with which we drive the inductor will
determine a current through the inductor. The voltage we choose at port one divided by the
current we measure through port one determines $Z_{11}$ and likewise for
$Z_{22}$. By inspection it is easy to determine that $Z_{11}$ and $Z_{22}$ are
both $j \omega L$. For the off diagonal terms we need to drive one port with
current and measure the voltage built up across the other port when it is open.
However, again, any current, $I_1$ with which we drive port 1 will determine a
voltage across port 2 that is $j \omega L I_1$. So, $Z_{21} = Z_{12} = Z_{11} =
Z_{22} = j \omega L$.

\[ 
    Z_I = \begin{pmatrix} j \omega L & j \omega L \\ j \omega L & j \omega L
    \end{pmatrix} 
\]

$Z_I$ is the impedance matrix of the inductor. To obtain the impedance
parameters for the transistor we consider the following equations:

\begin{align}
    Z_{11} &= Z_c \frac{\left( 1+S_{11} \right)\left( 1-S_{22}
        \right)+S_{12}S_{21}}{\left( 1-S_{11} \right)\left( 1-S_{22}
\right)-S_{12}S_{21}} \\
    Z_{12} &= Z_c\frac{2 S_{12}}{\left( 1-S_{11}
\right)\left( 1-S_{22} \right)-S_{12}S_{21}} \\
    Z_{21} &= Z_c \frac{2S_{21}}{\left( 1-S_{11} \right)\left( 1-S_{22}
\right)-S_{12}S_{21}} \\
Z_{22} &= Z_c \frac{\left( 1-S_{11} \right)\left( 1+S_{22}
\right)+S_{12}S_{21}}{\left( 1-S_{11} \right)\left( 1-S_{22}
\right)-S_{12}S_{21}}
\end{align}

Plugging in the numbers that were given for the transistor yields:

\[ 
        Z_T \approx \begin{pmatrix}
            26.6 \phase{-60.4\degree} & 0 \\ 0 & 98.6 \phase{-79.7 \degree}
        \end{pmatrix} 
\]

If the inductor is a \SI{100}{\pico\henry} inductor then it has an impedance of
$j (2\pi \cdot 10*10^{9}).1 \cdot 10^{-9} = 2\pi$~$\Omega$ at 10 GHz. Adding the
impedance matrix of the inductor, $Z_I$ to that of the transistor, $Z_T$,
yields:

\[ 
    Z_T + Z_I \approx \begin{pmatrix}
        20.8\phase{-62.1 \degree} & j2\pi \\
        j2\pi & 92.8 \phase{-68.3 \degree}
    \end{pmatrix} 
\]

Using the following relationships to convert these impedance parameters to
scattering parameters yields:

\[ 
        S_{I+T} \approx \begin{pmatrix}
            .717 \phase{-138\degree} & .083 \phase{153\degree} \\
            .083 \phase{153\degree} & .723 \phase{-55.0 \degree}
    \end{pmatrix} 
\]


