\section*{Introduction}
\addcontentsline{toc}{section}{Motivation}
Filters are an important part of the RF/Microwave signal chain. They can be used
to prevent aliasing of signals used with nonlinear devices (like mixers) and
also to prevent interference from signals with proximal frequency components.
Ever filter is qualified by a number of metrics: insertion loss, rejection,
selectivity, group delay, etc. The topology, number and quality of components
determines the filter's behavior and, as such, must be chosen very carefully.

To add to the filter designer's struggles, it can be shown that it is not
possible to design a filter with arbitrarily large bandwidth and arbitrarily high
gain. The derivation of the gain-bandwidth theorem can be found in reference
\cite{wcd}. Thus, it is up to the designer to push the gain-bandwidth
limits to the limit. This requires high-fidelity components and an optimal
design scheme. Material science and component design is under active research.
This will improve the reliability and performance of the components, themselves.

What is needed, then, is to consider a methodology with which the proper
components for an optimal filter design can be considered. The simplified real
frequency technique (SRFT), devised in 1982 by Herbert Carlin and Binboga Yarman
addresses this problem \cite{srft}. Essentially, the SRFT achieves an optimal
filter design by using computer-aided optimization. This report will describe
SRFT and a MATLAB\textsuperscript{\textregistered} implementation of the SRFT
currently under development. The goal of this report is to demonstrate the
efficacy of the SRFT and a particular microwave design which was implement using
the aformentioned MATLAB\textsuperscript{\textregistered} implementation.
