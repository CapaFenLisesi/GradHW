\documentclass{article}
\usepackage[]{amsmath}
\usepackage[]{braket}
\begin{document}
Jerry Knight 
Loudon.
Ulf Leonhardt
Garrison and Chow

\section{Quantization of EM Field}
\begin{align}
    \nabla \cdot E = 0
    \nabla \cdot V = 0
    \nabla \times E = - \frac{\partial B}{\partial t}
    \nabla \times B = \mu_0 \epsilon_0 \frac{\partial E}{\partial t}
\end{align}
\[ 
    c = \frac{1}{\sqrt{\mu_0 \epsilon_0}} 
\]

We can introduce potentials $\phi, \vec{A}$ such that $\vec{B} = \nabla \times
\vec{A}$ and $\vec{E} - \nabla \phi - \frac{\partial\vec{A}}{\partial t}$ . We
can choose some gauge that preserves these fields. 

\[ 
\vec{A} \rightarrow \vec{A} + \nabla \Gamma(r,t)
\phi \rightarrow \phi - \frac{\partial \Gamma}{\partial t}
\]

We typically elect to choose
the Coulumb gauge:

\[ 
    \nabla \cdot A = 0 
\]

None of the physical description changes. We can consider the wave equation:

\[ 
    \nabla \times B = \frac{1}{c^2} \frac{\partial E}{\partial t} 
\]

By taking the curl of both sides we can write

\[ 
    \nabla \cdot \nabla A = \frac{1}{c^2} \frac{\partial }{\partial t} \left(
    -\frac{\partial A}{\partial t} \right) \quad(\phi = 0~\text{indicates absence of
    charge})
\]

using

\[ 
    \nabla \times \nabla \times A = \nabla \left( \nabla \cdot F \right) -
    \nabla^2 F
\]

and realizing that $\nabla(\nabla \cdot A)$ is zero by the choice of gauge. So,
now,

\[ 
    \nabla^2 A = \frac{1}{c^2}\frac{\partial^2 E}{\partial t^2}
\]

We are going to try and solve the above expression in terms of plane waves
(because the solution is much nicer). We will use a ``mode expansion'' to solve
for A.

\[ 
    A(r,t) = \sum_k A_k(t)e^{i \vec{k}\cdot \vec{r}} + A_k^*(t)*e^{-i \vec{k}
    \cdot \vec{r}} 
\]


So, we're going to select some cubic volume of space (where each side is of
length L). This volume is a tool we're going to use to establish boundary
conditions in order to solve the wave equation. We will use periodic boundary
conditions. For the other common convention (perfectly reflecting walls), this
is covered in Loudon. The Casimir effect, for example, utilizes boundary
conditions that are perfectly reflecting. But, we will mostly use periodic
boundary conditions in Quantum Optics.

Note that for A to be singular-valued we need to place some restrictions on
$\vec{k}$.

\[ 
    k_{k,y,z} = \frac{2\pi n_{x,y,z}}{L} \quad,n_{x,y,z} \in \{0,1,2,ldots,\}
\]

Our gauge condition is satisfied as long as $\vec{k} \cdot A = 0$. There are two
free directions for $\vec{A}(t)$ for any given k. These correspond to the
polarizations of light. We can substitute our mode expansion into the wave
equation.

\[ 
    \nabla^2 \left(\sum_k A_k(t)e^{i \vec{k}\cdot \vec{r}} + A_k^*(t)*e^{-i
    \vec{k} \cdot \vec{r}} \right) 
    = 
    \frac{1}{c^2} \frac{\partial^2}{\partial t^2} \left( \sum_k A_k(t)e^{i
    \vec{k}\cdot \vec{r}} + A_k^*(t)*e^{-i \vec{k} \cdot \vec{r}} \right) 
\]

For this to be satisfied, for all r,t we require it to be satisfied for each
$A_k$. We can solve this wave equation one mode at a time.

\[ 
    \nabla^2 A_k(t) e^{ i vec{k} \cdot \vec{r} } = \frac{1}{c^2}
    \frac{\partial^2}{\partial t^2} A_k(t) e^{ i \vec{k}\cdot \vec{r} } 
\]

This has the solution:

\[ 
    A_k(t) = A_k e^{-i \omega_k t} \quad, \omega = c |\vec{k}|
\]

We obtained the above expression using $\lambda \nu = c$ and realizing that
$2\pi\nu = \frac{2\pi c}{\lambda} = kc$. 

\[
    A(r,t) = \sum_k A_k e^{ -i \omega_k t + i \vec{k} \cdot \vec{r} } +
    \text{c.c.}
\]

(c.c denotes ``complex conjugate). From this solution, we can write the electric
and magnetic fields. These fields involve the time derivative and spatial derivative
of A, respectively.

\begin{align*}
    E_k &= -\frac{\partial }{\partial t} \left( A_k e^{-i \omega_k t + i
    \vec{k}
\cdot \vec{r}} + \text{c.c.} \right) \\
    &= i \omega_k \left( A_k e^{-i \omega_k t + i \vec{k}
    \cdot \vec{r}} + \text{c.c.} \right)
\end{align*}

\begin{align*} 
    B_k &= \nabla \times \left( A_k e^{-i \omega_k t + i \vec{k}
\cdot \vec{r}} + \text{c.c.} \right)  \\
&=  i \vec{k} \left( A_k e^{-i \omega_k t + i \vec{k} \cdot \vec{r}} +
    \text{c.c.} \right)
\end{align*}

If we consider the energy of a single mode:

\[ 
    H_k = \frac{1}{2}\int_{V} dV \left( \epsilon_0 E_k^2 + \mu_0^{-1} B_k^2 \right) 
\]

Using our boundary conditions:

\[ 
    \int_0^L e^{\pm i k_x x} dx  = 
    \begin{cases} 
        L if k_x = 0 \\ 
        0 if k_x \ne 0
    \end{cases}
\]

Using
\[ 
    \int dV e^{\pm i \left( \vec{a} - \vec{a}' \right)\cdot \vec{r}} = V
    \delta_{a,a'}
\]

So, now, considering our expressions for E and B that we obtained earlier:

\[
    \frac{1}{2}\int_V dV \epsilon_0 E_k^2 = \epsilon_0 V \omega_k^2 A_k \cdot
    A_k^*
\]

and 

\[ 
    \frac{1}{2} \int_V \mu_0^-1 B_k^2 = \epsilon_0 V \omega_k^2 A_k \cdot A_k^* 
\]

where we have to use $\left( A \times B \right)\cdot \left( C \times D \right) =
\left( A\cdot C \right)\left( B\cdot D \right) - \left( A \cdot D \right) \left( B
\cdot C\right)$. So, our energy in a single mode is, now:

\[ 
    H_k = 2 \epsilon_0 V \omega_k^2 A_k  A_k^* 
\]

These amplitudes (or mode variables) can be written in terms of a mode position
and a mode momentum.

\[ 
    A_k = \frac{1}{4\epsilon_0 V \omega_k^2} \left( \omega_k X_k + i P_k
    \right)\hat{\epsilon_k} 
\]

Where $\hat{\epsilon_k}$ is a unit vector (the polarization of the mode). Of
course,


\[ 
    A_k^* = \frac{1}{4\epsilon_0 V \omega_k^2} \left( \omega_k X_k - i P_k
    \right)\hat{\epsilon_k} 
\]

The normalization constant has been used in $A_k$ and $A_k^*$ so that when we
express the energy of the mode in terms of these quantities:

\[ 
    H_k = \frac{1}{2} \left( P_k^2 + \omega_k^2 X_k^2 \right) 
\]

This is the form of a harmonic oscillator. So, we have shown that the energy in
each mode is that of a harmonic oscillator. To obtain the energy of the field
(that stored in all of the modes):

\[ 
    H = \sum_k H_k 
\]

To take this classical discussion into the realm of quantum mechanics we need to
express the classical harmonic oscillator in terms of the quantum harmonic
oscillator.

In quantum mechanics,

\[ 
    \hat{H} = \frac{1}{2m} \left( \hat{p}^2 + m^2\omega^2 \hat{x}^2 \right)
\]

where the commutation relation between $\hat{x}$ and $\hat{p}$ is $[x,p] =
i\hbar$. We now introduce two new operators:

\[
    \alpha = \frac{1}{\sqrt{2\hbar m \omega}} \left( i \hat{p} + m \omega
    \hat{x} \right)
    \alpha^\dagger = \frac{1}{\sqrt{2 \hbar m \omega}} \left( -i \hat{p} + m
    \omega \hat{x} \right)
\]

where, $[a,a^\dagger] = 1$. Now, we can write:

\[ 
    H = \hbar \omega \left( a^\dagger a + 1/2 \right) 
\]

Now, if we know any eigenstate of $H$ ($H\ket{\Psi_n} = E_n \ket{\Psi_n}$). It
can be shown that $a\ket{\Psi_n}$ and $a^\dagger \ket{\Psi_n}$ are also
eigenstates of $H$.

\[ 
    H(a\ket{Psi_n}) = (E_n - \hbar \omega) a \ket{\Psi_n} 
\]

and

\[ 
    H(a^\dagger \ket{Psi_n}) = (E_n + \hbar \omega) a^\dagger \ket{\Psi_n} 
\]

So, once we know one energy eigenstate we can derive a whole ladder of states.
One problem we can run into is that if we apply the lowering operator too many
times, the energy of the system will become negative. But, our Hamiltonian looks
like the sum of squares of two operators. We know that by orthonormality of
states $\braket{\phi_x} \ge 0$ so $\bra{\phi} x^2 \ket{\phi} \ge 0$ (the same
holds for $\hat{p}$.

To fix this issue we constrain the lowest energy state $\ket{\Psi_0}$ such that
$a \ket{\Psi_0} = 0$. What is the energy associated with $\ket{\Psi_0}$?

\[ 
    H \ket{\Psi_0} = \hbar \omega \left( a^\dagger a + \frac{1}{2} \right)
    \ket{\Psi_0} = \frac{\hbar \omega}{2} \ket{\Psi_0} 
\]

We can apply $a^\dagger$ as many times as we want.

\[ 
    H \ket{\Psi_n} = \hbar \omega \left( n + 1/2 \right) \ket{\Psi_n}
\]

We can now introduce a new operator $\hat{n} \equiv a^\dagger a$ which will
index our states. The only other piece of information that we need is how these
operators affect the normalization of states. Consider some state $\ket{\Psi_n}$
is normalized. We know that $a \ket{\Psi_n}$ and $a^\dagger \ket{\Psi_n}$ are
eigenstates but we are ignorant of the resulting normalization.

\begin{align*}
    a \ket{\Psi_n} =& c_n \ket{\Psi_{n-1}} \\
    \bra{\Psi_n} a^\dagger a \ket{\Psi_n} =& \left| c_n \right|^2 \\
    \intertext{So, $n = c_n^2$ and $c_n = \sqrt{n}$}
    a \ket{n} = \sqrt{n} \ket{n-1}
\end{align*}

Similarly,

\begin{align*}
    a^\dagger \ket{\Psi_n} =& d_n \ket{\Psi_{n+1}} \\
    \bra{\Psi_n} a a^\dagger \ket{\Psi_n} =& \left| d_n \right|^2 \\
    \intertext{Using the commutation relation: $a a^\dagger = a^\dagger a + 1$.}
    a^\dagger \ket{n} = \sqrt{n+1}\ket{n+1}
\end{align*}

Now, we can write an numbers state as $\ket{n} = \frac{(a^\dagger)^n}{\sqrt{n!}}
\ket{0}$. If we wanted a wave function, we could write the ground state in terms
of the position representation and solve a differential equation obtaining
Hermite polynomials times Gaussian functions as solutions.

The quantum harmonic oscillator is really simplified using the operator
approach. The other thing that's really nice about the operator approach is that
it make calculating expectation values really easy. Consider

\begin{align*}
    \hat{a} &= \frac{1}{2\hbar m \omega} \left( i \hat{p} + m \omega \hat{x} \right)	
    \hat{a^\dagger} &= \frac{1}{2\hbar m \omega} \left( -i \hat{p} + m \omega \hat{x} \right)	
\end{align*}

We can take linear combinations of these operators to form $\hat{x}$:

\[ 
    \hat{x} = \sqrt{\frac{\hbar}{2 m \omega}} \left( a + a^\dagger \right)
\]

and

\[ 
    \hat{p} =  i \sqrt{\frac{\hbar m \omega}{2}} \left( a + a^\dagger \right)
\]

In order to make our classical expression for the energy look like the quantum
harmonic oscillator we will make substitutions of operators in place of
variables. We could write $A_k = \frac{1}{\sqrt{4 \epsilon_0 V \omega_k}} \left(
\omega_k X_k + i P_k\right) \hat{\epsilon_k}$. To quantize $A_k$ we replace X and P
with their quantum mechanical operator equivalents:

\[ 
    \frac{1}{\sqrt{4 \epsilon_0 V \omega_k}} \left( \omega_k \hat{x}_k + i
    \hat{p}_k \right) \hat{\epsilon_k} = \sqrt{\frac{\hbar}{2 \epsilon_0
    V \omega_k}} a_k \hat{\epsilon_k}
\]

and, similarly, for $A_k^*$

\[ 
    \frac{1}{\sqrt{4 \epsilon_0 V \omega_k}} \left( \omega_k \hat{x}_k - i
    \hat{p}_k \right) \hat{\epsilon_k} = \sqrt{\frac{\hbar}{2 \epsilon_0
    V \omega_k}} a_k^\dagger \hat{\epsilon_k}
\]

The commutation properties for the field operators arise from the commutation
properties of position and momentum operators in different systems. Because, the
separate modes really indicate a separate configuration of the system. So,

\[
    [x_k, x_k'] = 0 = [p_k, p_k']
    [x,y] = 0
    [x_k , p_k'] = i \hbar \delta_{k,k'}
\]

From these relations we can discover:

\[ 
    [a_k, a_k' ] = 0 =[a_k^\dagger, a_{k'}^\dagger]
    [a_k, a_{k'}^\dagger] = \delta_{k,k'}
\]

Our vector potential operator $A = \sum_k \sqrt{\frac{\hbar}{2\epsilon_0 V
\omega_k}} \hat{\epsilon_k} \left( a_k e^{-i \omega_k t + i \vec{k}\cdot
\vec{r}} + a_k^\dagger e^{i \omega_k t - i \vec{k} \cdot \vec{r}} \right)$. Note
that A is time dependent. Thus, A is expressed in terms of the Heisenberg
picture. We will, at some times during this course, express A in terms of the
Schrodinger picture (but, not usually).

\[ 
    \hat{E_k} = i \sqrt{\frac{\hbar \omega_k}{2 \epsilon_0 V}} \hat{\epsilon_k}
    \left(a_k e^{-i \omega_k t + i \vec{k}\cdot
\vec{r}} + a_k^\dagger e^{i \omega_k t - i \vec{k} \cdot \vec{r}} \right)
\]

and

\[ 
    \hat{B_k} = i \sqrt{\frac{\hbar }{2 \epsilon_0 V \omega_k}}
    \left(\hat{k} \times \hat{\epsilon_k} \right)
    \left(a_k e^{-i \omega_k t + i \vec{k}\cdot
\vec{r}} + a_k^\dagger e^{i \omega_k t - i \vec{k} \cdot \vec{r}} \right)
\]

To demonstrate that this formalism is valid, we can consider the energy of a
single mode:

\begin{align*}
    H_k =& \frac{1}{2} \int_V dV \left( \epsilon_0 \hat{E_k}^2 + \mu_0^-1
\hat{B_k}^2 \right) \\
=& \hbar \omega_k \left( a_k^\dagger a_k + 1/2 \right) ,\quad \text(steps not
shown in lecture)
\end{align*}

The hamiltonian for the entire system will just be the sum of the hamiltonians
for the various modes.

Now, we have related the energy of a free electromagnetic mode to that of a
quantum harmonic oscillator. The excitations of this oscillator are what we
call photons.

The next thing that we will need are what are called ``quadrature operators''.
These are directly analogous to position and momentum in the one-dimensional
harmonic oscillator. The quadrature operators $\hat{q}$ and $\hat{p}$ are given
by:

\[ 
    \hat{q} = \frac{1}{\sqrt{2}} \left( a + a^\dagger \right) \quad,\quad
    \hat{p} = \frac{i}{\sqrt{2}} \left( a^\dagger - a \right)
\]

and

\[ 
    [\hat{q},\hat{p}] = i 
\]

Based on Roberton's uncertainty relation: $\Delta q \Delta p \ge \frac{1}{2}$.
Now, we define quadrature eigenstates:

\[ 
    \hat{q}\ket{q} = q \ket{q} \quad,\quad \hat{p} \ket{p} =p \ket{p}
\]

\[ 
    \braket{q}{q'} = \delta(q-q') \quad,\quad \braket{p}{p'} = \delta(p-p') 
\]

And, furthermore, $\ket{q}$ and $\ket{p}$ are related through a Fourier
transform.

\[ 
    \ket{p} = \frac{1}{\sqrt{2\pi}} \int e^{i p q} \ket{q} dq 
\]

This allows us to write:

\begin{align*} 
    \hat{E_k} =& i \sqrt{\frac{\hbar \omega_k}{2 \epsilon_0 V}} \hat{\epsilon_k}
    \left(a_k e^{-i \omega_k t + i \vec{k}\cdot
\vec{r}} + a_k^\dagger e^{i \omega_k t - i \vec{k} \cdot \vec{r}} \right) \\
=& \sqrt{2} \sqrt{\frac{\hbar \omega_k}{2 \epsilon_0 V}} \left( q \sin(\omega_k
t - \vec{k}\cdot \vec{r}) - p \cos(\omega_k t - \vec{k} \cdot \vec{r}) \right)
\end{align*}

So, these quadrature operators describe the amplitude of E. Let's introduce a
phase shift operator.

\[ 
    U(\theta) = e^{i \theta \hat{n}} 
\]

This is the exponentiated hamiltonian of the mode (excepting the constant factor
of 1/2). That is, it describes the free (unitary) evolution of the field. One
identity that we will need fairly often is the Baker-Campbell-Haussdorff (BCH) lemma:

\[
    e^A B E^-A = B + [A,B] + 1/2 [A,[A,B]] + \frac{1}{3!} [A,[A,[A,B]]] + ...
\]

This comes up a lot when we want to calculate expectation values. Consider some
ket $\ket{Psi_\theta} = U(\theta) \ket{\Psi}$. The position expectation of this
state would require the use of the BCH lemma.

Consider $U^\dagger a U$. We can show that this reduces to $a e^{-i \theta}$.
Similarly, $U^\dagger a^\dagger U = a^\dagger e^{i \theta}$. If we, instead,
apply U to the quadrature operators we can consider $q_\theta = U^\dagger q U =
q \cos(\theta) + p \sin(\theta)$. Similarly $p_\theta = U^\dagger p U = -q
\sin(\theta) + p\cos(\theta)$. This rotation of q and p is reminiscent of the
classical phase space depiction of a harmonic oscillator (a circle).

We will now introduce a phase operators. This is an operator that will attempt
to describe the phase of the electromagnetic field. In classical
electromagnetism we can write the field for a plane wave as an amplitude and
complex exponential as follows: $E = \frac{1}{2} \left( e^{-i \omega t + i
    \vec{k} \cdot \vec{r} + i \phi} + e^{i \omega t - i \vec{k} \cdot \vec{r} - i
\phi} \right)$. Here $\phi$ is the phase of the electromagnetic wave. This looks
similar to our Q operator:

\[ 
    E_k = i \sqrt{\frac{\hbar \omega}{2 \epsilon_0 V}} \hat{\epsilon_k} \left(
    a_k e^{-i \omega_k t + i k \cdot r} - 
    a_k^\dagger e^{i \omega t - i k \cdot l }\right)
\]

if $a_k$ was written in polar form the we could retrieve the phase $\phi$. This
is what Dirac did. $a = e^{i \phi} \sqrt{\hat{n}}$ and $a^\dagger =
\sqrt{\hat{n}} e^{-i \phi}$. So, now we consider:

\[ 
    a a^\dagger - a^\dagger a = 1 = e^{i \phi} \sqrt{n}\sqrt{n} e^{-i \phi} -
    \sqrt{n}e^{-i \phi} e^{i \phi} \sqrt{n} = e^{i \phi} n e^{-i \phi} - n  
\]

Multiplying by $e^{i\phi}$  from the right yields:

\[
    e^{i \phi} n - n e^{i\phi} = e^{i \phi}
\]

we can expand these exponentials and compare term-by-term.

\begin{align*}
    [1 + i \phi + \frac{(i\phi)^2}{2}+\ldots]n -n[1+i \phi + \frac{\left( i\phi
    \right)^2}{2} + \ldots] =& 1 + i \phi + \frac{\left( i\phi \right)^2}{2} +
        \ldots \\
        \intertext{ The zero order term is zero and the first order term is just
        $i \phi n - n i \phi = 1$ such that $[n,\phi] = i$. We obtain the exact
        same expression at third order.}
    \end{align*}

Using the Robertson uncertainty relation we can write $\sigma_n \sigma_\phi \ge
\frac{1}{2}$. There is a problem with this and we will address this later. So,
this is more of a ``rule of thumb'' than it is a hard and fast rule.

A state with an exactly determined number of photons has no phase. In order to
have a phase we need to not have an exactly determined number of photons. In
order to elevate this from a rule of thumb to a hard rule we need $\hat{\phi}$
to be hermitian such that $e^{i\hat{\phi}}$ is unitary.

\[ 
    e^{i \hat{\phi}} = a n^{-1/2} 
\]

\[ 
    e^{i \hat{\phi}} = a n^{-1/2} \ket{n} = \ket{n-1} 
\]

This operator should not operate on the ground state or else it will annihilate
the state and normalization will not be preserved by this unitary operator.

He writes a lot of braket algebra and realizes that $e^{i\hat{\phi}} \ket{0} =
0$. Thus, $\bra{0} ((e^{i\phi})^\dagger e^{i \phi} \ket{0} = 0$ so $e^{i\phi}$ is
not unitary and $\phi$ is not hermitian. Other attempts have been made to
determine a phase operator (Barnett-Pegg, Susskind-Glogower,
Noh-Fougeres-Mandel) but they fail for different reasons. We don't have a good
notion of a phase operator, yet.

\section*{Important Quantum States of Light}
The Fock states (number states) are eigenstates of the mode Hamliltonian with
energy $\hbar \omega \left( n + 1/2 \right)$. These states are not eigenstates
of $H = \sum_k H_k$:

\[ 
    \bra{n} E \ket{n} = i \sqrt{\frac{\hbar \omega}{2 \epsilon_0 V}} \bra{n}
    a e^{-i \omega t + i \ket{k}\cdot \ket{r}} - c.c. \ket{n} = 0
\]

\[ 
    \bra{n} E^2 \ket{n} = \frac{- \hbar \omega}{2 \epsilon_0 V} \bra{n} -a
    a^\dagger - a^\dagger a \ket{n} = \frac{\hbar \omega}{\epsilon_0 V} \left( n
    + 1/2 \right)
\]

uncertainty in E: $\Delta E = \sqrt{<E^2> - <E>^2} = \sqrt{\frac{\hbar
\omega}{\epsilon_0 V} \left( n+1/2 \right)}$. The uncertainty in E grows with
more photons. Note that even the vacuums state has nonzero uncertainty.
\end{document} 
