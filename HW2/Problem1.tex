\begin{homeworkProblem}

\begin{homeworkSection}{a}
\textbf{A hollow cube has conducting walls defined by six planes $x = 0$, $y = 0$, $z = 0$, and 
$x = a$, $y = a$, $z = a$. The walls $z = 0$ and $z = a$ are held at a constant potential V. 
The other four sides are at zero potential.}

\textbf{Find the potential $\Phi(x, y, z)$ at any point inside the cube.}

The solution to a problem of this type is found by solving Laplace's equation, $\nabla^2 = 0$ everywhere in space where charges are absent. Given the rectangular symmetry of the problem I will propose a solution in Cartesian coordinates. That is, I will propose a solution of the form: $V(x,y,z) = V_x(x)V_y(y)V_z(z)$. Inserting this into Laplace's equation above yields:

\begin{align}
\nabla^2 V &= V_y(y)V_z(z)\ppds{V_x(x)}{x} + V_x(x)V_z(z)\ppds{V_y(y)}{y} + V_x(x)V_y(y)\ppds{V_z(z)}{z} = 0 \nonumber \\
0 &= \dfrac{1}{V_x(x)}\ppds{V_x(x)}{x} + \dfrac{1}{V(y)}\ppds{V_y(y)}{y} + \dfrac{1}{V_z(z)}V_x(x)V_y(y)\ppds{V_z(z)}{z} \nonumber
\end{align}

This implies that each term in the sum is a constant such that the sum of all of the terms is zero for any particular x, y or z.
\[
\ppds{V_x(x)}{x} = -\alpha^2 V_x(x) \quad,\quad \ppds{V_y(y)}{y} = -\beta^2 V_y(y) \quad,\quad \ppds{V_z(z)}{z} = -\gamma^2 V_z(z) \nonumber
\]

\begin{align}
V_x(x) &= A \exp(-i \alpha x) + B \exp(i \alpha x) \nonumber \\
V_y(y) &= C \exp(-i \beta y) + D \exp(i \beta y) \nonumber \\
V_z(z) &= E \exp(\sqrt{\alpha^2+\beta^2} z) + F \exp(-\sqrt{\alpha^2+\beta^2} z) \nonumber
\end{align}

Solving for A and B using $V_x(x)$'s two boundary conditions:

\begin{align}
V_x(a) &= 0 = A + B \quad,\quad 0 = A (\exp(-i \alpha a) - \exp(i \alpha a)) = -2iA \sin(\alpha a) \quad,\enskip \alpha = \frac{n\pi}{a}\enskip,\enskip n=0,1,2,... \nonumber \\
\intertext{Solving for C and D using $V_y(y)$'s two boundary conditions:} \nonumber 
V_y(a) &= 0 = C + D \quad,\quad 0 = C (\exp(-i \alpha a) - \exp(i \alpha a)) = -2iC \sin(\alpha a) \quad,\enskip \alpha = \frac{m\pi}{a}\enskip,\enskip n=0,1,2,... \nonumber \\
\intertext{Solving for E and F using $V_z(z)$'s two boundary conditions:} \nonumber
\Phi(x,y,z)&=\sum\limits_{odd n > 0}\sum\limits_{odd m > 0} \sin(\frac{n \pi x}{a}) \sin(\frac{m \pi x}{a}) (A_{nm}\exp(-\theta z) + B_{nm}\exp(\theta z)) \nonumber \\
\Phi(x,y,0) &= V = \sum\limits_{odd n > 0}\sum\limits_{odd m > 0} \sin(\frac{n \pi x}{a}) \sin(\frac{m \pi x}{a}) (A_{nm}+ B_{nm}) \nonumber \\
\Phi(x,y,a) &= V = \sum\limits_{odd n > 0}\sum\limits_{odd m > 0} \sin(\frac{n \pi x}{a}) \sin(\frac{m \pi x}{a}) (A_{nm}\exp(-\theta a) + B_{nm}\exp(\theta a)) \nonumber 
\end{align}

\begin{align}
\intertext{Using the orthogonality of sine functions I can multiply both sides by two functions $f(x) = sin(\frac{k \pi x}{a})$ and $g(y) = sin(\frac{l \pi y}{a})$ and integrate over the domain of the box to obtain the following two results. Note that I have used the following two integral identities: $\int\limits_0^a sin(\frac{k \pi x}{a}) dx = \frac{2a}{k\pi}$ for odd k (the integral is zero for even k) and $\int\limits_0^a sin^2(\frac{k \pi x}{a}) dx = \frac{a}{2}$.}
\int\limits_0^a \int\limits_0^a V sin(\frac{k \pi x}{a}) sin(\frac{l \pi y}{a}) dx dy &= \int\limits_0^a sin^2(\frac{k \pi x}{a}) sin^2(\frac{l \pi y}{a}) (A_{kl} + B_{kl}) dx dy \nonumber \\
\frac{16 V^2}{k l \pi} &= A_{kl} + B_{kl} \nonumber \\
\intertext{By similar analysis I can write the following below:}
\frac{16 V^2}{k l \pi} &= A_{kl}\exp(-\theta a) + B_{kl}\exp(\theta a) \nonumber
\end{align}

\begin{align}
\intertext{Note above the change in variables $\sqrt{\alpha^2+\beta^2} = \sqrt{(\frac{n\pi x}{a})^2 + (\frac{m\pi y}{a})^2} = \theta$. Combining these two equations into a matrix equation and inverting:}
\left( \begin{array}{c} P \\ P \end{array} \right) = 
\begin{pmatrix} 1 & 1 \\ \exp(-\theta a) & \exp(\theta a) \end{pmatrix}
\left( \begin{array}{c} A_{kl} \\ B_{kl} \end{array} \right) \nonumber 
\intertext{Above I have used the value P to absorb all the relevant constants: $P = \frac{16 V}{k l \pi}$}
\end{align}

By inverting this 2x2 matrix we obtain:
\[
\left( \begin{array}{c} A_{kl} \\ B_{kl} \end{array} \right) = \frac{1}{exp(\theta a)-exp(-\theta a)}
\begin{pmatrix} exp(\theta a) & -1 \\ -\exp(-\theta a) & 1 \end{pmatrix}
\left( \begin{array}{c} P \\ P \end{array} \right)
\]

Now it is clear that $A_{kl} = P\frac{(\exp(\theta a) - 1)}{\exp(\theta a)-\exp(-\theta a)}$ and that $B_{kl} = P\frac{(1-\exp(-\theta a))}{\exp(\theta a)-\exp(-\theta a)}$. Exponential identities can be used to reduce these expressions: $A_{kl} = P\frac{\exp(\theta a)}{1+\exp(\theta a)}$ and $B_{kl} = P\frac{1}{1+\exp(\theta a)}$.

Thus, the potential can be written as : \\ \par \problemAnswer{\[ \Phi(x,y,z) = \frac{16 V}{k l \pi} \sum\limits_{odd n} \sum\limits_{odd m} \frac{1}{1+\exp(\theta a)} \sin(\frac{n \pi x}{a}) \sin(\frac{m \pi y}{a}) (\exp(\theta a)exp(-\theta z) + \exp(\theta z)) \]}.

\end{homeworkSection}

\begin{homeworkSection}{b}
\textbf{Evaluate the potential at the center of the cube numerically, accurate to three 
significant figures. How many terms in the series is it necessary to keep in 
order to attain this accuracy? Compare your numerical result with the average 
value of the potential on the walls. See Problem 2.28.}

\end{homeworkSection}

\begin{homeworkSection}{c}
\textbf{Find the surface-charge density on the surface z = a.}
\end{homeworkSection}

\end{homeworkProblem}