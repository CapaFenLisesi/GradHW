\begin{homeworkProblem}[Jackson 3rd ed. : 3.4a]
\textbf{The surface of a hollow conducting sphere of inner radius a is divided into an even 
number of equal segments by a set of planes; their common line of intersection is 
the z axis and they are distributed uniformly in the angle $\phi$. (The segments are like 
the skin on wedges of an apple, or the earth's surface between successive meridians 
of longitude.) The segments are kept at fixed potentials $\pm V$, alternately. 
\\ \\ Set up a series representation for the potential inside the sphere for the gen- 
eral case of 2n segments, and carry the calculation of the coefficients in the 
series far enough to determine exactly which coefficients are different from 
zero. For the nonvanishing terms, exhibit the coefficients as an integral over $\cos\theta$.}
\\ \par
The first thing to realize is that no longer is there azimuthal symmetry in this problem. Thus, $\Phi(r,\theta,\phi) = \sum\limits_{l=0}^\infty \sum\limits_{m=-l}^l (A_{lm} r^l + B_{lm} r^{-(l+1)})Y_{lm}(\theta,\phi)$. I have been given $V(R,\phi)$. Now, I need to use orthogonality and slick mathematical tricks to get my answer into a tractable form. $V(R,\phi)$ alternates sign based on how many ``slices'' the sphere has in it (there are 2n wedges for n slices). Thus, the potential can be expressed in the following form if the sphere is aligned such that at $\phi = 0$ the dividing line between V and -V is along the x-axis and for $\phi \in (0,\pi/n)$ the potential is positive.

\[
\Phi(\Phi,R) =
\begin{cases}
V \quad \phi \in \big(\frac{\pi 2j}{n},\frac{\pi (2j+1)}{n}\big) \\
V \quad \phi \in \big(\frac{\pi (2j+1)}{n},\frac{\pi (2j+2)}{n}\big)
\end{cases}
\text{for $j = 0,1,2,...,n-1$}
\]

Thus, in the usual way, we will use orthogonality to obtain the coefficients in terms of the boundary conditions. Note, though, that this solution will determine $\Phi$ inside of the sphere. Thus, all $B_l$ coefficients must drop out in order that $\Phi$ be normalizable (not divergent at $r=0$).

\begin{align}
	A_{ab} R^k &= \int_0^{2\pi}\int_0^\pi V(R,\phi) Y_{ab}^* \sin\theta d\theta d\phi \nonumber \\
	\intertext{These spherical harmonics, $Y_{km}$ can be expressed in terms of associated Legendre polynomials $P_{lm}$, some constants and a complex exponential by :} \nonumber
	Y_{ab} &= \sqrt{\frac{(2a+1)}{4\pi}\frac{(a-b)!}{(a+b)!}} P_{ab}(\cos\theta) \exp(i b\phi) \nonumber \\
	A_{ab} R^k &= \sqrt{\frac{(2a+1)}{4\pi}\frac{(a-b)!}{(a+b)!}} \int_0^{2\pi} V(R,\phi) \exp(-i b\phi) d\phi \int_0^\pi  P_{ab}(\cos\theta)\sin\theta d\theta d\phi
\end{align}

To be succinct, the first term will become a constant $\gamma_{ab}$. I will also deal with only solving the first integral for a while. This one deserves some treatment.

\begin{align}
  \int_0^{2\pi} V(R,\phi) \exp(-i b\phi) d\phi &= V\sum_{j=0}^{n-1} \Big( \int_{\frac{\pi}{n}(2j)}^{\frac{\pi}{n}(2j+1)} \exp(-i b\phi) d\phi - \int_{\frac{\pi}{n}(2j+1)}^{\frac{\pi}{n}(2j+2)} \exp(-i b\phi) d\phi \Big) \nonumber
\end{align}

\begin{align}
	&= \frac{iV}{b} \sum\limits_{j=0}^{n-1} \Big( \exp(-ib\frac{\pi}{n}(2j+1)) - \exp(-ib \frac{\pi}{n}(2j)) - \exp(-ib \frac{\pi}{n}(2j+2)) + \exp(-ib \frac{\pi}{n}(2j+1)) \Big) \quad \nonumber \\
	%&\text{} \nonumber \\
	&= \frac{iV}{b} \sum\limits_{j=0}^{n-1} \Big(2\exp(-iA(2j+1)) - \exp(-iA(2j)) -\exp(-iA(2j+2)) \quad\quad \text{,\;Allow $b \frac{\pi}{n} \equiv A$} \nonumber \\
	&= \frac{iV}{b} \Big(2\exp(-iA)-1-\exp(-2iA)\big) \sum\limits_{j=0}^{n-1} \exp(-2iAj) 
\end{align}

At this time, I will consider the sum over j, since I have isolated it to one term. Substituting the expression for A into the exponential yields: $\sum\limits_{j=0}^{n-1} \exp(-2\pi i j \frac{b}{n})$. Performing this sum over n yields the following expression:

\[
exp\big(-2\pi i (b-\frac{b}{n})\big) \frac{\exp(2\pi i b)-1}{\exp(2\pi i \frac{b}{n})-1} \nonumber
\]

Now, b and n are integers. b is the order of the spherical harmonic in the complex exponential term. n is the number of planes that I drive through the z axis. Thus, the fraction in the above expression will return 0 for all b. It also seems that the the same fraction is undefined for whenever $b = \gamma n$ ($\gamma$ being some integer). However, applying L'Hospital's rule to the above expression yields $n$ for the entire sum. Note that in order for $b \gamma n$ to be an integer that n must divide b. Thus, the magnitude of b is lower bounded by n. Physically, this means that for this potential, no value of b (in the spherical harmonics) can exist apart from those that are n or greater or -n and smaller.
\\ \par
Consider now the term in front of the sum in equation 2. If we identify $\exp(-iA)$ as some quantity we will call $x$ this expression can be rewritten as $2x-1-x^2$. This can easily be rewritten as $-(x-1)^2$. 
\begin{align}
x-1 &= \exp(-iA)-1  \nonumber \\
&= \exp(-iA/2)\big(\exp(iA/2)-\exp(-iA/2)\big) \nonumber \\
&= -2i\exp(-iA/2)\sin(A/2) \nonumber \\
&= -2i\exp(-i\pi\frac{b}{n})\sin(\frac{\pi b}{2n}) \nonumber \\
-(x-1)^2 &= 4\exp(-2\pi i \frac{b}{n})sin^2(\frac{\pi b}{2 n}) \nonumber
\end{align}

In order that this expression be nonzero, we require that $\sin(\frac{\pi b}{2n})=\sin(\frac{\pi \gamma}{2})$ not equal zero. This expression is zero whenever $\gamma$ is even. Thus, $\gamma$ must be odd. So, I can write b as $(2z+1)n$ where $n$ is the number of slices I put through the z axis and $z$ is some integer which can take on the values $-\text{something},-\text{something}+1,...,0,1,2,...,\text{something else},...$.

Finally, I can rewrite equation 1 as follows:

\[
A_{ab} R^k = \frac{iV}{b} \sqrt{\frac{(2a+1)}{4\pi}\frac{(a-b)!}{(a+b)!}} \int_0^\pi  P_{ab}(\cos\theta)\sin\theta d\theta d\phi
\]

Where b is limited to be odd and a multiple of n. So far, I have not done any work to constrain the values of a. In general, performing that integral to constrain a using $\theta$ (as was done with b using $\phi$) would be difficult as this integral does not have any analytic expression. However, if certain values of $Y_{ab}$ are odd, then we can immediately toss away those combinations of $a$ and $b$. So, consider the following property of the associated Legendre polynomials. $P_{ab}(x) = (-1)^b\frac{(a-b)!}{(a+b)!}P_{ab}(x)$. In order for $P_{ab}$ to be odd, then, it must be the case that $(a-b)!=(a+b)!$. This occurs when $b=0$. But, this can not be any instance of $b$ as $b$ must be odd and a multiple of n. Thus, there are so constraints on the value of $b$. Finally, writing this solution with the usual Legendre parameters $l$ and $m$:

\begin{align}
\Phi(r,\theta,\phi) &= \sum\limits_{l=0}^\infty \sum\limits_{-l \leq \text{odd m} \leq-(2j+1)n }^{ l \geq \text{odd m} \geq (2j+1)n} \Big(\frac{iV}{mR^l} \sqrt{\frac{(2l+1)}{4\pi}\frac{(l-m)!}{(l+m)!}} \int_0^\pi  P_{lm}(\cos\theta)\sin\theta d\theta \Big) r^l Y_{lm}(\theta,\phi) \nonumber \\
&\frac{-(l-n)}{2n} < j < \frac{(l-n)}{2n}\; , \; j \in \mathbb{Z} \nonumber
\intertext{I could use orthonormality conditions to reduce the amount of terms in this sum even more. I could even use the fact that $\Im{(V)} = 0$. But, it's easier to just leave the expression in this form and apply the orthonormality conditions and the ``reality condition'' once n is fixed.}
\end{align}

We now need to determine the first few cases where terms in this sum are nonzero. Consider the case when $n=0$. Then, no $m$'s can exist. Consider the case when $n=1$. For this case, $m$ can be $1$ (the smallest odd multiple of n) $3$, $5$, etc. as long as $l$ will allow it. So, the first few $(l,m)$'s that survive the sum are $(1,-1)$,$(1,1)$,$(2,-1)$,$(2,1)$,$(3,-3)$,$(3,-1)$,$(3,-1)$,$(3,3)$ , etc. Thus, it seems that the general form of a surviving term goes as $(l,m) = (l,(2i+1)n)$ as long as 
\begin{enumerate}
	\item 2i+1 is less than l+1 or greater than -l-1
	\item i is an integer such as to make (2i+1)n an odd multiple of n
\end{enumerate}
\end{homeworkProblem}