\begin{homeworkProblem}[Jackson 3rd ed. : 3.7]

\begin{homeworkSection}{a}
\textbf{Three point charges (q, -2q, q) are located in a straight line with separation a and 
with the middle charge (2q) at the origin of a grounded conducting spherical shell 
of radius b, as indicated in the sketch. }
\\ \par
\textbf{Write down the potential of the three charges in the absence of the grounded 
sphere. Find the limiting form of the potential as $a\rightarrow 0$, but the product 
$qa^2 = Q$ remains finite. Write this latter answer in spherical coordinates. }
\\ \par
To begin, the potential of a point charge can be expressed as \ignore{$V(\vec{r}) = k\frac{\rho(\vec{r'})}{|\vec{r}-vec{r'}} d\Tau'$} $V(\vec{r}) = k\frac{q}{|\vec{r}-\vec{r'}|}$. $\vec{r}$ is the point at which the potential is to be evaluated. $\vec{r'}$ is the point at which the charges exist. Thus, by inspection, the potential due to three point charges can be expressed as :

\[
V(\vec{r})=kq \Big(\frac{-2}{|\vec{r}|} + \frac{1}{|\vec{r} + a\hat{z}|} + \frac{1}{|\vec{r} - a\hat{z}|} \Big)
\]

To find the potential in the limiting case, let us first write $|\vec{r} + a\hat{z}|$ as $\sqrt{r^2+a^2-2ar\cos\theta}$. Here, $\theta$ is the polar angle referenced from the $+z$ axis. Doing so results in the following:

\begin{center}
\begin{align}
V(\vec{r})&=kq \Big(\frac{-2}{r} + \frac{1}{\sqrt{r^2+a^2-2ar\cos\theta}} + \frac{1}{\sqrt{r^2+a^2+2ar\cos\theta}} \Big) \nonumber \\
V(\vec{r})&=kq \Big(\frac{-2}{r} + \frac{1}{r\sqrt{1+(a/r)^2-2(a/r)\cos\theta}} + \frac{1}{r\sqrt{1+(a/r)^2+2(a/r)\cos\theta}} \Big) \nonumber \\
\intertext{Using the following expansion (to second order) $(x^2+2\beta x +1) \rightarrow^2 1+\beta x + (1.5\beta^2-.5)r^2$} \nonumber \\
V(\vec{r})&=kq \Big(\frac{-2}{r} + \frac{1-\cos\theta (a/r) + (1.5\cos^2\theta -.5)(a/r)^2}{r} + \frac{1+\cos\theta(a/r)+(1.5\cos^2\theta-.5)(a/r)^2}{r}\Big) \nonumber \\
\intertext{Note that all terms except for the $(1.5\cos^2\theta - .5)$ terms die out. But, there are two of them.} \nonumber
V(\vec{r})&=kq \Big(\frac{(3\cos^2\theta -1)(a/r)^2}{r} \Big) \nonumber 
\end{align}
\end{center}
Now, according to the problem statement $qa^2 \rightarrow Q$. \\ \par
\problemAnswer{\[V(\vec{r}) =kQ \frac{3\cos^2\theta-1}{r^3}\]}
\\ \par
This is already in spherical coordinates.

\end{homeworkSection}

\begin{homeworkSection}{b}
\textbf{
The presence of the grounded sphere of radius b alters the potential for $r<b$. 
The added potential can be viewed as caused by the surface-charge density 
induced on the inner surface at $r=b$ or by image charges located at $r > b$. 
Use linear superposition to satisfy the boundary conditions and find the potential everywhere inside the sphere for $r < a$ and $r > a$. Show that in the 
limit $a\rightarrow 0$, 
\[
\mathbf{\Phi(r,\theta,\phi) \rightarrow \frac{Q}{2\pi\epsilon_0 r^3}\bigg(1-\frac{r^5}{b^5}\bigg)P_2\cos\theta}
\]
}
\\ \par
To find the solution to part (b) of this problem it will be best to do as Jackson says and use ``linear superposition to satisfy the boundary conditions''. See, the charges set up a potential at radius $r = b$. If I can find this potential I will flip its sign and slap that potential on top of the potential generated by the point charges. Whatever, potential I have in $\Phi(r,\theta,\phi)$ at $r=b$ 
due to the point charges can be used along with the potential generated by the point charges, themselves to determine the potential everywhere inside and outside of the sphere. In general, because of the problem's symmetry, my solution will be of the form : $\Theta(r,\theta) = \sum\limits_{l=0}^{\infty} A_l r^l + B_l r^{-(l+1)} P_l\cos\theta$.
\\ \par
The nice thing about point charges is that when that axis is located on the z-axis it is easy to express the potential due to a point charge in spherical coordinates: $\Phi(r,\theta) = \sum\limits_{l=0}^{\infty} \frac{r_<^l}{r_>^{l+1}} P_l\cos\theta$ where $r_<$ is the smaller of the following two distances: the distance from the point charge to the origin and the distance away from the origin at which the potential is being considered. This equation is valid for $\vec{r} = z \hat{z}$ and $z > 0$ (i.e. the charge is located on the positive z-axis). For charges on the negative z-axis this must be slightly modified: $\Phi(r,\theta) = \sum\limits_{l=0}^{\infty} (-1)^l \frac{r_<^l}{r_>^{l+1}} P_l\cos\theta$.
\\ \par
Thus, the potential due to these three point charges at $\vec{r} = b \hat{z}$ is :

\[
kQ \sum\limits_{l=0}^{\infty}\Big(\frac{a^l}{b^{l+1}} + \frac{(-1)^l a^l}{b^{l+1}}\Big) P_l\cos\theta
\]

Note that in the above all of the odd terms in the sum die because of the symmetry of the placement of the two point charges.
\\ \par
Note also that in the above I have chosen to ignore the potential due to the charge at the center of the coordinate system (it just shifts the above potential by a constant). If this is the potential at $r = b$ then if I flip the sign of the above potential and add it to the potential generated by the three point charges then I will effectively ground a sphere of radius b, centered at the origin. Thus, allow the ``grounded sphere'' to be constructed from both a charged sphere (with potential distribution given by the negative of the above) and the charge distribution given in the problem.
\\ \par
Before we set up the grounded sphere, though, let us consider what the potential inside the sphere due to the fictitious sphere looks like. For $r<b$ we can assume that because our boundary conditions (those given by the fictitious sphere) exhibit azimuthal symmetry. Thus, the potential inside the fictitious sphere looks like

\begin{center}
\[
V_{fict} = -kQ \sum\limits_{l=0}^{\infty}\Big(\frac{a^l}{b^{l+1}} + \frac{(-1)^l a^l}{b^{l+1}}\Big) P_l\cos\theta
\]
\end{center}

However, this potential must be related to the general solution for Laplace's equation inside of the sphere. Namely, these are the boundary conditions on the potential inside of the sphere.

\begin{center}
\[
	V_{fict}(b,\theta) = \sum\limits_{k=0}^{\infty}\Big(A_k b^k + B_k b^{-k+1} P_k\cos\theta \Big) = -kq \sum\limits_{l=0}^{\infty}\Big(\frac{a^l+(-1)^l a^l}{b^{l+1}}\Big) P_l\cos\theta
\]
\end{center}

If this is to hold for all $r$ (even $r=0$) then it must be the case that all $B_k$s are zero. This forces $A_k$s $=-2\frac{kq a^l}{b^{2l+1}}$. I can now write the potential within the enclosed fictitious sphere as:

\begin{center}
\[
V_{fict}(r,\theta) = (-2kq) \sum\limits_{even k=0}^\infty \frac{a^k}{b^{2l+1}} r^{-(k+1)} P_k\cos\theta
\]
\end{center}

Now, I just need to superpose this with the potential due to the point charges. The result for $|\vec{r}|>a$ is:

\[
\Phi(r,\theta) = 2kq \bigg( \sum\limits_{even k=0}^\infty a^k \Big(r^{-(k+1)}-\frac{r^l}{b^{2l+1}}\Big) P_k\cos\theta \bigg)
\]

Note that as $r \rightarrow 0$ we allow the summand to go to zero for k = 0. Thus, for small a, the largest term that survives is the $k=2$ term:

\[
\Phi_2(r,\theta) = 2kq\bigg( a^2(r^{-3} - \frac{r^2}{b^5} \bigg)P_2\cos\theta
\]

This can be seen to readily reduce to the desired expression. 

Given that $k = \frac{1}{4\pi\epsilon_0}$, $Q = qa^2$: \\ \par
\problemAnswer{
\[ \Phi_2(r,\theta) = \frac{Q}{2\pi\epsilon_0 r^3}\bigg(1 - \frac{r^5}{b^5} \bigg)P_2\cos\theta \]
}
\\ \par
For $|\vec{r}<a|$ the potential inside the sphere only changes in the point charge term. The role of $r_<$ and $r_>$ switch. Thus, it is written:

\[
\Phi(r<a,\theta) = 2kq \sum\limits_{even k = 0}^{\infty} \bigg( \frac{r^k}{a^{k+1}} - \frac{a^k}{b^{2l+1}}r^{-(k+1)}\bigg) P_k\cos\theta
\]
\end{homeworkSection}

\end{homeworkProblem}