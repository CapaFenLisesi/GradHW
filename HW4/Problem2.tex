\begin{homeworkProblem}
\begin{homeworkSection}{a}
The trick to this problem is to understand that a finite (not infinity) conductivity $\sigma$ has the physical effect of introducing an imaginary component into the permittiviy $\epsilon = \epsilon_{real} + j\frac{\sigma}{\omega}$. All of the other equations remain the same. As $\epsilon$ is now a complex quantity let us write it as $\epsilon = A\exp(i\theta)$. This will have advantages over the polar form as will become apparent shortly.
\\

Light reflecting normally off of a surface can be shown to change amplitude as : $\frac{E_r}{E_0} = \frac{n_1-n_2}{n_2+n_1}$ where $n_2$ is the index of refraction of the material upon which the light impinges. If this index of refraction is larger than $n_1$ then the wave will experience a sign reversal. Now, $n = \sqrt{\frac{\mu \epsilon}{\mu_0 \epsilon_0}}$. Assuming the material has a magnetic permeability close to that of vacuum (a reasonably assumption for optical materials). $n \approx \sqrt{\epsilon/\epsilon_0}$.
\\
Substituting this expression for $n$ into the prior expression for the ratio of the electric field amplitudes allows me to write: $\frac{E_r}{E_I} = \frac{1-\sqrt{\epsilon/\epsilon_0}}{1+\sqrt{\epsilon/\epsilon_0}} = \frac{\sqrt{\epsilon_0}-\sqrt{\epsilon}}{\sqrt{\epsilon_0}+\sqrt{\epsilon}}$. Expressing $\epsilon$ as the complex quantity given earlier yields : $\frac{E_r}{E_I} = \frac{\sqrt{\epsilon_0}-\sqrt{A}\exp(i\theta/2)}{\sqrt{\epsilon_0}+\sqrt{A}\exp(i\theta/2)}$. I can manipulate this algebraically to obtain: $\frac{E_r}{E_I} = \frac{\epsilon_0 - A - 2i\sqrt{\epsilon_0 A} sin(\theta/2)}{\epsilon_0 + A + 2\sqrt{\epsilon_0 A}\cos(\theta/2)}$. The phase of the reflected wave can be obtained by taking the arctangent of the imaginary and the real portions of the previous expression. $\Phi = \arctan(-\frac{2\sqrt{\epsilon_0 A}\sin(\theta/2)}{\epsilon_0-A})$. Now, $\theta(\sigma)$ and $\A(\sigma)$ so let's consider their expansions to determine an expression for the phase that only contains to first order in $\sigma$. $A = \sqrt{(\epsilon^2_{real}+(\sigma/\omega)^2}$. Expanding $A(\sigma)$ about $\sigma = 0 $ yields $\epsilon_{real} + \frac{\sigma^2}{2\epsilon_{real}\omega^2} + O(\sigma^4)$. Since the assignment requires an expansion of the reflected phase to first order in $\sigma$ I can safely neglect the second term and keep the 0th order term $\epsilon_{real}$. Performing a similar analysis of $\theta(\sigma)$, expanding about zero, yields $\sigma/(\omega \epsilon_{real}) + O(\sigma^3)$. Thus, I am only required to keep the first order term and the reflecte phase, to first order is $\Phi = \arctan(-2\sqrt{\epsilon_0 \epsilon_{real}}\frac{\sigma}{\omega \epsilon_{real}}/(\epsilon_0-\epsilon_{real}))$.
\\
Now, to determine the intensity of the reflected wave I must find the squared magnitude of $\frac{E_r}{E_I}$. Utilizing my prowess in solving for squared magnitudes: $|\frac{E_r}{E_I}|^2 = \frac{\sqrt{(\epsilon_0-\epsilon_{real})^2+4\epsilon_0\epsilon_{real}\sin^2(\theta/2)}}{(\epsilon_0+\epsilon_{real}+2\sqrt{\epsilon_0 \epsilon_{real}}\cos\theta/2)^2}$. There may exist a simpler expression than this. But, I'm not going to try and find it. This is a solution to the problem.

\end{homeworkSection}

\begin{homeworkSection}{b}
This problem requires me using knowledge of the skin depth of a material. The next few lines are pretty much copy-pasted from 5.18 (a) of Jackson's Electrodynamics where he discusses this exact phenomenon.
\\

For a material with finite conductivity, imagine an H field (oriented in the $\hat{x}$ direction) is given by $H_x(z,t) = h(z)\exp(-i\omega T)$. By manipulating Maxwell's equations ($\nabla x H = J$ and $J = \sigma E$ - knowing $B = \nabla x A$) we can show that $\nabla^2A = \mu\sigma \frac{\partial{A}}{\partial{t}}$. In this particular situation $\big( \frac{d^2}{dz^2} + i\mu\sigma\omega \big) h(z) = 0$. A solution to this equation is given by $H_x(z,t) = A\exp(-z/\delta)\exp(i(z/\delta - \omega t)) + B\exp(z/\delta)\exp(-i(z/\delta + \omega t))$. Here, $\delta = \sqrt{\frac{2}{\mu\sigma\omega}}$: the ``skin depth'' of the material. In order to conserve energy it must be the case that $B = 0$. For a free wave $H = \hat{n} x E \frac{\epsilon_0}{\mu_0}$. Thus, we should expect the same qualitative behavior for $E(z,t)$. In this case, $E$ is oriented in the $-\hat{y}$ direction (in order to make the cross product return the right direction for $H$).

Thus, if we assume the amplitude of the wave upon encountering the surface is $E_0$ then after a distance d into the surface the amplitude is $E_0 \exp(-d/\delta)\exp(i(d/\delta -\omega t))$. Thus, the intensity of the wave is $E_0^2 \exp(-2d/\delta)$ (the magnitude squared of the amplitude).

\end{homeworkSection}
\end{homeworkProblem}
