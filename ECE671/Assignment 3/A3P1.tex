\section*{Problem 1: Low Pass Filter Design}

The first step in the design process is to select the appropriate
``g-values'' based on the tables relevant to the filter at hand. The filter
chosen for the design is a Butterworth/Maximally flat filter design. The
cut-off (\SI{3}{\deci\bel}) frequency is specified as $ \omega_c =
\SI{8}{\giga\hertz}$ and the design reference impedance is selected as
\SI{50}{\ohm} . The required required attenuation at $ \omega =
\SI{13.6}{\giga\hertz}$ is \SI{16}{\deci\bel} ( $ \omega/\omega_c = 1.7 $ ).
Based on tables that determine the attenuation of different order filters
over frequency, I will select a filter of order 5. It's worth noting that a
fourth order filter should suffice in the case of an ideal lumped element
design. However, in case there are problems with the implementation or in
case microstrip lines do not provide as much attenuation with varying
topologies I will opt to design a fifth order filter. The g-values
associated with this filter design are as follows:

\begin{table}[h]
    \centering
    \caption{g values}
    \label{tab:1a_g_value_table}
    \begin{tabular}{|c|c|c|c|c|c|}
        \hline $g_{1}$ & $g_{2}$  & $ g_{3} $ & $ g_{4} $ & $ g_{5} $ & $
        g_{6} $ \\ 
        \hline .618    & 1.618    & 2.0       &  1.618    & .618 & 1.0 \\
        \hline
    \end{tabular}
\end{table}

I will choose low-pass prototype beginning with a shunt element. Based on
the order of the filter and the design topology $g_{1}, g_{3}, g_{5}$ are
the low-pass prototype values for capacitors and $g_{2}, g_{4}$ are those
for inductors. $g_{6}$ is the low-pass prototype of the 50 ohm load
(termination). The following transformation relationships will aid in taking
the g-values to their appropriate component values.

\begin{equation}
g_{N+1} Z_c \rightarrow Z_{load} \quad \quad \quad \quad
    \frac{Z_c g_n}{\omega_c} \rightarrow L \quad \quad \quad \quad
\frac{g_n}{Z_C \omega_c } \rightarrow C
    \end{equation}

    Plugging in the appropriate values I obtain the following component values:

    \begin{table}[h]
        \centering
        \caption{Component Values}
        \label{tab:1a_comp_values}
        \begin{tabular}{|c|c|c|c|c|c|}
            \hline $C_1$ & $L_2$  & $C_3$ & $L_4$ & $C_5$ & $
            Z_{load}$ \\ 
            \hline \SI{.25}{\pico\farad}  & \SI{1.61}{\nano\henry} &
            \SI{.79}{\pico\farad}  & \SI{1.61}{\pico\farad}  &
            \SI{.25}{\pico\farad} &
            \SI{50}{\ohm}  \\
            \hline
        \end{tabular}
    \end{table}

    To verify the design a lumped-element model was simulated initially. The
    attenuation was found to be $\approx \SI{-23}{\deci\bel}$ of attenuation at
    \SI{13.6}{\giga\hertz} which far exceeds the \SI{16}{\deci\bel} requirement.
    Note that the component names $C_1, L_2, C_3$ differ from those on the
    schematic. The names are immaterial. The component values are what matter.

    \subsection*{Problem 1a: Shunt Stub Design}
    To realize a shunt stub design I will first convert the capacitors into
    shunt stubs, directly, and use Kuroda's identity to use unit
    elements ($\lambda/8$ lengths of transmission line) to convert the
    series inductors into shunt stubs. Note that by virtue of microstrip,
    the stubs are required to be open. Note, also, that the transmission
    line impedance will need to be different from \SI{50}{\ohm}, in general,
    in order to accommodate Kuroda's identity.

    In order to use Kuroda's identity I need to know the impedance of the
    various reactive elements at the design frequency. It can be shown that
    the reactance of the inductors and capacitors can be calculated as:

    \begin{align*}
        X_C^{\omega_c} = \frac{Z_c}{g_n} \quad\quad\quad\quad X_L^{\omega_c} = Z_c g_n	
    \end{align*}

    , where the reactance is calculated at the design frequency. This is the
    frequency for which we will select our transmission line stub
    impedances. Thus, constructing a table as before:

    \begin{table}[h]
        \centering
        \caption{Component Values}
        \label{tab:1a_comp_ractances}
        \begin{tabular}{|c|c|c|c|c|}
            \hline $X_{C_1}$ & $X_{L_2}$  & $X_{C_3}$ & $X_{L_4}$ & $X_{C_5}$ \\
            \hline \SI{80.9}{\ohm}  & \SI{80.9}{\ohm} & \SI{25}{\ohm}  &
            \SI{80.9}{\ohm}  & \SI{80.9}{\ohm} \\
            \hline
        \end{tabular}
    \end{table}

    In this case, the symmetry of the component values makes the design
    particularly easy. Kuroda's identity needs to be accomplished in two steps.
    First, two unit elements need to each be added to the network immediately
    following the generator and preceding the load. Then, those two unit
    elements that are ``facing the filter network'' need to each interact with
    first and last capacitor to swap the order of the reactive element and the
    transmission line. Those two capacitors, then, will become inductors. Then,
    the unit elements that have been moved to the interior of the network need
    to convert the inside inductors into capacitors. Finally, the remaining unit
    elements at the beginning and the end of the network need to turn the newly-
    fashioned inductors into capacitors. Then, the design will be completed with
    the following topology: capacitor, unit element, capacitor, unit element,
    capacitor, unit element, capacitor, unit element, capacitor. 
   
    Doing all of this yields the following design:

    \begin{table}[h]
        \centering
        \caption{Open Stub Implementation}
        \label{tab:1a_open_stub_ideal}
        \begin{tabular}{|c|c|c|c|c|c|c|c|c|}
            \hline $ \text{Stub}_1 $ & $ \text{Series TL}_2 $  & $ \text{Stub}_3
            $ & $ \text{Series TL}_4 $ & $ \text{Stub}_5 $ & 
            $ \text{Series TL}_6 $ & $ \text{Stub}_7 $  & $ \text{Series TL}_8 $
            & $ \text{Stub}_9 $  \\
            \hline \SI{181}{\ohm} & \SI{69.7}{\ohm} & \SI{42.7}{\ohm} &
            \SI{111.8}{\ohm} & \SI{25}{\ohm} & \SI{111.8}{\ohm} &
            \SI{42.7}{\ohm} & \SI{69.4}{\ohm} & \SI{181}{\ohm} \\
            \hline
        \end{tabular}
    \end{table}

    The next thing to do is to implement this in microstrip. Based on the design
    parameters (material and dimensions) the widths of the conductors to achieve
    these impedances are constrained. Instead of hand-calculating the impedances
    I will use line calculator tool ``LineCalc'' (bundled with ADS) to design my
    widths. In the table below I enumerate the chosen widths and lengths for the necessary
    impedance and electrical length:

    \begin{table}[h]
        \centering
        \caption{Lengths and Widths of Microstrip Sections}
        \label{tab:1a_microstrip_dimensions}
        \begin{tabular}{|c|c|c|}
            \hline Impedance ($\Omega$) & Width (mm) & Length (mm)  \\
            \hline \SI{181}{\ohm}       & .080       & 3.67         \\
            \hline \SI{69.7}{\ohm}      & 1.34       & 3.47         \\
            \hline \SI{42.7}{\ohm}      & 2.92       & 3.38         \\
            \hline \SI{111.8}{\ohm}     & .482       & 3.57         \\
            \hline \SI{25}{\ohm}        & 5.97       & 3.30         \\
            \hline
        \end{tabular}
    \end{table}

    Note that the electrical length does no change significantly with the
    characteristic impedance. This is to be expected. Note, also, that the
    larger impedance lines require much smaller widths than likely can be
    manufactured ($\sim \SI{80}{\micro\meter} $). This would be undesirable from
    a manufacturing standpoint. However, this design does not need to be
    feasible. So, it is given.

    \subsection*{Problem 1b: Stepped Impedance Design}

    The stepped impedance design is not as elegant of a design. To design a
    stepped impedance filter I begin using the component values that I had
    designed my lumped element filter for. Then, I take advantage of the fact
    that short high impedance transmission line sections look like inductors and
    short section low impedance transmission line sections look like inductors.
    ``Short'' here means a small fraction of a wavelength (typically, $ < 10\%$
    of the wavelength). If the amount of time it takes the electromagnetic wave
    to propagate through the transmission line is denoted as T, the capacitance
    of a short structure looks like:

    \[ 
        C = \frac{T}{Z_c} 
    \]
    
    where $Z_c$ is the characteristic impedance of the line. For inductors,

    \[ 
        L =Z_c T
    \]
    
    These can be understood by realizing that the speed of propagation down the
    line is:

    \[ 
        v_p = \frac{1}{\sqrt{lc}} = \frac{Z_c}{l}
    \]
    
    Where $l$ and $c$ are the inductance and capacitance of the transmission
    line per unit length, respectively.  The propagation delay time T is related
    to $v_p$ by the distance travelled down the line $d$ as follows:

    \[ 
        T = \frac{d}{v_p} = \frac{l d}{Z_c} = \frac{L}{Z_c}
    \]
    
    The last inequality holds since $l d$ is just $L$, the lumped element
    equivalent inductance. Thus, to a reasonable approximation, the delay is
    expressed in terms of the characteristic impedance as given above. I have
    the lumped-element inductances and capacitances of this filter, already. I
    just have to select an appropriate delay. No impedance above \SI{130}{\ohm}
    or below \SI{10}{\ohm} can be used. Thus, I can set up the following
    inequalities for the delay in terms of the lumped elements:

    \[
        \SI{10}{\ohm} \le Z_c = \frac{T}{C} \le \SI{130}{\ohm} 
    \]
   
    \[ 
        \SI{10}{\ohm} \le Z_c = \frac{L}{T} \le \SI{130}{\ohm}  
    \]
    
    Thus, for a fixed delay, the largest capacitance will lower bound
    characteristic impedance and the largest inductance will upper bound the
    impedance, for examples. However, knowing how long the lines can be
    (assuming $d = \lambda/10$) and the speed of the wave through the medium
    (determined by $v_p = c / \sqrt{\epsilon_{eff}}$) allows me to determine T ($
    \epsilon_{eff} = \frac{\epsilon_r+1}{2} = \frac{2.2+1}{2} = 1.6$, in
    microstrip).

    \[ 
        T = \frac{\lambda/10}{v_p} = \frac{\lambda/10}{\lambda f} = 
        \frac{1}{10 f}
    \]
    
    For, this problem, then, $T = \SI{12.5}{\pico\second} $, since $ f =
    \SI{8}{\giga\hertz} $. Plugging this into the above expression for $C$ and
    $L$ we find the bound on $C$ and $L$ to be:

    \[ 
        \SI{96.2}{\femto\farad} \le C \le \SI{1.25}{\pico\farad} 
    \]
    
    \[ 
        \SI{125}{\pico\henry} \le L \le \SI{1.63}{\nano\henry}  
    \]
    
    Note that we meet the design requirement by the skin of our teeth. Our
    capacitances ($ C_{max} = \SI{.79}{\pico\farad} $, $ C_{min} =
    \SI{.25}{\pico\farad}  $) are well within the specification. However, the
    largest inductance is $\SI{1.61}{\nano\henry} $ which corresponds to
    \SI{129}{\ohm} and barely falls in the specification. To give the
    manufacturing some tolerance I could specify the delay to be larger. This
    would allow for larger inductances. However, this design works. Since all of
    the lengths are $\lambda/10 = \frac{c}{10 f \sqrt{1.6} } \approx
    \SI{2.96}{mm} $ all that is needed to be done is to calculated the widths of
    each section. This will be done, again, using LineCalc once the impedances
    of the lines have been determined. The estimate of the line length is
    supported by using LineCalc which gives \SI{2.73}{\milli\meter} for $Z_c =
    \SI{50}{\ohm} $. And, actually, since I'm using LineCalc to calculate the
    widths of the lines and since it seems to be determining the effective
    length of line (accounting for fringing effects, possibly) I will use
    LineCalc's values for the lengths as well. The table below enumerates the
    impedances of the lines, calculated using the above formulae:

    \begin{table}[h]
        \centering
        \caption{Line Impedance Corresponding to Lumped Elements}
        \label{tab:1b_line_impedance_of_lumped_elements}
        \begin{tabular}{|c|c|}
            \hline Component Value & Corresponding Impedance $ \Omega $ \\
            \hline $C_1 = C_5 = \SI{.25}{\pico\farad} $ &
            $\frac{\SI{12.5}{\pico\second}}{\SI{.25}{\pico\farad}} =
            \SI{50}{\ohm}$ \\
            \hline $L_1 = L_4 = \SI{1.61}{\nano\henry}$ &
            $\frac{\SI{1.61}{\nano\henry}}{\SI{12.5}{\pico\second}} \approx
            \SI{129}{\ohm}$  \\
            \hline $ C_3 =  \SI{.79}{\pico\farad} $ &
            $\frac{\SI{12.5}{\pico\second}}{\SI{.79}{\pico\farad}} \approx
            \SI{15.8}{\ohm} $ \\
            \hline
        \end{tabular}
    \end{table}

    \begin{table}[h]
        \centering
        \caption{Dimensions of Stepped Impedances}
        \label{tab:1b_stepped_impedance_dimensions}
        \begin{tabular}{|c|c|c|}
            \hline Lumped Element & Width (mm)  & Length (mm) \\
            \hline $ C_1 = C_5 $ & \SI{2.31}{mm} & \SI{2.73}{mm} \\
            \hline $ L_2 = L_4 $ & \SI{.318}{mm} & \SI{2.87}{mm}  \\
            \hline $ C_3 $ & \SI{10.4}{mm} & \SI{2.6}{mm}  \\
            \hline
        \end{tabular}
    \end{table}

    Amazingly, this filter response has $\approx \SI{6}{\deci\bel} $ of
    attenuation at $ \SI{8}{\giga\hertz} $. This is not in accordance with the
    design. I don't know why this is happening. If I am to postulate a guess,
    however, it would be that the lengths are not small enough to be considered
    ideal, thus, the capacitance of the segments is contributing a small amount
    to the inductive transmission line segments and vice versa. The way in which
    to test this would be to design at a  lower frequency such that the
    length of the lines can be constructed as smaller fractions of the
    wavelength. If the $\SI{3}{\deci\bel}$ frequency approaches the designed
    cut-off frequency then this must be the problem.

    \subsection*{Problem 1c: 1a and 1b with Steps and T-Junctions}

    The last thing to do is to insert the transmission line tees into the open
    stub filter and insert the steps into the stepped impedance transmission
    line.
