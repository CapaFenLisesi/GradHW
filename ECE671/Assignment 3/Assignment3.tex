\documentclass{article}
\usepackage[]{amsmath}
\usepackage[]{fullpage}
\usepackage[]{siunitx}
\begin{document}
    \section*{Problem 1: Low Pass Filter Design}

    The first step in the design process is to select the appropriate
    ``g-values'' based on the tables relevant to the filter at hand. The filter
    chosen for the design is a Butterworth/Maximally flat filter design. The
    cut-off (\SI{3}{\deci\bel}) frequency is specified as $ \omega_c =
    \SI{8}{\giga\hertz}$ and the design reference impedance is selected as
    \SI{50}{\ohm} . The required required attenuation at $ \omega =
    \SI{13.6}{\giga\hertz}$ is \SI{16}{\deci\bel} ( $ \omega/\omega_c = 1.7 $ ).
    Based on tables that determine the attenuation of different order filters
    over frequency, I will select a filter of order 5. It's worth noting that a
    fourth order filter should suffice in the case of an ideal lumped element
    design. However, in case there are problems with the implementation or in
    case microstrip lines do not provide as much attenuation with varying
    topologies I will opt to design a fifth order filter. The g-values
    associated with this filter design are as follows:

    \begin{table}[h]
        \centering
        \caption{g values}
        \label{tab:1a_g_value_table}
        \begin{tabular}{|c|c|c|c|c|c|}
            \hline $g_{1}$ & $g_{2}$  & $ g_{3} $ & $ g_{4} $ & $ g_{5} $ & $
            g_{6} $ \\ 
            \hline .618    & 1.618    & 2.0       &  1.618    & .618 & 1.0 \\
            \hline
        \end{tabular}
    \end{table}

    I will choose low-pass prototype beginning with a shunt element. Based on
    the order of the filter and the design topology $g_{1}, g_{3}, g_{5}$ are
    the low-pass prototype values for capacitors and $g_{2}, g_{4}$ are those
    for inductors. $g_{6}$ is the low-pass prototype of the 50 ohm load
    (termination). The following transformation relationships will aid in taking
    the g-values to their appropriate component values.

    \begin{equation}
    g_{N+1} Z_c \rightarrow Z_{load} \quad \quad \quad \quad
    \frac{Z_c g_n}{\omega_c} \rightarrow L \quad \quad \quad \quad
    \frac{g_n}{Z_C \omega_c } \rightarrow C
    \end{equation}
    
    Plugging in the appropriate values I obtain the following component values:
    
    \begin{table}[h]
        \centering
        \caption{Component Values}
        \label{tab:1a_comp_values}
        \begin{tabular}{|c|c|c|c|c|c|}
            \hline $C_1$ & $L_2$  & $C_3$ & $L_4$ & $C_5$ & $
            Z_{load}$ \\ 
            \hline \SI{.25}{\pico\farad}  & \SI{1.61}{\nano\henry} &
            \SI{.79}{\pico\farad}  & \SI{1.61}{\pico\farad}  &
            \SI{.25}{\pico\farad} &
            \SI{50}{\ohm}  \\
            \hline
        \end{tabular}
    \end{table}

    To verify the design a lumped-element model was simulated initially. The
    attenuation was found to be $\approx \SI{-23}{\deci\bel}$ of attenuation at
    \SI{13.6}{\giga\hertz} which far exceeds the \SI{16}{\deci\bel} requirement.
    Note that the component names $C_1, L_2, C_3$ differ from those on the
    schematic. The names are immaterial. The component values are what matter.

    \subsection*{Problem 1a: Shunt Stub Design}
        To realize a shunt stub design I will first convert the capacitors into
        shunt stubs, directly, and use Richards Transformation to use unit
        elements ($\lambda/8$ lengths of transmission line) to convert the
        series inductors into shunt stubs. Note that by virtue of microstrip,
        the stubs are required to be open.

    \subsection*{Problem 1b: Stepped Impedance Design}
    \subsection*{Problem 1c: 1a and 1b with Steps and T-Junctions}
\end{document}
