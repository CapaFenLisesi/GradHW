\section*{Theory}
\addcontentsline{toc}{section}{Theory}

The theory behind the SRFT is simple and relies on 4 basic ideas:

\begin{enumerate}
    \item Optimization of Transducer Gain
    \item Kramers-Kronig Relations
    \item The Gewertz Method
    \item Darlington Synthesis (Pole Extraction at Infinity)
\end{enumerate}

\subsection*{Optimization of Transducer Gain}
\addcontentsline{toc}{subsection}{Optimization of Transducer Gain}

The concept of transducer gain optimization is simple: Consider the source and
the load and allow the introduction of some network that maximizes the power
transfer from the source to the lead. This power transfer will be maximum when
the source and the load are conjugately matched. Thus, the problem can be stated
as follows:

\[
\min_{\text{networks}} \left| \left| T_{\text{goal}}(\omega) -
T_{\text{network}}(\omega) \right| \right|
\]

The idea, stated above, is to declare some goal transducer gain function (that may be
physically impossible based on gain-bandwidth considerations). Then, considering
all possible networks, we minimize over the norm of the difference between the
two transducer gain functions. However, we need some way in which to perform
this minimization. So, we next express the transducer gain of the network in
terms of the impedance of the network and the load:

\[ 
T(\omega) = \frac{4 r R}{(x + X)^2 + (r + R)^2} 
\]

Where, above, lower case $r$ and $x$ indicate load quantities and $R$ and $X$
represent network quantities. It should be clear from this description that the
frequency dependence is captured by all of the four circuit quantities: $r, R,
x, X$. It is precisely this frequency dependence that determines the transducer
gain. It should also be noted that the gain appears to be maximized when the
reactance of the load is cancelled by the reactance of the introduced network.
This is a restatement of the conjugate matching principle stated earlier. Now,
for the purposes of this report, the load will always be considered fixed. The
network is introduce in the context of the load and the transducer gain is
maximized.

They way in which SRFT attempts to determine the optimum resistance
characteristic is determined by the relationship between the resistance and
reactance characteristic and will be discussed subsequently.

\subsection*{Kramers-Kronig Relations}
\addcontentsline{toc}{subsection}{Kramers-Kronig Relations}

A really interesting side-effect of having access to a linear, causal system is
that the real and imaginary parts of the transfer function of such a system are
not independent. It can be shown that the relationship between the real and
imaginary parts of a circuit network $R$ and $X$ are as follows (see \cite{wcd}):


\begin{align*}
    R(\omega) &= - \frac{2}{\pi} \int_{0}^{\infty} \frac{\Omega~X(\Omega)}{\Omega^2 -
\omega^2} d\Omega \\
    X(\omega) &= \frac{2\omega}{\pi} \int_{0}^{\infty}
\frac{R(\Omega)}{\Omega^2-\omega^2} d \Omega
\end{align*}

These are known as the Kramers-Kronig relations. It is at this point that the
numerical optimization procedure is introduced.  Imagine that the resistance as
a function of frequency is broken up into many piecewise segments. Then, for
each break frequency $\omega_k$ there exists some resistance $R_k$ at that break
frequency. $R(\omega)$ can now be expressed as:

\[ 
R(\omega) = R_0 + \sum^{n}_{k=1} D_k a_k(\omega) 
\]

where 

\begin{align*}
    a_k(\omega) =&
    \begin{cases}
        1 & \omega \ge \omega_k \\
        \frac{\omega - \omega_{k-1}}{\omega_k - \omega_{k-1}} &\omega_{k-1} \le
        \omega \le \omega_k \\
        0 &\omega < \omega_{k-1}
    \end{cases}
\end{align*}

It is clear from this, then, that $D_k$ represents the slope of the kth break
frequency and, as such:

\[ 
D_k = R_k - R_{k-1} 
\]

If this is the form of $R(\omega)$ that is assumed for the SRFT then it can be
shown that the reactance must be related to $R(\omega)$ by:

\[
X(\omega) = \sum^{n}_{k=1} D_k b_k(\omega)
\]

where
\begin{align*}
    b_k(\omega) = (\pi \left( \omega - \omega_{k-1} \right))^{-1} \big(
&\left( \omega + \omega_k \right) \ln \left| \omega + \omega_k \right| +
\\
&\left( \omega - \omega_k \right) \ln \left| \omega - \omega_k \right| - \\
&\left(\omega + \omega_{k-1} \right) \ln \left| \omega + \omega_{k-1} \right| - 
\\
&\left( \omega - \omega_{k-1} \right) \ln \left| \omega - \omega_{k-1} \right| 
\big)
\end{align*}

This expression for $X(\omega)$ is nothing more than the previous expressions
relating $R(\omega)$ to $X(\omega)$ assuming that $R(\omega)$ has a piecewise
linear form. This piecewise linearity makes it easy to calculate the reactance
on a computer. Since the goal is to reduce the difference between the goal gain
function and the network gain function, the algorithm will perform the
following: It will adjust the $D_k$s using some scheme (gradient ascent, random,
etc.) and will attempt to produce a set of $D_k$s that makes the difference
between $T_{network}$ and $T_{goal}$ a minimum. Those $D_k$s will correspond to
some $R(\omega)$ and some $X(\omega)$. Thus, it would seem that the $D_k$
correspond to some network. It is this point that we have to introduce the way
in which the network is deduced from $R(\omega)$ and $X(\omega)$.

\subsection*{The Gewertz Method}
\addcontentsline{toc}{subsection}{The Gewertz Method}

In order to determine a network from the obtained piecewise linear $R(\omega)$,
I can perform the following. It can be shown that the impedance of the network
and the resistance of the network are related:

\[ 
R(-s^2) = \frac{1}{2} \left( Z(s) + Z(-s) \right) 
\]

where $s = j\omega$. The reason for expression $R$ as a function of $s^2$ is
that the resistance function must be even about $\omega = 0$, by physical
considerations. So, now, consider that the piecewise linear $R(\omega)$ is
approximated by some rational function fit $\tilde{R}(-s^2) =
\frac{p(-s^2)}{q(-s^2)}$.  Now, consider $Z(s) = \frac{n(s)}{d(s)}$, where $n$
and $d$ are the numerator and denominator polynomials of $Z(s)$, respectively.
It can be shown, then, that by relating the coefficients of $p(-s^2)$ with those
of $d(s)$ and $n(s)$ that the following holds (note that $d(s)d(-s) = q(-s^2)$):

\[ 
    \begin{pmatrix} p_0 \\ p_2 \\ \vdots \\ p_{2m} \end{pmatrix} = 
        \begin{pmatrix}
            d_0 & 0    & \ldots & 0 & 0      \\
            d_2 & -d_1 & \ldots & 0 & 0      \\
            \vdots & \vdots & \vdots & \vdots & \vdots \\
            0   &   0  & \ldots & -d_{m-1} & d_{m-2} \\
            0   &   0  & \ldots & 0 & d_{m} 
        \end{pmatrix}
        \begin{pmatrix}
            n_0 \\ n_1 \\ \vdots \\ n_{m-1} \\ n_{m}
        \end{pmatrix}
\]

Above, it is assumed that $p(s) = \sum^{2m-2}_{i=1} p_i s^i$ and similarly with
the other polynomials.

Now, our goal is to solve for $Z(s) = \frac{n(s)}{d(s)}$. We can obtain the
coefficients of $d(s)$ from those of $q(s)$ (we have $q(-s^2)$ from our fit of
$R(-s^2)$) by using spectral factorization. According to \cite{wcd}, any
real non-negative function $A(-s^2)$ can be written as a product, $a_0^2
a(s)a(-s)$, where $a_0$ is real and $a(s)$ is a real monic polynomial. Thus, we
know the coefficients $d_0, d_1, d_2,\ldots, d_{2m}$ which form $d(s)$ since we
know $q(-s^2)$ from having had fit $R(s)$ to a rational function.

We can then invert the above equation to find the numerator polynomial
coefficients. Once we have $n(s)$ from the aforementioned procedure, we form
$Z(s) = \frac{n(s)}{d(s)}$. This is the Gewertz method.

What we need now is a way in which to obtain the network (resistors, inductors,
capacitors) from $Z(s)$. This is what Darlington synthesis enables us to do.

\subsection*{Darlington Synthesis}
\addcontentsline{toc}{subsection}{Darlington Sythesis}

It is shown in reference \cite{wcd} that any realizable impedance function can
be realized as a lossless LC ladder network terminated on a resistor. The proof
will not be given here but a demonstration of the technique will be described.
Assuming that $Z(s)$ is expressed as a rational function (which can always be
done to at least some approximation), then the network can be constructed as
follows:

\[ 
Z(s) = \frac{n_0 + n_1 s + n_2 s^2 + \ldots + n_m s^m}{d_0 + d_1 s + d_2 s^2 +
\ldots + d_m s^m} 
\]

Then, by iterative division, $Z(s)$ can be expressed as a continued fraction:

\[ 
Z(s) = \frac{1}{\frac{d_0 + d_1 s + d_2 s^2 +
\ldots + d_m s^m}{n_0 + n_1 s + n_2 s^2 + \ldots + n_m s^m}} 
\]

Then, using long division, this can be expressed in the form:

\[
Z(s) = \frac{1}{c_0(s) + \frac{1}{c_1(s) + \frac{1}{c_2(s) + \ldots}}}
\]

Now, this is exactly the form of an LC ladder network input impedance. By
construction, $c_0(s)$ is at most a first order polynomial in $s$. This holds
true for other $c_i(s)$ also. Consider the case, for example, when $c_0(s) =
.5s$. This is exactly a capactive admittance of value
\SI[parse-numbers=false]{.5s}{\siemens} which corresponds to a capacitance of
\SI{.5}{\farad}. Thus, the first element in the network would be a shunt
capacitance (since it's an admittance) of \SI{.5}{\farad}. So, apparently,
expressing the impedance in this form by repeating iterative long division is a
way in which the components of the network are made easily visible.

Once the components are identified, all that remains is to simulate the network
determined by the algorithm to ensure that it, indeed, satisfies the properties
desired by the designer; that is, that it approximates well the optimal
gain-bandwidth characteristic.

Now that the theory has been established, a practical example of a filter will
be undertaken and the network will by synthesized and simulated. The
shortcomings of the current implementation will be discussed subsequently.
