\section{Scattering Parameters of an Ideal Transformer}
\setcounter{equation}{0}
\addtocounter{section}{1}

The thing to understand about a transformer with an 1:n turns ratio is that the
relationship between port 2's voltage and port 1's voltage is:

\[ 
    \frac{V_2}{V_1} = n 
\]

Similarly, the currents through both ports are related in the following way:
\[ 
    \frac{I_1}{I_2} = n 
\]

Now, in order to find the scattering parameters I will need to use these
relationships in addition to the following idea: When I drive port 1, there will
be no incident voltage on port 2, because it's matched. Likewise, when I drive
port 2, there will be no incident voltage wave on port 1, because it's matched.
To start, we will find $S_{11}$ and $S_{21}$. When we do this we will drive port
1. This implies that $V_2^+ = 0$ so that we can write the following for the
transformer relationships:

\begin{align}
    n(V_1^+ + V_1^-) = V_2^- \text{\quad Voltage Relationship} \label{port1voltage} \\
    V_1^- - V_1^+ = n V_2^- \text{\quad Current Relationship}
    \label{port1current}
\end{align}


\[ 
        S_{11} = \frac{V_1^-}{V_1^+} = \frac{1+n^2}{1-n^2}
\]

\[ 
        S_{21} = \frac{V_2^-}{V_1^+} = \frac{n \left( V_1^+ + V_1^-
        \right)}{V_1^+} = n \left( 1+S_{11} \right) = \frac{2n}{1-n^2}
\]

To find $S_{22}$ and $S_{12}$ I will drive port 2 which implies that $V_1^+ =
0$. This gives me the following transformer relationships:

\begin{align}
    nV_1^- = V_2^- + V_2^+ \text{\quad Voltage Relationship} \label{port1voltage} \\
    V_1^- = n \left( V_2^- -  V_2^+ \right) \text{\quad Current Relationship}
    \label{port1current}
\end{align}

\[ 
        S_{22} = \frac{V_2^-}{V_2^+} = \frac{n^2+1}{n^2-1}
\]

\[ 
        S_{12} = \frac{V_1^-}{V_2^+} = \frac{n \left( V_2^- - V_2^+
        \right)}{V_2^+} = n \left(  S_{22} - 1 \right) = \frac{2n}{n^2-1}
\]

Thus, finally,

\[ 
        S_{\text{Transformer}} = \begin{pmatrix}
            \frac{1+n^2}{1-n^2} & \frac{2n}{1-n^2} \\ \frac{2n}{n^2-1} &
            \frac{n^2+1}{n^2-1}
        \end{pmatrix} 
\]


