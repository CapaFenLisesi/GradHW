\documentclass{article}
\usepackage[]{amsmath}
\usepackage[]{braket}
\usepackage{commath}
\begin{document}
Last time we had addressed some important states of light: Namely, the Fock
(number) states. So, we had introduced the operator
\[
   \hat{n} \Ket{n} = n \Ket{n}
\]

We had showed that this operator does not commute with the phase operator $
\hat{\phi} $.

\[
   [ \hat{n} , \hat{\phi} ] = i
\]

We had showed that for all Fock states $ \Braket{n | E | n } = 0 $ and $
\Braket{n | E^2 | n } \ne 0 $. These relationships express a lack of
correspondence with classical physics.

\section{Coherent States}
\label{sec:coherent_states}

Another important class of states are the coherent states. These were introduced
by Glauber in the 1960s. These states are characterized by being eigenstates of
the lowering/annihilation operator:

\[
   \hat{a} \Ket{\alpha} = \alpha \Ket{\alpha}
\]

$ \alpha $ is not necessarily real and $ \hat{a} $ is not hermitian.
Additionally, two different coherent states are not necessarily orthogonal:

\[
   \Ket{\alpha} \hat{a}^\dagger = \Ket{\alpha} \alpha^*
\]

Thus, we can consider the matrix elements of $ \hat{E} $ in terms of these
coherent states:

\[
   \Braket{\alpha | E | \alpha} = i \sqrt{\frac{\hbar \omega}{2 \epsilon_0 V}}
   \Braket{\alpha | a e^{-i \omega t + i k \cdot r} - a^\dagger e^{i \omega t -i
   k \cdot r} | \alpha  } = 2 \sqrt{\frac{\hbar \omega}{2 \epsilon_0 V}} |\alpha|
   \sin(\omega t - k \cdot r - \theta)
\]

We can calculate the variance of E:

\[
   \Braket{\alpha | E^2 | \alpha} = \frac{\hbar \omega}{2 \epsilon_0 V} \left( 2
   |\alpha| \sin(\omega t - k \cdot r - \theta) \right)^2 + \frac{\hbar \omega}{2
   \epsilon_0 V}
\]

Thus, we have$ \Delta E = \sqrt{ \Braket{E^2} - \Braket{E}^2 } =
\sqrt{\frac{\hbar \omega}{2 \epsilon_0 V}}$. This allows us to understand the
classical result in terms of the results obtained quantum mechanically. Note
that this expression is time independent.

\begin{align*}
   \Ket{\alpha} =& \sum_n c_n \Ket{n} \\
   a \Ket{\alpha} =& \sum_n c_n a \Ket{n} \\
   =& \sum_{n=0}^\infty c_n \sqrt{n} \Ket{n-1} \\
   =& \sum_{n=-1}^\infty c_{n+1} \sqrt{n+1} \Ket{n} \\
   a \Ket{\alpha} =& \sum_{n=0}^\infty c_n \alpha \Ket{n}
\end{align*}

This allows us to solve for the $ c_{n=1} $ values in terms of the coherent
states:

\[
   c_{n+1} = \frac{\alpha^n}{\sqrt{n!}} c_0
\]

Depending on normalization $ c_0 = \exp( -\alpha^2/2 ) $. Thus, we can express a
coherent state as:

\[
   \Ket{\alpha} = e^{-\left| \alpha \right|^2 / 2} \sum_{n=0}^\infty
   \frac{\alpha^n}{\sqrt{n!}} \Ket{n}
\]

This is a useful expression if we want to know things like the probability of
finding n photons given a coherent state $ \Ket{\alpha} $. This can be found to
be:

\[
   p_n = \frac{\left( \left| \alpha \right|^2 \right)^n e^{-\left| \alpha
   \right|^2} }{n!}
\]

This is the well-known Poisson distribution. Suppose we want to calculate the
expectation value of the number of photons: $ \Braket{n} $:

\begin{align*}
   \overline{n} = \Braket{n} =& \sum_n n p_n \\
   =& \sum_n n \frac{\left( \left| \alpha \right|^2 \right)^n e^{-\left|
\alpha \right|^2} }{n!} \\
=& \left| \alpha \right|^2
\end{align*}

So, $ \Braket{\alpha | a^\dagger a | \alpha} = \left| \alpha \right|^2 $. So,
we can re-express the distribution as:

\[
   p_n = \frac{\overline{n}^n e^{-\overline{n}} }{n!}
\]

!! Insert histogram of Poisson distribution !!

We can consider the width of the probability distribution:

\[
   \Braket{n^2} = \sum_n n^2 p_n = \left| \alpha \right|^2 + \left| \alpha
   \right|^4
\]

and $ \Delta n = \sqrt{\Braket{n^2}-\Braket{n}^2} = \sqrt{\left| \alpha
\right|^2} = \sqrt{\overline{n}} $. Thus, the fractional uncertainty: $
\frac{\Delta n}{n} \rightarrow 0 $ for large $ n $. Let's now consider the
quadrature operators:

\[
   \Braket{q} = \Braket{\alpha | \frac{1}{\sqrt{2}} \left( a + a^\dagger \right)
   | \alpha} = \frac{1}{\sqrt{2}} \left( a + a^\dagger \right)
\]

\[
   \Braket{q^2} = \frac{1}{2}\left( \alpha^2 + {a^*}^2 + 2 \left| \alpha
      \right|^2 + 1 \right)
\]

This implies that $ \Delta q = \frac{1}{\sqrt{2}} $ and $ \Delta p =
\frac{1}{\sqrt{2}} $. Recall that $ \Delta p \Delta q \ge \frac{1}{2} $. Thus,
coherent states are minimum uncertainty states. We can express the free-time
evolution of the annihilation operator as:

\[
a \rightarrow a e^{-i \omega t}  
\]

such that

\[
   a(t) \Ket{\alpha} = \alpha e^{-i \omega t} \Ket{\alpha} 
\]

\section{Poisson Distribution}

This distribution arises out of a study of the following scenario. Consider the
following conditions that will construct the scenario:

\begin{itemize}
   \item The probability of an event in some time $ \dif t$ is $\lambda \dif t$
   \item The probability of two events in the same time is negligible
   \item The events are independent
\end{itemize}

Such a problem has the following result:

\[
   p_n (t+dt) = p_n(t) \left( 1 - \lambda dt \right) + p_{n-1}(t) \lambda dt
\]

This can be understood by realizing that there are two ways in which $n$ events
can occur in time $ t $. They can all happen before time $ t $ or $ n-1 $ can happen
before time $ t $ and 1 can happen in the time $ \dif t $ . 

Above, $ p_n $ indicates the probability of n events occurring. Rearranging the
terms in this expression:

\[
   p_n(t+dt) - p_n(t) = \lambda \left[ - p_n(t) + p_{n-1}(t) \right] dt
\]

\[
   \od{p_n}{t} = \lambda \left[ p_{n-1}(t) - p_n(t) \right]
\]

This is recursive set of differential equations. The number of events that
occurs is non-negative. This yields the following initial conditions:

\[
   p_0(0) = 1 \quad \text{and} \quad p_n(0) = 0 \quad \text{and} \quad p_{-1}(t) = 0
\]

Solving this recurrence relation for $ p_n(t) $ yields:

\[
   p_n(t) = \frac{\left( \lambda t \right)^n}{n!} e^{-\lambda t}  
\]

This agrees with the previous expression for the Poisson distribution if
$\overline{n} = \lambda t$. So, if you're counting events and they don't happen
very often, then the count statistics that you can expect to obtain will be
Poissonian.

We can start to develop mathematical tools for working with these states.

\section{The Displacement Operator}

Consider some operator $ D $ defined in the following way $ D(\alpha) =
\exp(\alpha a^\dagger - \alpha^* a) $. To facilitate future math we will now
introduce Glauber's identity (disentangling theorem).

\subsection{Glauber's Identity (Disentangling Theorem)}
\[
   e^{\frac{1}{2}[A,B]} e^{A+B} = e^A e^B 
\]

This is true under the condition that $ [A,[A,B]] = 0 = [B,[A,B]] $. The proof
follows:

\[
   f(x) = e^{Ax} e^{Bx} \quad \text{$ A $ and $ B $ are operators}
\]

Consider $ \od{f}{x} = A e^{Ax} e^{Bx} + e^{Ax} B e^{Bx} $. By inserting
identity in the form $ e^{-Ax} e^{Ax} $ we can write:

\begin{align*}
   \od{f}{x} =& A e^{Ax} e^{Bx} + e^{Ax} B e^{-Ax} e^{Ax} B \\
   =& \left( A + e^{Ax} B e^{-Ax} \right) f(x) \\
   =&  \left( A + B + [A,B]x \right) f(x) \\
   =& \left( \alpha + \beta x \right) f(x)
\end{align*}

Now, we are assuming that $ [A,[A,B]] = [B,[A,B]] = 0 $ so we have:

\[
   f = e^{\left( A+B \right)x} e^{\frac{1}{2} [A,B]x^2}   
\]

When $ x = 1 $  we have $ e^{A}e^{B} = e^{A+B}e^{\frac{1}{2}[A,B]} $.

We can use this identity to write the displacement operator in a different way:

\[
   D(\alpha) = e^{-\frac{1}{2}\left| \alpha \right|^2} e^{\alpha a^\dagger} e^{-
   \alpha^* a} 
\]

This is in the so-called normal ordering form where all of the lower operators
are on the right and the raising operators are on the left.

The coherent states can be expressed in terms of this operator as:

\[
   \Ket{\alpha} = D(\alpha) \Ket{0}
\]

So,

\begin{align*}
   D(\alpha) =& e^{-\frac{1}{2} \left| \alpha \right|^2} e^{\alpha a^\dagger}
   \sum_n \frac{- { \alpha^* }^n }{n!} a^n \Ket{0}  \\
   =& e^{-\frac{1}{2} \left| \alpha \right|^2} \sum_n \frac{\alpha^n}{n!}
   {a^\dagger}^n \Ket{0} \\
   =& \Ket{\alpha}
\end{align*}

This displacement operator has some nice properties:

\begin{itemize}
   \item $ D^\dagger(\alpha) D(\alpha) = D(\alpha) D^\dagger(\alpha) = 1 $
   \item $ D^\dagger (\alpha) = D^{-1} (\alpha) = D(-\alpha) $ 
   \item $ D^\dagger(a) a D(\alpha) = a + \alpha \enskip \text{( BCH )} $ 
   \item $D^\dagger(\alpha) a^\dagger D(\alpha) = a^\dagger + a^*$
\end{itemize}

We can use these identities in some nice ways. Consider the following:

\begin{align*}
   \Braket{\alpha | {a^\dagger}^2 a^2 | \alpha} =& \Braket{0 | D^\dagger(\alpha)
   a^\dagger_n a^\dagger_n a_n a_n D(\alpha) | 0} \\
   =& {\alpha^*}^2 \alpha^2 = \left| \alpha \right|^4
\end{align*}

Let's now consider the effect of cascading multiple displacement operators:

\[
   D(\alpha)D(\beta) = e^{i \Im{(\alpha\beta^*)} } D(\alpha+\beta)
\]

The coherent states are not necessarily orthogonal so they are kind of awkward
to use as a basis.

\begin{align*}
   \Braket{\beta | \alpha} =& \Braket{0 | D^\dagger(\beta) D(\alpha) | 0} \\
   =& e^{i \Im{(-\beta \alpha^*)}} \Braket{0 | D(-\beta \alpha) | 0} \\
   =& e^{i \Im{\left( -\beta \alpha \right)}} e^{-\frac{\left| \alpha - \beta
\right|^2}{2}} \\
=& e^{-\left( \left| \alpha \right|^2 + \left| \beta \right|^2 \right)/2 +
\alpha\beta^*}  
\end{align*}

If $\left| \alpha - \beta \right| $ is large then the overlap will be small.
Analogous to $ \sum_n \ket{n} \bra{n} = 1 $ for coherent states is:

\[
   \int_{-\infty}^\infty \int_{-\infty}^\infty \dif \Im{(\alpha)} \dif
   \Re{(\alpha)} \Ket{\alpha}\Bra{\alpha} = \pi 1
\]

One consequence of all of this is that any operator $ X $ is completely
specified by all of the terms of the form :

\[
   \Braket{\alpha | X | \alpha}
\]

Note that this isn't typically the case. We normally need off-diagonal matrix
elements in addition to all of the diagonal elements. This can be seen by
considering:

\begin{align*}
   \Braket{\alpha | X | \alpha} =& e^{- \left| \alpha \right|^2} \sum_n \sum_m
   \frac{{\alpha^*}^m {\alpha^n}}{\sqrt{m!n!}} \Braket{m | X | n}  \\
   \intertext{Now, for two operators $ Y $ and $ Z $ that have the property that
   $ \Braket{\alpha | Y | \alpha } = \Braket{\alpha | Z | \alpha} $ consider $ X
   = Y - Z $.  Then, }
   \sum_n \sum_m \frac{{\alpha^*}^m \alpha^n}{\sqrt{m!n!}} \Braket{m | X | n} =
   0 \\
   \intertext{This is the form of a Taylor series where all of the coefficients
   are zero (such that the right hand side of the above expression is zero).}
   F(\alpha,\alpha^*) = \sum_n \sum_m \frac{{\alpha^*}^m \alpha^n}{\sqrt{m!n!}}
   \Braket{m | X | n}
\end{align*}

Now, we know that all of the matrix elements are zero, so, $X = Y-Z$ is zero
everywhere. Thus, $Y$ and $Z$ were specified completely just by their diagonal
elements (since $X$ was only formulated in terms of the diagonal elements of $Y$
and $Z$).

\section{Thermal States}
\label{sec:thermal_states}

\begin{itemize}
   \item Thermal states have no time evolution : $ \rho = \sum_n = p_n
      \Ket{n}\Bra{n} $ 
   \item Thermal light is the state of maximum entropy for a fixed amount of
      energy
\end{itemize}

The second property above can be shown by considering the Von-Neumann entropy of
a thermal state $\rho$:

\[
   S = -k_B \sum_n p_n \ln(p_n) - a (\sum_n p_n - 1) - b ( \sum_n p_n E_n -E )
\]

where $ a $ and $ b $ are Lagrange multipliers and $ E $ is the total energy.
The first term is used to normalize the expression of Von-Neumann entropy and
the second term is used to conserve energy. If we maximize this:

\[
   \pd{S}{p_n} = 0 = -k_b \left( \ln p_n + 1 \right) - a - b E_n \rightarrow 
   p_n = \frac{1}{e^{a/k_B + 1}}e^{\frac{-b E_n}{k_B}}
\]

We can recognize $ b = 1/\text{Temperature} $. So, now,

\[
   p_n = \exp(-\frac{E_n}{k_B T}) / \sum_n \exp(\frac{-E_n}{k_B T})
\]

For a harmonic oscillator $ E_n = \hbar \omega \left( n + \frac{1}{2} \right) $.
Using this we can express $ p_n $ as:

\[
   p_n = \frac{\overline{n}^n}{\left( \overline{n} + 1 \right)^{n+1}}
\]

This is the probability distribution for a thermal state. We can observe that:

\[
   p_{n+1} = \frac{\overline{n}}{\overline{n} +1 } p_{n-1}
\]

such that $p_n$ is monotonically decreasing function of n.
!! Insert histogram of $ p_n $ !!


We can calculate the uncertainty in $ n $:

\[
   \Braket{n^2} = \sum_n n^2 p_n = \overline{n} + 2 \overline{n}^2
\]

So, $ \Delta n = \sqrt{\Braket{n^2} - \Braket{n}^2} = \sqrt{ \overline{n}^2 +
\overline{n}}$ 

\section{Thermal Operator}
\label{sec:thermal_operator}
The goal of this operator is going to be to create thermal states out of the
vacuum state. But, it needs to create mixed states. So, it will act in a larger
Hilbert space. Then, we'll have to trace out some of the space. We will
introduce an ancillary mode $\tilde{a}$.

Defining the thermal operator:

\[
   T(\theta) = \exp(\theta(a^\dagger \tilde{a}^\dagger - a \tilde{a}))
\]

This is in the form of a two-mode squeezing operator. With coherent states we
could use Glauber's theorem because the two operators in the exponential
commuted with their commutator. Here, we don't have that. But, it still is
possible to write this operator in normal form:

\[
   T(\theta) = sech(\theta) \exp(a^\dagger \tilde{a}^\dagger \tanh(\theta))
   \exp((a^\dagger a + \tilde{a}^\dagger \tilde{a})\ln(sech(\theta))) \exp(-a
   \tilde{a} \tanh(\theta))
\]

We only have to consider the effect of the first two parts of this operator when
acting on the ground state. The last terms kill any ground state term.

\begin{align*}
   T(\theta) =& \Ket{0,0} \\
   =& sech(\theta) \exp(a^\dagger \tilde{a}^\dagger \tanh(\theta)) \Ket{0,0} \\
   =& sech(\theta) \sum_n \tanh^n(\theta) \Ket{n,n} = \rho \\
   Tr_{\tilde{a}}(\rho) =& sech^2(\theta)\tanh^{2n}(\theta) = \left( 1-
\tanh^2(\theta) \right) \left( \tanh^2(\theta) \right)^n \\
\end{align*}

This is in the form of a thermal distribution where $ \tanh^2(\theta) =
\exp(-\frac{\hbar \omega}{k_B T}$). The strength of the thermal operator is related
to the temperature of the system.

If we consider the following combination of operators $ T^\dagger(\theta) a
T(\theta) $ .

\begin{align*}
   T^\dagger(\theta) a T(\theta) =& a + \left[ \theta \left( a \tilde{a} -
   a^\dagger \tilde{a}^\dagger \right), a \right] + \frac{\theta^2}{2} \left[
\ldots \right] \\
=& a \left( 1 + \frac{\theta^2}{2} + \frac{\theta^4}{4!} + \ldots \right) +
\tilde{a}^\dagger \left( \theta + \frac{\theta^3}{3!} + \frac{\theta^5}{5!} +
\ldots \right) \\
=& cosh(\theta)a + sinh(\theta) \tilde{a}^\dagger
\end{align*}

Likewise,

\[
   T^\dagger(\theta) a^\dagger T(\theta) = cosh(\theta) a^\dagger +
   sinh(\theta) \tilde{a}
\]

These are the Bogoliubov transformations. They can describe the mode
transformations seen in Hawking radiation and the Unruh effect. If we want to
calculate the number of expected photons in a number state:

\begin{align*}
   \Braket{n}_{thermal} =& \Braket{0,0 | T^\dagger(\theta) a^\dagger a T(\theta)
   | 0,0} \\
   =& \Braket{0,0| T^\dagger a^\dagger T T^\dagger a T | 0,0} \\
   \intertext{The above holds because $T$ is formed from exponentiated hermitian
   operators and, as such, is unitary.}
   =& \Braket{0 | \left( cosh(\theta) a^\dagger + sinh(\theta) \tilde{a} \right)
   \left( cosh(\theta)a + sinh(\theta) \tilde{a}^\dagger \right) | 0,0}\\
   =& \Braket{0,0| cosh^2(\theta) a^\dagger a + sinh^2(\theta) \tilde{a}
   \tilde{a}^\dagger| 0,0} + \text{crossterms}\\
   =& sinh^2(\theta)
\end{align*}

Thermal state generation and analysis is difficult, so typically
experimentalists will use ``pseudothermal'' states. These states are generated
by emitting a monochromatic coherent light source at a piece of rotating ground
glass. The ground class is pitted and each pit emits coherent light. Some
distance away from these emitters is placed a collimator that transmits a sum of
all of coherent states. You can show that the probability of obtaining $\alpha$
out of that collimator is:

\[
   P(\alpha) = \frac{1}{\pi \overline{n}} \exp(-\frac{\left| \alpha
   \right|^2}{\overline{n}})
\]

Since we know the probability for every $ \alpha $ then we know everything (we
have already shown that an operator is uniquely defined if only the diagonal
matrix elements are given when using the coherent state basis). The effective
state, then, is

\[
   \rho = \int \dif^2 \alpha p(\alpha) \Ket{\alpha}\Bra{\alpha}
\]

If we now allow $ \alpha = r e^{i \theta} $ then we can write:

\[
   \rho = \frac{1}{\pi \overline{n}} \int_0^\infty \int_0^{2\pi} r \dif r \dif
   \theta e^{-r^2/\overline{n}}\sum_n e^{-r^2/2}\frac{r^n e^{i n \theta}
   }{\sqrt{n!}}\Ket{n} \sum_m e^{-r^2/2} \frac{r^m e^{-i m \theta} }{\sqrt{m!}} \Bra{m}  
\]

This can be re-expressed as:

\[
   \rho = \frac{1}{\pi \overline{n}} \sum_n \sum_m \int_0^\infty \dif r e^{-r^2 \left(
   \frac{1}{\overline{n}} +1 \right)}
   \frac{r^{n+m+1}}{\sqrt{n!m!}}\Ket{n}\Bra{m} \int_0^{2\pi} \dif \theta e^{i
   \left( n-m \right)\theta}  
\]

and, finally:

\[
   \rho = \frac{2}{\overline{n}} \sum_n \int_0^\infty \dif r e^{-r^2 \left(
   \frac{1}{\overline{n}} + 1 \right)} \frac{r^{2n+1}}{n!} \Ket{n}\Bra{n}
\]


\end{document}
