\begin{homeworkProblem}
A simple capacitor is a device formed by two insulated conductors adjacent to each 
other. If equal and opposite charges are placed on the conductors, there will be a 
certain difference of potential between them. The ratio of the magnitude of the 
charge on one conductor to the magnitude of the potential difference is called the 
capacitance (in SI units it is measured in farads). Using Gauss's law, calculate the 
capacitance of 

\begin{homeworkSection}{(a)}

two large, flat, conducting sheets of area A, separated by a small distance d; 

Using a small Gaussian surface on either side of the sheet I find that the electric field looks like $E * 2A = \frac{\sigma a}{\epsilon_0}$, where the factor of 2A on the left side came from the fact that there is flux through both sides of my pill box. Thus, the electric field from a sheet of charge can be approximated in the large A, small d limit as $E = \frac{Q}{2 A \epsilon_0}$ pointing perpendicular to the plates. Thus, the electric field inside, due to both the positive and negative charge is $E_{inside} = \frac{Q}{A \epsilon_0}$. The potential gained by traveling from one plate to another is just $\int \vec{E}\cdot \vec{dl} = \frac{2Q d}{a \epsilon_0}$. This was obtained by taking a path perpendicular to the plates (the direction in which the field faces). The ratio of the charge on one plate to the change in potential is : $\frac{Q}{V} = \frac{A \epsilon_0}{d}$.

\end{homeworkSection}

\begin{homeworkSection}{(b)}

two concentric conducting spheres with radii a, b (b > a); 

Using the results from Problem 1, I know that the potential outside of a sphere of charge is $\frac{k Q}{r}$, where r is my distance from the center of the sphere of charge and Q is the charge on the sphere. Thus, the change in potential between distances a and b with $ b > a $ is $|\Delta V| = \frac{k Q} (\frac{1}{a} - \frac{1}{b})$, where the difference has been taken in order to make the result a positive quantity. So, the capacitance, which is given by $ C = \frac{Q}{V} = \frac{1}{4\pi \epsilon_0}\frac{1}{\frac{1}{a}-\frac{1}{b}} $.

\end{homeworkSection}

\begin{homeworkSection}{(c)}
two concentric conducting cylinders of length L, large compared to their radii a,b (b > a). 

Using a Gaussian cylinder as my test volume and assuming that the field is radially-distributed (reasonable given the conditions of the problem statement). Thus $E(r) 2 \pi r \Delta z = \frac{\sigma 2 \pi a \Delta z}{\epsilon_0}$. Here, $\sigma$ is the charge per unit area on the surface of the cylinder. Thus, $E(r) = \frac{\sigma a}{\epsilon_0 r}$. But, if we consider the charge per unit length $\lambda$ that runs along the length of the cylinder : $\sigma = \frac{\lambda}{2 \pi a}$. Thus $E(r) = \frac{\lambda}{2\pi \epsilon_0 r}$. $\lambda = \frac{Q}{L}$. The potential gained from moving out along the radius will go as the integral of $E(r)$ over r. Thus, the expression obtained will be a natural log. $\Delta V(r) = \frac{Q}{2\pi L \epsilon_0} ln(b/a)$ where b is a distance chosen to be larger than a so as to make the change in potential a positive quantity. Now, $C = \frac{Q}{V} = \frac{Q}{\frac{Q}{2\pi L \epsilon_0} ln(b/a)} = \frac{2 \pi L \epsilon_0}{ln(b/a)}$. 

\end{homeworkSection}

\begin{homeworkSection}{(d)}
What is the inner diameter of the outer conductor in an air-filled coaxial cable 
whose center conductor is a cylindrical wire of diameter 1 mm and whose 
capacitance is:

\begin{itemize}
	\item $ 3*10^{-11} \frac{F}{m}?$
	\item $ 3*10^{-12} \frac{F}{m}?$
\end{itemize}

To be general I will solve for the inner diameter as a function of an arbitrary capacitance. From problem 5 (c) I know that $C/L = \frac{2 \pi \epsilon_0}{ln(b/a)}$. Thus, $ b = a e^{\frac{2 \pi \epsilon_0}{C/L}}$. I am given b. $2\pi \epsilon_0 \approx 5.563*10^-11 F/m$. Thus, substituting the proper values: $b_{3*10^{-11} F/m} \approx 1 mm * e^{1.85} \approx 6.36 mm$ and $b_{3*10^{-12} F=m} \approx 1 mm * e^{18.5} \approx 108 km $

\end{homeworkSection}

\end{homeworkProblem}