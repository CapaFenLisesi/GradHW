\begin{homeworkProblem}

Two point charges (each of charge Q) are positioned at z=R and z=-R on the z-axis. 
A circular ring of radius R is: centered at the origin, sits on the x-y plane and has a total 
charge of -2Q uniformly distributed on its circumference. 

\begin{homeworkSection}{(a)} 

Determine the potential along the z-axis (for r>R). 

In order to determine the potential everywhere in space I will first determine the electric field everywhere in space. This allows me to answer part b) immediately. Of course, I could take the gradient of the potential and get the electric field for part b). Sue me.

First, let's consider the two point charges (because they are super easy). The electric field generated by a point charge (in SI units, used for the remainder of this assignment) looks like $\vec{E} = k \frac{Q}{|\vec{r}-vec{r'}|^3}(\vec{r}-\vec{r'})$. Considering a point on the positive Z axis, where z > R. Both charges are of value +Q. Thus, the electric field points upward along the z-axis. The electric field from the point charges, then, above this z value is: $ \vec{E_{z_{pc}}}(z>R) = k \frac{Q}{(z-R)^2}\hat{z} + k \frac{Q}{(z+R)^2}\hat{z} $. When z < -R this expression must indicate that the field points downward along the negative z axis. Thus, $\vec{E_{z_{pc}}}(z<-R) = -k \frac{Q}{(z+R)^2} \hat{z} - k \frac{Q}{(z-R)^2} \hat{z}$. Note that the magnitude of this expression is the same as the first. Only the sign changes. This makes sense.

Now, let's consider the ring of charge. Given that this is a distribution of charge along a line segment (of length $2\pi R$) the charge density on the ring can be given by $\lambda = \frac{-2Q}{2\pi R}$. Let's consider the effect of an infinitesimal amount of charge $dq = \lambda * R d\theta$ located on the ring of charge. We will determine the field generated by this infinitesimal amount of charge at a position along the z axis where, still, z > R.

\begin{align}
\vec{dE} = k \frac{\lambda R d\theta}{|z\hat{z}-R\hat{rho}|^3}(z\hat{z} - R\hat{\rho}) \nonumber \\
\intertext{Here, $\hat{\rho}$ is the unit vector that points radially towards the center of the ring in the XY plane} \nonumber \\
E_z = \vec{dE_z} \cdot \hat{z} = k \frac{\lambda R d\theta}{(z^2+R^2)^1.5}z \nonumber \\
\vec{dE_{z_{ring}}}(z>R) = k \frac{\lambda R d\theta}{(z^2+R^2)^2.5}z \hat{z} \nonumber \\
\intertext{Now, we need to add up all of the charge. I'll use a polar integral around the circumference of the ring.} \nonumber \\
E_{z_{ring}}(z>R) = k \frac{\lambda R}{(z^2+R^2)^1.5}z \int_0^2\pi d\theta \nonumber \\
E_{z_{ring}}(z>R) = 2\pi k \frac{\lambda R}{(z^2+R^2)^1.5}z \nonumber \\
\intertext{Of course, $2\pi R \lambda  = -2Q$.} \nonumber \\
E_{z_{ring}}(z>R) = -k \frac{2Q}{(z^2+R^2)^1.5}z \nonumber \\
\intertext{$k \equiv \frac{1}{4\pi \epsilon_0}$ so I could reduce this expression. I won't. The field for z<-R is the same, but it is directed in the $\hat{-z}$ direction.} \nonumber
\end{align}

Thus, the total electric field in the z-direction is $\vec{E_z} = \vec{E_{z_{ring}}} + \vec{E_{z_{pc}}}$ which can be written as $\vec{E_z}(z>R,z<-R) = (-k \frac{2Q}{(z^2+R^2)^1.5}z + k \frac{Q}{(z-R)^2}\hat{z} + k \frac{Q}{(z+R)^2} ) \hat{z} (\theta(z+R)-1)$. Here, $\theta (z)$ is a Heaviside step function in z. That is, the function returns 1 when the argument is non-negative and 0 when the argument is negative. This toggles the unit vector symbol correctly. Note that this field expression is only valid (as stated) when z lies above R or below -R. The field in the region where z>-R and z<R has not been determined. It has not been requested.

By convention, the electric potential is zero at infinity. Also, by convention, the electric potential is defined in terms of the electric field as $V(x_1,x_2,x_3) = -\int_\infty^{x_1} \int_\infty^{x_2} \int_\infty^{x_3} \vec{E}(x'_1,x'_2,x'_3) \cdot dl \hat{dl}$. Here, $\vec{dl}$ is an infinitesimal displacement in the direction of the path that takes me from $(\infty,\infty,\infty)$ to $(x_1,x_2,x_3)$. Such an integral is called a ``path integral''. Let's consider a path integral along the z-axis to some position where z>R. I will integrate the electric field along that line and determine the electric potential along the z axis for all points where z > R. 

\begin{align}
\int_\infty^{z>R} (k \frac{-2Q}{(z^2+R^2)^1.5}z + k \frac{Q}{(z-R)^2}\hat{z} + k \frac{Q}{(z+R)^2} ) \hat{z} \cdot dz -\hat{z} \nonumber \\
\int_\infty^(z>R)  (-k \frac{2Q}{(z^2+R^2)^1.5}z) dz + \int_\infty^(z>R) k \frac{Q}{(z-R)^2} + k \frac{Q}{(z+R)^2} dz \nonumber \\
\intertext{At this time it is useful to cite the following integral identity: $\int_\infty^a \frac{x}{(x^2+b^2)^1.5}dx = -(b^2+a^2)^-.5$} \nonumber \\
-\frac{2Q}{(z^2+R^2}^1.5 + k\frac{Q}{z-R} + k\frac{Q}{z+R} \nonumber
\end{align}

However, given the symmetry of the problem, traversing the path from $-\infty \rightarrow z < -R$ would result in the same potential. I would be climbing electric field vectors from the positive point charges, still. I would be traveling in the direction of the field lines from the charged ring. So, I would have the same potential.

Thus, $V(z>R) = V(z<-R) = -\frac{2Q}{(z^2+R^2)^1.5} + k\frac{Q}{z-R} + k\frac{Q}{z+R}$.


\end{homeworkSection}

\begin{homeworkSection}{(b)}

Determine the E-field along the z-axis (for r>R). This solution can be taken immediately from part a). 

 $\vec{E_z} = \vec{E_{z_{ring}}} + \vec{E_{z_{pc}}}$ which can be written as $\vec{E_z}(z>R,z<-R) = (-k \frac{2Q}{(z^2+R^2)^1.5}z + k \frac{Q}{(z-R)^2}\hat{z} + k \frac{Q}{(z+R)^2} ) \hat{z} (\theta(z+R)-1)$

\end{homeworkSection}

\begin{homeworkSection}{(c)}
In the case that z>>R (but not infinity), what is a good approximation for the Electric 
Potential along the x-axis? (it�s not zero, write the potential as an expansion and keep 
the first none zero term). 

This problem requires that I expand the potential. I will consider a fixed z. I will then consider expanding about small R/z. I will expand about $R/z = 0$. I will then keep the lowest order terms. To make my life a little easier, though, I will utilize the following property of Maclaurin series expansions. Consider $h(\alpha) = f(\alpha) + g(\alpha)$. The expansion of $h(\alpha)$ about $\alpha = 0$ is $\sum_n \frac{d^n h}{d^n \alpha}|_{\alpha = 0} \alpha^n = \sum_n \frac{d^n (f+g)}{d^n \alpha}|_{\alpha = 0} \alpha^n = \sum_n \frac{d^n f}{d^n \alpha} \alpha^n + \sum_n \frac{d^n g}{d^n \alpha}|_{\alpha = 0} \alpha^n$. The last expression is simply the sum of the two expansions, though.

Thus, $V(z) = -\frac{2Q}{z(1+(\frac{R}{z})^2)} + k \frac{Q}{z(1-\frac{R}{z})} + k\frac{Q}{z(1+\frac{R}{z}}$. It can easily be shown that the expansion of a function of the form $(1+x^2)^{-.5}$ for sufficiently small x is $1-\frac{x^2}{2}+\frac{3}{8}x^4$. Similarly, a function of the form $(1 \pm x)^{-1}$ for sufficiently small x resembles $1 \mp x \mp x^2$.

Expanding the potential about $\frac{R}{z} = 0$ yields, then, to second order in $\frac{R}{z}$: $V_{O(2)}(z) = -\frac{2Q}{z}(1-(\frac{\frac{R}{z})^2}{2}+\frac{kQ}{z}(1-\frac{R}{z}-(\frac{R}{z})^2)+\frac{kQ}{z}(1+\frac{R}{z}+(\frac{R}{z})^2)=3k\frac{Q}{z}(\frac{R}{z})^2$.

\end{homeworkSection}

\end{homeworkProblem}