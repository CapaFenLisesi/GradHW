\begin{homeworkProblem}

\textbf{Using Dirac delta functions in the appropriate coordinates, express the following 
charge distributions as three-dimensional charge densities p(x). }

\begin{homeworkSection}{(a)} 

\textbf{In spherical coordinates, a charge Q uniformly distributed over a spherical 
shell of radius R. }

Without loss of generality I will assume that the sphere is centered at the origin of my spherical coordinate system. To solve this problem I just need to meet the following requirements: $Q = \int_{r>R} \rho(x')d\tau'$ and that $0 = \int_{r<R} \rho(x')d\tau '$. I will propose the following charge distribution $\rho(x') = \frac{Q}{4\pi R^2} \delta(|\vec{r}|-R)$. Sticking this into a spherical integral: $\int_0^\pi \int_0^{2\pi} \int_0^\infty = \frac{Q}{4\pi R^2} r^2 sin(\theta) dr d\phi d\theta = 4\pi \frac{Q}{4\pi R^2}R^2 = Q.$. Here, I have chosen $\phi$ to be my azimuthal angle and $\theta$ to be my colatitude or polar angle. This result is good. My delta function is positioned at the right place.  If I was to integrate over a sphere within the charged sphere, I would pick up no charge (delta function is zero, there). If I was to integrate over a sphere larger than the charged sphere then I would pick up Q charge (delta function would force $\int r^2 \delta(|\vec{r}|-R) dr \rightarrow R^2$).

\end{homeworkSection}

\begin{homeworkSection}{(b)}

\textbf{In cylindrical coordinates, a charge $\lambda$ per unit length uniformly distributed 
over a cylindrical surface of radius b. }

Let's propose the following solution and just see if it works. I don't have a great way to explain my mental process for generating these charge densities. I'm not stealing them. I promise. My roommate helped me think through these problems but I used no solutions manuals or internet resources. Consider $\rho(\vec{r'}) = \frac{\lambda}{2\pi b} \delta(|\vec{r'}|-b)$. Sticking this into an integral over a cylinder of $\Delta z$ height and infinite radius: $\int_{all xy space, and some \Delta z height} \frac{\lambda}{2\pi b} \delta(\vec{r'}|-R) r' dr' d\theta' dz = \frac{\lambda}{2\pi b} b 2\pi \Delta z = \lambda \Delta z$. This is the amount of charge I expect to get when I do this integral. When my radius is too small I grab no charge. When my radius is just past b then I pick up $\lambda \Delta z$ charge.

\end{homeworkSection}

\begin{homeworkSection}{(c)}

\textbf{In cylindrical coordinates, a charge Q spread uniformly over a flat circular 
disc of negligible thickness and radius R. }

Again, proposing a volume charge density: $\rho(\vec{r'}) = \frac{Q}{\pi R^2} \delta(z) \int_{-\infty}^{R-|\vec{r}|}\delta(t) dt$. The integral of the delta function ($ \int_{-\infty}^{R-|\vec{r}|}\delta(t) dt $) is just a representation of the Heaviside step function. That is: $\int_{-\infty}^{R-|\vec{r}|}\delta(t) dt = H(R-|\vec{r}|)$. I have used a new notation $H(x)$ for a Heaviside step function of x instead of the more traditional $\theta(x)$ to avoid confusion with $\theta$ being used as an angular coordinate.

Sticking this volume charge density into an integral over a cylinder of infinite height and radius a < R: $\int_{-\infty}^\infty \int_0^{2\pi} \int_0^a  \frac{Q}{\pi R^2} \delta(z) (\int_{-\infty}^{R-|\vec{r}|}\delta(t) ) r dr d\theta dz dt = \frac{Q}{\pi R^2} \frac{a^2}{2} 2\pi = Q \frac{a^2}{R^2}$. This is exactly what I would expect. Once, $a = R$ I collect all of the charge. 

\end{homeworkSection}

\begin{homeworkSection}{(d)}

\textbf{The same as part (c), but using spherical coordinates.}

A snarky solution would just have me replace $r \rightarrow r sin(\theta)$ and $z \rightarrow r cos(\theta)$ and $\theta \rightarrow \phi$. This would result in the expression : $\rho(\vec{r'}) = \frac{Q}{\pi R^2} \delta(R cos(\theta)) \int_{-\infty}^{R-|\vec{r sin(\theta)}|}\delta(t) dt$. Technically, this is correct. However, evaluating functions of direct delta distributions is non-trivial. Therefore, I will try to recast this expression in a form which utilizes products of dirac delta distributions over the various coordinates.
\\
Consider, $\rho(\vec{r'}) = \frac{Q}{\pi R^2 r'} \delta(\theta - \frac{\pi}{2}) (\int_{-\infty}^{R-|\vec{r'}|} \delta(t) dt)$. Sticking this into a volume integral similar to the one in (c) yields: $\int_0^\pi \int_0^{2\pi} \int_0^a \frac{Q}{\pi R^2 r'}\delta(\theta - \frac{\pi}{2}) (\int_{-\infty}^{R-|\vec{r'}|} \delta(t) dt) r'^2 sin(\theta) dr d\phi d\theta = \int_0^a \frac{Q}{\pi R^2} r' 2\pi dr' = 2\pi \frac{Q}{\pi R^2} \frac{a^2}{2} = \frac{Qa^2}{R^2}$. This is just as expected. When $a \rightarrow R$ I obtain the total charge Q built up on the surface. 

\end{homeworkSection}

\end{homeworkProblem}