 \documentclass[17pt]{article}
\usepackage{fancyhdr} % Required for custom headers
\usepackage{lastpage} % Required to determine the last page for the footer
\usepackage{extramarks} % Required for headers and footers
\usepackage{graphicx} % Required to insert images
\usepackage{lipsum} % Used for inserting dummy 'Lorem ipsum' text into the template
\usepackage{amsmath,amsthm,amsxtra}

% Margins
\topmargin=-0.45in
\evensidemargin=0in
\oddsidemargin=0in
\textwidth=6.5in
\textheight=9.0in
\headsep=0.25in 

\linespread{1.1} % Line spacing

% Set up the header and footer
\pagestyle{fancy}
\lhead{\hmwkAuthorName} % Top left header
\chead{\courseTitle\ : \hmwkTitle} % Top center header
\rhead{\firstxmark} % Top right header
\lfoot{\lastxmark} % Bottom left footer
\cfoot{} % Bottom center footer
\rfoot{Page\ \thepage\ of\ \pageref{LastPage}} % Bottom right footer
\renewcommand\headrulewidth{0.4pt} % Size of the header rule
\renewcommand\footrulewidth{0.4pt} % Size of the footer rule

\setlength\parindent{0pt} % Removes all indentation from paragraphs

%----------------------------------------------------------------------------------------
%	DOCUMENT STRUCTURE COMMANDS
%	Skip this unless you know what you're doing
%----------------------------------------------------------------------------------------

% Header and footer for when a page split occurs within a problem environment
\newcommand{\enterProblemHeader}[1]{
\nobreak\extramarks{#1}{#1 continued on next page\ldots}\nobreak
\nobreak\extramarks{#1 (continued)}{#1 continued on next page\ldots}\nobreak
}

% Header and footer for when a page split occurs between problem environments
\newcommand{\exitProblemHeader}[1]{
\nobreak\extramarks{#1 (continued)}{#1 continued on next page\ldots}\nobreak
\nobreak\extramarks{#1}{}\nobreak
}

\setcounter{secnumdepth}{0} % Removes default section numbers
\newcounter{homeworkProblemCounter} % Creates a counter to keep track of the number of problems
\newcommand{\homeworkProblemName}{}
\newenvironment{homeworkProblem}[1][Problem \arabic{homeworkProblemCounter}]{ % Makes a new environment called homeworkProblem which takes 1 argument (custom name) but the default is "Problem #"
\stepcounter{homeworkProblemCounter} % Increase counter for number of problems
\renewcommand{\homeworkProblemName}{#1} % Assign \homeworkProblemName the name of the problem
\section{\homeworkProblemName} % Make a section in the document with the custom problem count
\enterProblemHeader{\homeworkProblemName} % Header and footer within the environment
}{
\exitProblemHeader{\homeworkProblemName} % Header and footer after the environment
}

\newcommand{\problemAnswer}[1]{ % Defines the problem answer command with the content as the only argument
\noindent\framebox[\columnwidth][c]{\begin{minipage}{0.98\columnwidth}#1\end{minipage}} % Makes the box around the problem answer and puts the content inside
}

\newcommand{\homeworkSectionName}{}
\newenvironment{homeworkSection}[1]{ % New environment for sections within homework problems, takes 1 argument - the name of the section
\renewcommand{\homeworkSectionName}{#1} % Assign \homeworkSectionName to the name of the section from the environment argument
\subsection{\homeworkSectionName} % Make a subsection with the custom name of the subsection
\enterProblemHeader{\homeworkProblemName} % Header and footer within the environment
}{
\enterProblemHeader{\homeworkProblemName} % Header and footer after the environment
}

%----------------------------------------------------------------------------------------
%	NAME AND CLASS SECTION
%----------------------------------------------------------------------------------------

\newcommand{\hmwkTitle}{Assignment 1} % Assignment title
\newcommand{\hmwkDueDate}{Wendesday,\ January\ 30,\ 2013} % Due date
\newcommand{\courseTitle}{Physics 760: Electricity and Magnetism} % Course/class
\newcommand{\hmwkClassInstructor}{Dr. Stefan Kycia} % Teacher/lecturer
\newcommand{\hmwkAuthorName}{John Rinehart} % Your name
\newcommand{\sudentNumber}{20503440} % Your name
\newcommand{\position}{PhD student at ECE departent}

%----------------------------------------------------------------------------------------
%%-----------------------------------------------------------------------------------------

%%%%%%%%%%
\newcommand{\red}[1]{\textcolor[rgb]{1,0,0}{#1}}
\newcommand{\blu}[1]{\textcolor[rgb]{0,0,1}{#1}}
\newcommand{\bs}[1]{\boldsymbol{#1}}
%\newcommand{\V}[1]{\bm{#1}}
\newcommand{\V}[1]{\Vec{#1}}
\newcommand{\A}[1]{\Hat{#1}}
\newcommand{\W}[1]{\widehat{#1}}
\newcommand{\T}[1]{\widetilde{#1}}
\newcommand{\pd}[2]{\dfrac{\partial #1}{\partial #2}}
\newcommand{\fpd}[2]{\frac{\partial #1}{\partial #2}}
\newcommand{\pds}[1]{\dfrac{\partial}{\partial #1}}
\newcommand{\fpds}[1]{\frac{\partial}{\partial #1}}
\newcommand{\pdss}[1]{\dfrac{\partial^2}{\partial {#1}^2}}
\newcommand{\pdsss}[2]{\dfrac{\partial^2}{\partial #1 \partial #2}}
\newcommand{\pdt}[2]{\dfrac{\partial^2 {#1}}{\partial {#2}^2}}
\newcommand{\pdtt}[3]{\dfrac{\partial^2 {#1}}{\partial {#2} \partial {#3}}}
\newcommand{\dif}[2]{\frac{d{#1}}{d{#2}}}
\newcommand{\vt}[1]{\Vec{\mathcal{#1}}}
\newcommand{\VP}[1]{\Vec{\mathbf{#1}}}
\newcommand{\vp}[1]{\mathbf{#1}}
\newcommand{\phas}[1]{\angle{#1}^{\circ}}
\newcommand{\er}{\epsilon_{r}}
\newcommand{\mr}{\mu_{r}}
\newcommand{\Lrw}{\Longrightarrow}
\newcommand{\refeq}[1]{(\ref{#1})}
\newcommand{\abs}[1]{\left| #1\right|}
\newcommand{\ket}[1]{|#1\rangle}
\newcommand{\bra}[1]{\langle #1| }
\newcommand{\bracket}[2]{\langle#1|#2\rangle }


%%%---
\newcommand\ointint{\begingroup
\displaystyle \unitlength 1pt
\int\mkern-7.2mu
\begin{picture}(0,3)
\put(0,3){\oval(10,8)}
\end{picture}
\mkern-7mu\int\endgroup}
%%%----
\providecommand{\abs}[1]{\lvert#1\rvert}
\providecommand{\norm}[1]{\lVert#1\rVert}

%%%%%%%%%%%%%%%

%---Packeges------------------------------------------------------------------
 %------------------------------------------------------
\usepackage[utf8]{inputenc}
\usepackage[T1]{fontenc}
\usepackage[english]{babel}
\usepackage[latin1]{inputenc}
\usepackage[T1]{fontenc}
\usepackage{pstricks}
\usepackage[usenames,dvipsnames]{pstricks}
\usepackage{epsfig}
\usepackage{pst-grad} % For gradients \usepackage{pst-plot} % For axes
\usepackage{pifont}
\usepackage{amsfonts}
\graphicspath{{IMG/}}
\usepackage[utf8]{inputenc}
 \usepackage[OT1]{fontenc}
 \usepackage[absolute,overlay]{textpos}
 \usepackage{graphicx}
 \usepackage[bookmarks=false,pdffitwindow]{hyperref}
 \usepackage{tikz}
 \usepackage{xcolor}
 \usepackage{calc}
\usepackage{chngcntr}
%----------------------------------------------------------------
\numberwithin{equation}{section}
\renewcommand{\theequation}{\arabic{equation}}



%%------------------------------------------------------------------------------------------
%	TITLE PAGE
%----------------------------------------------------------------------------------------

\title{
\vspace{2in}
\textmd{\textbf{\courseTitle\\ \vspace{0.5in}\hmwkTitle}}\\
\vspace{0.5in}\large{{\hmwkClassInstructor}}
\vspace{3in}
}
\author{\textbf{\hmwkAuthorName}\\ 
}\\

\date{January 1, 2000} % Insert date here if you want it to appear below your name

%----------------------------------------------------------------------------------------

\begin{document}

\maketitle


\newpage
\tableofcontents
\newpage


%----Problems-------
%%%\begin{homeworkProblem}

\begin{homeworkSection}{(a)}
To construct the Hamiltonian describing time evolution of the dynamical varibales we can simply consider the total energy of the system as a constant of motion however we firmly follow a more general approach. we first try to redrive the second order differential equation which determines equation of motion based on Lagrange equation. Primary KVL and KCL equations manifest themself in defining conjugate varibales. Quite generally we should first pick up minimum number of varibles which can completely describe the working point in configuration space. We choose $\mathcal{Q}=v$ as the sole dynamical quantity. $i$ is related to $v$ by:
\begin{equation}
i=C\frac{dv}{dt}=C\dot{\mathcal{Q}}
\end{equation}
\begin{figure}[!h]
\centering
% Generated with LaTeXDraw 2.0.8
% Thu Feb 28 09:11:00 GMT-06:00 2013
% \usepackage[usenames,dvipsnames]{pstricks}
% \usepackage{epsfig}
% \usepackage{pst-grad} % For gradients
% \usepackage{pst-plot} % For axes
\scalebox{0.7} % Change this value to rescale the drawing.
{
\begin{pspicture}(0,-1.5479687)(4.9146876,1.5879687)
\psline[linewidth=0.028222222cm](3.8764584,0.9753869)(3.8864996,0.56300914)
\rput{-88.605156}(3.507715,4.24715){\psarc[linewidth=0.028222222](3.9297698,0.32649234){0.2}{0.0}{180.0}}
\rput{-88.605156}(3.916978,3.8667367){\psarc[linewidth=0.028222222](3.9395065,-0.07338913){0.2}{0.0}{180.0}}
\rput{-88.605156}(4.3262405,3.486323){\psarc[linewidth=0.028222222](3.9492435,-0.4732706){0.2}{0.0}{180.0}}
\rput{-88.605156}(4.7366757,3.073829){\psarc[linewidth=0.028222222](3.9431307,-0.8897928){0.18375}{0.0}{180.0}}
\psline[linewidth=0.028222222cm](3.9272726,-1.1114945)(3.9373138,-1.5238723)
\psline[linewidth=0.04cm](3.936875,-1.5279688)(1.356875,-1.5279688)
\psline[linewidth=0.04cm](3.876875,0.97203124)(1.296875,0.97203124)
\psline[linewidth=0.054cm](1.756875,-0.098767914)(0.916875,-0.098767914)
\psline[linewidth=0.054cm](1.756875,-0.38796875)(0.916875,-0.38796875)
\psline[linewidth=0.04cm](1.316875,0.95203125)(1.336875,-0.04796875)
\psline[linewidth=0.04cm](1.376875,-1.5279688)(1.376875,-0.38796875)
\psline[linewidth=0.04cm,arrowsize=0.133cm 2.36,arrowlength=1.4,arrowinset=0.4]{<-}(2.236875,0.9521436)(2.576875,0.9519189)
\usefont{T1}{ptm}{m}{n}
\rput(2.1842186,-0.27296874){\large $C$}
\usefont{T1}{ptm}{m}{n}
\rput(4.534219,-0.29296875){\large $L$}
\usefont{T1}{ptm}{m}{n}
\rput(2.5142188,1.3670312){\large $i$}
\usefont{T1}{ptm}{m}{n}
\rput(0.28421876,-0.25296876){\large $v$}
\end{pspicture} 
}

\caption{\small parallel LC network}
\end{figure}
Please note that we could also choose the cahrge of the capacitor ($Q$) and total magnetic flux inisde the inductor ($\Phi$) as the dynamical variables. We define the action as the difference of magnetic and electrical energies as:
\begin{equation}
\mathcal{L}=\frac{1}{2}Li^2-\frac{1}{2}Cv^2=\frac{1}{2}C\left[\frac{1}{\omega_0^2}\dot{\mathcal{Q}}^2-\mathcal{Q}^2\right]
\end{equation}  
where $\omega_0$ is defined as 
$$\omega_0=\frac{1}{\sqrt{LC}}$$
Lagrange equation for this action is:
\begin{equation}
\frac{d}{dt}\pd{\M{L}(\M{Q},\dot{\M{Q}})}{\dot{\M{Q}}}-\pd{\M{L}(\M{Q},\dot{\M{Q}})}{\M{Q}}=0\quad\Lrw\quad \ddot{\M{Q}}=-\omega_0^2\M{Q}\quad\Lrw\quad \frac{d^2v}{dt^2}=-\omega_0^2 v
\end{equation}
This equation is exactly same as the equation which can be derived based on KVL and KCL equations. Thus our Lagrangian works well and we can go ahead to cunstruct the Hamiltonian. First note that the conjugate varible corresponde to  $\M{Q}$ is:
\begin{equation}
\M{P}=\pd{\M{L}}{\dot{\M{Q}}}=\frac{C\dot{\M{Q}}}{\omega_0^2}=LC i
\end{equation}
Now the Hamiltonian is:
\begin{equation}\label{P1-H}
\M{H}=\left.\M{P}\dot{\M{Q}}-\M{L}(\M{Q},\dot{\M{Q}})\right|_{\dot{\M{Q}}=\omega_0^2\M{P}/C}
=\frac{\omega_0^2\M{P}^2}{2C}+\frac{1}{2}C\M{Q}^2
\end{equation}
Interestingly canonical equations of motion in phase space are:
\begin{align}
\dot{\M{Q}}=\pd{\M{H}(\M{Q},\M{P})}{\M{P}}=\frac{\omega_0^2\M{P}}{C}\quad  &\Lrw \quad   C\frac{dv}{dt}=i \quad\text{(KCL)}\\
\dot{\M{P}}=-\pd{\M{H}(\M{Q},\M{P})}{\M{Q}}=-C\M{Q}\quad &\Lrw \quad  L\frac{di}{dt}=-v \quad \text{(KVL)}
\end{align}
\end{homeworkSection}
\begin{homeworkSection}{(b)}
The Hamiltonian describing time evolution in a one dimensional SHO is:
\begin{equation}\label{P1-SHO}
H=\frac{p^2}{2m}+\frac{1}{2}m\omega^2x^2
\end{equation}
where $x$ and $p$ are conjugate varibles. In both \eqref{P1-H} and \eqref{P1-SHO} Hamiltonians are elipitical functions of two conjugate varibles. In fact both Hamiltonians are equal if we do the following replacements:
\begin{equation}\label{P1-Sub}
H\leftrightarrow \M{H}\qquad x\leftrightarrow\M{Q}\qquad p\leftrightarrow\M{P}\qquad m\leftrightarrow LC^2\qquad\omega\leftrightarrow \omega_0
\end{equation}

 \end{homeworkSection}

\begin{homeworkSection}{(c)}
 The conh=jugate varible to $\M{Q}=v$ has been calculated in part (a) as we showed:
 \begin{equation}
 \M{P}=\pd{\M{L}}{\dot{\M{Q}}}=\frac{C\dot{\M{Q}}}{\omega_0^2}=LC i
 \end{equation}
 \end{homeworkSection}

%--------d--------------
\begin{homeworkSection}{(d)}
To quantize the LC circuit we start form the Driac's quantization rules. We first take a look at fundamental Poison backets in the classical problem:
\begin{equation}
\left\{\M{Q},\M{P}\right\}=\pd{\M{Q}}{\M{Q}}\pd{\M{P}}{\M{P}}-\pd{\M{P}}{\M{Q}}\pd{\M{Q}}{\M{P}}=1
\end{equation}   
According to the Dirac's quantization rule we should just replace:
\begin{equation}
\frac{1}{j\hbar}\left[\quad ,\quad \right]\leftrightarrow\left\{\quad,\quad\right\}
\end{equation}
Please note that we have used $j$ as the unit imaginary number. This quantization implies that $\M{Q}$ and $\M{P}$ should be treated as operators. So we have:
\begin{equation}
\left[\A{\M{Q}},\A{\M{P}}\right]=\left[\A{v},LC\A{i}\right]=j\hbar
\end{equation}
This commutation relation is the starting point to develop the whole \textit{Lie algebra}. So we should have the same quantization and generally algebra in a SHO and quantum LC circuit.
\end{homeworkSection}



%------------e-------------
\begin{homeworkSection}{(e)} 
As it's explained in the previous part we have the same algebra in SHO and quantum LC network. So we can simply use the result of our discussions in SHO. we should just use the substitutions given in \eqref{P1-Sub}. So we have:
\begin{align}
&a_{SHO}=\sqrt{\frac{m\omega}{2\hbar}}\left(x+\frac{jp}{m\omega}\right)\quad\Lrw\quad 
\M{A}=\sqrt{\frac{LC^2\omega_0}{2\hbar}}\left(\M{Q}+\frac{j\M{P}}{LC^2\omega_0}\right)\\
&a^\dagger_{SHO}=\sqrt{\frac{m\omega}{2\hbar}}\left(x-\frac{jp}{m\omega}\right)\quad\Lrw\quad 
\M{A}^{\dagger}=\sqrt{\frac{LC^2\omega_0}{2\hbar}}\left(\M{Q}-\frac{j\M{P}}{LC^2\omega_0}\right)
\end{align}
 or
 \begin{align}
 \M{A}&=\sqrt{\frac{C}{2\hbar\omega_0}}\left(\A{v}+\frac{j\A{i}}{C\omega_0}\right)\qquad \text{annihilation operator}\label{P1-111}\\
 \M{A}^{\dagger}&=\sqrt{\frac{C}{2\hbar\omega_0}}\left(\A{v}-\frac{j\A{i}}{C\omega_0}\right)\qquad\text{creation operator}\label{P1-112}
 \end{align}
The Hamiltonian can be rewriten in terms of creation and annihilation operators as:
\begin{equation}
\M{H}=\hbar \omega_0\left(\M{A}^{\dagger}\M{A}+\frac{1}{2}\right)
\end{equation}
Numbering operator is defined as:
\begin{equation}
\M{N}=\M{A}^{\dagger}\M{A}
\end{equation}

\end{homeworkSection}
%-----f-----------
\begin{homeworkSection}{(f)}
from \eqref{P1-111} and \eqref{P1-112}
\begin{align}
\A{v}&=\sqrt{\frac{\hbar\omega_0}{2C}}\left(\M{A}+\M{A}^{\dagger}\right)\\
\A{i}&=-j\sqrt{\frac{\hbar\omega_0}{2L}}\left(\M{A}-\M{A}^{\dagger}\right)
\end{align}

\end{homeworkSection}
%-----g-----
\begin{homeworkSection}{(g)}
Using the results of the quantization of SHO we can write:
\begin{equation}
E_n=\left(n+\frac{1}{2}\right)\hbar\omega_0=\left(n+\frac{1}{2}\right)\hbar\frac{1}{\sqrt{LC}}
\end{equation}
\end{homeworkSection}

%------h------
\begin{homeworkSection}{(h)}
Current and voltage as conjugate dynamical varibles are incompatible. So the following uncertainty relations prohibite the precise measurment of both $v$ and $i$ simultaniously:
\begin{equation}
\langle(\Delta v)^2\rangle\langle (\Delta i)^2\rangle\geq\frac{1}{4}\left|\langle\left[\A{v},\A{i}\right]\rangle\right|^2=\frac{\hbar^2}{4L^2C^2}
\end{equation}
Paricularly uncertainty in nth energy eigenstate is:
\begin{align}
&\langle(\Delta v)^2\rangle_n=\bra{n} v^2\ket{n} -\bra{n} v\ket{n}^2=\frac{\hbar\omega_0}{2C}\bra{n}\M{A}^2+\M{A}^{\dagger}^2+\M{A}\M{A}^{\dagger}+\M{A}^{\dagger}\M{A}\ket{n}-\frac{\hbar\omega_0}{2C}\bra{n}\M{A}+\M{A}^{\dagger}\ket{n}^2\\
&\langle(\Delta i)^2\rangle_n=\bra{n} i^2\ket{n} -\bra{n} i\ket{n}^2=\frac{\hbar\omega_0}{2L}\bra{n}\M{A}^2+\M{A}^{\dagger}^2+\M{A}\M{A}^{\dagger}+\M{A}^{\dagger}\M{A}\ket{n}-\frac{\hbar\omega_0}{2L}\bra{n}\M{A}+\M{A}^{\dagger}\ket{n}^2
\end{align}
Hence:
\begin{equation}\label{P1-220}
\left.
\begin{array}{l}
\langle(\Delta v)^2\rangle_n =\frac{\hbar\omega_0}{2C}(2n+1)\\
\langle(\Delta i)^2\rangle_n =\frac{\hbar\omega_0}{2L}(2n+1)
\end{array}
\right\}\quad\Lrw\quad \langle(\Delta v)^2\rangle_n\langle(\Delta i)^2\rangle_n=\frac{\hbar^2}{4L^2C^2}(2n+1)^2
\end{equation}
Clearly qunatum uncertainty which reprsents quantum mechanical effects is proportional to $\omega_0^4$ this menas that quantum mechanical effects are more important at higher frequencies comparable to $\omega_0$. Dimensional analysis is a direct method to show this limit.  
\end{homeworkSection}

\begin{homeworkSection}{(i)}
To evaluate quantum flactuation level in an eigenstate of energy we use equation \eqref{P1-220}. We have:
\begin{align}
&\sqrt{\langle(\Delta v)^2\rangle_n}=\sqrt{\frac{\hbar\omega_0}{2C}(2n+1)}\approx 1.28 \mathrm{\mu V}\sqrt{2n+1}\\
&\sqrt{\langle(\Delta i)^2\rangle_n}=\sqrt{\frac{\hbar\omega_0}{2L}(2n+1)}\approx 40.7 \mathrm{nA}\sqrt{2n+1}
\end{align}
\end{homeworkSection}
%-----j----------------------------
\begin{homeworkSection}{(j)}
We assume that quanta of thermal radiation is $k_BT$. $k_B$ is Boltzman's constant. To measure the the quantized energy levels we should conduct our experiment in an enviornment in which termal energy quanta is lower than the half of the first energy level:
\begin{equation}
k_BT<\frac{\hbar\omega_0}{2}\quad\Lrw\quad T<0.12\mathrm{K}
\end{equation} 
%--------
\end{homeworkSection}
%----k----
\begin{homeworkSection}{(k)}
Assume that the voltage wave function is reptresented by $\Psi(v,t)$. since $\M{Q}$ and $\M{P}$ commutation relation is the same as $x-p$ commutation relation we can use the same math. So we can just replace $\M{P}\leftrightarrow -j\hbar\fpds{\M{Q}}$. Hamiltonian is also generator of time translation so we can write:
\begin{equation}
j\hbar\pd{\Psi(v,t)}{t}=-\frac{\omega_0^2\hbar^2}{2C}\pdt{\Psi(v,t)}{v}+\frac{1}{2}Cv^2\Psi(v,t)
\end{equation}
\end{homeworkSection}
%-----l------
\begin{homeworkSection}{(l)}
The voltage wavefunction associated with the ground state of the LC network is similar to x-space wave function of SHO. we should just use \eqref{P1-Sub}. Actually we have:
\begin{equation}
\M{A}\ket{0}=0\quad\Lrw\quad v\Psi_0(v)+\frac{\hbar}{LC^2\omega_0}\pd{\Psi_{0}(v)}{v}=0
\end{equation}
And explicitly:
\begin{equation}
\Psi_0(v)=\left({\frac{LC^2\omega_0}{\pi\hbar}}\right)^{1/4}\exp\left[-\frac{1}{2}\frac{LC^2\omega_0 v^2}{\hbar}\right]
\end{equation}

\end{homworkSection}


\end{homeworkProblem}

%%%\setcounter{equation}{0}
%---------------------
%%%\begin{homeworkProblem}
\begin{homeworkSection}{(a)}
Strating with
\begin{equation*}
[\A{a},\A{a}^\dagger]=1
\end{equation*}
we arrive at:
\begin{equation}\label{P2-1}
[\A{N},\A{a}^\dagger]=[\A{a}^\dagger\A{a},\A{a}^\dagger]=\A{a}^\dagger[\A{a},\A{a}^\dagger]+[\A{a}^\dagger,\A{a}^\dagger]\A{a}=\A{a}^\dagger
\end{equation}
Note that in \eqref{P2-1} we have used the following identity:
\begin{equation}\label{P2-i}
[\A{A}\A{B},\A{C}]=\A{A}[\A{B},\A{C}]+[\A{A},\A{C}]\A{B}
\end{equation}
In the same line of reasoning we get:
\begin{equation}
[\A{N},\A{a}]=[\A{a}^\dagger\A{a},\A{a}]=\A{a}^\dagger[\A{a},\A{a}]+[\A{a}^\dagger,\A{a}]\A{a}=-\A{a}
\end{equation}
\end{homeworkSection}
\begin{homeworkSection}{(b)}
In the prsence of nonlinearity the Hamiltonian is:
\begin{equation}
\A{H}=\hbar\omega_0\A{a}^\dagger\A{a}-\frac{\hbar\kappa}{2}\A{a}^\dagger\A{a}^\dagger\A{a}\A{a}
\end{equation}
hence:
\begin{equation}
[\A{N},\A{H}]=[\A{a}^\dagger\A{a},\hbar\omega_0\A{a}^\dagger\A{a}-\frac{\hbar\kappa}{2}\A{a}^\dagger\A{a}^\dagger\A{a}\A{a}]=
-\frac{\hbar\kappa}{2}[\A{N},\A{a}^\dagger\A{a}^\dagger\A{a}\A{a}]
\end{equation}
Successive application of \eqref{P2-i} we can write:
\begin{align}
[\A{N},\A{a}^\dagger\A{a}^\dagger\A{a}\A{a}] &=\A{a}^\dagger [\A{N},\A{a}^\dagger\A{a}\A{a}]+[\A{N},\A{a}^\dagger]\A{a}^\dagger\A{a}\A{a}\\
[\A{N},\A{a}^\dagger\A{a}\A{a}]  &=  \A{a}^\dagger[\A{N},\A{a}\A{a}]+[\A{N},\A{a}^\dagger]\A{a}\A{a}\\
[\A{N},\A{a}\A{a}] &=\A{a}[\A{N},\A{a}]+[\A{N},\A{a}]\A{a}
\end{align} 
So we have:
\begin{equation}
[\A{N},\A{a}^\dagger\A{a}^\dagger\A{a}\A{a}]=-2\A{a}^\dagger\A{a}^\dagger\A{a}\A{a}+2\A{a}^\dagger\A{a}^\dagger\A{a}\A{a}=0
\end{equation}
combining all we get:
\begin{equation}\label{P2-NH}
[\A{N},\A{H}]=0
\end{equation}
We can employ a simpler analysis as well. If we write the Hamiltonian in the following form:
\begin{equation*}
\A{H}=\hbar\omega_0\A{N}-\frac{\hbar\kappa}{2}\A{a}^\dagger\A{N}\A{a}
\end{equation*}
using \eqref{P2-1} we can write:
\begin{equation}\label{P2-500}
\A{H}=\hbar\omega_0\A{N}-\frac{\hbar\kappa}{2}\left(\A{N}\A{a}^\dagger-\A{a}^\dagger\right)\A{a}=
\hbar\omega_0\A{N}-\frac{\hbar\kappa}{2}\left(\A{N}^2-\A{N}\right)
\end{equation}
We can readily see that the Hamiltonian is a function number operator and equation \eqref{P2-NH} holds. 
\end{homeworkSection}
%----------------------c---------
\begin{homeworkSection}{(c)}
Since $\A{N}$ and $\A{H}$ commutate , they can be simultaneously diagonalize. Eigenstates of the number operator are not degenerate so we can use number operator basis function as the eigenstates of the Hamiltonian:
\begin{equation}
\A{H}\ket{n}=\hbar\omega_0 n\ket{n}-\frac{\hbar\kappa}{2}\A{a}^\dagger\A{a}^\dagger\A{a}\A{a}\ket{n}
\end{equation} 
Using
\begin{equation*}
\A{a}\ket{n}=\sqrt{n}\ket{n-1}\qquad\A{a}^\dagger\ket{n}=\sqrt{n+1}\ket{n+1}
\end{equation*}
we obtain:
\begin{equation}
\A{a}^\dagger\A{a}^\dagger\A{a}\A{a}\ket{n}=\sqrt{n(n-1)(n-1)n}\ket{n}=n(n-1)\ket{n}
\end{equation}
So we have:
\begin{equation}
\A{H}\ket{n}=\left\{\hbar\omega_0 n-\frac{\hbar\kappa}{2}n(n-1)\right\}\ket{n}
\end{equation}
we could also use \eqref{P2-500} which leads to the same result:
\begin{equation}\label{P2-550}
\A{H}\ket{n}=\hbar\omega_0\A{N}\ket{n}-\frac{\hbar\kappa}{2}\left(\A{N}^2-\A{N}\right)\ket{n}=
\left\{\hbar\omega_0 n-\frac{\hbar\kappa}{2}n(n-1)\right\}\ket{n}
\end{equation}
\end{homeworkSection}
%----------d-----------
\begin{homeworkSection}{(d)}
We have already proved that $\A{N}$ and $\A{H}$ commute. So we can immediately conclude that the number operator is a constant of motion. The Hesenberg time evolution for number operator is:
\begin{equation}
\frac{d\A{N}}{dt}=\frac{1}{i\hbar}[\A{N},\A{H}]=0\quad\Lrw\quad \A{N}=\A{N}(0)
\end{equation}
Since $\A{N}$ is a constant motion the number photons inside the cavity remains constant during time evolution.
\end{homeworkSection}
\begin{homeworkSection}{(e)}
The Heisenberg time evolution equation for the annihilation operator is:
\begin{equation}
\frac{d\A{a}}{dt}=\frac{1}{i\hbar}[\A{a},\A{H}]=-i[\A{a},\omega\A{N}-\frac{\kappa}{2}\A{a}^\dagger\A{a}^\dagger\A{a}\A{a}]=-i\omega_0\A{a}+\frac{i\kappa}{2}[\A{a},\A{a}^\dagger\A{a}^\dagger\A{a}\A{a}]
\end{equation}
Using \eqref{P2-i} identity we obtain:
\begin{equation}
[\A{a},\A{a}^\dagger\A{a}^\dagger\A{a}\A{a}]=\A{a}^\dagger\A{a}^\dagger[\A{a},\A{a}\A{a}]+[\A{a},\A{a}^\dagger\A{a}^\dagger]\A{a}\A{a}=2\A{a}^\dagger\A{a}\A{a}=2\A{N}\A{a}
\end{equation}
Since $\A{N}$ is a constant of motion we can write:
\begin{equation}\label{P2-a}
\frac{d\A{a}(t)}{dt}=-i\omega_0\A{a}(t)+i\kappa\A{N}(0)\A{a}(t)
\end{equation} 
To solve this first order differential equation we define am auxiliary operator:
\begin{equation}
\A{\zeta}(t)=\A{a}(t)\exp(i\omega_0 t)
\end{equation}
Inserting in \eqref{P2-a} we obtain:
\begin{equation}
\frac{d\A{\zeta}(t)}{dt}=i\kappa \A{N}(0)\A{\zeta}(t)
\end{equation}
So we arrive at:
\begin{equation}
\A{\zeta}(t)=e^{i\kappa\A{N}(0)t}\A{\zeta(0)}
\end{equation}
finally we get:
\begin{equation}\label{P2-at}
\A{a}(t)=\exp\left[-i\omega_0t+i\kappa\A{N}(0)t\right]\A{a}(0)
\end{equation}
\end{homeworkSection}
%-----------------f----------------------------
\begin{homeworkSection}{(f)}
Applying transpose conjugate operator on both sides of \eqref{P2-at} we obtain:
\begin{equation}
\A{a}^\dagger(t)=\A{a}^\dagger(0)\exp\left[i\omega_0t-i\kappa\A{N}^\dagger(0)t\right]=\A{a}^\dagger(0)\exp\left[i\omega_0t-i\kappa\A{N}(0)t\right]
\end{equation}
From operator algebra we know that :
\begin{equation}
[A,B]=\lambda A\quad\Lrw\quad Ae^{B}=e^\lambda e^B A
\end{equation}
Since $[\A{a}^\dagger(0),\A{N}(0)]=-\A{a}^\dagger(0)$ then:
\begin{equation}
\A{a}^\dagger(t)=\exp\left[i\kappa t+i\omega_0t-i\kappa\A{N}(0)t\right]\A{a}^\dagger(0)
\end{equation}
\end{homeworkSection}
%----------------------g------------
\begin{homeworkSection}{(g)}
If just one photon is lost from the cavity total energy of the system inside the cacity would change. We can simply calculate the energy difference  in the number state. Before $t=T$ the quantum state of the field is given by $\ket{n}$ and  by photon annihilation the state in the nuumber state would be $\ket{n-1}$ so we have:  
\begin{equation}
\hbar\omega_{ph}=\Delta E=E_1-E_2=\bra{n}\A{H}\ket{n}-\bra{n-1}\A{H}\ket{n-1}
\end{equation}
Using \eqref{P2-550} we can write:
\begin{eqnarray}
\Delta E&=&\left\{\hbar\omega_0 n-\frac{\hbar\kappa}{2}n(n-1)\right\}-\left\{\hbar\omega_0 (n-1)-\frac{\hbar\kappa}{2}(n-1)(n-2)\right\}\nonumber\\
&=&\hbar\omega_0-\hbar\kappa(n-1)
\end{eqnarray}
So the spectrometer measures:
\begin{equation}
\omega_{ph}=\frac{\Delta E}{\hbar}=\omega_0-\kappa(n-1)
\end{equation}
Note that we can measure the frequency with certainty.
\end{homeworkSection}


\end{homeworkProblem}
%%%\setcounter{equation}{0}
%--------------------- 
%%%\begin{homeworkProblem}
The spinor part of the Hamiltonian describing time evolution of  an half spin system such as electron inside a uniform but time-varying magnetic filed is:
\begin{equation}
H=-\B{\mu}.\vp{B}=-\frac{gq}{2m}\vp{S}.\vp{B}=-\mu_0\B{\sigma}.\vp{B}
\end{equation}
 where $g$ is gyromagnetic ration and $\mu_0$ is defined as:
 \begin{equation}
 \mu_0=\frac{qg}{2m}
 \end{equation}
Explicitly the Hamiltonian is:
\begin{equation}
H=-\mu_0B_0
\begin{pmatrix}
\cos\theta & \sin\theta\left[\cos\omega t-i\sin\omega t\right]\\
\sin\theta\left[\cos\omega t+i\sin\omega t\right] & -\cos\theta
\end{pmatrix}
\end{equation} 
For notational convenience we write the Hamiltonian in terms of two new parameters:
\begin{equation}
H=-\hbar
\begin{pmatrix}
\omega_1              & \gamma e^{-i\omega t}\\
\gamma e^{i\omega t} & -\omega_1 
\end{pmatrix}
\end{equation}
where:
\begin{align*}
\hbar\omega_1 &=\mu_0B_0\cos\theta\\
 \hbar\gamma &=\mu_0B_0\sin\theta
\end{align*}
The Schr\"odinger equation now reads:
\begin{equation}\label{P3-S}
i\hbar\frac{d}{dt}
\begin{pmatrix}
c_1(t)\\
c_2(t)
\end{pmatrix}
=
-\hbar\begin{pmatrix}
\omega_1              & \gamma e^{-i\omega t}\\
\gamma e^{i\omega t} & -\omega_1 
\end{pmatrix}
\begin{pmatrix}
c_1(t)\\
c_2(t)
\end{pmatrix}
\end{equation}
In the absense of the time dependent part of the Hamiltonian the solution is simply $c_1(t)=A\exp(i\omega_1 t)$ and $c_2(t)=B\exp(-i\omega_1 t)$. It's reasonable to keep the exponential part for the new Hamiltonian. Two coupled first order differential equations given in \eqref{P3-S} can be extremely simplified by the following change of varibles:
\begin{equation}
c_1(t)=A(t)\exp(i\omega_1 t)\qquad  c_2(t)=B(t)\exp(-i\omega_1 t)
\end{equation}   
Two new coupled equations are:  
\begin{equation}
\left\{
\begin{array}{l}
i\frac{dA(t)}{dt}=-\gamma \exp\left[-i(\omega +2\omega_1)t\right]B(t)\\
i\frac{dB(t)}{dt}=-\gamma \exp\left[+i(\omega +2\omega_1)t\right] A(t)
\end{array}\right.
\end{equation}
If we solve $B(t)$ from the first equation and put it in the second one, we obtain:
\begin{equation}
\frac{d^2A}{dt^2}+i\alpha \frac{dA}{dt}+\gamma^2 A=0
\end{equation}
where $\alpha$ has been defined for notational convenience:
\begin{equation}
\alpha=\omega+2\omega_1
\end{equation}
because of symmetry the same equation holds for $B(t)$ we should change the sign of $\alpha$:
\begin{equation}
\frac{d^2B}{dt^2}-i\alpha \frac{dB}{dt}+\gamma^2 B=0
\end{equation}  
The general solution for two differential equations are simply:
\begin{align}
&A(t)=A_{+} \exp(i\Omega_+t)+A_-\exp(i\Omega_-t)\\
&B(t)=B_+\exp(-i\Omega_+t)+B_-\exp(-i\Omega_-t)
\end{align}
where $\Omega_+$ and $\Omega_-$ are defined as:
\begin{equation}
\Omega_\pm=-\frac{\alpha}{2}\pm\sqrt{\frac{\alpha^2}{4}+\gamma^2}
\end{equation}
The constant coefficients can be writen in terms of $A(0)$ and $B(0)$ :
\begin{equation}
\left\{
\begin{array}{rcl}
A(0) &=&A_++A_- &\\
B(0) &=&B_++B_- &\\
\left.\frac{dA}{dt}\right|_{t=0}=i\gamma B(0)&=&+i\Omega_+A_++i\Omega_-A_-\\
\left.\frac{dB}{dt}\right|_{t=0}=i\gamma A(0)&=&-i\Omega_+B_+-i\Omega_-B_-
\end{array}\right.
\end{equation}
These eqquation lead to:
\begin{align}
A_+&=\frac{\Omega_+ A(0)-\gamma B(0)}{\Omega_+-\Omega_-}\\
A_-&=-\frac{\Omega_+ A(0)-\gamma B(0)}{\Omega_+-\Omega_-}\\
B_+&=-\frac{\Omega_- B(0)+\gamma A(0)}{\Omega_+-\Omega_-}\\
B_-&=\frac{\Omega_+ B(0)+\gamma a(0)}{\Omega_+-\Omega_-}
\end{align}
This problem is a general spin procession problem which shows that a time varying magnetic filed which rotates along a prescribed axis changes the spin state by two diferent frequency components. 
\end{homeworkProblem}
%%%\setcounter{equation}{0}
%----------------------
%%% ... %%%
%%%\input{answern}
%%%\setcounter{equation}{0}

%%%\newpage
%%%%-------------------------------------------
\begin{thebibliography}{9}

\bibitem{sakurai}
  J.~J.Sakurai,
  \emph{Modern Quantum Mechanics}.
  Addison Wesley, Massachusetts,
  Revised Edition.
%------------------------------------
\bibitem{greiner-relativistic_QM}
  W.~Gereiner,
  \emph{Relativistic Quantum Mechanics}.
  Springer,
  Third Edition,
  2000.
%--------------------------------------
\bibitem{taflove}
  A.~Taflove and S.C.~Hagness
  \emph{Computational Electrodynamics: The Finite Difference Time Domain Method}.
  Artech House,
  Second Edition, 2000.
  
  %----------------------------
  \bibitem{antonio004}
  A.~Soriano et.al,
  \emph{Analysis of the finite difference time domain technique to solve the Schr\"odinger's equation for quantum devices}
  Journal of App. Phys,
  vol 95,N 12,2004.
  %-------------------------------
  \bibitem{shibata}
  T.~Shibata
  \emph{Absorbing boundary conditions for the finite-difference time-domain calculation of the one-dimensional Schr\"odinger's equation}.
  Phys Rev B,
  vol 43, N 8, 1991.
  %--------------------------------------
\bibitem{kosloff}
 R.~ Kosloff and D.~Kosloff
  \emph{Absorbing boundaries for wave propagating problems}.
  Journal of Computational Phys,
  vol 63, 1986.
  %---------------------------
 \bibitem{majd}
 B.~Engquist and A.~Majd
  \emph{Absorbing boundaries for numerical solutions of the waves}.
  Math of Computation,
  vol 31, N 139, 1977. 


\end{thebibliography}


%----------------------------------------
\end{document}
