\documentclass{article}
\usepackage[]{hyperref}
\usepackage[]{amsmath}
\usepackage[makeroom]{cancel}
\begin{document}
\section{Fluctuation-Dissipation Theorem}
Last time we had discussed the fluctuation-dissipation theorem as it applied to
the problem of 1D particle interaction (Brownian motion). We had discussed two
time scales $ \tau $ (the time between interactions of particles) and $ \tau^{*}
$, the time scale over which the applied force is relevant. We had the fact that
$ \tau \gg \tau^{*} $.

We can cast the problem of 1D particle motion in the general way using the
differential equation
\[
   m \frac{dv}{dt} = \mathcal{F}(t) + F(t) = \mathcal{F}(t)t- \alpha v(t) + F'(t).
\]
This is the oft-cited Langevin equation. We can simplify this expression, such
as to make it tractable, in the following way: Assume $ \mathcal{F}(t) = 0 $, $
\langle v \rangle_{0} = 0 $ and $ v(t) \ne \langle v \rangle_{0} $. By
integrating the above expression from time $ t $ to time $ t + \tau $ (and
considering the slowly varying parts as constant over the integral) we have
\[
   m \left( v(t+\tau)-v(t) \right) = \mathcal{F}(t)\tau + \int_{t}^{t+\tau}dt'
   F(t'),
\]
where $ F(t') = - \alpha v(t) + F'(t) $. We can take an ensemble average
(\textbf{Why is this allowed?}), to obtain, finally:
\[
   m \langle v(t+\tau) - v(t) \rangle = \mathcal{F}(t) \tau +
   \int_{t}^{t+\tau}dt' \langle F(t') \rangle
\]
By considering the work-energy theorem and the way in which a unit of work
changes the energy of the system, we can determine the probability function of
the system.  $ W_{r}(t) $ is the probability function of the system at the rth
state. Through some analysis, we can write that
\[
   \frac{W_{r}(t+\tau)}{W_{r}^{0}} = e^{\beta \Delta E}.
\]
Now, we can write
\begin{align*}
   \langle F \rangle &= \sum_{r} P_{r}(t+\tau)F_{r}(t+\tau) \\
   &= \sum^{}_{r} P_{r}^{0}e^{\beta \Delta E}F_{r}
   \intertext{Then, through a Taylor expansion:}
   &\approx \sum^{}_{r} P_{r}^{0}\left( 1+\beta \Delta E \right)F_{r} \\
   &= \sum^{}_{r} P_{r}^{0} F_{r} + \beta \sum^{}_{r} P_{r}^{0} \Delta E F_{r} \\
   &= \cancelto{0}{\langle F \rangle_{0}} + \langle \beta  \Delta E F_{r}\rangle_{0} \\
   &= \langle \beta \Delta E F_{r} \rangle_{0}
\end{align*}
Now, if we want to discuss the change in energy, we can write
\begin{align*}
   \Delta E &= - \int_{t}^{t'} F(t'') dx(t'') \\
   &= - \int_{t}^{t'}F(t'') v(t'') dt'' \\
   \langle \Delta E F_{r} \rangle &= - \Bigl[\int_{t}^{t'}
F(t'')v(t'')dt''\Bigr] F(t') \\
&= - \langle v(t) \rangle \int_{t}^{t'} \langle F(t') F(t'') \rangle dt'' \\
m \bigl(\langle v(t+\tau) - v(t) \rangle\bigr) &= -\beta \langle v(t) \rangle
\int_{t}^{t+\tau} dt' \int_{t}^{t'} \langle F(t')F(t'') dt'' \rangle +
\mathcal{F}(t)\tau
\intertext{Then, we identify a bunch of the first part of the expression as $
\alpha = \beta \int_{t}^{t'}\langle F(t') F(t'') \rangle dt'' $. Then, we can
write}
&= -\alpha \langle v(t) \rangle \tau + \mathcal{F}(t)\tau
\end{align*}

\section{Johnson-Nyquist Noise (Thermal Noise)}
\begin{enumerate}
   \item This noise was studied in the early 1900s at Siemens by Walter Schottky
(1918 Ann. d. Phys 57, 541, ``tube noise'' in vacuum tube amplifiers).
\item It was studied later by J.B. Johnson in 1927 (``Thermal agitation of
   electricity in conductors'', Nature 119.50).
\item 1928 Phys. Rev. 32, 97
\item 1928 H. Nyquist ``Thermal agitation of electric charge in conductors''
\end{enumerate}
Imagine two resistors, $ R_{1} $ and $ R_{2} $ connected by an ideal wire. In
the macroscopic theory of thermal noise we can use the second law of
thermodynamics, to relate the voltage across one resistor to that across the
other resistor. The energy flow into $ R_{1} $ is the energy generated by $
R_{2} $. The opposite is true $ R_{2} $. The energy flow into $ R_{2} $ can be
expressed as
\[
   \frac{R_{2}}{\left( R_{1} + R_{2} \right)^2} = \langle v_{1}^{2} \rangle.
\]
Similarly, the power flow into $ R_{2} $ can be written as
\[
   \frac{R_{1}}{\left( R_{1} + R_{2} \right)^2} = \langle v_{2}^{2} \rangle.
\]
If $ R_{1} = R_{2} = R $ and $ \Theta $ is fixed for both resistors then we can
write
\[
   \langle v_{1}^{2}\rangle = \langle v_{2}^{2}\rangle
\]
The power spectral density $ S_{v}(\omega) $ of the voltage fluctuations:
\begin{itemize}
   \item should be independent of the detailed structure of the resistor material.
   \item should depend of $ R,\Theta, \omega $
\end{itemize}
In the case $ R_{1} \ne R_{2} $
\begin{align*}
   \frac{R_{2}}{\left( R_{1} + R_{2} \right)^2}\langle v_{1}^{2} \rangle &=
   \frac{R_{1}}{\left( R_{1} + R_{2} \right)^2}\langle v_{2}^{2} \rangle \\
   \frac{\langle v_{1}^{2}\rangle}{\langle v_{2}^{2}\rangle} =
   \frac{R_{1}}{R_{2}}.
\end{align*}
That is, we realize that the power spectral density should be proportional to $
R $. In the case that the resistors are at different temperatures $ \Theta_{1}
\ne \Theta_{2} $ the energy flow will be proportional to the difference in
temperatures between the two resistors. Mathematically, in this case, $
S_{v_{1}}(\omega) - S_{v_{2}}(\omega) \propto \Theta $.

\subsection{Thermal noise of an LCR circuit}
Imagine an LCR circuit in series with a source. We can write the energy stored
in the inductor as
\[
   \frac{1}{2} L \langle I^{2} \rangle = \frac{1}{2} L \int_{0}^{\infty} \langle
   S_{I}(\omega) \rangle \frac{d\omega}{2 \pi}
\]
By Parseval's theorem we can write
\[
   \langle S_{I}(\omega) \rangle =\frac{ \langle S_{v}(\omega) \rangle}{R^2 +
   \left( \omega L - \frac{1}{\omega C} \right)^{2}}.
\]
From this, we have
\[
   \int_{0}^{\infty} \langle S_{I}(\omega) \rangle \frac{d\omega}{2 \pi} =
   \int_{0}^{\infty} \frac{\langle S_{v}(\omega) \rangle}{R^{2} + \left( \omega
   L - \frac{1}{\omega C}\right)^{2}} \frac{d\omega}{2 \pi} \rightarrow
   \frac{1}{4 L^{2}} \frac{1}{\frac{R}{L}} = \frac{1}{4 R L }.
\]
By the equipartition theorem we can write
\[
 \frac{1}{2}L \langle I^{2} \rangle =
\frac{1}{2} L \frac{\langle S_{v}(\omega) \rangle}{4 R L} =
\frac{1}{2}k_{B}\Theta.
\]
Thus, at the end, we can write $ S_{v}(\omega) = 4 k_{B}\Theta R\big|_{\omega=0}
$.

\subsection{Microscopic Theory of Particle Interactions}
Consider electron motion when electrons are subject to some external E field.
\[
   a = \frac{q E}{m}
\]

and
\[
   v(t) = a t.
\]
Considering K collisions, we can write that the velocity as a function of time
is:
\[
   \langle v(t) \rangle = \frac{\text{Total displacement}}{\text{Total time}} =
   \frac{\frac{1}{2} \frac{q E}{m} \langle \tau_{f}^{2} \rangle K}{\langle
   \tau_{f} \rangle K} = \frac{q \langle \tau_{f}^{2}\rangle}{2 m \langle
\tau_{f} \rangle} E \equiv \mu E .
\]

Consider the following expression
\[
   m \frac{dv(t)}{dt} = -\frac{m}{\langle \tau_{f}\rangle} v(t) + F'(t).
\]

\end{document}
