\begin{homeworkProblem}[Problem 1: Wiener-Khintchine Theorem]
   \problemStatement{
      The average power of a noisy function $ x_{T}(t) $ is defined by
      \[
         \lim_{T \to \infty} \frac{1}{T} \int_{-\infty}^{\infty} \left[ x_{T}(t)
         \right]^2 dt = \lim_{T \to \infty} \frac{1}{2\pi} \int_{0}^{\infty}
         \frac{2 |X_{T}(i\omega)|^2}{T} d\omega
      \]
      where $ x_{T}(t) $ a gated function is defined by
      \[
         x_{T}(t) = \begin{cases}
            x(t), &\text{ if } |t| < T/2; \\
            0 &\text{ if otherwise.}
         \end{cases}
      \]
   $ T $ is a measurement time interval and $ X_{T}(i\omega) $ is the Fourier
   transform of $ x_{T}(t) $.
   }
   % Problem 1.1
   \subsection{Stationary Process}
   \label{sub:stationary_process}
   \problemStatement{
      If $ x_{T}(t) $ is a statistically stationary process, show that the average
      power of a noisy function is independent of $ T $ and is a constant universal
      quantity.
   }
   I don't have a good answer for this question. I have thought a lot about it.
   Ultimately, the answer to this question boils down to the following logic: If
   $ x_{T}(t) $ is a stationary process, then, by definition, the process has
   the property that all statistical quantities related to this process are time
   independent. So, by virtue of being stationary, the process satisfies the
   requirement that the average power of a noisy function, as defined above, is
   independent of $ T $ and is a constant universal quantity.
   % Problem 1.2
   \subsection{Nonstationary Process}
   \label{sub:nonstationary_process}
   \problemStatement{
      If $ x_{T}(t) $ is as statistically stationary process, show that the average
      power is dependent on $ T $.

      For the statistically nonstationary process, we are not allowed to take the
      limit of $ T \to \infty $. In this case, we introduce ensemble averaging which is taken
      first for many identical gated functions $ x_{T}(t) $. Then, the order of $
      \lim_{T \to \infty} $ and $ \int_{0}^{\infty} d\omega $ can
      be interchanged. Now, we can define the unilateral power spectral density $
      S_{x}(\omega) $
      \[
         S_{x}(\omega) = \lim_{T \to \infty} \frac{2 \langle |X_{T}(i\omega)|^2
         \rangle}{T}
      \]
   }
   My answer to this question is very similar to the previous question. To be a
   statistically non-stationary process, it means that we can not take the limit
   of statistical quantities as $ T \to \infty $. This procedure of taking
   the limit doesn't make sense for non-stationary processes. The limit doesn't
   exist because the statistical quantities related to the stochastic process
   change as a function of $ t $. Thus, the average power is dependent on $ T $.
   \clearpage
   % Problem 1.3
   \subsection{Autocorrelation in Terms of Spectral Density}
   \label{sub:autocorrelation_in_terms_of_spectral_density}
   \problemStatement{
   Recall the formula in Problem Set 1, 3(2)
   \begin{equation}
      \int_{-\infty}^{\infty}x_{T}(t+\tau)x_{T}(t) dt =
      \frac{1}{2\pi} \int_{-\infty}^{\infty} |X_{T}(i\omega)|^2 \exp(i \omega
      \tau) d \omega \label{eq:autocorr_spectrum}
   \end{equation}
   Suppose $ \tau \ne 0 $. One can also divide both sides of
   Eq.~\ref{eq:autocorr_spectrum} by $ T $, take an ensemble average, and take a
   limit of $ T \to \infty $. Show your steps to reach the following relation,
   \begin{equation}
      \lim_{T \to \infty} \frac{1}{T} \int_{-\infty}^{\infty}
      \langle x_{T}(t+\tau)x_{T}(t) \rangle dt =
      \lim_{T \to \infty} \frac{1}{2\pi} \int_{-\infty}^{\infty}
       |X_{T}(i\omega)|^{2} \cos(\omega \tau) d\omega \label{eq:autocorr_limit}
   \end{equation}
   }
   Let's do as the problem statement indicates and, first, divide the left hand
   side of Eq.~\ref{eq:autocorr_spectrum} by $ T $ and then take the limit as $
   T \to \infty$. Doing so, yields
   \begin{align*}
      \lim_{T \to \infty} \frac{1}{T} \int_{-\infty}^{\infty}
      \langle x_{T}(t+\tau)x_{T}(t) \rangle dt &=
      \lim_{T \to \infty} \frac{1}{2\pi T} \int_{-\infty}^{\infty}
       |X_{T}(i\omega)|^{2} \exp(i \omega \tau) d\omega \label{eq:autocorr_limit}
       \intertext{Now, this can be split into two terms.}
       &= \lim_{T \to \infty} \frac{1}{2\pi T} \int_{-\infty}^{\infty}
       |X_{T}(i\omega)|^{2} \cos(\omega \tau) d\omega
       + i \lim_{T \to \infty} \frac{1}{2\pi T} \int_{-\infty}^{\infty}
       |X_{T}(i\omega)|^{2} \sin(\omega \tau) d\omega
       \intertext{Now, in general, the second integral can not be said to go to
       zero. However, if the process is purely real ($ x_{T}(t) \in \mathds{R}
       ,\enskip \forall t$) then the imaginary part is identically zero.
       Additionally, if the bounds are chosen to be symmetric about $ \omega $
       then, since $ |X_{T}(i\omega)|^2 $ is even about $ \omega = 0 $ (shown
       in next problem), then that integral can be made to disappear. Thus, we
       can say, for either or both of these two cases:}
       &= \lim_{T \to \infty} \frac{1}{2\pi} \int_{-\infty}^{\infty}
       \frac{|X_{T}(i\omega)|^2}{T} \cos(\omega \tau) d\omega
       \intertext{Note that due to the evenness of $ |X_{T}(i \omega)|^2 $ we
       can re-express the above integral.}
       &= \lim_{T \to \infty} \frac{1}{2\pi} \int_{0}^{\infty}
       \frac{2|X_{T}(i\omega)|^2}{T} \cos(\omega \tau) d\omega
   \end{align*}
   The last step to yield Eq.~\ref{eq:autocorr_limit} would be to take an
   ensemble average of both sides. This would yield:
   \[
      \lim_{T \to \infty} \frac{1}{T} \int_{-\infty}^{\infty}
      \langle x_{T}(t+\tau)x_{T}(t) \rangle dt =
      \lim_{T \to \infty} \frac{1}{2\pi} \int_{-\infty}^{\infty}
       \langle |X_{T}(i\omega)|^{2} \rangle \cos(\omega \tau) d\omega
   \]
   Note that no assumptions needed to be made about the stationarity/ergodicity
   of the process in order to take the ensemble average.
   There is one other way we can prove the above. By breaking the integral up
   into that over negative frequencies and positive frequencies we can change $
   t \to -t$ in the integral over negative frequencies, use the property that $
   |X_{t}(i\omega)|^2$ is an even function of frequency and recombine the
   integrands to negate the sine term. However, this amounts to the same
   argument as given previously, since it assumes symmetric bounds and the
   evenness of $ |X_{T}(i\omega)|^2 $.
   \clearpage
   % Problem 1.4
   \subsection{Spectral Density in Terms of Autocorrelation}
   \label{sub:spectral_density_in_terms_of_autocorrelation}
   \problemStatement{We know that the left-hand side of
      Eq.~\ref{eq:autocorr_limit} is the ensemble averaged autocorrelation
      function $ \phi_{x}(\tau) $. Now we obtain the relation of
      the ensemble averaged autocorrelation $\phi_{x}(\tau)$ and
      the unilateral power spectral density $ S_{x}(\omega) $,
 \begin{equation}
    \phi_{x}(\tau) = \frac{1}{2\pi} \int_{0}^{\infty}S_{x}(\omega)
       \cos(\omega \tau) d\omega \label{eq:unilateral_psd}.
 \end{equation}
 Show that the inverse relation of Eq.~\ref{eq:unilateral_psd} is written as
 \begin{equation}
    S_{x}(\omega) = 4 \int_{0}^{\infty}  \phi_{x}(\tau)
    \cos(\omega \tau) d\tau \label{eq:autocorr}.
 \end{equation}
 Equations \ref{eq:unilateral_psd} and \ref{eq:autocorr} are known as the
 Wiener-Khintchine theorem.}
The solution to this problem is similar to that of a Fourier transform. To
establish the desired equality I will integrate the autocorrelation function
over all non-negative frequencies.
\begin{align*}
\int_{0}^{\infty}  \phi_{x}(\tau)  \cos(\omega \tau) d\tau
&= \frac{1}{2\pi} \int_{0}^{\infty} \left( \int_{0}^{\infty}S_{x}(\omega')
\cos(\omega' \tau) d\omega' \right) \cos(\omega \tau) d \tau
\intertext{Assuming $ S_{x}(\omega') $ is mathematically well-behaved, I
exchange the order of integration.}
&= \frac{1}{2\pi} \int_{0}^{\infty} S_{x}(\omega') d\omega'\left( \int_{0}^{\infty}
\cos(\omega' \tau) \cos(\omega \tau) d \tau \right)
\intertext{As shown in the appendix, the integral over $ \tau $ results in
$\frac{\pi}{2} \delta(\omega'-\omega) + \frac{\pi}{2}\delta(\omega+\omega')$.}
&= \frac{1}{4} \int_{0}^{\infty} S_{x}(\omega')\left(
\delta(\omega'-\omega) + \delta(\omega+\omega') \right) d\omega'
\intertext{Now, because we are only integrating over non-negative frequencies,
one of the delta distributions will not be integrated over. The other one will
be integrated over. Thus, only one contributes. Without loss of generality,
assume that $ w > 0 $. Then, the answer to the integral is}
&=\frac{S_{x}(\omega)}{4}
\intertext{Note, that in the case that $ \omega < 0 $, the answer would be}
&= \frac{S_{x}(-\omega)}{4}
\end{align*}
But, by construction, $ S_{x}(\omega) $ is an even function of $
\omega$. Observe the following for proof that $ S_{x}(\omega) $ is even in $
\omega $ about $ \omega = 0 $.
\begin{align*}
S_{x}(\omega) &=
   \lim_{T \to \infty} \frac{2 \langle |X_{T}(i\omega)|^2 \rangle}{T} = \lim_{T
   \to \infty} \frac{2 \langle |X_{T}(i\omega)|^2 \rangle}{T} \\
   &= \lim_{T \to \infty}\frac{2 \langle X_{T}(i\omega) X_{T}(- i \omega)
   \rangle }{T}
\end{align*}
   Since this function has the property that $ f(-a) = f(a) $ then it is even.
   So, $ S_{x}(\omega) = S_{x}(-\omega) $ and the result is that
\[
    S_{x}(\omega) = 4 \int_{0}^{\infty}  \phi_{x}(\tau)
    \cos(\omega \tau) d\tau
\]
 \end{homeworkProblem}
