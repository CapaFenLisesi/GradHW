% Problem 1.9
\begin{homeworkProblem}[Problem 9]
   \begin{homeworkSection}{a)}
      To show this, we'll determine the matrix elements of $ E $ in the number
      basis.
      \begin{align}
         E &\equiv (n+1)^{-1/2}a \label{eq:Ecreation}\\
         E_{i,j} &= \bra{i} (n+1)^{-1/2} a \ket{j} \\
                 &= \bra{i} (n+1)^{-1/2} \sqrt{j} \ket{j-1}
         \intertext{By expanding $ (n+1)^{-1/2} $ into a power series of $ n $ it can
         be shown that $ (\hat{n}+1)^{-1/2}\ket{m} = (m+1)^{-1/2} \ket{m} $.}
         &= \bra{i} \frac{\sqrt{j}}{\sqrt{j}} \ket{j-1} \\
         &= \bra{i} \ket{j-1}
      \end{align}
      So, the $ i $,$ j $th element is only non-zero for those values of $ j = i+1 $.
      The eigenvalue associated with the $ \ket{n} $ Fock state of the $ E $ operator
      is 1 for all $ n $. So, this operator can be written as
      \[
         E = \sum_{n=0}^{\infty} \ket{n}\bra{n+1}
      \]
      since this operator has the same action on the (complete) basis of Fock states
      as that given in Eq.~\ref{eq:Ecreation}.
   \end{homeworkSection}
   \begin{homeworkSection}{b)}
      \begin{align}
         E \ket{\phi} &= \sum_{n=0}^{\infty} \ket{n}\bra{n+1} \sum_{m}^{\infty}
         e^{i m \phi} \ket{m} \\
         &= \sum_{n=0}^{\infty} e^{i (n+1) \phi} \ket{n} \\
         &= e^{i \phi} \sum_{n=0}^{\infty} e^{i n \phi} \ket{n} \\
         &= e^{i \phi} \ket{\phi}
      \end{align}
      \begin{align}
         \braket{\phi_{1} | \phi_{2}} &= \sum^{\infty}_{m=0} e^{-i m \phi_1}
         \bra{m} \sum^{\infty}_{n=0} e^{i n \phi_{2}} \ket{\phi_{n}} \\
         &= \sum^{\infty}_{m = 0} e^{i m \left( \phi_2 - \phi_1 \right)}
         \braket{m | m} \\
         &= \sum^{\infty}_{m = 0} e^{i m \left( \phi_2 - \phi_1 \right)} \\
      \end{align}
      So, for $ \theta_2 = \theta_1 $ this sum diverges. For $ \theta_2 \ne
      \theta_1 $ this is not zero. So, these states are neither orthogonal nor
      normalizable.
   \end{homeworkSection}
   \begin{homeworkSection}{c)}
      \begin{align}
         P(\phi) &= \frac{1}{2 \pi} \braket{\phi | \rho_{\ket{n}} | \phi} \\
                 &= \frac{1}{2 \pi}
         \left(\sum^{\infty}_{i=0} e^{-i n \phi} \bra{i}\right)
         \ket{n} \bra{n}
         \left(\sum^{\infty}_{j=0} e^{j n \phi} \ket{j}\right) \\
         &= \frac{1}{2 \pi}
      \end{align}
      %The uncertainty of the phase can be calculated as $ \sqrt{\braket{\phi^{2}} -
      %\braket{\phi}^{2}} $.
      %\begin{align}
      %\braket{\phi} &= \braket{n | \phi | n } \\
      %&= \braket{n | \sum^{\infty}_{j=0} \ket{j} \bra{j+1} | n} \\
      %&= 0
      %\end{align}
      %\begin{align}
      %\braket{\phi^{2}} &= \braket{n | \phi^{2} | n } \\
      %\end{align}
      The uncertainty (variance) of the phase can be calculated as $ \left(\int_{0}^{2
            \pi} P(\phi) \phi^{2} d \phi - \left( \int_{0}^{2 \pi} P(\phi) d \phi
      \right)^{2}\right)^{1/2}$. Since $ P(\phi) = \frac{1}{2\pi}$ both integrals are easy
      to calculate.
      \[
         \int_{0}^{2 \pi} P(\phi) \phi^{2} d \phi  =
         \frac{1}{2\pi}\frac{(2\pi)^3}{3} = \frac{(2\pi)^2}{3}
      \]
      and
      \[
         \int_{0}^{2 \pi} P(\phi) \phi d \phi  =
         \frac{1}{2\pi}\frac{(2\pi)^2}{2} = \pi \enskip.
      \]
      So, the variance is
      \[
         \sigma_{\phi}^{2} = \frac{(2\pi)^2}{3} - \pi^2 = \frac{\pi^2}{3} \enskip.
      \]
      This is the same variance as that of a uniform distribution for some random
      variable X uniformly distributed over $ [a,b] $ which is $ \sigma^2_{X}(b-a)^2/12 $.

      This, and the fact that the probability distribution over $ \phi $ has no
      dependence on $ \phi $ indicates that the uncertainty of the phase for a Fock
      state is uniform over the phase (it's completely uncertain).

      To calculate the same uncertainty for the coherent states I can repeat the
      above approach where $ P(\phi) $ is now given by the corresponding expression
      for coherent states.
      %\begin{align}
      %P_{\alpha}(\phi) &= \frac{1}{2 \pi} \braket{\phi | \alpha | \phi}  \\
      %=&
      %\frac{1}{2 \pi}
      %\left(\sum^{\infty}_{n=0} e^{-i n \phi} \bra{n}\right)
      %\left(\sum^{\infty}_{k=0} e^{{-r^2}/2} \frac{r^k e^{i k \theta}}{\sqrt{k!}}
      %\ket{k}\right)
      %\left(\sum^{\infty}_{l=0} e^{{-r^2}/2} \frac{r^l e^{-i l \theta}}{\sqrt{l!}}
      %\bra{l}\right)
      %\left(\sum^{\infty}_{m=0} e^{i m \phi} \ket{m}\right) \\
      %&=
      %\frac{1}{2 \pi}
      %\left(\sum^{\infty}_{n=0} e^{-i n \phi}
      %e^{{-r^2}/2} \frac{r^n e^{i n \theta}}{\sqrt{n!}}
      %\right)
      %\left(\sum^{\infty}_{m=0} e^{{-r^2}/2} \frac{r^m e^{-i m \theta}}{\sqrt{m!}}
      %e^{i m \phi}\right) \\
      %&=
      %\frac{1}{2 \pi}
      %e^{-r^{2}}
      %\left(\sum^{\infty}_{n=0} \frac{r^n e^{i n (\theta- \phi)}}{\sqrt{n!}} \right)
      %\left(\sum^{\infty}_{m=0} \frac{r^m e^{-i m (\theta- \phi)}}{\sqrt{m!}} \right)
      %\intertext{This product can be split into two separate sums, one in which
      %$ n=m $ and another in which $ n\ne m $.}
      %&=
      %\frac{e^{-r^{2}}}{\pi}
      %\left(\sum^{\infty}_{n=0} \cos\left(n\left(\theta-\phi\right)\right) \frac{r^{2n}}{n!} \right) +
      %\frac{e^{-r^{2}}}{2 \pi}
      %\left(\sum^{\infty}_{n=0} \sum^{\infty}_{m=0, m \ne n}
      %\frac{r^{m+n}}{\sqrt{m!n!}}
      %\left( e^{i n \left( \theta- \phi \right)} + e^{-i m \left( \theta- \phi \right)} \right)
      %\right)
      %\end{align}
      \begin{align}
         P_{\alpha}(\phi) &= \frac{1}{2 \pi} \braket{\phi \ket{\alpha} \bra{\alpha}
         \phi} \\
         & \propto | \braket{\alpha | \phi} |^2 \\
         &= | \sum^{\infty}_{n=0}  \braket{\alpha \ket{n} \bra{n} \phi} |^2 \\
         &= | e^{{-r^{2}}/2} \sum^{\infty}_{n=0}
         \frac{r^{n}e^{-i n \theta}}{\sqrt{n!}} e^{i n \phi}|^2 \\
         &=
         \left(
            e^{{-r^{2}}/2} \sum^{\infty}_{n=0}
            \frac{r^{n}e^{-i n \theta}}{\sqrt{n!}} e^{i n \phi}
         \right)
         \left(
            e^{{-r^{2}}/2} \sum^{\infty}_{m=0}
            \frac{r^{m}e^{i m \theta}}{\sqrt{m!}} e^{-i m \phi}
         \right) \\
         &= e^{{-r^2}} \sum^{\infty}_{n=0} \sum^{\infty}_{m=0}
         \frac{r^{n+m} (e^{i n (\phi- \theta)}e^{-i m (\phi- \theta)})}{\sqrt{n!}\sqrt{m!}}
         \intertext{This can be broken into two sums, one in which $ n=m $ and one
         in which $ n \ne m $.}
         &= e^{{-r^2}} \sum^{\infty}_{n=0} \frac{r^{2n} 2 \cos(\theta - \phi) }{n!}
         +  e^{{-r^2}} \sum^{\infty}_{n=0} \sum^{\infty}_{m=0, m \ne n}
         \frac{r^{n+m} (e^{i n (\phi- \theta)}e^{-i m (\phi-
         \theta)})}{\sqrt{n!}\sqrt{m!}} \\
         &= 2 \cos(\theta - \phi)
         +  e^{{-r^2}} \sum^{\infty}_{n=0} \sum^{\infty}_{m=0, m \ne n}
         \frac{r^{n+m} (e^{i n (\phi- \theta)}e^{-i m (\phi- \theta)})}{\sqrt{n!}\sqrt{m!}}
         \intertext{Back-substituting the factor of $ 1/{2 \pi} $:}
         &= \frac{\cos(\theta - \phi)}{\pi}
         +  \frac{e^{{-r^2}}}{2\pi} \sum^{\infty}_{n=0} \sum^{\infty}_{m=0, m \ne n}
         \frac{r^{n+m} (e^{i n (\phi- \theta)}e^{-i m (\phi-
         \theta)})}{\sqrt{n!}\sqrt{m!}} \\
      \end{align}
      The variance associated with the phase is calculated as
      $ \int_{0}^{2\pi} P(\phi) \phi^{2} d\phi - (\int_{0}^{2\pi} P(\phi) \phi
      d\phi)^2 $. Calculating the first term:
      \begin{align}
         &\int_{0}^{2\pi} \frac{\cos(\theta - \phi)}{\pi} \phi^{2} d \phi +
         \frac{e^{{-r^2}}}{2\pi} \sum^{\infty}_{n=0} \sum^{\infty}_{m=0, m \ne n}
         \frac{r^{n+m}}{\sqrt{n!}\sqrt{m!}}
         \int_{0}^{2\pi}
         e^{i (n-m) (\phi- \theta)} \phi^{2} d\phi \\
         &= 4 (\cos(\theta) - \pi \sin(\theta)) +
         \frac{e^{{-r^2}}}{2\pi} \sum^{\infty}_{n=0} \sum^{\infty}_{m=0, m \ne n}
         \frac{r^{n+m}}{\sqrt{n!}\sqrt{m!}} e^{-i(n-m)\theta}
         \int_{0}^{2\pi}
         e^{i (n-m) \phi} \phi^{2} d\phi \\
         &= 4 (\cos(\theta) - \pi \sin(\theta)) +
         \frac{e^{{-r^2}}}{2\pi} \sum^{\infty}_{n=0} \sum^{\infty}_{m=0, m \ne n}
         \frac{r^{n+m}}{\sqrt{n!}\sqrt{m!}} e^{-i(n-m)\theta}
         4 \pi \frac{-i \pi (n-m) + 1}{(n-m)^2} \\
         &= 4 (\cos(\theta) - \pi \sin(\theta)) +
         2 e^{{-r^2}} \sum^{\infty}_{n=0} \sum^{\infty}_{m=0, m \ne n}
         \frac{r^{n+m}}{\sqrt{n!}\sqrt{m!}} e^{-i(n-m)\theta}
         \frac{-i \pi (n-m) + 1}{(n-m)^2}
         \intertext{This sum can be simplified a bit further by rewriting the double
         sum as a single sum over just the values of m that are greater than n.}
         &= 4 (\cos(\theta) - \pi \sin(\theta)) +
         4 e^{{-r^2}} \sum^{\infty}_{n=0} \sum^{\infty}_{m=0, m > n}
         \frac{r^{n+m}}{\sqrt{n!}\sqrt{m!}} \frac{\cos((n-m)\theta)}{(n-m)^2}
      \end{align}
      Calculating the second term (just the integral which we'll square, later):
      \begin{align}
         &\int_{0}^{2\pi} \frac{\cos(\theta - \phi)}{\pi} \phi d \phi +
         \frac{e^{{-r^2}}}{2\pi} \sum^{\infty}_{n=0} \sum^{\infty}_{m=0, m \ne n}
         \frac{r^{n+m}}{\sqrt{n!}\sqrt{m!}}
         \int_{0}^{2\pi} e^{i (n-m) (\phi- \theta)} \phi d\phi \\
         &= -2 \sin(\theta) +
         \frac{e^{{-r^2}}}{2\pi} \sum^{\infty}_{n=0} \sum^{\infty}_{m=0, m \ne n}
         \frac{r^{n+m}}{\sqrt{n!}\sqrt{m!}} e^{-i(n-m)\theta}
         \int_{0}^{2\pi} e^{i (n-m) \phi} \phi d\phi \\
         &= -2 \sin(\theta) +
         \frac{e^{{-r^2}}}{2\pi} \sum^{\infty}_{n=0} \sum^{\infty}_{m=0, m \ne n}
         \frac{r^{n+m}}{\sqrt{n!}\sqrt{m!}} e^{-i(n-m)\theta}
         \frac{-2\pi i (n-m) }{(n-m)^2}\\
         &= -2 \sin(\theta) -
         e^{-r^2} \sum^{\infty}_{n=0} \sum^{\infty}_{m=0, m \ne n}
         \frac{r^{n+m}}{\sqrt{n!}\sqrt{m!}}
         \frac{e^{-i(n-m)\theta}}{(n-m)}\\
         &= -2 \sin(\theta) -
         2 e^{-r^2} \sum^{\infty}_{n=0} \sum^{\infty}_{m=0, m > n}
         \frac{r^{n+m}}{\sqrt{n!}\sqrt{m!}}
         \frac{\cos((n-m)\theta)}{(n-m)}\\
      \end{align}
      Squaring this term and subtracting it from the first term calculated for the
      variance will result in a lot of ugliness. However, we can see, immediately,
      that the 
   \end{homeworkSection}
\end{homeworkProblem}
