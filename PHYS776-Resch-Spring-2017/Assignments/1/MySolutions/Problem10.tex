% Problem 1.10
\begin{homeworkProblem}[Problem 10]
   \begin{homeworkSection}{a)}
      This derivation is documented online in a number of sources (e.g.
      \cite{schrodinger}. This work is my best attempt to reproduce my
      understanding of those s
   \end{homeworkSection}
   \begin{homeworkSection}{b)}
      Schrodinger's relation can be written in terms of expectation values of
      the various operators (over a coherent state $ \ket{\alpha} $) as follows:
      \begin{align}
         \Delta A^{2} \Delta B^{2} &=
         \left( \braket{q^2} - \braket{q}^2 \right)
         \left( \braket{p^2} - \braket{p}^2 \right) \\
         &\ge \left( \frac{1}{2} \braket{qp + pq} - \braket{q} \braket{p}
      \right)^2
      + \left( \frac{1}{2i} \braket{[q,p]} \right)^2
      \intertext{But, $ [q,p] = i $ and $ qp + pq = qp + (qp + [q,p])
      = i + 2qp$.}
      &=
      \left( \frac{1}{2} \braket{2qp + i } - \braket{q} \braket{p}
      \right)^2
      + \left( \frac{1}{2} \right)^2 \\
      &= \left(
   \braket{qp} + \frac{i}{2} - \braket{q} \braket{p} \right)^2 +
   \left( \frac{1}{2} \right)^2 \\
\end{align}
So, the problem has been reduced to calculating $ \braket{qp} $, $
\braket{q^2} $, $ \braket{p^2} $, $ \braket{q} $, and $ \braket{p} $.
\begin{align}
   % <qp>
   \braket{qp}_{\alpha} &= \braket{\alpha | qp | \alpha} \\
                        &=
   \braket{\alpha | \left(\frac{i(a+a^{\dagger})(a^{\dagger}-a)}{2} \right) |
   \alpha} \\
   &= \frac{i}{2} \braket{\alpha | a a^{\dagger} - a^{2} +
(a^{\dagger})^{2} - a^{\dagger}a | \alpha} \\
&= \frac{i}{2} \left( |\alpha|^2 - \alpha^{2} + \bar{\alpha}^2 - |\alpha|^{2}
\right) \\
&= \frac{i}{2} \left( -r^{2}e^{2 i \theta} + r^{2} e^{-2 i \theta} \right)
\text{\quad where $\alpha = r e^{i \theta}$ } \\
&= \frac{i}{2} \left( -2 i r^2 \sin(2\theta) \right) \\
&= r^{2} \sin(2\theta)
\end{align}
\begin{align}
   % <q^2>
   \braket{q^2} &= \frac{1}{2}
   \braket{\alpha |
   a^{2} + a^{\dagger}a + aa^{\dagger} + (a^{\dagger})^{2} | \alpha } \\
   &= \frac{1}{2}(\alpha^{2} + 2 \left| \alpha \right|^{2} + \bar{\alpha}^2 )\\
   &= r^{2} \cos(2\theta) + r^{2} \\
   &= r^{2} \left( 1 + \cos(2\theta) \right) \\
   &= 2 r^{2} \cos^{2}(\theta) \\ \nonumber \\
   % <p^2>
   \braket{p^2}
   &= \frac{-1}{2}
   \braket{\alpha |
   (a^\dagger)^{2} - a^{\dagger}a - aa^{\dagger} + a^{2} | \alpha } \\
   &= \frac{-1}{2}( \bar{\alpha}^2 - 2 \left| \alpha \right|^{2} + \alpha^{2}) \\
   &= r^{2} \left( 1 - \cos(2\theta) \right) \\
   &= 2 r^{2} \sin^{2}(\theta)
\end{align}
\begin{align} % <q>
   \braket{q}_{\alpha} &= \braket{\alpha | q | \alpha} \\
                       &= \braket{\alpha |
   \left(\frac{(a+a^{\dagger})}{\sqrt{2}} \right)
| \alpha} \\
&= \frac{1}{\sqrt{2}} \left( r e^{i \theta} + r e^{- i \theta} \right)
\text{\quad where $\alpha = r e^{i \theta}$ } \\
&= \frac{1}{\sqrt{2}} \left( 2 r \cos(\theta) \right) \\
&= \sqrt{2} r  \cos(\theta) \\ \nonumber \\
% <p>
\braket{p}_{\alpha} &= \braket{\alpha | q | \alpha} \\
                    &= \braket{\alpha |
\left(i\frac{(a^{\dagger}-a)}{\sqrt{2}} \right) | \alpha} \\
&= \frac{i}{\sqrt{2}} \left( r e^{-i \theta} - r e^{i \theta} \right)
\text{\quad where $\alpha = r e^{i \theta}$ } \\
&= \frac{i}{2} \left( - 2 i r \sin(\theta) \right) \\
&= \sqrt{2} r  \sin(\theta) \\ \nonumber \\
   \end{align}
   We are in a position, now, to evaluate the left-hand side of Schrcodinger's
   relation.
   \begin{align}
      \left( \braket{q^2} - \braket{q}^2 \right)
      \left( \braket{p^2} - \braket{p}^2 \right)  &=
      0
%      \left(
%         2 r^{2} \cos^{2}(\theta) - \left(\sqrt{2} r  \cos(\theta) \right)^2
%      \right)
%      \left(
%         2r^{2}\sin^{2}(\theta) - \left(\frac{r \sin(\theta)}{\sqrt{2}}\right)^2
%      \right) \\
%      &= r^4 \left(
%      4 \cos^{2}(\theta) \sin^{2}(\theta) -
%      \cos^{2}(\theta) \sin^{2}(\theta) -
%      \cos^{2}(\theta) \sin^{2}(\theta) -
%      \frac{\sin^{2}(\theta)\cos^{2}(\theta)}{4}
%   \right) \\
%   &= \frac{7r^{4}\sin^2(\theta)\cos^2(\theta)}{4}
   \end{align}
   Evaluating the right-hand side now:
   \begin{align}
      \left(
      \braket{qp} + \frac{i}{2} - \braket{q} \braket{p} \right)^2 +
      \left( \frac{1}{2} \right)^2  &=
      \left( r^{2} \sin(2\theta) + \frac{i}{2} - \sqrt{2}r \cos(\theta) \sqrt{2} r
      \sin(\theta) \right)^2 + \left( \frac{1}{2} \right)^{2} \\
      &= \left( r^{2}\left(\sin(2\theta) - 2 \cos(\theta)\sin(\theta)\right) +
   \frac{i}{2} \right)^2 + \left(\frac{1}{2}\right)^2 \\
   &= 0
   \end{align}
   So, the uncertainty (the left-hand side of the above expression) is zero and,
   as such, is minimum (the uncertainty is as strictly non-negative quantity.
   Additionally, equality of the Schr\"{o}dinger inequality is satisified implying these states are the
   states of minimum uncertainty.
   \begin{homeworkSection}{c)}
      To determine which single mode squeezed states are minimum in their
      uncertainties of $ p \text{and} q $ it is necessary to calculate those
      states which satisfy equality in Robertson's relation. Thus, we need to
      calculate
      \begin{align}
         \Delta q^2 \Delta p^2 \ge \left( \frac{1}{2i} \braket{[q,p]}
         \right)^2 \enskip.
      \end{align}
      But, $ [p,q] = i $. So, we just need to calculate
      \begin{align}
         \frac{1}{4} &=
         \left( \braket{q^2} - \braket{q}^2 \right)
         \left( \braket{p^2} - \braket{p}^2 \right) \\
         &= \left( \braket{0|S^{\dagger}(\xi)q^{2}S(\xi)|0} -
      \braket{0|S^{\dagger}(\xi)qS(\xi)|0} \right)
      \left( \braket{0|S^{\dagger}(\xi)p^{2}S(\xi)|0} -
         \braket{0|S^{\dagger}(\xi)pS(\xi)|0}
         \right) \enskip.
      \end{align}
      Now, calculating each of these terms involves much more algebra. It's
      possible (not difficult) to do, but it takes some time to write.
      Therefore, these results will be taken directly from the course notes. The
      results are listed below.
      \begin{align}
         \braket{q} &= \braket{0| S^{\dagger}(\xi) q S(\xi) | 0} = 0 \\
         \braket{p} &= \braket{0| S^{\dagger}(\xi) p S(\xi) | 0} = 0 \\
         \braket{q^2} &= \braket{0| S^{\dagger}(\xi) q^2 S(\xi) | 0} \\
                      &= \frac{1}{2}
         \left( \cosh(2r) - \sinh(2r)\cos(\theta) \right) \text{where $\xi
         \equiv r e^{i \theta}$}\\
         \braket{p^2} &= \braket{0| S^{\dagger} p^2 S | 0} \\
                      &= \frac{1}{2}
         \left( \cosh(2r) + \sinh(2r)\cos(\theta) \right)
      \end{align}
      So, the condition of interest can be written, now, as
      \begin{align}
         \frac{1}{4} &= \frac{1}{4} \left( \cosh^{2}(2r)
         -\sinh^2(2r)\cos^2(\theta) \right)
         \label{eq:robertson}
         \intertext{or}
         1 &= \left( \cosh^{2}(2r) -\sinh^2(2r)\cos^2(\theta) \right) \\
         \cosh^{2}(2r) - 1 &= \sinh^{2}(2r)\cos^2(\theta) \\
         \intertext{But, $ \cosh^2(2r) - 1 = \sinh^2(2r) $}
         \sinh^2(2r) &= \sinh^2(2r)\cos^2(\theta)
      \end{align}
      So, $ \theta = n \pi \enskip, n \in \mathcal{Z}$
   \end{homeworkSection}
   Thus, the squeezed states are of minimum uncertainty as long as the phase of
   the squeezing parameter is a multiple of $ \pi $. We can understand this as
   squeezing the state completely into the $ q $ (for $ \theta = 2 \pi n $) or $
   p $ (for $\theta = (2n+1) \pi$) quadrants.
\end{homeworkSection}
\begin{homeworkSection}{d)}
This problem strongly represents the most recently performed algebra. We can take
advantage of that to avoid some more work. The left-hand side of the
Schr\"{o}dinger equation is the same as the left-hand side of the Robertson
relation.
\begin{align}
   \frac{1}{4} \left( \cosh^{2}(2r) -\sinh^2(2r)\cos^2(\theta) \right) &\ge
   \left( \frac{1}{2} \left( 2 \braket{qp} - i \right) -
   \braket{q}\braket{p} \right)^2 + \left( \frac{i}{2i} \right)^2 \\
   &= \left( \frac{1}{2} \left( 2 \braket{qp} - i \right) -
\braket{q}\braket{p} \right)^2 + \left( \frac{i}{2i} \right)^2 \\
&= \frac{1}{4} \left( 2 \braket{qp} - i \right)^2 + \frac{1}{4} \\
&= \frac{1}{4} \left( 4 \braket{qp}^2 - 4i \braket{qp} - 1 \right)
+ \frac{1}{4} \\
&= \braket{qp}^2 - i \braket{qp} \\
\end{align}
Solving for $ \braket{qp} $:
\begin{align}
   \braket{qp} &= \braket{0 | S^{\dagger} qp S | 0} \\
               &= \braket{0 | S^{\dagger} p S S^{\dagger} q S | 0} \quad\text{by
inserting $ SS^{\dagger} = \mathds{1} $.} \\
\end{align}
Now, it will be important to evaluate $ S^{\dagger} p q S $.
\newcommand{\aTransformed}{a\cosh(r) - a^{\dagger} e^{i \theta}\sinh(r)}
\newcommand{\aDaggerTransformed}{a^{\dagger}\cosh(r) - a e^{-i \theta}\sinh(r)}
\begin{align}
   S^{\dagger}p q S &= S^{\dagger} \left( \frac{a + a^{\dagger}}{\sqrt{2}} \right)
   \left( \frac{i \left( a^{\dagger} - a \right)}{\sqrt{2}} \right) S \\
   &= \frac{i}{2}
   \left(
      S^{\dagger} \left(a a^{\dagger} - a^{2} + (a^{\dagger})^2 - a^{\dagger}a\right) S
   \right) \\
   &= \frac{i}{2}
   \left(
   S^{\dagger} a S S^{\dagger} a^{\dagger} S - S^{\dagger} a S S^{\dagger} a S +
S^{\dagger}a^{\dagger} S S^{\dagger} a^{\dagger} S - S^{\dagger} a^{\dagger} S
S^{\dagger} a S
   \right)
   \intertext{Using the results from the notes, again (3.121 and 3.122):}
   = \frac{i}{2} (
   &\left(\aTransformed\right)\left(\aDaggerTransformed\right) - \\
   &\left(\aTransformed\right)\left(\aTransformed\right) + \\
   &\left(\aDaggerTransformed\right)\left(\aDaggerTransformed\right) - \\
   &\left(\aDaggerTransformed\right)\left(\aTransformed\right)
   )\\
   &= \frac{i}{2}
   \bigg(\mathbf{a a^{\dagger} \cosh^{2}(r)} - a^{2} e^{-i \theta} \sinh(r) \cosh(r)
   - \left( a^{\dagger} \right)^{2} e^{i\theta} \sinh(r) \cosh(r)
   + \mathbf{a^{\dagger} a \sinh^2(r)} \\
   & - a^{2} \cosh^2(r) + \mathbf{a a^{\dagger} e^{i \theta} \sinh(r) \cosh(r)}
   + \mathbf{a^{\dagger} a e^{i\theta} \sinh(r) \cosh(r)}
   - (a^{\dagger})^2 e^{2 i \theta} \sinh^2(r) \\
   & + (a^{\dagger})^2 \cosh^2(r)
   - \mathbf{a^\dagger a e^{-i \theta} \cosh(r)\sinh(r)}
   - \mathbf{a a^{\dagger} e^{-i \theta} \sinh(r) \cosh(r)} + a^2 e^{-2 i \theta}
   \sinh^2(r) \\
   & - \mathbf{a^\dagger a \cosh^2(r)} + \left( a^{\dagger} \right)^2 e^{i \theta}
   \sinh(r) \cosh(r) + a^{2} e^{-i \theta} \sinh(r) \cosh(r)
- \mathbf{a a^\dagger \sinh^2(r)} \bigg)
\end{align}
Now, only the bolded terms survive an inner product like
$ \braket{n | S^{\dagger} qp S | n } $. So, we evaluate that expression, now:
\begin{align}
   \braket{ 0 | S^{\dagger} qp S | )}
   &=
   \frac{i}{2}
   \braket{0 |
      a a^{\dagger} \cosh^2(r) + a^\dagger a \sinh^2(r)
   | 0} \\
   &+
   \frac{i}{2}
   \braket{0 |
      a a^\dagger e^{i \theta} \sinh(r) \cosh(r)
      + a^{\dagger} a e^{i \theta} \sinh(r) \cosh(r)
   | 0} \\
   &-
   \frac{i}{2}
   \braket{0 | a^\dagger a e^{-i \theta} \cosh(r) \sinh(r)
      + a a^\dagger e^{-i \theta} \sinh(r) \cosh(r)
   | 0} \\
   &-
   \frac{i}{2}
   \braket{0 | a^\dagger a \cosh^2(r) + a a^\dagger \sinh^2(r) | 0}
   \intertext{Only the terms acting first with a creation operator survive.}
   &=
   \frac{i}{2}
   \braket{0 | a a^{\dagger} \cosh^2(r) | 0} \\
   &-
   \frac{i}{2}
   \braket{0 | a a^\dagger e^{i \theta} \sinh(r) \cosh(r) | 0} \\
   &+
   \frac{i}{2}
   \braket{0 | a a^\dagger e^{-i \theta} \sinh(r) \cosh(r) | 0} \\
   &-
   \frac{i}{2}
   \braket{0 | a a^\dagger \sinh^2(r) | 0} \\
   &= \frac{i}{2} \left( \cosh^2(r) - \sinh^2(r) \right)
   - \frac{i}{2} \left( 2 i \sinh(r) \cosh(r) \sin(\theta) \right) \\
   &= \frac{i}{2} + \frac{1}{2}\sinh(2\theta) \sin(\theta)
\end{align}
So, evaluating the right-hand side of Schr\"{o}dinger's inequality, now:
\begin{align}
   \braket{qp}^2 - i \braket{qp} &=
   \left( - \frac{1}{4} + \frac{i}{2} \sinh(2\theta) \sin(\theta) +
   \frac{1}{4} \sinh^2(2\theta)\sin^2(\theta) + \frac{1}{2} - \frac{i}{2} \sinh(2\theta)\sin(\theta) \right)
   \\
   &= \frac{1}{4} \left( \sinh^2(2\theta) \sin^2(\theta) + 1 \right) \\
   &= \frac{1}{4} \left( \sinh^2(2\theta) \left( 1 - \cos^2(\theta) \right) + 1
\right) \\
   &= \frac{1}{4} \left( \left( \sinh^2(2\theta) - \sinh^2(2\theta)
   \cos^2(\theta) \right) + 1 \right) \\
   &= \frac{1}{4} \left( \cosh^2(2\theta) - \sinh^2(2\theta) \cos^2(\theta) \right)
\end{align}
This last expression is the exact same as the left-hand side of Robertson's
relation in Eq.~\ref{eq:robertson} (the right-hand side of
Eq.~\ref{eq:robertson} is the left-hand side of Robertson's relation). This is
the same as the left-hand side of Schr\"{o}dinger's relation. So, the squeezed
states are minimum uncertainty states for all $ \xi $.
\end{homeworkSection}
\end{homeworkProblem}
